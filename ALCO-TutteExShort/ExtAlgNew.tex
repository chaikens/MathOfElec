%% The class cedram-ALCO is just a wrapper around amsart.cls (version 2)
%% implementing the layout of the journal, and some additionnal
%% administrative commands. 
%% You can place one option:
%% * "Unicode" if the file is UTF-8 encoded.
\documentclass[Unicode]{cedram-alco}


\usepackage{xcolor}


%% Here you might want to add some standard packages if those
%% functionnalities are required.
%\usepackage[matrix,arrow,tips,curve]{xy}
% ...
%% The production will anyway use amsmath (all ams utilities except
%% amscd for commutative diagrams which you need to load explicilty if
%% required), hyperref, graphicx, mathtools, enumitem...

%% User definitions if necessary...  Such definitions are forbidden
%% inside titles, abstracts or the bibliography.
\DeclarePairedDelimiter\abs{\lvert}{\rvert} %Something useful only for this sample's sake: you can erase this line in your file (or find it useful...)
%% The title of the paper: amsart's syntax. 
\title
%% The optionnal argument is the short version for headings.
[Exterior Algebra Tutte Functions]
%% The mandatory argument is for the title page, summaries, headings
%% if optionnal void.
{An Exterior Algebra Valued Tutte Function on Linear Matroid Pairs}

%% Authors according to amsart's syntax + distinction between Given
%% and Proper names:
\author[\initial{S.} \middlename{} Chaiken]{\firstname{Seth} \middlename{} \lastname{Chaiken}}

%% Do not include any other information inside \author's argument!
%% Other author data have special commands for them:
\address{University at Albany\\
  State University of New York\\
Computer Science Department\\
1400 Washington Avenue\\
Albany, NY 12222 (USA)}

%% Current address, if different from institutionnal address
\curraddr{18 Eileen Street\\
Albany, NY 12203 (USA)}


%% e-mail address
\email{schaiken@albany.edu}

%% possibly home page URL (not encouraged by journal's style)
%\urladdr{https://en.wikipedia.org/wiki/Marin\_Mersenne}

%% Acknowledgements are not a footnote in
%% \author, but are given apart:
%\thanks{The author was partially supported by a special grant for Junior Woodchucks.}


%% If co-authors exist, add them the same way, in alaphabetical order
%\author{\firstname{Joseph}  \lastname{Fourier}}
%\address{Universit\'e de  Grenoble\\
% Institut Moi-m\^eme\\
% BP74, 38402 SMH Cedex (France)}
%\email{fourier@fourier.edu.fr}

% Key words and phrases:
\keywords{Tutte Functions, Exterior Algebra, Laplacian, Electrical Networks}
  

%% Mathematical classification (2010)
%% This will not be printed but can be useful for database search
\subjclass{10X99, 14A12, 11L05}

%mycommands

%Matroid Variable--converts a matroid to a variable.
\newcommand{\MVar}[1]{
  \setlength\fboxrule{1.5pt}
  \setlength\fboxsep{1.5pt}
  {\;\framebox{\ensuremath{#1}}\;}}


\newcommand{\ext}[1]{\ensuremath{\mathbf{#1}}}
%\newcommand{\extvee}{\ensuremath{\mathbf{\vee}}}
\newcommand{\extvee}{\;\;}
\newcommand{\Plucker}{Pl\"{u}cker\ }

\newcommand{\Nal}{\ensuremath{N_{\alpha}}}
\newcommand{\NbePe}{\ensuremath{N_{\beta}^{\perp}}}
\newcommand{\eNal}{\ensuremath{\ext{N}_{\alpha}}}
\newcommand{\eNbePe}{\ensuremath{\ext{N}_{\beta}^{\perp}}}
\newcommand{\eNbe}{\ensuremath{\ext{N}_\beta}}
\newcommand{\Nbe}{\ensuremath{N_\beta}}

\newcommand{\Is}{\ensuremath{\iota}}
\newcommand{\Vs}{\ensuremath{\upsilon}}


%Emphasize in color!
\newcommand{\Remph}[1]{{\color{red}#1}}
\newcommand{\Bemph}[1]{{\color{blue}#1}}

\newcommand{\alert}[1]{{\color{red}#1}}



%\newcommand{\dunion}{\uplus}
\newcommand{\dunion}{\coprod}

\newcommand{\extLVert}[2]{\ext{L}\left( \begin{array}{c} {#1}\\ {#2} \end{array} \right)}
\newcommand{\extLVertSub}[3]{\ext{L}_{#1}\left( \begin{array}{c} {#2}\\ {#3} \end{array} \right)}
\newcommand{\extLHor}[2]{\ext{L}\left( {#1}; {#2} \right)}
\newcommand{\extLHorSub}[3]{\ext{L}_{#1}\left(  {#2}; {#3}  \right)}

\newcommand{\LVert}[2]{\ext{L}\left( \begin{array}{c} {#1}\\ {#2} \end{array} \right)}
\newcommand{\LVertSub}[3]{\ext{L}_{#1}\left( \begin{array}{c} {#2}\\ {#3} \end{array} \right)}
\newcommand{\LHor}[2]{\ext{L}\left( {#1}; {#2} \right)}
\newcommand{\LHorSub}[3]{\ext{L}_{#1}\left(  {#2}; {#3}  \right)}




\definecolor{Blue}{rgb}{.255,.41,.884} % RoyalBlue of svgnames
\definecolor{Red}{rgb}{1, 0, 0} % Red of svgnames
\definecolor{Green}{rgb}{.196,.804,.196} % LimeGreen of svgnames
\definecolor{Yellow}{rgb}{1,.648,0} % Orange of svgnames





\begin{document}
%% Abstracts must be placed before \maketitle
\begin{abstract}
 The matrix tree theorem expresses the basis enumeration Tutte function
(spanning tree count of connected graphic matroids) as a
determinant--a minor of the graph's Laplacian. We will generalize the
choice of a root vertex for the trees by distinguishing a fixed set $P$
of $p$ matroid elements, which we will call ports, a word from
engineering. We then define a function from $K$-linear matrix
representations of matroid pairs ($N_{\alpha},N_{\beta})$
(often $N_{\alpha}=N_{\beta}$) whose ground sets
include $P$, into pure (decomposible)
exterior algebra elements which represent points in the Grassmannian of $p$ dimensional $K$-linear subspaces
of $K^{(2p)}$. We express this subspace value as a decomposible
(i.e. product of vectors) element in the exterior algebra (of
anti-symmetric tensors) over $K^{(2p)}$, in a standard way.  The result is
a restricted or set-pointed Tutte function $\ext{L}_E(N_{\alpha},N_{\beta})$
that
satisfies sign-corrected forms of the two familiar identities for
deletion/contraction of elements not in $P$, and for disjoint
union. Note that these identities are in exterior algebra, whose
multiplication is anti-commutative. This generalizes the all-minors
directed graph matrix tree theorem (proved combinatorially by the
author) because those minors are the coefficients in the expansion of
$T$'s value over the standard basis--they are the standardized \Plucker
coordinates for the dimension $p$ subspace $T$'s value represents.  An
unusual feature for this Tutte-like function is that when $K$ is ordered
so the matroids are oriented, $T$ can distinguish different orientations
of the same matroid.  
\end{abstract}


\maketitle

% First paragraph after a section is not indented. If there is text
% below the title before the first section, it should be unindented
% like this.
\noindent
\input{gitcommit}



\section{Introduction}


The subject of Tutte functions is one of the most prominent
in classical and contemporary discrete mathematics. Recent
developments have been made in algebraic frameworks which provide
grounds for various function domains and analogies,
and for analysis of convolutions.
It might therefore be interesting to see some Tutte functions valued
in exterior algebras.  Unlike the classical and recent frameworks
we are aware of, 
exterior algebras \cite{MarcusFDMuAlPt2} have \emph{anticommutative} multiplication.

As usual, a matroid representation is a subspace
of a vector space equipped with a particular basis $S$
identified with the matroid's ground set.  Let $P \subseteq S$.
Our exterior
algebra valued Tutte function emerges as a restriction, to represented matroids,
of a specialization of
the so-called relative, set-pointed, or what we call ``$P$-ported''
Tutte functions \cite{MR0419272,SetPointedLV,sdcPorted,TutteEx,RelTuttePolyDiaoHetyei}.
These functions obey the deletion-contraction
identity only for $e\not\in P$.  When the matroid is regular and
the representation is the row space of a full-row rank unimodular matrix $N$, and $P=\emptyset$,
our function reduces to the basis counting Tutte function; it is the determinant
of $N N^t$.
Further, let $N$ be the reduced $\pm 1$-entry incidence matrix of a graph so the
matroid is graphic.  Now well-known Matrix-Tree theorem
is fact that the determinant $|N N^t|$ of the
reduced Laplacian matrix equals the number of bases, i.e. maximal forests.

The directed graph version of the Matrix-Tree theorem asserts rooted maximal directed forests
(sometimes called branchings of arborescences) are emumerated by 
$|N_1 (N_2)^t|$.  Here, the 0-1 matrix $N_2$ encodes the incidences of tails of edges to
vertices except for the root vertices, and $N_1$ is the $\pm 1$-entry incidence matrix with
the rows corresponding to the root vertices deleted.  These facts about the Laplacian and
its directed analog naturally lead us to construct our
Tutte functions for \emph{pairs} of matroid representations.  

Let the pure tensor $\mathbf{N}$
determined $\mathcal{N}$ be defined
by the exterior product 
of the rows of $N$, up to non-zero $K$ multiples.  Its 
decomposition begins but does not completely fulfill a Tutte function analogy:
\begin{equation}\label{extDaC}
  \text{for all } e \in E, \mathbf{N} =
  \mathbf{e}\wedge(\mathbf{N}/\mathbf{e}) +
  (\mathbf{N}\setminus \mathbf{e}).
\end{equation}
These \emph{contraction} and \emph{deletion} operations are defined up to sign
by extending the vector space operations of projecttion and quotient
(by subspace $<\mathbf{e}>$) to exterior algebra.  (We will adopt comventions
to make (\ref{extDaC}) an identity.)


\bibliographystyle{amsplain-ac}
\bibliography{ExtAlgTFLMatPairs}


\end{document}

