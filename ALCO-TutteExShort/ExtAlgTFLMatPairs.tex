%% The class cedram-ALCO is just a wrapper around amsart.cls (version 2)
%% implementing the layout of the journal, and some additionnal
%% administrative commands. 
%% You can place one option:
%% * "Unicode" if the file is UTF-8 encoded.
\documentclass[Unicode]{cedram-alco}
%% Here you might want to add some standard packages if those
%% functionnalities are required.
%\usepackage[matrix,arrow,tips,curve]{xy}
% ...
%% The production will anyway use amsmath (all ams utilities except
%% amscd for commutative diagrams which you need to load explicilty if
%% required), hyperref, graphicx, mathtools, enumitem...

%% User definitions if necessary...  Such definitions are forbidden
%% inside titles, abstracts or the bibliography.
\DeclarePairedDelimiter\abs{\lvert}{\rvert} %Something useful only for this sample's sake: you can erase this line in your file (or find it useful...)
%% The title of the paper: amsart's syntax. 
\title
%% The optionnal argument is the short version for headings.
[Exterior Algebra Tutte Functions]
%% The mandatory argument is for the title page, summaries, headings
%% if optionnal void.
{An Exterior Algebra Valued Tutte Function on Linear Matroid Pairs}

%% Authors according to amsart's syntax + distinction between Given
%% and Proper names:
\author[\initial{S.} \middlename{} Chaiken]{\firstname{Seth} \middlename{} \lastname{Chaiken}}

%% Do not include any other information inside \author's argument!
%% Other author data have special commands for them:
\address{University at Albany\\
  State University of New York\\
Computer Science Department\\
1400 Washington Avenue\\
Albany, NY 12222 (USA)}

%% Current address, if different from institutionnal address
\curraddr{18 Eileen Street\\
Albany, NY 12203 (USA)}


%% e-mail address
\email{schaiken@albany.edu}

%% possibly home page URL (not encouraged by journal's style)
%\urladdr{https://en.wikipedia.org/wiki/Marin\_Mersenne}

%% Acknowledgements are not a footnote in
%% \author, but are given apart:
%\thanks{The author was partially supported by a special grant for Junior Woodchucks.}


%% If co-authors exist, add them the same way, in alaphabetical order
%\author{\firstname{Joseph}  \lastname{Fourier}}
%\address{Universit\'e de  Grenoble\\
% Institut Moi-m\^eme\\
% BP74, 38402 SMH Cedex (France)}
%\email{fourier@fourier.edu.fr}

% Key words and phrases:
\keywords{Tutte Functions, Exterior Algebra, Laplacian, Electrical Networks}
  

%% Mathematical classification (2010)
%% This will not be printed but can be useful for database search
\subjclass{10X99, 14A12, 11L05}


\newcommand{\ext}[1]{\ensuremath{\mathbf{#1}}}
%\newcommand{\extvee}{\ensuremath{\mathbf{\vee}}}
\newcommand{\extvee}{\;\;}
\newcommand{\Plucker}{Pl\"{u}cker\ }

\newcommand{\Nal}{\ensuremath{N_{\alpha}}}
\newcommand{\NbePe}{\ensuremath{N_{\beta}^{\perp}}}
\newcommand{\eNal}{\ensuremath{\ext{N}_{\alpha}}}
\newcommand{\eNbePe}{\ensuremath{\ext{N}_{\beta}^{\perp}}}

%\newcommand{\dunion}{\uplus}
\newcommand{\dunion}{\coprod}








\begin{document}
%% Abstracts must be placed before \maketitle
\begin{abstract}
 The matrix tree theorem expresses the basis enumeration Tutte function
(spanning tree count of connected graphic matroids) as a
determinant--a minor of the graph's Laplacian. We will generalize the
choice of a root vertex for the trees by distinguishing a fixed set $P$
of $p$ matroid elements, which we will call ports, a word from
engineering. We then define a function from $k$-linear matrix
representations of matroid pairs ($N_{\alpha},N_{\beta})$ (often
$N_{\alpha}=N_{\beta}$ whose ground sets
include $P$, into pure (decomposible)
exterior algebra elements which represent points in the Grassmannian of $p$ dimensional $k$-linear subspaces
of $k^{(2p)}$. We express this subspace value as a decomposible
(i.e. product of vectors) element in the exterior algebra (of
anti-symmetric tensors) over $k^{(2p)}$, in a standard way.  The result is
a restricted or set-pointed Tutte function $\ext{F}_E(N_{\alpha},N_{\beta})$
that
satisfies sign-corrected forms of the two familiar identities for
deletion/contraction of elements not in $P$, and for disjoint
union. Note that these identities are in exterior algebra, whose
multiplication is anti-commutative. This generalizes the all-minors
directed graph matrix tree theorem (proved combinatorially by the
author) because those minors are the coefficients in the expansion of
$T$'s value over the standard basis--they are the standardized Plucker
coordinates for the dimension $p$ subspace $T$'s value represents.  An
unusual feature for this Tutte-like function is that when $k$ is ordered
so the matroids are oriented, $T$ can distinguish different orientations
of the same matroid.  
\end{abstract}


\maketitle

% First paragraph after a section is not indented. If there is text
% below the title before the first section, it should be unindented
% like this.
\noindent
The aim ... .

\section{Introduction}

\noindent
Tutte functions are a well-known, long-standing idea
for which new applications, variations and generalizations
continue to be active research subjects. For example,
Krajewski, Moffatt and Tanasa give a common Hopf algebra framework
for the deletion and contraction operations for many of
these variations\cite{KRAJEWSKI2018271}.
See also Crapo and Schmitt's work
on the Hopf algebra of matroids and call
for attention to ``naturally occurring algebraic
structures'' in matroid theory\cite{CRAPO20051066}.  The target for us
is exterior algebras, which occur naturally when problems of
Kirchhoff's and Ohm's law electrical network solving and their
generalizations are studied from a matroid theory point of view,
including a view from oriented matroids.  Besides its application
to differential forms, exterior algebra is
the algebra of the posing and solution of equations in elementary
linear algebra.
Our most compelling reason for connecting
Tutte functions with an exterior algebra in lieu of a commutative algebra is that
products in the former represent linear subspaces. The examples of such subspaces
we give comprise solutions to problems posed from graphs generalizing the inversion
of graph Laplacians.  To connect these with Tutte decomposition, we will make
refinements to the machinery of exterior algebra, also called Grassmann algebra,
developed by Marcus\cite{MarcusFDMuAlPt2}.  Our refinements are used to
deal with the fact that the ground set of a matroid's minor is a proper subset
of that matroid's ground set.

One of the earliest Tutte functions is the basis enumerator for matroids. It
is a determinant in the case of graphic, and more generally, regular matroids;
this is a consequence of the Cauchy-Binet theorem.
The well-known Matrix Tree theorem tells us this determinant is (up to sign)
any full-rank minor of the graph's Laplacian matrix. More generally,
every minor is an enumerator for forests (signed, in some non-principal minor
cases) satisfying the following condition: Each vertex indexing
a deleted row is in exactly one of the trees, and similary for each
vertex indexing a deleted column\cite{sdcMTT}. (Each forest's sign depends on the sign of the
matching the forest determines between the deleted row vertices and
deleted column vertices; of course when the minor is principal,
the only such matching is the identity.)
We give a key observation: \emph{Each} such
minor, when $e$ is a given edge, equals the sum of the corresponding
minor when $e$ is deleted plus the determinant expansion terms for
forests that contain $e$.  The latter can be also be expressed as a determinant.
In the following, we will develop a formulation in which
\emph{all these minors together},
when they are constituted as an element in an exterior algebra, satisfy the
well-known Tutte identities.  The formulation applies to pairs of
row-spaces of matrices, with rank conditions for non-triviality,
and our Tutte decomposition operates on matrix
columns as does (possibly oriented)
matroid deletion and contraction.

In order for a tuple of determinants to be a Tutte function value, one
must label them in a combinatorial way.  For this purpose, we use
the \emph{set-pointed}\cite{TPMorphMatI99,TPMorphMatI99},
also called the
\emph{relative}\cite{RelTuttePolyDiaoHetyei} or
\emph{ported}\cite{sdcPorted,TutteEx}
variation of Tutte functions.
Let $P$ be a set.  A function $F$ is a $P$-ported
when the identity for deletion and contraction of $e$ is asserted,
and rules for reducing loops or coloops apply, only when $e \not\in P$.
$P$ will play the role of vertices in our all-minors of the Laplacian
matrix tree theorem specialization.
We prefer the engineering term ``port'' because it eludes to both an object through
which systems can be interconnected,
and can be manipulated and observed by their
environments\cite{Recski,narayanan1997submodular}.
Engineers would call our graph theory example (\ref{ExamK4}) a ``$p$-port'', with $p=2$.

Let $P_{\alpha}$ and $P_{\beta}$ be disjoint copies of $P$.  Let $K$ be a field.
We will see that our Tutte function's value will be an exterior product of
$|P|$ vectors in the vector space over $K$ with basis $P_\alpha \cup P_\beta$.
Such exterior algebra products are sometimes called ``decomposibles''\cite{MarcusFDMuAlPt2}
and are ``pure'' when considered to be anti-symmetric tensors.  Our value will
have grade level or ``step''  $|P|$ in the exterior algebra over our
vector space.
As such, (when non-zero)
it is a \emph{Grassmann representitive}\cite{MarcusFDMuAlPt2}[p 18]
for some
dimension $P$ subspace, a point in the Grassmannian $Gr(|P|,2|P|)$.
However, in order for Tutte's additive identity to make sense, the value cannot be
a point in the $\binom{2|P|}{|P|}-1$ dimensional projective space
containing the variety $Gr(|P|,2|P|)$.  It must be in a module, which for
our construction is a linear space.

A possibly novel feature is that unlike the usual commutative ring values for Tutte
functions, our functions' values, being in an exterior algebra, are graded and have
\emph{anticommutative} multiplication.
Hence a sign correction must be included in Tutte's multiplicative
identity.  Our Tutte function merely reduces to the parametrized
basis enumerator obtained from setting $u=v=0$ in Tutte
decomposition with coloop $e$ valued at $g_e+r_eu$ and loop $e$ valued at $r_e+g_ev$.
``Porting'' is necessary for our approach to yield anything
interesting. To find a Tutte function, like the Whitney rank-nullity
generating function form in variables $u,v$, that reduces to our exterior
algebra version when $u=v=0$ is an interesting question.
Our
references\cite{MR0419272,SetPointedLV,sdcPorted,TutteEx,RelTuttePolyDiaoHetyei}
document that all the fundamental features
(expansions, universality, etc.)
of Tutte function theory extend to porting/set-pointing/relativization.  Throughout,
we restrict our attention to the ``normal'' parametrized Tutte functions which, according
to classifications given by Zaslavsky\cite{MR93a:05047},
and Bollobas and Riordan\cite{BollobasRiordanTuttePolyColored},
have rank-nullity generating function interpretations. 

Another possibly interesting feature is that, in a rather trivial way,
our Tutte function can distinguish different orientations of an oriented
matroid.  When decomposition defined by our Tutte identities is applied to
a represented oriented matroid, the irreducibles are representations of
some of its minors for which their ground set only contains elements in $P$.   Hence different
orientations of the same matroid minor are distinguished.  Recently,
a new Tutte function variation that distinguishes properties of orientations
more finely was found\cite{AwanBernardiOMTuttePre}.


The theory works when parameters $r_e$, $g_e$ are given for each element $e\not\in P$. We
will have, formally,
\[
   F(N) = g_eF(N/e)+r_eF(N\backslash e) \text{ if\ }e\text{ can be deleted and contracted, and }e\not\in P.
   \]
Indeed, our chief motivation is to give an algebraic combinatorial description of solutions
to electrical network problems, where the parameters will represent the coefficient
in Ohm's law.  It is convenient (and elegant) to write that law in homogeneous form:
For each resistor $e$ there
are two parameters $r_e,g_e$
(``proresistance'', ``proconductance'' apparently originated in \cite{SmithElec} and  used in
\cite{TutteEx,CirThProjHomo2019}) such that $i_e$ is the current through the resistor and $v_e$ is
the voltage drop across the resistor (in the direction of the current) if and only if there
exists a value $x_e$ for which $i_e=g_ex_e$ and $v_e=r_ex_e$.  In other words, when they
are non-zero and not infinite, the conventional
resistance is $r_e/g_e$ and conductance is $g_e/r_e$. Shorts (zero resistance) and opens (zero
conductance) are thus accomodated.   To make this long story short (see section ()),
our Tutte function's value
will encode the set of voltage drops across and current flows through each of
the network edges in $P$ that are compatible with Kirchhoff's and Ohm's laws applied to $E\cup P$. This set
is the linear solution space of an underdetermined system projected into the
dimension $2|P|$ space with a voltage ($v_{p}$) and a current ($i_{p}$) coordinate for each $p\in P$.

Thus, the subspace represented by our Tutte function's value is parametrized.  This opens
questions for research into the topological and geometric properties of the set of those subspaces
as the parameters range say over $Reals$ or $Complexes$, say for a given graph or pair of
matrices.



\section{Setup and Theorem}

\noindent
We begin by defining how we index columns of matrices (with a set of
linear space basis elements) and then
present specific
exterior algebra products that encode their row spaces and support
combinatorial operations where a distinguished set of indices $P$ is given.
We demonstrate this with the example
of the electrical network equations and their solutions.
The Tutte function is defined
on those pairs of exterior products, which might have been given abstractly.
It is important to recognize that two exterior algebra elements are given,
possibly by a suitable $\binom{n}{r}$ tuple of components, not the point in the Grassmannian
whose \Plucker coodinates are equivalence classes of non-zero multiples of
such a tuple.  We \emph{do} distinguish different such multiples.  Another way
to put it is that our data and constructed objects are matrices modulo left multiplication
by $SL_r$, not by $GL_r$ the latter which gives the Grassmannian.  Of course all such matrix
classes surject onto the Grassmannian.



Let $\Nal$ and $\NbePe$ be two full row rank matrices (over field $K$)
with
columns indexed by $P\cup E$, $P\cap E=\emptyset$, for which
$\text{rank}(\Nal)+\text{rank}(\NbePe)=|E|+|P|$.  Let
$G=\text{diag}\{g_e\}_{e\in E} $ and $R=\text{diag}\{r_e\}_{e\in E} $
be diagonal matrices of parameters.  For any such matrix $N$, $N(E)$ and
$N(P)$ denote the submatrices with columns $E$ and $P$ respectively. With
two disjoint copies of $P$ and bijections
$P \leftrightarrow P_{\alpha}\leftrightarrow P_{\beta}$, define the matrix
with columns indexed by $P_\alpha \dunion P_\beta \dunion E$, where each entry
is in $K$ or is a $K$ multiple of a parameter:
\[
    L = L\left( \begin{array}{c} \Nal\\ \NbePe \end{array} \right)
    = \left[\begin{array}{c|c|c} \Nal(P)  &  0  &  \Nal(E)G \\  \hline
        0  & \NbePe(P)  &  \NbePe(E)R \end{array}\right].
\]

The remaining two steps below will simply take the exterior product of the rows of $L$, and
then, for the value, extract the terms that are divisible by $\wedge_{e \in E}\ext{e}$ when the product
is expanded in terms of the basis generated by $P_\alpha \dunion P_\beta \dunion E$.
This is too simple because it remains unclear how to define the construction for the
``minors'' (to be defined) which will have smaller sets in place of $E$. The
construction must be consistent on the minors in order that the values
of our function on those minors satisfy the Tutte identities. 

An equivalent simplified construction is to project the solution space $\{z\}$ of $Lz=0$ onto
the $P_\alpha\cup P_\beta$ coordinates and then take the exterior product of some basis
for that projection's orthogonal complement.  In other words, we eliminate all the variables
in $Lz=0$ indexed by $E$ and take the exterior product of the resulting matrix rows.
Unfortunately, these views define an exterior product only
up to a $K$ multiple---nice to solve for some of the remaining $z$'s in terms of
the others, or to get a point in a Grassmannian, but a failure
for additive or multiplicative identities.


\subsection{Definition of a Tutte Function of $P$-Ported Pairs}


We need 
some careful formalism so we can algebraically relate the resulting Tutte function value
to its values for our to-be-defined minors and direct product factors.  Those 
will have a proper subset for $E$ and possibly so for $P$.

Let $U$ be the vector space $K(g_e,\ldots;r_e,\ldots)P_\alpha
\dunion P_\beta \dunion E$ with distingushed basis $P_\alpha
\dunion P_\beta \dunion E$.  Vectors and other elements of the
exterior algebra $\wedge U$ will be denoted by boldface
characters, such as $\ext{e}_j$, $\ext{p_\alpha}_k$,
$\ext{p_\beta}_l$ $\in$ $P_\alpha \dunion P_\beta \dunion E$.
    
Let $p = |P|$, $n=|E|$ and $L_i$ be row $i$ of $L$ for $i\in\{1,\ldots,n+p\}$.
We form $\ext{L}$ by exterior multiplication (in order) of the vectors each given by the
rows of $(n+p)\times (n+2p)$ matrix $L$ for the coefficients when written in terms of
basis $P_\alpha \dunion P_\beta \dunion E$:
\begin{equation}\label{defExtL}
  \ext{L} = \wedge_{i=1}^{n+p}L_i[\ext{p_\alpha}_1\ \cdots\ \ext{p_\alpha}_{p}\ \ext{p_\beta}_1\ \cdots\ \ext{p_\beta}_{p}\
    \ext{e}_1\ \cdots\ \ext{e}_{n}]^t. 
\end{equation}

We can now define what will turn out to be our Tutte function.  One cannot say what it means for
it to be a Tutte function until we get to defining the deletion and contraction operations!

Let us fully expand $\ext{L}$ into a polynomial in anti-commuting variables
$P_\alpha \dunion P_\beta \dunion E$.  Since $\ext{v}\wedge\ext{w}=-\ext{w}\wedge\ext{v}$ for vectors in
exterior algebra, we can write
\[
\ext{L} = (\ext{L}_E)\wedge (\ext{e_1}\wedge\ext{e_2}\cdots\wedge\ext{e_n}) + \ext{J}.
\]
We will abbreviate expressions like $(\ext{e_1}\wedge\ext{e_2}\cdots\wedge\ext{e_n})$ by
$(\ext{e_1}\ext{e_2}\cdots\ext{e_n})$ or $\ext{E}$ where \emph{set $E$ is specified as a sequence of its elements}.
So, given $(N_\alpha,N_\beta^\perp)$ as above, $\ext{L}_E(N_\alpha,N_\beta^\perp)$ is defined by (\ref{defExtL}) where
$\ext{J}$ is the sum of terms none of which is a multiple of $\ext{E}$.
In other words, $\ext{L}_E$ is the image of the homomorphism
$\Gamma_E:\wedge^{(2p+n)}U\rightarrow\wedge^{(2p)}U$ defined on the basis of
products of vectors corresponding to $P_\alpha \dunion P_\beta \dunion E$ to that
of $P_\alpha \dunion P_\beta$ as follows:
\[
   \begin{array}{ccc}
     \ext{Z}\ext{E} & \longmapsto & \ext{Z} \\
     \ext{Z}\ext{E'} & \longmapsto & 0
   \end{array}
\]
where $Z\subseteq P_\alpha \dunion P_\beta$ and $E'\subsetneqq E$.

\begin{prop}
  $\ext{L}_E(N_\alpha,N_\beta^\perp)$ is invariant under left actions of $SL_r$ on $N_\alpha$ or $N_\beta^\perp$.
\end{prop}
\begin{proof}
  All the coefficients are products of an all row minor of $N_\alpha$, an all-row minor of $N_\beta^\perp$, a sign, and
  parameters.
  By definition, the actions of $SL_r$ will multiply those determinants by 1. 
\end{proof}

We can therefore use the similarly formed exterior products
(respecting order) of the rows of $N_\alpha$ and $N_\beta^\perp$
for our data, denote them by 
$\ext{N_\alpha}$ and $\ext{N_\beta^\perp}$
and justify

\begin{defi}
  Given $E$, $P$ and $(\ext{N_\alpha},\ext{N_\beta^\perp})$ as above, $\ext{L}_E(\ext{N_\alpha},\ext{N_\beta^\perp})$ is the
  unique exterior algebra element satisfying
  \[
  \ext{L}(\ext{N_\alpha},\ext{N_\beta^\perp}) =
  (\ext{L}_E)\wedge (\ext{e_1}\wedge\ext{e_2}\cdots\wedge\ext{e_n}) + \ext{J} =
  (\ext{L}_E(\ext{N_\alpha},\ext{N_\beta^\perp}))\ext{E}+\ext{J},
  \]
where $\ext{J}$ is the sum of terms none of which is a multiple of
$\ext{E}=\ext{e_1}\wedge\ext{e_2}\cdots\wedge\ext{e_n}$, when
$\ext{L}$ is expanded using the basis generated by $P_\alpha \dunion
P_\beta \dunion E$.  In other words,
  \[ 
  \ext{L}_E(\ext{N_\alpha},\ext{N_\beta^\perp}) =\Gamma_E(\ext{L}(\ext{N_\alpha},\ext{N_\beta^\perp})).
  \]
\end{defi}


\subsection{Example and an interpretation of $\ext{L}_E(\ext{N_\alpha},\ext{N_\beta^\perp})$}\label{ExamK4}

Let $P=\{p_1, p_2\}$ and $E=\{e_1, e_2, e_3, e_4, e_5\}$ together be the edges in the (oriented) graph
below representing an electrical network, where $E$ represents the resistors.  The currents and voltages
in all these edges are variables $i_t$ and $v_t$ for $t\in P\cup E$.  Ohm's law
asserts that for $e\in E$, the ratio voltage-to-current ratio
$v_e:i_e$ $=$ $r_e:g_e$, the ratio of the parameters. The matrix $N_\alpha$ is the
usual ``signed'' vertex-edge incidence matrix which represents our graph's graphic matroid, with one
row deleted so $N_\alpha$ has full row rank.  The rows of $N_\alpha$ are the coefficients in
Kirchhoff's current law, asserting that the edge currents constitute a 1-cycle, (i.e., a flow.)
Matrix $N_\beta^\perp$ is a totally unimodular matrix whose
rows are a basis for the 1-cycle space, obtained by coding the incidences of edges with the oriented fundamental
circuits each associated to an edge not in a fixed spanning tree.  It's rows are the coefficients in
Kirchhoff's voltage law, asserting that the edge
voltages drops constitute a 1-coboundary, (i.e., differences
of a potential across edges.)
This well-known construction gives
two full row rank totally unimodular matrices whose row spaces are orthogonal complements,
representing the graph's
graphic and cographic matroids respectively.

We form the following system of equations from ().
\[
cool sys of eq
\]
The electrical network problem at hand is to determine linear constraints on the
port variables $i_p, v_p$, $p\in P = \{p_1, p_2\}$ imposed by the system.  In other words,
we want a linear map on the $U$-vector space with basis $P_\alpha, P_\beta$ whose kernel
is the projection of this system's solution space.


\subsection{An easy but weak Tutte function variation}

This is apparent:  Take any $e\in E$, any parameter
values and $E'=E\setminus e$. Generalizing the example's discussion tells us that all three of
(1) $\ext{L}_E(\ext{N_\alpha},\ext{N_\beta^\perp})$,
(2) $\ext{L}_{E'}(\ext{N_\alpha}/e,\ext{N_\beta^\perp\setminus e})$ and
(3) $\ext{L}_{E'}(\ext{N_\alpha}\setminus e,\ext{N_\beta^\perp/e})$ represent
\Plucker coordinates, because they are representatives for the \Plucker coordinates
of the row spaces of the $F$'s from (1), (2) and (3) respectively.  Since \emph{for each determinant
$D_1$, $D_2$ and $D_3$ separately} in their expansions $c_1 D_1 = c_2 g_e D_2 + c_3 r_e D_3$ for
some coefficients, we can say:

\begin{prop}
Let $G_1$, $G_2$ and $G_3$ be points on the Grassmannian $Gr(p,2p)$ represented by
$\ext{L}_E(\ext{N_\alpha},\ext{N_\beta^\perp})$,
$\ext{L}_{E'}(\ext{N_\alpha}/e,\ext{N_\beta^\perp\setminus e})$ and
$\ext{L}_{E'}(\ext{N_\alpha}\setminus e,\ext{N_\beta^\perp/e})$ respectively. Then
$G_1$, $G_2$ and $G_3$ are colinear in the projective space containing this Grassmannian, and
all points on the line determined by $G_2$, $G_3$ represent points in this Grassmannian.
\end{prop}
\begin{proof}
  As $(r_e:g_e)$ ranges from $(1:0)$ to $(0:1)$, $\ext{L}_E(\ext{N_\alpha},\ext{N_\beta^\perp})$ represents
  the row space of $F$ from the above argument, which varys with $(r_e:g_e)$.
\end{proof}




\section{Final remarks}

\longthanks{I wish to thank the whole community}


%\nocite{*}
\bibliographystyle{amsplain-ac}
\bibliography{ExtAlgTFLMatPairs}
\end{document}
