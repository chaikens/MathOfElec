%% The class cedram-ALCO is just a wrapper around amsart.cls (version 2)
%% implementing the layout of the journal, and some additionnal
%% administrative commands. 
%% You can place one option:
%% * "Unicode" if the file is UTF-8 encoded.
\documentclass[Unicode]{cedram-alco}
%% Here you might want to add some standard packages if those
%% functionnalities are required.
%\usepackage[matrix,arrow,tips,curve]{xy}
% ...
%% The production will anyway use amsmath (all ams utilities except
%% amscd for commutative diagrams which you need to load explicilty if
%% required), hyperref, graphicx, mathtools, enumitem...

%% User definitions if necessary...  Such definitions are forbidden
%% inside titles, abstracts or the bibliography.
\DeclarePairedDelimiter\abs{\lvert}{\rvert} %Something useful only for this sample's sake: you can erase this line in your file (or find it useful...)
%% The title of the paper: amsart's syntax. 
\title
%% The optionnal argument is the short version for headings.
[Exterior Algebra Tutte Functions]
%% The mandatory argument is for the title page, summaries, headings
%% if optionnal void.
{An Exterior Algebra Valued Tutte Function on Linear Matroid Pairs}

%% Authors according to amsart's syntax + distinction between Given
%% and Proper names:
\author[\initial{S.} \middlename{} Chaiken]{\firstname{Seth} \middlename{} \lastname{Chaiken}}

%% Do not include any other information inside \author's argument!
%% Other author data have special commands for them:
\address{University at Albany\\
  State University of New York\\
Computer Science Department\\
1400 Washington Avenue\\
Albany, NY 12222 (USA)}

%% Current address, if different from institutionnal address
\curraddr{18 Eileen Street\\
Albany, NY 12203 (USA)}


%% e-mail address
\email{schaiken@albany.edu}

%% possibly home page URL (not encouraged by journal's style)
%\urladdr{https://en.wikipedia.org/wiki/Marin\_Mersenne}

%% Acknowledgements are not a footnote in
%% \author, but are given apart:
%\thanks{The author was partially supported by a special grant for Junior Woodchucks.}


%% If co-authors exist, add them the same way, in alaphabetical order
%\author{\firstname{Joseph}  \lastname{Fourier}}
%\address{Universit\'e de  Grenoble\\
% Institut Moi-m\^eme\\
% BP74, 38402 SMH Cedex (France)}
%\email{fourier@fourier.edu.fr}

% Key words and phrases:
\keywords{Tutte Functions, Exterior Algebra, Laplacian, Electrical Networks}
  

%% Mathematical classification (2010)
%% This will not be printed but can be useful for database search
\subjclass{10X99, 14A12, 11L05}


\newcommand{\ext}[1]{\ensuremath{\mathbf{#1}}}
%\newcommand{\extvee}{\ensuremath{\mathbf{\vee}}}
\newcommand{\extvee}{\;\;}
\newcommand{\Plucker}{Pl\"{u}cker\ }

\newcommand{\Nal}{\ensuremath{N_{\alpha}}}
\newcommand{\NbePe}{\ensuremath{N_{\beta}^{\perp}}}
\newcommand{\eNal}{\ensuremath{\ext{N}_{\alpha}}}
\newcommand{\eNbePe}{\ensuremath{\ext{N}_{\beta}^{\perp}}}

%\newcommand{\dunion}{\uplus}
\newcommand{\dunion}{\coprod}








\begin{document}
%% Abstracts must be placed before \maketitle
\begin{abstract}
 The matrix tree theorem expresses the basis enumeration Tutte function
(spanning tree count of connected graphic matroids) as a
determinant--a minor of the graph's Laplacian. We will generalize the
choice of a root vertex for the trees by distinguishing a fixed set $P$
of $p$ matroid elements, which we will call ports, a word from
engineering. We then define a function from $k$-linear matrix
representations of matroid pairs ($N_{\alpha},N_{\beta})$ (often
$N_{\alpha}=N_{\beta}$ whose ground sets
include $P$, into pure (decomposible)
exterior algebra elements which represent points in the Grassmannian of $p$ dimensional $k$-linear subspaces
of $k^{(2p)}$. We express this subspace value as a decomposible
(i.e. product of vectors) element in the exterior algebra (of
anti-symmetric tensors) over $k^{(2p)}$, in a standard way.  The result is
a restricted or set-pointed Tutte function $\ext{F}_E(N_{\alpha},N_{\beta})$
that
satisfies sign-corrected forms of the two familiar identities for
deletion/contraction of elements not in $P$, and for disjoint
union. Note that these identities are in exterior algebra, whose
multiplication is anti-commutative. This generalizes the all-minors
directed graph matrix tree theorem (proved combinatorially by the
author) because those minors are the coefficients in the expansion of
$T$'s value over the standard basis--they are the standardized Plucker
coordinates for the dimension $p$ subspace $T$'s value represents.  An
unusual feature for this Tutte-like function is that when $k$ is ordered
so the matroids are oriented, $T$ can distinguish different orientations
of the same matroid.  
\end{abstract}


\maketitle

% First paragraph after a section is not indented. If there is text
% below the title before the first section, it should be unindented
% like this.
\noindent
The aim ... .

\section{Introduction}

\noindent

Tutte functions are a well-known, long-standing idea
for which new applications, variations and generalizations
continue to be active research subjects. For example,
Krajewski, Moffatt and Tanasa give a common Hopf algebra framework
for the deletion and contraction operations for many of
these variations\cite{KRAJEWSKI2018271}.
One of the earliest Tutte function is the basis enumerator for matroids. It
is a determinant in the case of graphic, and more generally, regular matroids;
this is a consequence of the Cauchy-Binet theorem.
The well-known Matrix Tree theorem tells us this determinant (up to sign)
is any full-rank minor of the graph's Laplacian matrix. More generally,
every minor is an enumerator for (signed, in some non-principal minor
cases) forests satisfying the following condition: Each vertex indexing
a deleted row is in exactly one of the trees, and similary for each
vertex indexing a deleted column\cite{sdcMTT}. (Each forest's sign depends on the sign of the
matching the forest determines between the deleted row vertices and
deleted column vertices; of course when the minor is principal,
the only such matching is the identity.)
We give a key observation: \emph{Each} such
minor, when $e$ is a given edge, equals the sum of the corresponding
minor when $e$ is deleted plus the determinant expansion terms for
forests that contain $e$.  The latter can be also be expressed as a
determinant.
In the following, we will develop a formulation in which
\emph{all these minors together},
when they are constituted as an element in an exterior algebra, satisfy the
well-known Tutte identities.

In order for a tuple of determinants to be a Tutte function value, one
must label them in a combinatorial way.  For this purpose, we use
the \emph{set-pointed}, also called the \emph{relative} or \emph{ported}
variation of Tutte functions.  Let $P$ be a set.  A function $F$ is a set $P$ pointed or $P$-ported
Tutte function, or a relative Tutte function with respect to $P$, when the
identity for deletion and contraction of $e$ is asserted only when $e \not\in P$.
$P$ will play the role of vertices in our all-minors matrix tree theorem specialization.

Let $P_{\alpha}$ and $P_{\beta}$ be disjoint copies of $P$.  Let $K$ be a field.
We will see that our Tutte function's value will be an exterior product of
$|P|$ vectors in the vector space over $K$ with basis $P_\alpha \cup P_\beta$.
Such exterior algebra products are sometimes called ``decomposibles''\cite{MarcusFDMuAlPt2}
and are ``pure'' when considered to be anti-symmetric tensors.  Our value will
have grade level or ``step''  $|P|$ in the exterior algebra over our
vector space.
As such, (when non-zero)
it \emph{represents} some
dimension $P$ subspace, a point in the Grassmannian $Gr(|P|,2|P|)$.
However, in order for Tutte's additive identity to make sense, the value cannot be
a point in a projective space like the one containing the variety $Gr(|P|,2|P|)$.


The theory works when parameters $r_e$, $g_e$ are given for each element $e\not\in P$. We
will have, formally,
\[
   F(N) = g_eF(N/e)+r_eF(N\backslash e) \text{ if\ }e\text{ can be deleted and contracted, and }e\not\in P.
   \]
Indeed, our chief motivation is to give an algebraic combinatorial description of solutions
to electrical network problems, where the parameters will represent the coefficient
in Ohm's law.  It is convenient (and elegant) to write that law in homogeneous form:
For each resistor $e$ there
are two parameters $r_e,g_e$ such that $i_e$ is the current through the resistor and $v_e$ is
the voltage drop across the resistor (in the direction of the current) if and only if there
exists a value $x_e$ for which $i_e=g_ex_e$ and $v_e=r_ex_e$.  In other words, when they
are non-zero and not infinite, the conventional
resistance is $r_e/g_e$ and conductance is $g_e/r_e$. Shorts (zero resistance) and opens (zero
conductance) are thus accomodated.   To make this long story short (see section ()),
our Tutte function's value
will encode the set of voltage drops across and current flows through each of
the network edges in $P$ that are compatible with Kirchhoff's and Ohm's laws. This set
is the linear solution space of an underdetermined system projected into the
dimension $2|P|$ space with a voltage and a current coordinate for each $p\in P$.



\section{Setup and Theorem}


\subsection{Definition of a Tutte Function of $P$-Ported Pairs}

We begin by defining how we index columns of matrices (with a set of
linear space basis elements) and then
present specific
exterior algebra products that encode their row spaces and support
combinatorial operations where a distinguished set of indices $P$ is given.
We demonstrate this with the example
of the electrical network equations and their solutions.
The Tutte function is defined
on those pairs of exterior products, which might have been given abstractly.
It is important to recognize that two exterior algebra elements are given,
possibly by a suitable $\binom{n}{r}$ tuples of components, not the point in the Grassmannian
whose \Plucker coodinates are equivalence classes of non-zero multiples of
such a tuple.  We \emph{do} distinguish different such multiples.  Another way
to put it is that our data and constructed objects are matrices modulo left multiplication
by $SL_r$, not by $GL_r$ the latter which gives the Grassmannian.  Of course all such matrix
classes surject onto the Grassmannian.






Let $\Nal$ and $\NbePe$ be two full row rank matrices (over field $K$)
with
columns indexed by $P\cup E$, $P\cap E=\emptyset$, for which
$\text{rank}(\Nal)+\text{rank}(\NbePe)=|E|+|P|$.  Let
$G=\text{diag}\{g_e\}_{e\in E} $ and $R=\text{diag}\{r_e\}_{e\in E} $
be diagonal matrices of parameters.  For any such matrix $N$, $N(E)$ and
$N(P)$ denote the submatrices with columns $E$ and $P$ respectively. With
two disjoint copies of $P$ and bijections
$P \leftrightarrow P_{\alpha}\leftrightarrow P_{\beta}$, define the matrix
with columns indexed by $P_\alpha \dunion P_\beta \dunion E$, where each entry
is in $K$ or is a $K$ multiple of a parameter:
\[
    L = L\left( \begin{array}{c} \Nal\\ \NbePe \end{array} \right)
    = \left[\begin{array}{c|c|c} \Nal(P)  &  0  &  \Nal(E)G \\  \hline
        0  & \NbePe(P)  &  \NbePe(E)R \end{array}\right]
\]
    

Now let $U$ be the vector space $K(g_e,\ldots;r_e,\ldots)P_\alpha
\dunion P_\beta \dunion E$ with distingushed basis $P_\alpha
\dunion P_\beta \dunion E$.  Vectors and other elements of the
exterior algebra $\wedge U$ will be denoted by boldface
characters, such as $\ext{e}_j$, $\ext{p_\alpha}_k$,
$\ext{p_\beta}_l$ $\in$ $P_\alpha \dunion P_\beta \dunion E$.
    
Let $p = |P|$, $n=|E|$ and $L_i$ be row $i$ of $L$ for $i\in\{1,\ldots,n+p\}$.
We form $\ext{L}$ by exterior multiplication (in order) of the vectors each given by the
rows of $(n+p)\times (n+2p)$ matrix $L$ for the coefficients when written in terms of
basis $P_\alpha \dunion P_\beta \dunion E$:
\begin{equation}\label{defExtL}
  \ext{L} = \wedge_{i=1}^{n+p}L_i[\ext{p_\alpha}_1\ \cdots\ \ext{p_\alpha}_{p}\ \ext{p_\beta}_1\ \cdots\ \ext{p_\beta}_{p}\
    \ext{e}_1\ \cdots\ \ext{e}_{n}]^t. 
\end{equation}

We can now define what will turn out to be our Tutte function.  One cannot say what it means for
it to be a Tutte function until we get to defining the deletion and contraction operations!

Let us fully expand $\ext{L}$ into an anti-commutative polynomial in
$P_\alpha \dunion P_\beta \dunion E$.  Since $\ext{v}\wedge\ext{w}=-\ext{w}\wedge\ext{v}$ for vectors in
exterior algebra, we can write
\[
\ext{L} = (\ext{L}_E)\wedge (\ext{e_1}\wedge\ext{e_2}\cdots\wedge\ext{e_n}) + \ext{J}.
\]
We will abbreviate expressions like $(\ext{e_1}\wedge\ext{e_2}\cdots\wedge\ext{e_n})$ by
$(\ext{e_1}\ext{e_2}\cdots\ext{e_n})$ or $\ext{E}$ where \emph{set $E$ is specified as a sequence of its elements}.
So, given $(N_\alpha,N_\beta^\perp)$ as above, $\ext{L}_E(N_\alpha,N_\beta^\perp)$ is defined by (\ref{defExtL}) where
$\ext{J}$ is the sum of terms none of which is a product of $\ext{E}$.

\begin{prop}
  $\ext{L}_E(N_\alpha,N_\beta^\perp)$ is invariant under left actions of $SL_r$ on $N_\alpha$ or $N_\beta^\perp$.
\end{prop}
\begin{proof}
  All the coefficients are products of an all row minor of $N_\alpha$, an all-row minor of $N_\beta^\perp$, a sign, and
  parameters.
  By definition, the actions of $SL_r$ will multiply those determinants by 1. 
\end{proof}

We can therefore use the similarly formed exterior products we denote by $\ext{N_\alpha}$ and $\ext{N_\beta^\perp}$
of the rows (in order) of $N_\alpha$ and $N_\beta^\perp$ for our data, and justify

\begin{defi}
  Given $E$, $P$ and $(\ext{N_\alpha},\ext{N_\beta^\perp})$ as above, $\ext{L}_E(\ext{N_\alpha},\ext{N_\beta^\perp})$ is the
  unique exterior algebra element satisfying
  \[
  \ext{L}(\ext{N_\alpha},\ext{N_\beta^\perp}) =
  (\ext{L}_E)\wedge (\ext{e_1}\wedge\ext{e_2}\cdots\wedge\ext{e_n}) + \ext{J} =
  (\ext{L}_E(\ext{N_\alpha},\ext{N_\beta^\perp}))\ext{E}+\ext{J}.
  \]
  where $\ext{J}$ is the sum of terms none of which is a product of $\ext{E}=\ext{e_1}\wedge\ext{e_2}\cdots\wedge\ext{e_n}$.
\end{defi}

    


\section{Final remarks}

\longthanks{I wish to thank the whole community}


%\nocite{*}
\bibliographystyle{amsplain-ac}
\bibliography{ExtAlgTFLMatPairs}
\end{document}
