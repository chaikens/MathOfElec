

Let $P=\{p_1, p_2\}$ and $E=\{e_1, e_2, e_3, e_4, e_5\}$ together be the edges in the (oriented) graph
below representing an electrical network, where $E$ represents the resistors.  The currents and voltages
in all these edges are variables $i_t$ and $v_t$ for $t\in P\cup E$.  Ohm's law
asserts that for $e\in E$, the ratio voltage-to-current ratio
$v_e:i_e$ $=$ $r_e:g_e$, the ratio of the parameters. The matrix $N_\alpha$ is the
usual ``signed'' vertex-edge incidence matrix which represents our graph's graphic matroid, with one
row deleted so $N_\alpha$ has full row rank.  The rows of $N_\alpha$ are the coefficients in
Kirchhoff's current law, asserting that the edge currents constitute a 1-cycle, (i.e., a flow.)
Matrix $N_\beta^\perp$ is a totally unimodular matrix whose
rows are a basis for the 1-cycle space, obtained by coding the incidences of edges with the oriented fundamental
circuits each associated to an edge not in a fixed spanning tree.  It's rows are the coefficients in
Kirchhoff's voltage law, asserting that the edge
voltages drops constitute a 1-coboundary, (i.e., differences
of a potential across edges.)
This well-known construction gives
two full row rank totally unimodular matrices whose row spaces are orthogonal complements,
representing the graph's
graphic and cographic matroids respectively.

We form the following system of equations from ().
\[
cool sys of eq
\]
The electrical network problem at hand is to determine linear constraints on the
port variables $i_p, v_p$, $p\in P = \{p_1, p_2\}$ imposed by the system.  In other words,
we want a linear map on the $U$-vector space with basis $P_\alpha, P_\beta$ whose kernel
is the projection of this system's solution space.

