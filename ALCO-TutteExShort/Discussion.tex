
Among the many currently active research subjects
are algebraic theories for variations of
Tutte decomposition and functions, and convolutions for
Tutte functions of various compositions of objects.  
For example,
Krajewski, Moffatt and Tanasa give a common Hopf algebra framework
for the deletion and contraction operations for many of
these variations\cite{MoffatHopfTutte}.
This approach was generalized and further systemized by
Dupont, Fink and Moci\cite{ALCO_2018__1_5_603_0}.
See also Crapo and Schmitt's work
on the Hopf algebra of matroids and call
for attention to ``naturally occurring algebraic
structures'' in matroid theory\cite{CRAPO20051066}.  

Our motivating linear algebra construction, applied to
electrical netwoks, and its the matroid
abstraction
doesn't appear to be addressed in the above cited literature.
Current research along these lines is in Laplacians and their
determinants of simplicial complexes of dimensions greater than 1.

We see our contribution is to Tutte functions of objects like these
when they have ports.  


One of the
first applications of combinatorial enumeration
appeared in a little known 1847 paper by Kirchhoff\cite{Kirchhoff}
in which a solution was given in terms of tree enumerations,
and proven by a graphic matroid basis exchange argument, rather
than by the matrix tree theorem.
Our linear algebra construction and its application to
electrical networks was studied early on by
Bott and Duffin's\cite{BottDuffinAlgNetworks} study of the ``constrained
inverse''.  When the $g_e$, $r_e$ parameters are algebraically independent,
$\LVert{\eNal}{\eNbePe}$\ref{LDefs} represents a minor of a matroid union--
a discusson that relates this to the constrained
inverse appears in \cite{sdcBDIMatroid}.  A product formula for
the $P$-ported Tutte polynomial of a matroid union appears in
\cite{sdcPorted}.  

---------------------------------------
\newpage

The target for us
is exterior algebras, which occur naturally when problems of
partial solution of systems of linear equations\cite{ExteriorAlgInLinalgRef},
i.e., Schur complements,
%of linearly parametrized systems,
are studied from a matroid theory, sometimes oriented, point of view.
In exterior algebra, unlike commutative algebras,
linear subspaces (as matroid representations and
solutions to systems of homogeneous linear equations)
are represented by exterior products of vectors; also called
\emph{decomposible} elements, \emph{decomposibles},
or ``pure'' antisymmetric tensors $\ext{N}$.
When $\ext{N}$ is expanded in terms of the basis
generated by the elements corresponding to a matroid
ground set, the result is a basis enumeration.


Our motivating example of such subspaces occur
in problems posed from graphs that generalize inversion
of graph Laplacians.  These are defined in
terms of linear resistive linear electrical networks,
under Kirchhoffs' and Ohm's laws.





An examination shows that our notion of  $\ext{N}$ and definitions $\ext{N}^\perp$ and $\ext{L}_E$ make
sense even when $\ext{N}$ is a arbitrary element of the exterior algebra, not necessarilly a pure element.
Furthermore, the properties and Tutte identity theorems hold in this wider context.   Of course,
non-pure elements do not represent subspaces, so our matroid representation and subspace
interpretions no longer apply.  On the other hand, a chirotope of any oriented matroid can be encoded by
the sum over the bases of an ordered product of the basis elements.



-------------------------------------

