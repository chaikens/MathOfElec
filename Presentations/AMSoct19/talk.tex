\documentclass{beamer}
%\mode<presentation>
\usepackage{cite}
\usepackage{amsmath}
\title{Restricted or Ported Tutte Decomposion and Analogs
  of All-Minors Laplacian Expansions}
\author{Seth Chaiken\\
  Assoc. Prof. Emeritus Dept. of Computer Science\\
Univ. at Albany\\
\url{schaiken@albany.edu}
}
\date{October 13, 2019}
%%%%%%%%%%%%%%%%%%%%%%%%%%%%%%%%%%%%%%%%%%%%%%%%%%%%%%%%%%%%%
% Specialized Symbols
%
%   Disjoint Union
%\newcommand{\dunion}{\uplus}
\newcommand{\dunion}{\coprod}
%{\mbox{\hbox{\hskip4pt$\cdot$\hskip-4.62pt$\cup$\hskip2pt}}}
%{\mbox{\hbox{\hskip6pt$\cdot$\hskip-5.50pt$\cup$\hskip2pt}}}
%best choice: {\mbox{\hbox{\hskip0.45em$\cdot$\hskip-0.44em$\cup$\hskip0.2em}}}
%{\mbox{\hbox{\hskip0.45em$+$\hskip-0.70em$\cup$\hskip0.3em}}}
%
% Dot inside a cup.
% If there is a better, more Latex like way 
% (more invariant under font size changes) way,
% I'd like to know.

\newcommand{\Bases}[1]{\ensuremath{{\mathcal{B}}(#1)}}
\newcommand{\Reals}{\ensuremath{\mathbb{R}}}
\newcommand{\FieldK}{\ensuremath{K}}
\newcommand{\Perms}{\ensuremath{\mathfrak{S}}}
%\newcommand{\rank}{{\rho}}% {{\mbox{rank}}}
%\newcommand{\Rank}{{\rho}}% {{\mbox{rank}}}
\newcommand{\rank}{{\mbox{r}}}% {{\mbox{rank}}}
\newcommand{\Rank}{{\mbox{r}}}% {{\mbox{rank}}}
\newcommand{\Card}[1]{\ensuremath{{\left|#1\right|}}}
\newcommand{\ext}[1]{\ensuremath{\mathbf{#1}}}
%\newcommand{\extvee}{\ensuremath{\mathbf{\vee}}}
\newcommand{\extvee}{\;\;}
\newcommand{\Plucker}{Pl\"{u}cker\ }

% Set Complement
% command to mess with overline, bar or custom 
% alternatives for sequence or set complement
%
\newcommand{\scomp}[1]{\ensuremath{\overline{#1}}}
%\newcommand{\scomp}[1]{\ensuremath{\bar{#1}}}
%
%   Put a symbol for a matroid in a box, or brackets
%\newcommand{\MVAR}[1]{{\boxed{#1}\;}}
\newcommand{\MVAR}[1]{{[#1]\;}}
%
\newcommand{\UNION}{\cup} %try to make this bold.
%
%Emphasize in color!
\newcommand{\Remph}[1]{{\color{red}#1}}
%
%%%%%%%%End of specialized symbols%%%%%%%%%%%%%%%%%%%%%%



\begin{document}
\begin{frame}
 \titlepage
\end{frame}

\note{(here we go, geronimo)}

\begin{frame}
  \begin{itemize}
  \item
    Matrices $N_1$, $N_2$; full row rank, columns indexed by
    $P\dunion E$.  $P_1,P_2\leftrightarrow P$, $P_1\cap P_2=\emptyset$.
  \item
    Weight (parameter) matrices\\
    $G=\text{diag}\{g_e\}_{e\in E} $,
    $R=\text{diag}\{r_e\}_{e\in E} $.
  \item
    Matrix with columns $P_1 \dunion P_2 \dunion E$
    \[
M = \left[\begin{array}{c|c|c} N_1(P)  &  0  &  N_1(E)G \\  \hline
0  & N_2(P)  &  N_2(E)R \end{array}\right]
    \]
  \end{itemize}

  Define
  \[
  F(M)=((\binom{2p}{p})-\text{tuple of determinants\ } M[Q_1Q_2E])
  \]
  where $Q_1\subseteq P_1$, $Q_2\subseteq P_2, |Q_1Q_2|=p=|P|$.
    
\end{frame}

\begin{frame}
  Column $e$ of $M$ is
\[
\left[\begin{array}{c}
    N_{1,1,e}g_e\\
    N_{1,2,e}g_e\\
    ...\\
    N_{1,r_1,e}g_e\\
    N_{2,1,e}r_e\\
    N_{2,2,e}r_e\\
    ...\\
    N_{2,r_2,e}r_e
  \end{array}\right]
=
\left[\begin{array}{c}
    N_{1,1,e}\\
    N_{1,2,e}\\
    ...\\
    N_{1,r_1,e}\\
    0\\
    0\\
    ...\\
    0
  \end{array}\right]g_e
+
\left[\begin{array}{c}
    0\\
    0\\
    ...\\
    0\\
    N_{2,1,e}\\
    N_{2,2,e}\\
    ...\\
    N_{2,r_2,e}
  \end{array}\right]r_e
\]
So, using $M\left(\begin{array}{c} N_1\\ N_2 \end{array} \right)$
to denote how $M$ is formed from the $N_i$,
\[
F(M)_{Q_1Q_2E} = M[Q_1Q_2E] =
\]
\[
g_eM\left(\begin{array}{c} N_1/e\\ N_2\backslash e \end{array} \right)[Q_1Q_2E]
+
r_eM\left(\begin{array}{c} N_1\backslash e\\ N_2/e \end{array} \right)[Q_1Q_2E].
\]
\end{frame}

\begin{frame}
  Each determinant $M[Q_1Q_2E]$ is one of $\binom{2p}{p}$ components, so
  \[
  F(M) =
  g_eM\left(\begin{array}{c} N_1/e\\ N_2\backslash e \end{array} \right)
+
r_eM\left(\begin{array}{c} N_1\backslash e\\ N_2/e \end{array} \right)
\]
where
\[
N_i/e \text{\ means remove the\ } g_e \text{\ or\ } r_e \text{\ but otherwise keep column\ }e
\]
\[
N_i\backslash e \text{\ means replace column\ }e\text{\ by\ }0.
\]
\end{frame}

\begin{frame}
    \[
  F(M) =
  g_eM\left(\begin{array}{c} N_1/e\\ N_2\backslash e \end{array} \right)
+
r_eM\left(\begin{array}{c} N_1\backslash e\\ N_2/e \end{array} \right)
\]
Real deletion/contraction removes $e$ from the ground set of the matroid or other
object, but $N_i/e,N_i\backslash e$ still have column $e$.  But (*) holds for all $e\in E$, so
we have a basis expansion:
\[
M[Q_1Q_2E] = \sum_{A\subseteq E}g_Ar_{\overline{A}}N_1[Q_1A]N_2[Q_2\overline{A}]
\epsilon(Q_1A,Q_2\overline{A})
\]
The $A$ term is $\neq 0$ iff $Q_1A$ is a column basis for $N_1$\\
and $Q_2\overline{A}$ is a column basis for $N_2$.
So, for each $Q_1Q_2$ (all $|A|$ for non-zero terms are equal)
\[
M[Q_1Q_2E] = \pm\sum_{A\subseteq E}g_Ar_{\overline{A}}N_1[Q_1A]N_2[Q_2\overline{A}]
\epsilon(A,\overline{A})
\]


\end{frame}





\end{document}
