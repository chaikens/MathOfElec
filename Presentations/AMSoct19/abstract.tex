\documentclass[12pt]{article}

\usepackage{amsmath}


%\title{The Exterior Algebra Form of All-Minors Laplacion Tutte
%  Decomposition and Similar Theorems}

\title{Restricted or Ported Tutte Decomposion and Analogs
  of All-Minors Laplacian Expansions}

\author{Seth Chaiken\\
\small Computer Science Department\\[-0.8ex]
\small University at Albany, State Univ. of New York, USA\\[-0.8ex]
\small \texttt{schaiken@albany.edu}
}

\date{
Formatted:\today\\
\small Mathematics Subject Classifications: 52C40, 15A75, 05C50}

\begin{document}
\maketitle

\begin{abstract}
  The all-minors matrix tree theorem's expansions of a graph's
  Laplacian minors together comprise one special case of an exterior
  algebra valued Tutte-type function.  For graphs it is constructed
  after gluing a pointed 1-dim star whose edges (called (``ports'')
  identify the original vertices.  The function's value has all the
  Laplacian's minors for its Pl{\"u}cker coordinates.  Each coordinate
  satisfies Tutte-Grothendieck equations when deletion/contraction is
  restricted to non-port elements (therefore, so do the exterior
  algebra values.)  How the setup, including homogeneous element
  weights $(g_e:r_e)$ for each non-port element $e$, applies to
  Kirchhoff's electrical networks, gain graphs studied by Reff,
  Zaslavsky and others, linear matroids and Laplacians of simplicial
  complexes is described.  It also extends to a single exterior
  algebra expansion that encapsulates the Cauchy-Binet expansions of
  all the minors of any non-symmetric weighted Laplacian-type matrix
  $A\operatorname{diag}(g_e)B^{t}$.  Thus it applies to digraph
  directed spanning forest enumeration and solving for the behavior of
  linear electronic circuits with ideal operational amplifiers.
\end{abstract}

\end{document}

old last sentence:
It also extends to non-symmetric
  Laplacian-type matrices $A\operatorname{diag}(g_e)B^{t}$, and so
  applies to directed graphs and linear electronic circuits with ideal
  operational amplifiers.

  old:
  %  It (therefore
%  each coordinate) satisfies Tutte-Grothendieck equations when
  %  deletion/contraction is restricted to non-port elements.
