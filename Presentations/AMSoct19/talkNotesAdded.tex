%\documentclass{article}
%\usepackage{beamerarticle}
\documentclass{beamer}
%\mode<presentation>
\usepackage{cite}
\usepackage{amsmath}
\title{Restricted or Ported Tutte Decomposion and Analogs
  of All-Minors Laplacian Expansions}
\author{Seth Chaiken\\
%  Assoc. Prof. Emeritus Dept. of Computer Science\\
Univ. at Albany\\
\url{schaiken@albany.edu}
}
\date{October 13, 2019\\
  The occasion of Thomas Zaslavsky's 75th Birthday\\
  Special Session on Algebraic Combinatorics, AMS meeting\\
at Binghamton University}
%%%%%%%%%%%%%%%%%%%%%%%%%%%%%%%%%%%%%%%%%%%%%%%%%%%%%%%%%%%%%
% Specialized Symbols
%
%   Disjoint Union
%\newcommand{\dunion}{\uplus}
\newcommand{\dunion}{\coprod}
%{\mbox{\hbox{\hskip4pt$\cdot$\hskip-4.62pt$\cup$\hskip2pt}}}
%{\mbox{\hbox{\hskip6pt$\cdot$\hskip-5.50pt$\cup$\hskip2pt}}}
%best choice: {\mbox{\hbox{\hskip0.45em$\cdot$\hskip-0.44em$\cup$\hskip0.2em}}}
%{\mbox{\hbox{\hskip0.45em$+$\hskip-0.70em$\cup$\hskip0.3em}}}
%
% Dot inside a cup.
% If there is a better, more Latex like way 
% (more invariant under font size changes) way,
% I'd like to know.

\newcommand{\Bases}[1]{\ensuremath{{\mathcal{B}}(#1)}}
\newcommand{\Reals}{\ensuremath{\mathbb{R}}}
\newcommand{\FieldK}{\ensuremath{K}}
\newcommand{\Perms}{\ensuremath{\mathfrak{S}}}
%\newcommand{\rank}{{\rho}}% {{\mbox{rank}}}
%\newcommand{\Rank}{{\rho}}% {{\mbox{rank}}}
\newcommand{\rank}{{\mbox{r}}}% {{\mbox{rank}}}
\newcommand{\Rank}{{\mbox{r}}}% {{\mbox{rank}}}
\newcommand{\Card}[1]{\ensuremath{{\left|#1\right|}}}
\newcommand{\ext}[1]{\ensuremath{\mathbf{#1}}}
%\newcommand{\extvee}{\ensuremath{\mathbf{\vee}}}
\newcommand{\extvee}{\;\;}
\newcommand{\Plucker}{Pl\"{u}cker\ }

% Set Complement
% command to mess with overline, bar or custom 
% alternatives for sequence or set complement
%
\newcommand{\scomp}[1]{\ensuremath{\overline{#1}}}
%\newcommand{\scomp}[1]{\ensuremath{\bar{#1}}}
%
%   Put a symbol for a matroid in a box, or brackets
%\newcommand{\MVAR}[1]{{\boxed{#1}\;}}
\newcommand{\MVAR}[1]{{[#1]\;}}
%
\newcommand{\UNION}{\cup} %try to make this bold.
%
%Emphasize in color!
\newcommand{\Remph}[1]{{\color{red}#1}}
\newcommand{\Bemph}[1]{{\color{blue}#1}}
%

% Subscripts denoting Current and Voltage
%\newcommand{\Is}{\ensuremath{I}}
%\newcommand{\Vs}{\ensuremath{V}}
\newcommand{\Is}{\ensuremath{\iota}}
\newcommand{\Vs}{\ensuremath{\upsilon}}


%%%%%%%%End of specialized symbols%%%%%%%%%%%%%%%%%%%%%%



\begin{document}


\begin{frame}
 \titlepage
\end{frame}

\begin{frame}{What is a Strong Tutte function?}
  
  Some history.
  Zaslavsky (1992) in ``Strong Tutte Functions of Matroids and Graphs'' showed what happens
  with Tutte equations (on a field), with 4 parameters (or weights) (Different notation!)
  $g_e$, $r_e$,
  $i_{\ext{loop(e)}}$ and $i_{\ext{coloop(e)}}$
  for element $e$:

  \begin{enumerate}
  \item For all $\ext{N}$ with separator (neither loop nor coloop) $e\in S(\ext{N})$,
    \[
    F(\ext{N}) = g_eF(\ext{N}/e) + r_e(\ext{N}\backslash e)
    \]
  \item When $\ext{N}=\ext{N_1}\oplus\ext{N_2}$,
    \[
    F(\ext{N}) = F(\ext{N_1})F(\ext{N_2})
    \]
  \item When $\ext{N}$ is a loop or coloop on $e$, an initial value is given:
    \[
    F(\ext{N}) = i_{\ext{N}}
    \]
    (Z.'s term: Point values.)
    This means there are two parameters (besides $g_e$, $r_e$) for each $e$, so
    \[
    F(\ext{loop(e)})=i_{\ext{loop(e)}} \text{\ and\ } F(\ext{coloop(e)})=i_{\ext{coloop(e)}}
     \]
  \end{enumerate}
\end{frame}

\begin{frame}{What happens?}
  \begin{block}{Those equations might not have a solution!}
    For (typically \emph{lots of}) equations involving a common \emph{function} $F$, for them
    ``to have a solution'' MEANS there exists a function $F$ on some domain of matroids
    so all the equations are satisfied with that $F$.\\
 This MEANS $F(\ext{N})$ is what is computed by applying Tutte
 equations \emph{in any order they are applicable}.
  \end{block}
  \begin{block}{Z.'s result}
    Strong Tutte functions are classified into seven types, each given by
    conditions on the weights and the initial values.\\
    Amazingly, the conditions for there to be a solution are all derived by
    requiring all Tutte decompositions of \textbf{2 or 3 point matroids} in the domain
    to compute \emph{the same value}.
      \end{block}
\end{frame}


\begin{frame}{Maybe more history}
  All things matroids and Tutte polynomial were around Zaslavsky and
  the rest of the 1970's MIT gang.  \\
  After a couple of years, I tried my hand at drawing algorithms for planar
  graphs, and was led to Tutte's ``How to draw a graph'', and Brook, Smith, Stone
  and Tutte's ``Dissecting a square into squares.''  Both inverted submatrices of
  a graph's Laplacian; both had the Matrix Tree Theorem to prove this was possible.
  Harmonic functions on vertices were used to place vertices (after fixing places of
  some) and to find sizes of squares so they tiled a square in a given combinatorial
  pattern.  
 \end{frame}
\begin{frame}{Solving electrical problems by counting trees}
   \begin{block}{Very shocking fact-Maxwell's or Kirchhoff's rule}
    The equivalent resistance $R_{uv}$
    between nodes $u$ and $v$ of a resistor network $N$
    with edge conductances $g_e$ ($=r_e^{-1}$ is
    \[
    R_{uv}=\frac{\sum_{F\text{\ a spanning tree in\ } N/(uv) \text{\ with\ } u,v \text{\ identified}}\Pi_{e\in F}g_e}
    {\sum_{F\text{\ a spanning tree in\ } N}\Pi_{e\in F}g_e}
    \]
    So weighted tree enumeration don't just tell us some matrices are invertable.
   \end{block}
   Thinking matroids, $N/(uv)$ is $N$ with a different kind of edge, an interface
   edge $p=uv$ added.  Then,
   \begin{center}
     Numerator is $\sum g_F$ over $F$ bases in $N/p$.\\
     Denominator is $\sum g_F$ over $F$ bases in $N\backslash p$.\\
     BOTH of these sums are weighted Tutte functions.
   \end{center}
\end{frame}

\begin{frame}{Why call $(uv)$ a port?}

  \begin{block}{from ``The Tutte Polynomial of a Ported Matroid'' sdc 1989}
    We have been motivated by electrical network considerations where the
branches used to connect the network to other networks are distinguished
from the branches or variables associated with devices such as resistors or
capacitors..
  \end{block}
\end{frame}


\begin{frame}{Ported/Set Pointed/Relative Tutte Functions}
  \begin{definition}[sdc 1989]
     (\alert{easily updated with weights and oriented matroids})
    Let $M(E,P)$ be a $P$-ported \alert{oriented} matroid with rank function $\rho$.
    The $P$-ported \alert{weighted} rank generating function $r_P(M)$ is
    \[
    r_P(M) = \sum_{A\subseteq E}[M/A|P]\alert{g_Ar_{\overline{A}}}x^{\rho(M)-\rho(M/A|P)-\rho(A)}y^{|A|-\rho(A)}
    \]
    Here $S(M)=P(M)\dunion E(M)$\alert{, and $r_e,g_e$ are weights for each $e\in E$.}\\
    For any \alert{oriented} matroid $M$ for which $E(M)=\emptyset$, $[\emptyset]=1$ and
    \[
      [M] = [M_1][M_2]\cdots[M_k]
      \]
      where  $M=M_1\oplus M_2 \cdots M_k$ and each $M_i$ is connected.  These bracket
      \emph{oriented} matroid symbols comprise, with (the well-known) Tutte Polynomial
      variables $x$ and $y$, the variables in $r_P(M)$.
  \end{definition}
\end{frame}

\begin{frame}{$P$-ported parametrized Tutte Equations}
  They are the usual, except deletion/contraction of $e$ is
  \alert{forbidden} when $e\in P$.
  \begin{definition}
     An \alert{oriented} matroid $M(P,E)$ is $P$-ported when its
    ground set $S(M)=P\dunion E$.
    A function $F$ on \alert{oriented} matroids is a 
    $P$-ported \alert{weighted} Tutte function if
    \begin{itemize}
    \item Whenever $e\in E(M)$, and $e$ is a non-separator in $M$,
      $F(M) = g_eF(M/e)+r_eF(M\backslash e)$.
    \item Whenever $M=M_1\oplus M_2$, $F(M)=F(M_1)F(M_2)$.
    \end{itemize}
    \begin{theorem} 
     $r_P$ defined above is (such) a Tutte function.
    \end{theorem}
  \end{definition}
\end{frame}



\begin{frame}{Ported Tutte Decomposition (not all of it)}
\hspace{-0.2in}\mbox{\input{K4Decomp.pspdftex}}
\end{frame}




\begin{frame}{Zaslavsky 1992: It doesn't all work. $r_P$ is not universal.}
  \[
 \begin{split}
   & r_P(\ext{loop(e)}) = g_ey + r_e\\
   & r_P(\ext{coloop(e)}) = r_ex + g_e
 \end{split}
  \]
  The two Tutte decompositions of the circuit $U^1_{ef}$ on $e,f$ to compute a prospective Tutte function $F$ give
  \[
  \begin{split}
    e \text{\ first\ } F(U^1) &= g_e F(loop(f)) + r_e F(coloop(f)) \\
    f \text{\ first\ } F(U^1) &= g_f F(loop(e)) + r_f F(coloop(e))
  \end{split}
  \]
  We still need those 4 point values, and they can't be chosen independently
  of the 4 weights.  Zaslavsky called the class where there are arbitrary $x$, $y$ values
  and the point values are  $g_ey + r_e$ and $r_ex + g_e$ \emph{normal Tutte functions}.
\end{frame}
\begin{frame}{Related work.}
\begin{itemize} 
\item
  $|P|=1$ and series/parallel connections on pointed matroids (Brylawsky (1971)), extended
  to unions and dual-unions over $P$ (also sdc 1989).
  \item
    Matroids called set-pointed on $P$ encoded by products of many variables (Las Vergnas 1975).
  \item
    Weighs/colors/parameters (Zaslavsky 1992, Bollob\'{a}s and Riordan 1999, Ellis-Monaghan and Traldi 2006.
    Much motivation from maps on surfaces and knot theory.
  \item
    Dao and Hetyei (2012) named carried out BRZs classfication program,
    called the matroids relative.  Motivated by knots with ports for virtual crossings.
    Easy to see this extends to oriented matroids.
\end{itemize}
\end{frame}


\begin{frame}{What this talk is about.}
  \begin{center}\large
    Some ways determinants make Tutte functions.\\
    How the graph and other Laplacians ACTUALLY ARE Tutte functions, not just
    a partcular determinant.\\
    Ported Tutte functions are needed to tell this story.\\
    The only Tutte function I know valued in a \textbf{non-commutative ring} (the signed
    commutative exterior algebra).\\
    Only normal Tutte functions are relevant, and we only need them with
    $x=y=0$ (which does $P$-subbasis enumeration).
  \end{center}
\end{frame}





    
   


\newcommand{\Nal}{\ensuremath{N_{\alpha}}}
\newcommand{\NbePe}{\ensuremath{N_{\beta}^{\perp}}}
\newcommand{\eNal}{\ensuremath{\ext{N}_{\alpha}}}
\newcommand{\eNbePe}{\ensuremath{\ext{N}_{\beta}^{\perp}}}


\begin{frame}{Tutte Functions using determinants: Our setup}
  \begin{itemize}
  \item
    Matrices $\Nal$, $\NbePe$; full row rank, columns indexed by
    $P\dunion E$. $\text{rank}(\Nal)+\text{rank}(\NbePe)=|E|+|P|$.\\
    $P_{\alpha},P_{\beta}\leftrightarrow P$, $P_{\alpha}\cap P_{\beta}=\emptyset$.
  \item
    Weight (parameter) matrices\\
    $G=\text{diag}\{g_e\}_{e\in E} $,
    $R=\text{diag}\{r_e\}_{e\in E} $.
  \item
    Matrix with columns $P_\alpha \dunion P_2 \dunion E$
    \[
    L = L\left( \begin{array}{c} \Nal\\ \NbePe \end{array} \right)
    = \left[\begin{array}{c|c|c} \Nal(P)  &  0  &  \Nal(E)G \\  \hline
0  & \NbePe(P)  &  \NbePe(E)R \end{array}\right]
    \]
  \end{itemize}

  Define
  \[
  F(L)=((\binom{2p}{p})-\text{tuple of determinants\ } L[Q_\alpha \overline{Q_\beta} \alert{E(\text{all of }E)}])
  \]
  indexed by length $p=|P|$ sequences $Q_\alpha \overline{Q_\beta} \subseteq P_\alpha P_\beta$ where
  $Q_\alpha\subseteq P_\alpha$ and $\overline{Q_\beta}\subseteq P_\beta$.
    
\end{frame}

\begin{frame}
  Column $e$ of $L$ when $e\not\in P$ is
\[
\left[\begin{array}{c}
    N_{\alpha,1,e}g_e\\
    N_{\alpha,2,e}g_e\\
    ...\\
    N_{\alpha,r_1,e}g_e\\
    N_{\beta,1,e}^\perp r_e\\
    N_{\beta,2,e}^\perp r_e\\
    ...\\
    N_{\beta,r_2,e}^\perp r_e
  \end{array}\right]
=
\left[\begin{array}{c}
    N_{\alpha,1,e}\\
    N_{\alpha,2,e}\\
    ...\\
    N_{\alpha,r_1,e}\\
    0\\
    0\\
    ...\\
    0
  \end{array}\right]g_e
+
\left[\begin{array}{c}
    0\\
    0\\
    ...\\
    0\\
    N_{\beta,1,e}^\perp \\
    N_{\beta,2,e}^\perp\\
    ...\\
    N_{\beta,r_2,e}^\perp
  \end{array}\right]r_e
\]
So, for all $e\in E$\alert{, that is $e\not\in P$}:
\[
F(L)_{Q_\alpha \overline{Q_\beta} } = L[Q_\alpha \overline{Q_\beta} E ] =
\]
\[
g_eL\left(\begin{array}{c} \Nal/e\\ \NbePe\backslash e \end{array} \right)[Q_\alpha \overline{Q_\beta} E]
+
r_eL\left(\begin{array}{c} \Nal\backslash e\\ \NbePe/e \end{array} \right)[Q_\alpha \overline{Q_\beta} E].
\]
\end{frame}

\begin{frame}
  \begin{block}{Since deletion and contraction are done only for $e\not\in P$}
    we get a \textbf{Ported} (sdc) or \textbf{Set-pointed} (Las Vergnas) or
    \textbf{relative} (Dao and Hetyei) Tutte Function.
  \end{block}

  
  %  Each determinant $L[Q_\alpha \overline{Q_\beta} E]$ is one of $\binom{2p}{p}$ components, so
  $|Q_\alpha\overline{Q_\beta}|=p$, so $\binom{2p}{p}$ determinants $L[Q_\alpha \overline{Q_\beta} E]$ make the tuple:
  \[
  F(L) =
  g_eFL\left(\begin{array}{c} \Nal/e\\ \NbePe\backslash e \end{array} \right)
+
r_eFL\left(\begin{array}{c} \Nal\backslash e\\ \NbePe/e \end{array} \right)
\]
where
\[
N/e \text{\ means remove the\ } g_e \text{\ or\ } r_e \text{\ but otherwise keep column\ }e
\]
\[
N\backslash e \text{\ means replace column\ }e\text{\ by\ }0.
\]
\begin{block}{Pl\"{u}cker coordinates}
  These determinants can be considered an \emph{affine} version of the
  (projective) Pl\"{u}cker coordinates for the row space of $L$ projected
  into $K^{P_\alpha\dunion P_\beta}$.  We need affine so Tutte's $\mathbf{+}$ identity makes sense.
\end{block}
\end{frame}

\begin{frame}
    \begin{equation*}\tag{*}
  FL\left( \begin{array}{c} \Nal\\ \NbePe \end{array} \right) =
  g_eFL\left(\begin{array}{c} \Nal/e\\ \NbePe\backslash e \end{array} \right)
+
r_eFL\left(\begin{array}{c} \Nal\backslash e\\ \NbePe/e \end{array} \right)
    \end{equation*}
Real deletion/contraction removes $e$ from the ground set of the matroid or other
object, but $N/e,N\backslash e$ still have column $e$.  But (*) holds for all $e\in E$, so
Laplace's expansion is
%we have
a basis expansion:
\[
L[Q_\alpha \overline{Q_\beta} E] = \sum_{A\subseteq E}g_Ar_{\overline{A}}\Nal[Q_\alpha A]\NbePe[\overline{Q_\beta}\overline{A}]
\epsilon(Q_\alpha A,\overline{Q_\beta}\overline{A})
\]
The $A$ term is $\neq 0$ iff $Q_\alpha A$ is a column basis for $\Nal$ and \\
$\overline{Q_\beta} \overline{A}$ is a column basis for $\NbePe$.
So, for each $Q_\alpha \overline{Q_\beta} $
\[
L[Q_\alpha \overline{Q_\beta} E] =
\pm\sum_{A\subseteq E}g_Ar_{\overline{A}}\Nal[Q_\alpha A]\NbePe[\overline{Q_\beta}\overline{A}]
\epsilon(A,\overline{A})
\]
(The non-zero terms all have $|A|=\text{rank}(\Nal)-|Q_\alpha|$.)
\end{frame}


\begin{frame}
  \frametitle{Quick and dirty fix}
  \begin{enumerate}
  \item Drag column $e$ to the far right.\\
    Changes sign of $F(L)$ by $\epsilon(E'e)$.
  \item Left multiply by a determinant 1 matrix that sends the last column
    to $(0,...,1g_e,0,...,1r_e)^{\mathbf{t}}$ (if the top or bottom submatrix has just 1 row,
    do the hack: $\mathbf{N}/e$ is number $\mathbf{N}_{1,e}$ that acts like a matrix with
    columns $E'$ and no rows.)
  \item Drag the row with the $1g_e$ to the bottom.\\
    Changes sign of $F(L)$ by $(-1)^{\rank{\mathbf{\NbePe}}}$
  \item With $e$ deleted/contracted from the $\mathbf{N}$s defining $L$, define $F$ by
    $FL_{Q_\alpha \overline{Q_\beta}} = L[Q_\alpha \overline{Q_\beta} E']$
  \end{enumerate}
  \begin{block}{Result}
    \[
    FL\left( \begin{array}{cc} \Nal \\ \NbePe \end{array} \right)
    =
    \epsilon(E'e)\left(
    g_e (-1)^{\rank(\NbePe)}FL \left( \begin{array}{cc} \Nal/e \\ \NbePe\backslash e \end{array} \right)  +
    r_e FL \left( \begin{array}{cc} \Nal\backslash e \\ \NbePe/e \end{array} \right) \right)
    \]
  \end{block}
\end{frame}


\begin{frame}
  \frametitle{Simplify calculations /w minors via Exterior Algebra}
  \[
  \begin{array}{cc}
     \begin{minipage}{2in}Full $r$-row minors of matrix $N$ with columns indexed by $S$: 
     \end{minipage}
    &
     \begin{minipage}{2.3in}Coefficients when the exterior product of $N$'s row vectors $\ext{N}$
       are expressed in basis\\
       $\{\ext{e}_{i_1}\wedge\ext{e}_{i_2}\cdots\ext{e}_{i_r}|i_1<i_2\cdots<i_r\}$:
     \end{minipage}\\
%    \\
    \begin{array}{ccl}
      (e_1) & (e_2) & (e_3) \\ 
      \alert{a_1} & a_2 & \alert{a_3} \\
      \alert{b_1} & b_2 & \alert{b_3} \\ \hline
      \multicolumn{3}{c}{\ext{N}[e_1e_3]=(\alert{a_1b_3-a_3b_1})}
    \end{array} 
    &
    \begin{array}{cc}
             & (\alert{a_1\mathbf{e}_1}+a_2\mathbf{e}_2+\alert{a_3\mathbf{e}_3}) \\
      \wedge & (\alert{b_1\mathbf{e}_1}+b_2\mathbf{e}_2+\alert{b_3\mathbf{e}_3}) \\ \hline
 %   \multicolumn{2}{c}{((a_1b_2-a_2b_1)\mathbf{e}_1\mathbf{e}_2 +
 %   (a_1b_3-a_3b_1)\mathbf{e}_1\mathbf{e}_3 +
 %     (a_2b_3-a_3b_2)\mathbf{e}_2\mathbf{e}_3)}
      \multicolumn{2}{c}{( 
      \alert{(a_1b_3-a_3b_1)\mathbf{e}_1\mathbf{e}_3} +
      \cdots
      )}
    \end{array} 
  \end{array}
  \]
  We sometimes omit the $\wedge$ and we can always write:
  \[
  \text{(Exterior product)} \ext{N} = \sum_{A\subseteq S; |E|=r}\ext{N}[A]\ext{A}
  \]
  Each subset $A$ is ordered $a_1 a_2 \ldots a_r$ \textbf{arbitrarilly} but $\ext{A}$
  denotes the exterior product of (row coordinate vectors) \textbf{in the same order}
  \[
  \ext{A} = \ext{a}_1\ext{a_2}\ldots\ext{a_r}
  \]
\end{frame}

\begin{frame}{Catalogs of Oriented Matroid operations\\
    on the OM of matrix $N$ and on
    %THE EXTERIOR PRODUCT
    $\ext{N}=\wedge(\text{rows}(N))$}

\hspace*{-0.35in}$
    \begin{array}{ccc}
\text{domain $D$ of operations:} &  \text{chirotopes}            & \text{exterior products} \\
\text{$D$ is which functions:}                & \pm\chi:B\mapsto\text{sign}(N[B]) & \ext{N}:B\mapsto\ext{N}[B] \\
\text{type of fun. value}                   & \text{sign}\in\{0,+,-\} &  \text{\bf field value; number}\\ \hline
\text{deletion\ } \bullet\backslash A  & \pm\chi':B\mapsto\chi(B) & \ext{N}\backslash A:B\mapsto\ext{N}[B]  \\
\text{contraction\ }\bullet / A             & \pm\chi':B\mapsto\chi(BA) & \ext{N}/A:B\mapsto\ext{N}[BA] \\

\text{duality\ }\bullet^{\perp} & \pm\chi^{\perp}:B\mapsto\chi(\overline{B})\epsilon(\overline{B}B) &
\ext{N}^{\perp}:B\mapsto\ext{N}[\overline{B}]\epsilon(\overline{B}B) \\ \\
\multicolumn{3}{c}{\parbox{\textwidth}{We must choose some global orientation $\epsilon$ in order to define duality as an exterior alg. operation!
    \\
    $\epsilon$ is an alternating sign function on all finite sequences of elements. }}\\ \\ 
\text{This implies} & \multicolumn{2}{c}{(\ext{N}\backslash X)^\perp = \epsilon(S')\epsilon(S'X)(\ext{N}^\perp/X)}  \\
\text{commutations} & \multicolumn{2}{c}{(\ext{N}/X)^\perp = \epsilon(S')\epsilon(S'X)(-1)^{|X|r\ext{N}^\perp}(\ext{N}^\perp\backslash X)}\\

    \end{array}
    $
\end{frame}


\begin{frame}{Our setup - again}
  \begin{itemize}
  \item
    Matrices $\Nal$, $\NbePe$; full row rank, columns indexed by
    $P\dunion E$. $\text{rank}(\Nal)+\text{rank}(\NbePe)=|E|+|P|$.\\
    $P_{\alpha},P_{\beta}\leftrightarrow P$, $P_{\alpha}\cap P_{\beta}=\emptyset$.
  \item
    Weight (parameter) matrices\\
    $G=\text{diag}\{g_e\}_{e\in E} $,
    $R=\text{diag}\{r_e\}_{e\in E} $.
  \item
    Matrix with columns $P_\alpha \dunion P_2 \dunion E$
    \[
    L\left( \begin{array}{c} \Nal\\ \NbePe \end{array} \right)
    = \left[\begin{array}{c|c|c} \Nal(P)  &  0  &  \Nal(E)G \\  \hline
0  & \NbePe(P)  &  \NbePe(E)R \end{array}\right]
    \]
  \end{itemize}

  Define
  \[
  F(L)=((\binom{2p}{p})-\text{tuple of determinants\ } L[Q_\alpha\overline{Q_\beta}E])
  \]
  indexed by sequences $Q_\alpha \overline{Q_\beta} \subseteq P_\alpha P_\beta$ where
  $Q_\alpha\subseteq P_\alpha$, $\overline{Q_\beta}\subseteq P_\beta, |Q_\alpha \overline{Q_\beta}|=p=|P|$.
    
\end{frame}

\begin{frame}

\hspace{-0.2in}{\small
    $
    L\left( \begin{array}{c} \Nal\\ \NbePe \end{array} \right)
    = \left[\begin{array}{c|c|c} \Nal(P)  &  0  &  \Nal(E)G \\  \hline
0  & \NbePe(P)  &  \NbePe(E)R \end{array}\right]
    $\ \ \ \
    $
    F(L)=\text{tuple\ } (L[Q_\alpha \overline{Q_\beta} E])
    $
}

\vspace{0.1in}Translate into exterior algebra definitions:
\[
\begin{split}
  \ext{L}\left( \begin{array}{c} \eNal\\ \eNbePe \end{array} \right)
   & := (\Is(\eNal)(P_\alpha) + \Is_G(\eNal(E)))\wedge(\Vs(\eNbePe)(P_\beta) + \Vs_R(\eNbePe)(E)) \\
  &  = (\Is_G(\eNal)\wedge\Vs_R(\eNbePe))
\end{split}
\]

\[
\begin{split}
  \ext{F}_E(\ext{L})& := \ext{L}/E = \sum_{Q_\alpha,\overline{Q_\beta}}\ext{L}[Q_\alpha \overline{Q_\beta} E]\ext{Q}_\alpha\overline{\ext{Q}_\beta} \\
  & =   ((\Is(\eNal)\backslash   e\alert{(\text{no\ }\ext{e})} + g_e(\Is(\eNal)/e)\wedge\alert{\ext{e}})                   \\
  & \;\;\;\;  \wedge  (\Vs(\eNbePe)\backslash e\alert{(\text{no\ }\ext{e})} + r_e(\Vs(\eNbePe)/e)\wedge \alert{\ext{e}}) ) / E \\
\alert{\text{2 of 4 terms}}  & = \Big( r_e  \;\;\; \;\;\;\;\;\;\;\;\;\;\;\;\;\; \Is(\ext{\eNal})\backslash e\wedge (\Vs(\eNbePe)/e)      \wedge  \alert{\ext{e}} \\
\alert{\text{vanish}}  & \;\;+ g_e (-1)^{r(\eNbePe)}(\Is(\eNal)/e)\wedge(\Vs(\eNbePe)\backslash e)    \wedge  \alert{\ext{e}}\Big) / E 
  \end{split}
\]
 

  %((\binom{2p}{p})-\text{tuple of determinants\ } \ext{L}[Q_\alpha \overline{Q_\beta} E])
\end{frame}


\newcommand{\eNbe}{\ensuremath{\ext{N}_\beta}}
\newcommand{\Nbe}{\ensuremath{N_\beta}}

\begin{frame}

\hspace{-0.2in}{\small
    $
    L\left( \begin{array}{c} \Nal\\ \NbePe \end{array} \right)
    = \left[\begin{array}{c|c|c} \Nal(P)  &  0  &  \Nal(E)G \\  \hline
0  & \NbePe(P)  &  \NbePe(E)R \end{array}\right]
    $\ \ \ \
    $
    F(L)=\text{tuple\ } (L[Q_\alpha \overline{Q_\beta} E])
    $
}

\[
\begin{split}
  \ext{F}_E(\ext{L}) = \ext{L}/E 
   = \Big( r_e  \;\;\; \;\;\;\;\;\;\;\;\;\;\;\;\;\; & \Is(\eNal\backslash e)  \wedge (\Vs(\eNbePe/e))      \wedge  \ext{e} \\
   \;\;+ g_e (-1)^{r(\eNbePe)} ( & \Is(\eNal/e))\wedge(\Vs(\eNbePe\backslash e))   \wedge  \ext{e}\Big) / E \\
%  =  \Big( r_e  \Is(\eNal\backslash e)\wedge & (\Vs(\eNbePe/e))      
%  + g_e (-1)^{r(\eNbePe)}(\Is(\eNal/e))\wedge(\Vs(\eNbePe\backslash e))   \Big) \\
   %  & \wedge  \ext{e} / E \\
\end{split}
\]
\[  = r_e\left(\ext{L}\left(\begin{array}{c} \eNal\backslash e \\
    \eNbePe/e  \end{array} \right)  \wedge \ext{e} /E \right) +
   g_e(-1)^{r(\eNbePe)}\left(\ext{L}\left(\begin{array}{c} \eNal /e \\
    \eNbePe \backslash e \end{array} \right) \wedge \ext{e} /E \right)
   \]
   
\hspace*{-0.3in}
\mbox{\begin{minipage}{6in}${(\ext{N}\backslash e)^\perp = \epsilon(S')\epsilon(S'e)(\ext{N}^\perp/e)}$ ;
         ${(\ext{N}/e)^\perp = \epsilon(S')\epsilon(S'e)(-1)^{|\{e\}|r\ext{N}^\perp}(\ext{N}^\perp\backslash e)}$\end{minipage}}


%\vspace{0.3in}
\begin{block}{Result}
\hspace*{-0.3in}  \mbox{\begin{minipage}{6in}$ = \epsilon(S)\epsilon(S'e)(r_e\left(\ext{L}\left(\begin{array}{c} \eNal\backslash e \\
    (\eNbe\backslash e)^\perp  \end{array} \right)  \wedge \ext{e} /E \right) +
   g_e\left(\ext{L}\left(\begin{array}{c} \eNal /e \\
    (\eNbe / e)^\perp \end{array} \right) \wedge \ext{e} /E \right))
   $\end{minipage}}
\end{block}

\end{frame}




\begin{frame}
  With $\ext{L}(\eNal\;\; \eNbe)=\ext{L}\left(\begin{array}{c}\eNal \\ \eNbePe \end{array} \right)$, and more sign calculations:
    \begin{definition}
      For $E$, $P$ sets written as ordered sequences,
      \[
      \ext{F}_E(\eNal\;\;\eNbe) = \ext{L}(\eNal\;\;\eNbe)/E
      \]
    \end{definition}
    \begin{theorem}
      \begin{multline*}
        \epsilon(PE)\ext{F}_E(\eNal\;\;\eNbe)=\\
        \epsilon(PE')
        \left(g_e\ext{F}_{E'}(\eNal/e\;\; \eNbe/e) +
      r_e\ext{F}_{E'}(\eNal\backslash e\;\; \eNbe\backslash e)\right)
      \end{multline*}
  \end{theorem}

\end{frame}

\begin{frame}
  \begin{corollary}
       $
       \ext{F}=\ext{F}_E(\eNal\;\;\eNbe) =
          \pm\sum_{H\subset E}g_H r_{\overline{H}}
       \ext{F}_\emptyset
       (\ext{N}_\alpha /H \backslash\overline{H} \;\;
        \ext{N}_\beta/H\backslash\overline{H})
        $
  \end{corollary}
  \begin{block}{Applying the Tutte Polynomial}
    \begin{itemize}
      \item
    THEREFORE:  We can obtain $\ext{F}$ by doing a ported Tutte decomposition, keeping track of the
    contraction and deletion order $H$,$\overline{H}$. Then, when we get nodes with no more $e\in E$,
    substitute the exterior product
     $\ext{F}_\emptyset
       (\ext{N}_\alpha /H \backslash\overline{H}
    \ext{N}_\beta/H\backslash\overline{H})$ which is
    $\ext{N_\alpha}/H|P\wedge\ext{N_\beta}/H|P$
  \item When the $\ext{N}_\alpha=\ext{N}_\beta$ represent regular matroids by unimodular matrices, we can do the
    familiar substition of the Tutte function value(s) $F([M/H|P])$ for the matroid variable (product) $[M/H|P]$.
    (More research needed to develop how to make sure proper $\pm$ signs are maintained everywhere.)
    \end{itemize}
  \end{block}
\end{frame}

\begin{frame}
      \begin{corollary}
    \begin{enumerate}
     \item
       Componentwise, $\sum_{Q_\alpha,Q_\beta} \ext{F}_E[Q_\alpha \overline{Q}_\beta]
       \ext{Q}_\alpha\overline{\ext{Q}_\beta}=$
       \[
       \begin{split}
       =&\pm\sum_{Q_\alpha,Q_\beta}\sum_{H\in E} g_H r_{\overline{H}}
       \ext{N}_\alpha[Q_\alpha H]
       \ext{N}_\beta^\perp[\overline{Q_\beta}\overline{H}]\\
       =&
       \pm\sum_{Q_\alpha,Q_\beta}\sum_{H\in E}g_H r_{\overline{H}}
       \ext{N}_\alpha[Q_\alpha H]
       \ext{N}_\beta[Q_\beta H]
       \end{split}
       \]
     \item
       Two expr. for  products of numbers  $\ext{N}_\alpha[Q_\alpha H]
       \ext{N}_\beta[Q_\beta H]$:
       \[
       (\ext{N}_\alpha/Q_\alpha)[H]\cdot (\ext{N}_\beta/Q_\beta)[H]
       =
        (\ext{N}_\alpha/H)[Q_\alpha]\cdot (\ext{N}_\beta/H)[Q_\beta]
       \]
     \item
       It's non-zero iff $H$  is a common basis (in the matroids of)
       $\ext{N}_\alpha/Q_\alpha$ and $\ext{N}_\beta/Q_\beta$\\
       iff $Q_\alpha$ is a basis in $\ext{N}_\alpha/H$ and
       $Q_\beta$ is a basis in $\ext{N}_\beta/H$
    \end{enumerate}
  \end{corollary}
\end{frame}



\begin{frame}{Weighted Laplacian-like matrices}

  Generalize a graph's incidence matrix: Make $P$ label the rows,
  $E$ the columns of any matrices $A_\alpha,A_\beta$.  Take all $r_e\neq 0$.
  Then,
  $N_\alpha = ( I(P)\;\; A_\alpha(E))$ and $N_\beta= (I(P)\;\; A_\beta(E))$, and
  \vspace{-0.1in}
   \[
    L\left( \begin{array}{c} \Nal\\ \NbePe \end{array} \right)
    = \left[\begin{array}{c|c|c} I  &  0  &  A_\alpha  G \\  \hline
0  & -A_\beta^t  &  IR \end{array}\right] = L\left( \Nal\;\;\; \Nbe \right). \text{\ Do row ops:}
    \]
    \vspace{-0.1in}
    \[
    \left(\begin{array}{cc} I & -A_\alpha  G R^{-1} \\
      0 & R^{-1} \end{array} \right) L
    = \left(\begin{array}{ccc}
        I & A_\alpha GR^{-1}A_\beta^t& 0 \\
        0 & -R^{-1}A_\beta ^t    & I
        \end{array} \right), \text{\ and therefore}
    \]

    \[
    \epsilon(Q_\alpha\overline{Q_\alpha})\ext{F}_E(\ext{L})[Q_\alpha \overline{Q_\beta}] =
       \frac{1}{r_E} \sum_{B\in E}g_B r_{\overline{B}} A_\alpha[\overline{Q_\alpha} B] A_\beta[{\overline{Q_\beta}}B]  
    \]
    is the Cauchy-Binet expansion of any  minor $(\overline{Q_\alpha},\overline{Q_\beta})$ of the
    weighted graph Laplacian-like matrix $A_\alpha GR^{-1}A_\beta^t$.\\
    (Note $\frac{1}{r_E}r_{\overline B}=(r^{-1})_B$.)
\end{frame}

\begin{frame}{Examples}
  
 \hspace*{-0.25in}$
  \begin{array}{lc}
    \begin{array}{c}
      N_\alpha=N_\beta=N; A = \text{\ graph's}\\
      \text{incidence matrix w/ columns}\\
      (0,..0,1,0,..,-1,0,..,0)^t\text{ for}\\
      \text{each edge; reps. graphic matroid.}
    \end{array}
    &
    \begin{array}{c}
      \text{The all-minors}\\
      \text{Matrix Tree Theorem}\\
    \text{for weighted undirected graphs}
    \end{array}
    \\ \\
    \begin{array}{c}
      A_\alpha, N_\alpha \text{ as above. }A_\beta=\\
      \text{only the\ } +1\text{\ entries of\ }A\text{ for a}\\
      \text{directed graph, so }+1\text{\ is for}\\
      \text{an edge head on a vertex.}\\
    \end{array}
    &
    \begin{array}{c}
      \text{The all-minors}\\
      \text{ Matrix Tree Theorem} \\
    \text{for weighted directed graphs}
    \end{array}
  \\ \\
    \begin{array}{c}
      N_\alpha=N_\beta=N; A = \text{\ gain graph's}\\
      \text{incidence matrix w/ columns}\\
      (0,..,0,1,0,..,-\gamma_e,0,..,0)^t\text{ for}\\
      e\text{ with gain } \gamma_e \in \mathbf{C}.
    \end{array}
    &
    \begin{array}{c}
      \text{All-minors expansions of a}\\
      \text{signed, genr. gain graph's Laplacian}
    \end{array}\\
    \multicolumn{2}{c}{\text{NB: Edge Gains\ }\gamma_e\text{\ are DIFFERENT ATTRIBUTES}}\\
    \multicolumn{2}{c}{\text{from weights/parameters\ }g_e,r_e}
  \end{array}
  $
\end{frame}


\begin{frame}{All-Minors Digraph Matrix Tree Theorem Example}
  \begin{center}
    \input{DirMTTDemo.pspdftex}
  \end{center}
  \begin{center}
    This contributes the term\\
    ${\color{magenta}g_F r_{\overline{F}}}{\color{blue}\ext{N}_\alpha[Q_\alpha F]}
    {\color{red}\ext{N}_\beta[Q_\beta F]}$.\\
    The {$\color{blue}\mathbf{q_{\alpha 1},q_{\alpha 2}}$ port edges} $\cup$  the
    {$\color{purple}f_i$} elements {\color{blue}as edges in the graphic matroid} comprise
    a spanning tree.\\
     The {$\color{red}\mathbf{q_{\beta 1},q_{\beta 2}}$ port arrows} $\cup$  the
    {$\color{magenta}f_i$} elements {\color{red} as arrows in a partition matroid} comprise
    a basis.  Each {\color{red}part (a red cirle)} of the partition is the
    set of {\color{red}arrows} incident to a vertex, except the star vertex.
  \end{center}
\end{frame}

\begin{frame}
  \begin{center} \input{DirMTTDemoSmaller.pspdftex}
  \end{center}
  \begin{center}
    \mbox{\begin{minipage}{0.45\textwidth}
          Contract the ports.\\
          \input{DirMTTCrPorts.pspdftex}\\
          Count the bases in common.
          ${\color{magenta}g_F r_{\overline{F}}}{\color{blue}\ext{N}_\alpha/Q_\alpha[F]}
    {\color{red}\ext{N}_\beta/Q_\beta[F]}$.
           \end{minipage}}
    \mbox{\begin{minipage}{0.35\textwidth}
        Contract the non-ports.\\
        \input{DirMTTCrElts.pspdftex}\\
        $\alpha$ and $\beta$ ports are bases in the
        contracted $N_\alpha$ and $N_\beta$ matroids.
              ${\color{magenta}g_F r_{\overline{F}}}{\color{blue}\ext{N}_\alpha/F[Q_\alpha]}
    {\color{red}\ext{N}_\beta/F[Q_\beta]}$.
          \end{minipage}}
  \end{center}
\end{frame}





\begin{frame}{Resistive Network style problems Solved by Tutte Functions}
  With the $\binom{2p}{p}$ tuple of $(p+n)\times(p+n)$
  minors of $\ext{L}(\ext{N}\;\;\ext{N})$ all including columns $E$, every electrical style
  problem can be analyzed.
  \begin{block}{Input}
    Choose $1\le k\le p$, and choose from among the set of $2p$ variables
    $\{v_1,...,v_p;i_1,...,i_p\}$ these 4 subsets:
    \begin{itemize}
    \item $k$ ``source'' variables $S=\{s_1,...,s_k\}$.
    \item $p-k$ ``zero'' variables $Z=\{z_1,...,z_{p-k}\}$ so $S\cap Z=\emptyset$, ie. $|S\cup Z|=p$.
    \item $k'$ ``response'' variables $R=\{r_1,...,r_{k'}\}$
    \item $p-k'$ ``don't care'' variables $D=\{d_1,...,d_{p-k}\}$ 
    \end{itemize}
  \end{block}
\end{frame}

\begin{frame}{Question and Answer}
%  \begin{block}{Question}
    Does there exist a $k'\times k$ matrix $\Xi$ for all source values $s_i$,
    \[
    \Xi(s_1,...,s_k)^t = (r_1,...,r_{k'})^t \text{\ is the unique solution in}
    \]
    \[
    \begin{split}    \{ (r_1,...,r_{k'}) |& L( \ext{N}\;\;\ext{N} )(v_1,...,v_p;i_1,...,i_p)^t = 0,\\
      & s_i \text{\ are given\ }, z_i=0,\\
      &\text{and there exist\ } d_1,...,d_{p-k'}\}??
    \end{split}
    \]
%  \end{block}
  \begin{block}{Answer for $S,Z,R,D$}
    \begin{itemize}
     \item
        If $\ext{F}_E(\ext{N}\;\;\ext{N})[SZ]\not=0$, then $\Xi$ exists.

      \item
        If so, every minor of $\Xi$ is, for some
        $Q_{\text{Num1}},Q_{\text{Num2}},Q_{\text{Den1}},Q_{\text{Den2}}\subset P_\alpha P_\beta$
        \[
        \frac{\sum_{F\subseteq E}\ext{N}[Q_{\text{Num1}}F]\ext{N}[Q_{\text{Num2}}F]g_Fr_{\overline{F}}}
          {\sum_{F\subseteq E}\ext{N}[Q_{\text{Den1}}F]\ext{N}[Q_{\text{Den2}}F]g_Fr_{\overline{F}}}
          \]
          Remember, each (field valued) sum, being a component of
          $\ext{F}_E(\ext{N}\;\;\ext{N})$,  IS A TUTTE FUNCTION.
    \end{itemize}
  \end{block}
\end{frame}
          

\begin{frame}{Ported Tutte Decomposition (incomplete)}
\hspace{-0.2in}\mbox{\input{K4Decomp.pspdftex}}
\end{frame}




\begin{frame}{Conditions (what sets $F$ are enumerated by one det. $C_i$)
}
The \textbf{conditions ...}
are on the rank, nullity of $F$ and, WHAT ORIENTED MINOR is \
$G/F\setminus (E\setminus F)$, the minor
with ONLY PORT EDGES from contracting $F$
and deleting the other resistor edges, leaving the
ports.

The conditions for a given $C_k$ \textit{sometimes}
make all the signs the same (eg: $C_i$ and 
$C_j$ in 1-port equivalent resistance $R=C_i/C_j$)

\textit{Othertimes}, the oriented \textbf{P-minors}
in the completed Tutte decomposition of $C_k$ determine
some + and some - signs.

\begin{center}
\begin{minipage}{0.3\textwidth}
\begin{tabular}{c}
When $[G/F|P]$ is \\
\input{c2plus.pspdftex} \\
the term is \\
\Remph{{\LARGE\bf +}}$g_Fr_{E\setminus F}$ \\
\end{tabular}
\end{minipage}
\begin{minipage}{0.3\textwidth}
\begin{tabular}{c}
When $[G/F|P]$ is\\
\input{c2minus.pspdftex}\\
the term is\\
\Remph{{\LARGE\bf -}}$g_Fr_{E\setminus F}$\\
\end{tabular}
\end{minipage}
\end{center}

\end{frame}







\begin{frame}{Known to EEs: Linear electrical networks with IDEAL AMPLIFIERS}

  $N_\alpha i(P,E)=0$  expresses Kirchhoff's current law on currents $i_e$ in the network edges (along edge direction)
  and currents $i_p$ into vertices from external connections.\\[.1in]
  $N_\beta^\perp v(P,E)=0$ expresses Kirchoff's voltage law: The voltage rise along a network edge 
  $v_e=v_h-v_t$ is the difference of the head and tail vertex potentials.  (Sometimes the vertex potentials are
  imposed by external connections.)\\[0.1in]
  $N_\alpha=N_\beta$ in ordinary resistor networks.

  \begin{block}{Different Graphs for $N_\alpha$ and $N_\beta$}
    W. K. Chen models networks with ideal amplifiers by $N_\alpha$ by one graph on $(P,E)$ called the
    \textbf{Current Graph} and another graph also on $(P,E)$ called the \textbf{Voltage Graph}.
  \end{block}

\end{frame}
    

\begin{frame}
\frametitle{Voltage and Current graphs $G_V$, $G_I$}

\framebox{\begin{minipage}{0.52\textwidth}
``Voltage graph'' $G_V$ (EEs Hasler and Neirynck 1986, not Gross, et. al.)
$\mathbf{v}\in \text{Coboudaries}$ W/ SOME $v_e \equiv 0$
\end{minipage}}
\framebox{
\begin{minipage}{0.4\textwidth}
``Current graph'' $G_I$ represents KCL
$\mathbf{i}\in \text{Cycles}$ WITH SOME FLOWS $\equiv 0$
\end{minipage}}

\begin{itemize}
\item 
They are EQUAL GRAPHS for resistor networks.

\item
For networks with idealized amplifiers, they are not 
equal.  

\input{idealAmp.pspdftex}

The output voltage and current are whatever makes the input
voltage and current BOTH BE zero.

\item
(More) realistic amp. model $=$ idealized amp. $+$ 
resistors.

\end{itemize}


\framebox{
\begin{minipage}{0.2\textwidth}
\Remph{open}
\vspace{-0.1in}
\[
G_v = G\backslash e
\]
\[ 
G_I = G\backslash e
\]
\end{minipage}}
\framebox{\begin{minipage}{0.2\textwidth}
\Remph{short}
\vspace{-0.1in}
\[
G_v = G/e
\]
\[
G_I = G/e
\]
\end{minipage}}
\framebox{
\begin{minipage}{0.2\textwidth}
\Remph{nullator}
\input{nullator.pspdftex}
\vspace{-0.1in}
\[
G_v = G/e
\]
\[ 
G_I = G\backslash e
\]
\end{minipage}}
\framebox{\begin{minipage}{0.2\textwidth}
\Remph{norator}
\input{norator.pspdftex}
\vspace{-0.1in}
\[
G_v = G\backslash e
\]
\[
G_I = G/e
\]
\end{minipage}}

\end{frame}


\newcommand{\COEF}{\Reals}

\begin{frame}{Chain Complexes View (Alg. Topology, Homological Alg.)}

 A graph is a $k$-dim simplicial complex $X$ with $k=1$.\\
  In general, for us,
  the $k$-chains $C_k=Z[P\dunion E]$ $=\{ \sum_{x\in P\dunion E} c_e e\}$ are
  the free abelian group with basis $P\dunion E$.\\
  The $k$-cochains $C^k = \text{Hom}(C_k,\COEF)$ is the $\COEF$-module of linear maps
  from $C_k$ to a coefficient ring $\Reals$.\\
  The $k$-complex $X=\dunion_{j=0}^{k}X_j$ ($X_j$ is the set of $j$-simplices) determines,
  (or the chain complex might just be subspaces given with)
  \textbf{boundary maps} $\partial_j:C_j\rightarrow C_{j-1}$ for $j=0,...,k$ that satisfy
  $\partial_{j-1}\circ\partial_{j}=0$ for each $j$.\\
  The dual $\delta^{j}:C^{j-1}\rightarrow C^{j}$ is defined by
  $(\delta^j(u^*))(v)=u^*(\partial_j(v))$.
\end{frame}
\begin{frame}
  In the case $N_\alpha=N_\beta$, generalizing:

  \begin{itemize}
  \item
    $\ext{N}$ ($\wedge$ of the rows on $N_\alpha$) represents the $k$-coboundary group
    $B^k=\text{img}(\delta_k)$.
  \item
    The equation $N_\alpha\left(\begin{array}{c} I \\ G  \end{array}\right)(J_P\;\;X_E)^t = 0$
    says $\left(\begin{array}{c} I \\ G  \end{array}\right)(J_P\;\;X_E)\in Z_1$, is a $k$-cycle.
     (Electrically, a flow of currents in edges.)
  \item
    $\ext{N}^\perp$ ($\wedge$ of the rows of $N^\perp$) represents the $k$-cycle group
    $Z_k=\text{ker}(\partial_k)$. 
  \item
    The equation $N^\perp\left(\begin{array}{c} I \\ R  \end{array}\right)(V_P\;\;X_E)^t = 0$
    says $\left(\begin{array}{c} I \\ R  \end{array}\right)(V_P\;\;X_E)\in Z_1$, is a $k$-coboundary
    $\delta_k \psi$.
    (Electrically, $\delta_1 \psi$ maps each edge (1-simplex) to the difference of electrical
    potential assigned to vertices (a 1-cochain) $\delta_1(\psi)(v_0v_1) = \psi(v_1)-\psi(v_0)$.
  \end{itemize}
  
\end{frame}

\begin{frame}{Electribraic Topology--Happy Birthday, have fun Tom.}
  \begin{minipage}{0.5\textwidth}
    The left edge is a port containing an electric source.\\
    \Remph{Red:1-coboundary} Diffs of a potential $\psi$ (0-cobdy).  Coeffs \Remph{$v_e$} are $\ge 0$
    for the edge $e=u_0u_1$ dirs indicated, so \Remph{$v_e=\psi(u_1)-\psi(u_0)$}.\\
    \Bemph{Blue:1-cycle} Current (charge flow), $\ge 0$ with arrow. In resistor edge w/ conductance $g_e$,
    \Bemph{$c_e=g_e(-(\psi(u_1)-\psi(u_0)))$}.\\
    In the port, either the pot. diff. or the current is set by the source.  Other
    coefficents are determined by Kirchhoff's and Ohm's laws.
  \end{minipage}
  \begin{minipage}{0.40\textwidth}
    \input{CobCyc.pspdftex}
  \end{minipage}
\end{frame}


\begin{frame}{Research Notes Accumulated in Frames}
\end{frame}

\begin{frame}{Codomain of Our Tutte Function--Equivalent Definitions}
  \begin{itemize}
  \item Rank 1 Step $p$ antisymmetric tensors.
  \item Exterior products of $p$ vectors.
  \item Dim. $p \times 2p$ full rank $p$ matrices modulo left action
    of SL$_p$.
  \item All length $p C (2p)$ tuples representing (projective space) points in Grassmannian
    $G(p,2p)$.
    \item All length $p C (2p)$ tuples satisfying the Grassmann-Plucker relations.
  \end{itemize}
  It is a subset of the grade  $p$ subspace of the exterior algebra.  It is closed under
  linear combinations of the form $g F(N/e) + r F(N\setminus e)$.
\end{frame}
  


\begin{frame}{Unused, maybe wrong slides}
\end{frame}
 


 


\begin{frame}{Tutte equations, functions and Good Questions}
  \begin{enumerate}
  \item For all $\ext{N}$ with separator $e\in S(\ext{N})$,
    \[
    F(\ext{N}) = g_eF(\ext{N}/e) + r_e(\ext{N}\backslash e)
    \]
  \item When $\ext{N}=\ext{N_1}\oplus\ext{N_2}$,
    \[
    F(\ext{N}) = F(\ext{N_1})F(\ext{N_2})
    \]
  \item When $\ext{N}$ is indecomposible,
    \[
    F(\ext{N}) = i_{\ext{N}}
    \]
  \end{enumerate}
  $F$ is Tutte function when all the Tutte equations are satisfied.

  
  
  Good Questions: When does $\mathcal{N}$ and parameters ACTUALLY
  HAVE a Tutte function?  If so, what is a \emph{universal} Tutte function?

\end{frame}

\begin{frame}{Some answers--for Graphs and Matroids}

  \begin{block}{Only loops and coloops need initial values}
    The only
    $\ext{N}$ with no separators and no $\ext{N}=\ext{N}_1\oplus\ext{N_2}$
    for $\ext{N}_i\neq\emptyset$ are $\mathbf{loop}(e)$
    and $\mathbf{coloop}(e)$.
  \end{block}


  \begin{block}{The famous Tutte Polynomial}
  Adding all $g_e=r_e=1$, the Tutte polynomial $F(\ext{N})(x,y)$
  obtained from $i_{\mathbf{loop}(e)}=x$, 
  $i_{\mathbf{coloop}(e)}=y$ and $i_{\mathbf{\emptyset}}=1$.
  is a universal Tutte function.
  \end{block}
  
  \begin{block}{Normal Tutte Functions for Matroids}
    (Zaslavsky, Bollob\'{a}s/Riordan) With arbitrary $g_e,r_e$, and $x$,$y$, the \emph{normal} Tutte functions
    for matroids are obtained with $i_{\mathbf{coloop(e)}}=g_ey + x$,
  $i_{\mathbf{loop(e)}}=r_ex + y$ and  $i_{\mathbf{\emptyset}}=1$.
    They are exactly the ones with a weighted rank-nullity generating function.\\
    There's a big story about what relationships among the
    $g_e, r_e, i_{\mathbf{coloop(e)}} ,i_{\mathbf{loop(e)}}, i_\emptyset$ give others.
  \end{block}

\end{frame}

\begin{frame}{Hopf Alg. from Minor Systems (Krajewski, Moffatt, Tanasa 2017)}
  \begin{definition}[Minor System]
    \begin{itemize}
      \item
    Finite combinatorial objects $\{N\}$  w/ ground sets $E(N)$, graded by
    $|E(N)|$; unique $1$ with $E(1)=\emptyset$; $E(N)$ consists of objects at level $|E(N)|$.
  \item
    For distinct $e,f\in E(N)$, deletion \& contraction ops so both ($\backslash\backslash e$ or $// e$) commute
    with both ($\backslash\backslash f$ or $// f$).
    \end{itemize}.
  \end{definition}


\end{frame}

\begin{frame}{Some generalization..}
  Tutte equations are satisfied in a very general setup:
  \begin{enumerate}
  \item Elements $\{e\}$ each with parameters $g_e, r_e$.
  \item A category $\mathcal{N}$
    of objects $\ext{N}$ each with ground set $S=S(\ext{N})$
    of elements.
  \item For some \emph{decomposible}
    $\ext{N}$, for one or more \emph{separators} $e\in S(\ext{N})$, the
    \emph{contraction} and \emph{deletion} operations are defined with
    results $\ext{N}/e$ and $\ext{N}\backslash e$ in $\mathcal{N}$,
    with ground sets
    $S(\ext{N})\backslash\{e\}$
  \item Some $\ext{N}=\ext{N}_1\oplus\ext{N_2}$ are direct sums, where
    $S(\ext{N}_1)\cap S(\ext{N}_2)=\emptyset$.
  \item For each indecomposible $\ext{N}$ with no separators there is
    an additional parameter $i_{\ext{N}}$ called the \emph{initial value}.
  \end{enumerate}
\end{frame}
\end{document}

     Weighted/parametrized/colored Tutte fun. by Bollob\'{a}s/Riordan (1999), 
weighted/parametrized/colored Tutte fun



