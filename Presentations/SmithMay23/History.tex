\block{History}{
  \alert{Not well-known 1847 graph theory paper!} Kirchhoff's\cite{Kirchhoff}  ``Matrix Tree Theorem'' paraphrased:
  The solution to the linear resistive electrical network (Kirchhoff's and Ohm's laws)
  problem is comprised of ratios of certain tree or forest enumerators.

  Maxwell\cite{MaxR}
  expressed those problems with matrices.  EEs call Kirchhoff's result ``Maxwell's rule.''
  

  The theorem only for $\ext{N}_\alpha=\ext{N}_\beta$, with a lot of background and motivation
  appears in \cite{TutteEx}.  In \cite{sdcOMP} the setup is related to oriented matroids and their relation to
  non-linear electrical network problem well-posedness: Is
      $L\left( \begin{array}{c} \Nal\\ \NbePe \end{array} \right)$ non-singular for all positive $r_e$, $g_e$?

}
