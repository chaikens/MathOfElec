\documentclass[24pt, landscape, margin=0mm, innermargin=15mm, blockverticalspace=15mm, colspace=15mm, subcolspace=8mm]{tikzposter}
%default fontsize is 25pt

\geometry{paperwidth=36in, paperheight=47in}



%%%%%%%%%%%%%%%%%%%%%%%%%%%%%%%%%%%%%%%%%%%%%%%%%%%%%%%%%%%%%%%%%%%%%%
% \tableau{} sets up an alignment with zero spacing, boxing each entry
% with \tableaucell{}.
%%%%%%%%%%%%%%%%%%%%%%%%%%%%%%%%%%%%%%%%%%%%%%%%%%%%%%%%%%%%%%%%%%%%%%

\usepackage{amsthm, amsmath, amssymb, amsopn}
\usepackage{lipsum}
\usepackage{pdfpages}

\usepackage{cite}
\usepackage{hyperref}

\definecolor{Blue}{rgb}{.255,.41,.884} % RoyalBlue of svgnames
\definecolor{Red}{rgb}{1, 0, 0} % Red of svgnames
\definecolor{Green}{rgb}{.196,.804,.196} % LimeGreen of svgnames
\definecolor{Yellow}{rgb}{1,.648,0} % Orange of svgnames

\def\B{{\mathcal{B}}}
\newcommand{\calp}{{\cal P}}
\newcommand{\cals}{{\cal S}}
\newcommand{\calw}{{\cal W}}
\newcommand{\calh}{{\cal H}}
\newcommand{\bF}{{\mathbb F}}
\newcommand{\fs}{\mathfrak S}
\newcommand{\bari}{\overline{\imath}}
\newcommand{\barj}{\overline{\jmath}}
\newcommand{\fg}{\mathfrak G}
\newcommand{\arr}{\begin{picture}(30,5) \put(3,2.5){\vector(1,0){24}} \end{picture}}
\newcommand{\arrl}[1]{\stackrel{#1}{\arr}}

%Dave's macros for fixing running heads, xarrows...
%\addtolength\headheight{2pt}

%\setlength{\parskip}{5pt}
%%\setlength{\parindent}{0pt}
%\setlength{\textheight}{9in}
%\setlength{\topmargin}{-.50in}
%\setlength{\oddsidemargin}{.125in}
%\setlength{\evensidemargin}{.125in}
%\setlength{\textwidth}{6.25in}

\newcommand{\casetwo}[3]{\left\{ \begin{array}{ll} #1 &\mbox{if $#2$} \\ [0.04in] #3 &\mbox{otherwise}\,. \end{array} \right.}
\DeclareMathOperator{\arm}{arm}
\DeclareMathOperator{\leg}{leg}
\DeclareMathOperator{\comaj}{comaj}
\DeclareMathOperator{\inv}{inv}
\DeclareMathOperator{\Inv}{Inv}
\DeclareMathOperator{\des}{des}
\DeclareMathOperator{\Des}{Des}
\DeclareMathOperator{\Diff}{Diff}
\DeclareMathOperator{\maj}{maj}
\DeclareMathOperator{\cinv}{cinv}
\DeclareMathOperator{\mdes}{mdes}
\DeclareMathOperator{\mcinv}{mcinv}
\DeclareMathOperator{\south}{r}

\newlength{\cellsize}
\cellsize=2.5ex


% \tableau{} sets up an alignment with zero spacing, boxing each entry
% with \tableaucell{}.

\newcommand\tableau[1]{
\vcenter{
\let\\=\cr
\baselineskip=-16000pt
\lineskiplimit=16000pt
\lineskip=0pt
\halign{&\tableaucell{##}\cr#1\crcr}}}

% \tableaucell{} generates a square box of side length \cellsize.  If
% its argument is non-void, it is typeset in math mode, centered in the
% box, and a frame is drawn.

\newcommand{\tableaucell}[1]{{%
\def \arg{#1}\def \void{}%
\ifx \void \arg
\vbox to \cellsize{\vfil \hrule width \cellsize height 0pt}%
\else
\unitlength=\cellsize
\begin{picture}(1,1)
\put(0,0){\makebox(1,1){$#1$}}
\put(0,0){\line(1,0){1}}
\put(0,1){\line(1,0){1}}
\put(0,0){\line(0,1){1}}
\put(1,0){\line(0,1){1}}
\end{picture}%
\fi}}

\makeatletter
\let\cal\mathcal
\makeatother

\newcommand{\setp}[2]{\mbox{$\{#1\::\:#2\}$}}
\newcommand{\stackprod}[2]{\prod_{\begin{array}{c}\vspace{-6mm}\;\\ \vspace{-1mm}\scriptstyle{#1}\\ \scriptstyle{#2}\end{array}} }


\newtheorem{conj}[equation]{Conjecture}
\newtheorem{prop}[equation]{Proposition}
\newtheorem{dfn}[equation]{Definition}
\newtheorem{alg}[equation]{Algorithm}
\newtheorem{thm}[equation]{Theorem}
\newtheorem{rem}[equation]{Remarks}
\newtheorem{cor}[equation]{Corollary}
\newtheorem{lem}[equation]{Lemma}
%\newtheorem{exa}[theorem]{Example}
%\renewcommand{\theequation}{\thesection.\arabic{equation}}
%\makeatletter\@addtoreset{equation}{section}\makeatother

\newcommand{\bN}{{\mathbb N}}
\newcommand{\bC}{{\mathbb C}}
\newcommand{\spa}{\;\: }
\newcommand{\bQ}{{\mathbb Q}} 
\newcommand{\bZ}{{\mathbb Z}}
\newcommand{\case}[6]{#1#2 \left\{ \begin{array}{ll} #3 &\mbox{if $#4$} \\#5 & #6\,. \end{array} \right.}
\newcommand{\cased}[5]{#1#2 \left\{ \begin{array}{ll} #3 &\mbox{if $#4$} \\ \;\\#5 &\mbox{otherwise\,.} \end{array} \right.}
\newcommand{\map}[3]{\mbox{$#1 \colon #2 \rightarrow #3$}}
\newcommand{\vsp}{}
\newcommand{\vsph}{}
\newcommand{\vsphh}{}
\newcommand{\boldt}[1]{\vsph{\bf #1}\vsph }
%\newcommand{\subsection}[1]{\vsphh{\bf #1}\vsphh}
\newcommand{\newl}{\newline}

\newcommand{\oh}{\overline{h}}
\newcommand{\eg}{{\rm end}(\gamma)}
\newcommand{\lra}{\relbar\joinrel\longrightarrow}
\newcommand{\llra}{\relbar\joinrel\relbar\joinrel\longrightarrow}
\newcommand{\coe}{\mbox{coeff}}
\def\R{\mathbb{R}}
\def\Z{\mathbb{Z}}
\def\C{\mathbb{C}}
\def\Q{\mathbb{Q}}
\def\T{\mathbb{T}}
\def\O{\mathcal{O}}
\def\SL{\mathit{SL}}
\def\L{\mathcal{L}}
\def\weight{\mathrm{weight}}
\def\Waff{W_{\mathrm{aff}}}
\def\ds{\displaystyle}
\def\hvee{h^\vee}
\def\h{\mathfrak{h}}
\def\hR{\mathfrak{h}^*_\mathbb{R}}
\newcommand{\stacksum}[2]{\sum_{\begin{array}{c}\vspace{-6mm}\;\\ \vspace{-1mm}\scriptstyle{#1}\\ \scriptstyle{#2}\end{array}} }
\def\jdot{\overline{i}}
\newcommand{\jd}[1]{\overline{i_{#1}}}




%\newcommand{\smallskip}{\vspace{1mm}}
%\newcommand{\medskip}{\vspace{2.5mm}}
%\newcommand{\bigskip}{\vspace{4mm}}

%%%%%%%%%%%%%%%%%%%%%%%% sdc Special Symbols %%%%%%%%%%%%%%%%%%%%%%%%%
%%%%%%%%%%%%%%%%%%%%%%%%%%%%%%%%%%%%%%%%%%%%%%%%%%%%%%%%%%%%%
% Specialized Symbols
%
%   Disjoint Union
%\newcommand{\dunion}{\uplus}
\newcommand{\dunion}{\coprod}
%{\mbox{\hbox{\hskip4pt$\cdot$\hskip-4.62pt$\cup$\hskip2pt}}}
%{\mbox{\hbox{\hskip6pt$\cdot$\hskip-5.50pt$\cup$\hskip2pt}}}
%best choice: {\mbox{\hbox{\hskip0.45em$\cdot$\hskip-0.44em$\cup$\hskip0.2em}}}
%{\mbox{\hbox{\hskip0.45em$+$\hskip-0.70em$\cup$\hskip0.3em}}}
%
% Dot inside a cup.
% If there is a better, more Latex like way 
% (more invariant under font size changes) way,
% I'd like to know.

\newcommand{\Bases}[1]{\ensuremath{{\mathcal{B}}(#1)}}
\newcommand{\Reals}{\ensuremath{\mathbb{R}}}
\newcommand{\FieldK}{\ensuremath{K}}
\newcommand{\Perms}{\ensuremath{\mathfrak{S}}}
%\newcommand{\rank}{{\rho}}% {{\mbox{rank}}}
%\newcommand{\Rank}{{\rho}}% {{\mbox{rank}}}
\newcommand{\rank}{{\mbox{r}}}% {{\mbox{rank}}}
\newcommand{\Rank}{{\mbox{r}}}% {{\mbox{rank}}}
\newcommand{\Card}[1]{\ensuremath{{\left|#1\right|}}}
\newcommand{\ext}[1]{\ensuremath{\mathbf{#1}}}
%\newcommand{\extvee}{\ensuremath{\mathbf{\vee}}}
\newcommand{\extvee}{\;\;}
\newcommand{\Plucker}{Pl\"{u}cker\ }

% Set Complement
% command to mess with overline, bar or custom 
% alternatives for sequence or set complement
%
\newcommand{\scomp}[1]{\ensuremath{\overline{#1}}}
%\newcommand{\scomp}[1]{\ensuremath{\bar{#1}}}
%
%   Put a symbol for a matroid in a box, or brackets
%\newcommand{\MVAR}[1]{{\boxed{#1}\;}}
\newcommand{\MVAR}[1]{{[#1]\;}}
%
\newcommand{\UNION}{\cup} %try to make this bold.
%
%Emphasize in color!
\newcommand{\Remph}[1]{{\color{red}#1}}
\newcommand{\Bemph}[1]{{\color{blue}#1}}
%

% Subscripts denoting Current and Voltage
%\newcommand{\Is}{\ensuremath{I}}
%\newcommand{\Vs}{\ensuremath{V}}
\newcommand{\Is}{\ensuremath{\iota}}
\newcommand{\Vs}{\ensuremath{\upsilon}}


\newcommand{\alert}[1]{{\color{red}#1}}
    
   


\newcommand{\Nal}{\ensuremath{N_{\alpha}}}
\newcommand{\NbePe}{\ensuremath{N_{\beta}^{\perp}}}
\newcommand{\eNal}{\ensuremath{\ext{N}_{\alpha}}}
\newcommand{\eNbePe}{\ensuremath{\ext{N}_{\beta}^{\perp}}}
\newcommand{\eNbe}{\ensuremath{\ext{N}_\beta}}
\newcommand{\Nbe}{\ensuremath{N_\beta}}

\newtheorem*{theorem}{Theorem}
\newtheorem*{definition}{Definition}

%%%%%%%%End of specialized symbols%%%%%%%%%%%%%%%%%%%%%%







\title{An Exterior Algebra Valued Tutte Function on Linear Matroids or their Pairs\cite{TutteEx}}

\author{Seth Chaiken, University at Albany \url{schaiken@albany.edu}}
% Only used if using @institute in settitle below
%\institute{State University of New York at Albany}

%%%%%%%%%%%%%%%%%%%%%%%%%%%%%%%%%%%%%%%%%%%%%%%%%%%%%%%%%%%%%%%%%%%%%%%
%% NOTE: Custom settings (themes/colors/title)
%%%%%%%%%%%%%%%%%%%%%%%%%%%%%%%%%%%%%%%%%%%%%%%%%%%%%%%%%%%%%%%%%%%%%%%%{{{

%\hspace{-15cm}
%\includegraphics{fpsac2015logo}
%\hspace{36.5cm}

%\titlegraphic{
%\vspace{2cm}
%\includegraphics[scale=1.2]{banner}
%}

\makeatletter
\def\TP@titlegraphictotitledistance{-1cm}
\settitle{ 
    \centering
    \vspace{-2cm}
    \vbox{
    \color{titlefgcolor}
    \hspace*{-11em}
    {\bfseries \Huge \sc  \@title \par}
    \vspace*{1em}
    {\huge \@author} 
    %\hspace{10em}  
    %\vspace*{1em} 
    %{\LARGE \@institute}
    \\[\TP@titlegraphictotitledistance]
    {\@titlegraphic}

}
}
\makeatother

% define a user theme
\definecolorstyle{mystyle} {
    \definecolor{colorOne}{named}{purple}
    \definecolor{colorTwo}{named}{yellow}
    \definecolor{colorThree}{named}{white}
}{
    % Background Colors
    \colorlet{backgroundcolor}{white}
    \colorlet{framecolor}{yellow!}
    % Title Colors
    \colorlet{titlefgcolor}{black}
    \colorlet{titlebgcolor}{white}
    % Block Colors
    %\colorlet{blocktitlebgcolor}{yellow}
    \colorlet{blocktitlebgcolor}{white}
    \colorlet{blocktitlefgcolor}{purple}
    \colorlet{blockbodybgcolor}{white}
    \colorlet{blockbodyfgcolor}{black}
    % Innerblock Colors
    \colorlet{innerblocktitlebgcolor}{white}
    \colorlet{innerblocktitlefgcolor}{black}
    \colorlet{innerblockbodybgcolor}{colorThree!30!white}
    \colorlet{innerblockbodyfgcolor}{black}
    % Note colors
    \colorlet{notefgcolor}{black}
    \colorlet{notebgcolor}{colorTwo!50!white}
    \colorlet{noteframecolor}{colorOne}
}


%%%%%%%%%%%%%%%%%%%%%%%%%%%%%%%%%%%%%%%%%%%%%%%%%%%%%%%%%%%%%%%%%%%%%%
% Theme selection
%%%%%%%%%%%%%%%%%%%%%%%%%%%%%%%%%%%%%%%%%%%%%%%%%%%%%%%%%%%%%%%%%%%%%%

% Default, Basic, Envelope, Wave, VerticalShading, Filled, Empty.
\usetitlestyle{Basic}

% Default, Australia, Britain, Sweden, Spain, Russia, Denmark, Germany
\usecolorstyle{mystyle}
%\usecolorstyle{Australia}

% Default, Rays, VerticalGradation,  BottomVerticalGradation, Empty
\usebackgroundstyle{VerticalGradation}

% Default, Basic, Minimal, Envelope, Corner, Slide, TornOut
%\useblockstyle[titleleft ]{Slide}
%titleinnersep=0cm
\useblockstyle[titleleft, titleinnersep=0cm]{Envelope}
% Default, Table
\useinnerblockstyle[titleleft, titleinnersep=.5cm]{Default}

%%%%%%%%%%%%%%%%%%%%%%%%%%%%%%%%%%%%%%%%%%%%%%%%%%%%%%%%%%%%%%%%%%%%%%

%\usetheme{Autumn}\usecolorstyle[colorPalette=BrownBlueOrange]{Germany}

%%%%%%%%%%%%%%%%%%%%%%%%%%%%%%%%%%%%%%%%%%%%%%%%%%%%%%%%%%%%%%%%%%%%%%
% NOTE: end custom settings
%%%%%%%%%%%%%%%%%%%%%%%%%%%%%%%%%%%%%%%%%%%%%%%%%%%%%%%%%%%%%%%%%%%%%%

\begin{document}
\maketitle[titletotopverticalspace=5.5cm, titletoblockverticalspace=2.5cm]

\begin{columns}
\column{0.58}
\block{Overview}{
\input{overviewBigger.tikz}
}
  
\column{0.42}


\block{Theorem}{ 
  Given $N_\alpha$, $N_\beta$, of equal full row rank with columns labelled by
  $P\dunion E$, let $\eNal$, $\eNbe$ be the exterior products of their rows
  and dual $\eNbePe$ be constructed using (2).

  {\Large REMARK: $\ext{F}_E(\eNal\;\;\eNbe)$ is the exterior product of the rows of
  matrix $F$ illustrated in the Overview.}

{\LARGE $\longleftarrow\longleftarrow\longleftarrow$}
  
  For all $e\in E$ such that $e\not\in P$, with $E'=E\backslash e$,
  {\Large
     \[
     \epsilon(PE)\ext{F}_E(\eNal\;\;\eNbe)=
      \epsilon(PE')
        \left(g_e\ext{F}_{E'}(\eNal/e\;\; \eNbe/e) +
      r_e\ext{F}_{E'}(\eNal\backslash e\;\; \eNbe\backslash e)\right)
      \]
  }

  Similar identity for direct sums.
 }
\end{columns}


\begin{columns} 
\column{0.33}


%\block{}

\block{Summary}{
  \begin{enumerate}
  \item
    Usual parametrized Tutte functions $F$ are valued in comm. rings.
  \item
    Matrix Tree Theorem: The tree enumerator Tutte function is a determinant.
  \item
    Our generalization of the represented matroid basis enumerating determinant
    is a restricted Tutte function valued in exterior algebras (ie., anti-symmetric
    tensor spaces.)
  \item
    Relative (with respect to set $P$)\cite{RelTuttePolyDiaoHetyei}
    aka set $P$ pointed\cite{TPMorphMatI99}, $P$-``ported'' $F$\cite{sdcPorted}
    \[
    F(N,P) = r_e F(N\setminus e)+ g_e F(N/e)\]
    only when non-loop non-coloop $e\not\in P$.
  \item
    $P$ will play the role of graph vertices, to generalize deleting any
    equal sized row and col. sets from the Laplacian in the All-Minors Matrix
    Tree Theorem.
  \item
    Our Tutte function's values are {exterior products of vectors}. Thus
    they carry $\binom{2p}{p}$ determinants--each one is a Tutte function!
    Multiplication, occuring when $N$ is a direct sum, is anti-commutative.
    \item Unlike ordinary Tutte functions, these distinguish (at least trivial)
    orientations of the same matroid:
    $F$ matrices and distinct $\ext{F}_E(\eNal\;\;\eNbe)$, $E=\emptyset$, $|P|=2$
    
    \begin{tabular}{ c c c }
    \input{parPP.pdf_t} &
    $\begin{bmatrix}
    1 & 1 & 0 & 0 \\
    0 & 0 & 1 & -1
    \end{bmatrix}$
     &
     $(\ext{p}_{\alpha 1}+\ext{p}_{\alpha 2}\wedge(\ext{p}_{\beta 1}-\ext{p}_{\beta 2})=
     \alert{-\ext{p}_{\alpha 1}\ext{p}_{\beta 2} + \ext{p}_{\alpha 2}\ext{p}_{\beta 1}}+ \text{(equal)}\cdots$
     \\
    \input{parPN.pdf_t} &
    $\begin{bmatrix}
    1 & -1 & 0 & 0 \\
    0 & 0 & 1 & 1
    \end{bmatrix}$
    &
     $(\ext{p}_{\alpha 1}-\ext{p}_{\alpha 2}\wedge(\ext{p}_{\beta 1}+\ext{p}_{\beta 2})=
     \alert{\ext{p}_{\alpha 1}\ext{p}_{\beta 2} - \ext{p}_{\alpha 2}\ext{p}_{\beta 1}}+ \text{(equal)}\cdots$
      \end{tabular}
  \end{enumerate}
}


\block{Catalog of Oriented Matroid operations  %\\
    on  OM($N$) of matrix $N$ and on
    %THE EXTERIOR PRODUCT
    $\ext{N}=\wedge(\text{rows}(N))$}{
  \hspace*{-0.35in}
  \[
    \begin{array}{ccc}
      \text{Op is on:}         &  \text{chirotopes}
                                      & \text{exterior products} \\
      \text{which are:}        & \chi:B\rightarrow \{0,\pm \}
                                      & \text{decomposibles in\ } \wedge \\ 
      \text{case we use:}      & \chi:B\mapsto\text{sign}(N[B])
                                      &  \ext{N}:B\mapsto\ext{N}[B] \\
      \hline
      \text{OPERATION}         &  & \\
\text{deletion\ } \bullet\backslash A  & \text{restriction} & \text{restriction}  \\
\text{contraction\ }\bullet / A             & \pm\chi':B\mapsto\chi(BA) & \ext{N}/A:B\mapsto\ext{N}[BA]\text{\;\;(1)} \\

\text{duality\ }\bullet^{\perp} & \pm\chi^{\perp}:B\mapsto\chi(\overline{B})\epsilon(\overline{B}B) &
\ext{N}^{\perp}:B\mapsto\ext{N}[\overline{B}]\epsilon(\overline{B}B)\text{\;\;(2)} \\ \\
    \end{array}
    \]
$\ext{N}[B]$ is one component of a \emph{distinguished representative vector} of the \Plucker coordinates
(which are projective!) for the row space of $N$.

\setcounter{equation}{2}

We must choose some global orientation $\epsilon$ in order to define duality as an exterior alg. operation!
    $\epsilon$ is any alternating sign function on all finite sequences of elements.\\ 
    The definitions of deletion, contraction and dual imply these commutations:\\
    ($S$ is a ground set for exterior product analogs of matroids, needed so the analog of
    matroids with loops will have the analog of coloops. $X\subseteq S$ and $S'=S\backslash X$.)
    \begin{equation} (\ext{N}\backslash X)^\perp = \epsilon(S')\epsilon(S'X)(\ext{N}^\perp/X)
    \end{equation}
    \begin{equation} (\ext{N}/X)^\perp = \epsilon(S')\epsilon(S'X)(-1)^{|X|r\ext{N}^\perp}(\ext{N}^\perp\backslash X)
    \end{equation}
}


\column{0.33}

\block{Setup and proof outline}{
  \begin{itemize}
  \item
    See the statement of the Theorem.
    $\text{rank}(\Nal)+\text{rank}(\NbePe)=|E|+|P|$.   %\\
    $P_{\alpha},P_{\beta}\cong P$, $P_{\alpha}\cap P_{\beta}=\emptyset$.
  \item
    Weight (parameter) matrices  %\\
    $G=\text{diag}\{g_e\}_{e\in E} $,
    $R=\text{diag}\{r_e\}_{e\in E} $.
  \item
    Matrix with columns $P_\alpha \dunion P_\beta \dunion E$
    \[
    L\left( \begin{array}{c} \Nal\\ \NbePe \end{array} \right)
    = \left[\begin{array}{c|c|c} \Nal(P)  &  0  &  \Nal(E)G \\  \hline
0  & \NbePe(P)  &  \NbePe(E)R \end{array}\right]
    \]
  \end{itemize}

  Define
  \[
  F(L)=((\binom{2p}{p})-\text{tuple of determinants\ } L[Q_\alpha\overline{Q_\beta}E])
  \]
  indexed by sequences $Q_\alpha \overline{Q_\beta} \subseteq P_\alpha P_\beta$ where
  $Q_\alpha\subseteq P_\alpha$, $\overline{Q_\beta}\subseteq P_\beta, |Q_\alpha \overline{Q_\beta}|=p=|P|$.


%\vspace{0.1in}
Translate into exterior algebra definitions:
\[
\begin{split}
  \ext{L}\left( \begin{array}{c} \eNal\\ \eNbePe \end{array} \right)
   & := (\Is(\eNal)(P_\alpha) + \Is_G(\eNal(E)))\wedge(\Vs(\eNbePe)(P_\beta) + \Vs_R(\eNbePe)(E)) \\
  &  = (\Is_G(\eNal)\wedge\Vs_R(\eNbePe))
\end{split}
\]

\[
\begin{split}
  \ext{F}_E(\ext{L})& := \ext{L}/E = \sum_{Q_\alpha,\overline{Q_\beta}}\ext{L}[Q_\alpha \overline{Q_\beta} E]\ext{Q}_\alpha\overline{\ext{Q}_\beta} \\
  & =   ((\Is(\eNal)\backslash   e\alert{(\text{no\ }\ext{e})} + g_e(\Is(\eNal)/e)\wedge\alert{\ext{e}})                   \\
  & \;\;\;\;  \wedge  (\Vs(\eNbePe)\backslash e\alert{(\text{no\ }\ext{e})} + r_e(\Vs(\eNbePe)/e)\wedge \alert{\ext{e}}) ) / E \\
\alert{\text{2 of 4 terms}}  & = \Big( r_e  \;\;\; \;\;\;\;\;\;\;\;\;\;\;\;\;\; \Is(\ext{\eNal})\backslash e\wedge (\Vs(\eNbePe)/e)      \wedge  \alert{\ext{e}} \\
\alert{\text{vanish}}  & \;\;+ g_e (-1)^{r(\eNbePe)}(\Is(\eNal)/e)\wedge(\Vs(\eNbePe)\backslash e)    \wedge  \alert{\ext{e}}\Big) / E 
  \end{split}
\]
 

  %((\binom{2p}{p})-\text{tuple of determinants\ } \ext{L}[Q_\alpha \overline{Q_\beta} E])


\[
\begin{split}
  \ext{F}_E(\ext{L}) = \ext{L}/E 
   = \Big( r_e  \;\;\; \;\;\;\;\;\;\;\;\;\;\;\;\;\; & \Is(\eNal\backslash e)  \wedge (\Vs(\eNbePe/e))      \wedge  \ext{e} \\
   \;\;+ g_e (-1)^{r(\eNbePe)} ( & \Is(\eNal/e))\wedge(\Vs(\eNbePe\backslash e))   \wedge  \ext{e}\Big) / E \\
%  =  \Big( r_e  \Is(\eNal\backslash e)\wedge & (\Vs(\eNbePe/e))      
%  + g_e (-1)^{r(\eNbePe)}(\Is(\eNal/e))\wedge(\Vs(\eNbePe\backslash e))   \Big) \\
   %  & \wedge  \ext{e} / E \\
\end{split}
\]
\[  = r_e\left(\ext{L}\left(\begin{array}{c} \eNal\backslash e \\
    \eNbePe/e  \end{array} \right)  \wedge \ext{e} /E \right) +
   g_e(-1)^{r(\eNbePe)}\left(\ext{L}\left(\begin{array}{c} \eNal /e \\
    \eNbePe \backslash e \end{array} \right) \wedge \ext{e} /E \right)
   \]
   
%\hspace*{-0.3in}

%\mbox{\begin{minipage}{6in}${(\ext{N}\backslash e)^\perp = \epsilon(S')\epsilon(S'e)(\ext{N}^\perp/e)}$ ;
%         ${(\ext{N}/e)^\perp = \epsilon(S')\epsilon(S'e)(-1)^{|\{e\}|r\ext{N}^\perp}(\ext{N}^\perp\backslash e)}$\end{minipage}}

\[{(\ext{N}\backslash e)^\perp = \epsilon(S')\epsilon(S'e)(\ext{N}^\perp/e)}\]


\[{(\ext{N}/e)^\perp = \epsilon(S')\epsilon(S'e)(-1)^{|\{e\}|r\ext{N}^\perp}(\ext{N}^\perp\backslash e)}\]
         

%\vspace{0.3in}


%%%\begin{block}{Result}
%\hspace*{-0.3in}

\[
= \epsilon(S)\epsilon(S'e)
\left[
        r_e\left(\ext{L}\left(
        \begin{array}{c} \eNal\backslash e \\
    (\eNbe\backslash e)^\perp
    \end{array}
    \right)  \wedge \ext{e} /E \right)
+
        g_e\left(\ext{L}\left(
        \begin{array}{c} \eNal / e \\
    (\eNbe / e)^\perp \end{array} \right) \wedge \ext{e} /E \right)
\right]
\]
%%%%\end{block}

%%%\end{frame}




%%%%\begin{frame}
  With $\ext{L}(\eNal\;\; \eNbe)=\ext{L}\left(\begin{array}{c}\eNal \\ \eNbePe \end{array} \right)$, and more sign calculations:
    \begin{definition}
      For $E$, $P$ sets written as ordered sequences,
      \[
      \ext{F}_E(\eNal\;\;\eNbe) = \ext{L}(\eNal\;\;\eNbe)/E
      \]
    \end{definition}

    %%%\end{frame}
}




\column{0.34}
\block{History}{
  \alert{Not well-known 1847 graph theory paper!} Kirchhoff's\cite{Kirchhoff}  ``Matrix Tree Theorem'' paraphrased:
  The solution to the linear resistive electrical network (Kirchhoff's and Ohm's laws)
  problem is comprised of ratios of certain tree or forest enumerators.

  Maxwell\cite{MaxR}
  expressed those problems with matrices.  EEs call Kirchhoff's result ``Maxwell's rule.''
  

  The theorem only for $\ext{N}_\alpha=\ext{N}_\beta$, with a lot of background and motivation
  appears in \cite{TutteEx}.  See \cite{sdcOMP} for our setup related to oriented matroids and their relation to
  non-linear electrical network problem well-posedness.

}



\block{Bibliography}{

  \bibliographystyle{abbrv}
  \bibliography{../../bib/MathOfElec}{}
  
  
%\begin{itemize}

%\item C. Lenart, A. Lubovsky,  \textit{J. Algebraic Combin.}, 2015

%\item C. Lenart, S. Naito, D. Sagaki, A. Schilling, M. Shimozono,  \textit{Int. Math. Res. Not.}, 2016

%\end{itemize}
}




\end{columns}
\end{document}




\end{columns}
\end{document}


\block{The Tableau Model}{


 With the removal of the $\tilde{f_0}$ arrows, in types $A_{n-1}$ and $C_n$, we have 
$\textbf{B}^{k,1}\cong \textbf{B}(\omega_k)$
and in types $C_n$ and $D_n$, we have 
$$\textbf{B}^{k,1}\cong \textbf{B}(\omega_k) \sqcup \textbf{B}(\omega_{k-2}) \sqcup \textbf{B}(\omega_{k-4})\sqcup\hdots$$
where each $B(\omega_k)$ is given by $KN$ columns of height $k$.  These are strictly increasing fillings of the columns with entries $1<2<\hdots <n$ in type $A_{n-1}$.  With some additional conditions, they are fillings with entries $1<\hdots <n<\overline{n} <\hdots <\overline{1}$ in type $C_n$. Types $B_n$ and $D_n$ are similar.

}

\block{Type $A_4$ Crystal Graph of $\textbf{B}^{3,1}\otimes\textbf{B}^{2,1}$}{


%\includepdf[pages={-}]{KR_example.pdf}
%\includegraphics[scale=.65]{KR_example.pdf}
MISSING GRAPHIC HERE
} 

\column{.25}
\block{The Quantum Alcove Model for $\textbf{B}^{\textbf{p}}$}{

The main ingredient is the Weyl group $\textbf{W} = \langle s_\alpha :\alpha\in\Phi\rangle$.

The \textit{quantum Bruhat graph} on \textbf{W} is the directed graph with labeled edges $w \xrightarrow{\alpha} ws_\alpha,$ where 

\hspace{24pt} $l(ws_\alpha) = l(w) + 1$ (Bruhat graph), or 

\hspace{24pt} $l(ws_\alpha) = l(w) + 1 - 2\langle\rho,\alpha^{\vee}\rangle.$

\vspace{12pt}\textcolor{Blue}{Definition.} Given a dominant weight $\lambda=\omega_{p_1}+\hdots + \omega_{p_r}$, we associate with it a sequence of roots, called a $\lambda - chain$ (many choices possible): $$\Gamma = (\beta_1,\beta_2,\hdots,\beta_m).$$
Let $r_i := s_{\beta_i}$.  We consider subsets of positions in $\Gamma$ $$J = (j_1<j_2<\hdots <j_s)\subseteq \{1,\hdots,m\}.$$

\textcolor{Blue}{Definition.} A subset $J = \{j_1<j_2<\hdots < j_s\}$ is \textit{admissible} if we have a path in the quantum Bruhat graph $$Id = w_0\xrightarrow{\beta_{j_1}} r_{j_1}\xrightarrow{\beta_{j_2}} r_{j_1}r_{j_2} \hdots \xrightarrow{\beta_{j_s}} r_{j_1}\hdots r_{j_s}.  $$

\textcolor{blue}{Theorem [LNSSS, 2016]:} The collection of all admissible subsets, $\cm{A}(\Gamma)$,is a combinatorial model for $\textbf{B}^{\textbf{p}}$.


}

\block{The Two Realizations}{

\begin{itemize}
\item The Tableaux model is simpler and has less structure.
\item The Quantum Alcove model has extra structure which makes it easier to do several computations (energy function, combinatorial R-Matrix, charge statistic$\hdots$)

\end{itemize}

}

\block{Relating the Two Models}{

 We build a forgetful map $fill:\mathcal{A}(\Gamma)\rightarrow Tableau(\lambda)$ where $\lambda=\omega_{p_1}+\hdots\omega_{p_r}$.  

\vspace{12pt}\textcolor{blue}{Definition:} For any $k = 1,\hdots,n-1$ we define $\Gamma(k)$ to be the following chain of roots: 
$$((k,k+1),(k,k+2),\hdots,(k,n)\hdots$$
$$(2,k+1),(2,k+2),\hdots,(2,n)$$
$$(1,k+1),(1,k+2),\hdots,(1,n))$$

\vspace{12pt}\textcolor{blue}{Definition:} We construct a \textit{$\lambda$-chain} as a concatenation

 $\Gamma :=\Gamma^{\mu_1}\hdots\Gamma^1$ where $\Gamma^j = \Gamma(p_j)$.

\vspace{12pt}\textcolor{blue}{Example} Consider $n=4$ and $\lambda = (3,2,1,0)$.  Then the associated $\lambda$-chain is $ \Gamma = \Gamma^3\Gamma^2\Gamma^1 =$ $$ ((3,4),(2,4),(1,4) | (2,3),(2,4),(1,3),(1,4) | (1,2),(1,3),(1,4)).$$


\vspace{12pt}\textcolor{blue}{Example}
$J=\{1,2,4,5,8\}\in\mathcal{A}(\Gamma)$.
$$({\underline{(3,4)}},\underline{(2,4)},(1,4)|\underline{(2,3)},\underline{{(2,4)}},(1,3),{(1,4)}|{\underline{(1,2)}},(1,3),(1,4))$$


We get the corresponding path in the Bruhat order/quantum Bruhat graph
$$id=\begin{array}{l}\tableau{{1}\\{2}\\{\textcolor{Red} 3}} \\ \\ \tableau{{\textcolor{Red} 4}} \end{array} \!\begin{array}{c} \\ \xrightarrow{3,4} \end{array}\! \begin{array}{l}\tableau{{ 1}\\{\textcolor{Red} 2}\\{ 4} \\ \\ {\textcolor{Red}  3}} \end{array} \begin{array}{c} \\ {\xrightarrow{2,4}} \end{array}\! \begin{array}{l}\tableau{{\textcolor{Blue} 1}\\{\textcolor{Blue}{ 3}}}  \\ \tableau{{\textcolor{Blue}{ 4}}\\ \\{ 2}} \end{array}\!  \:|\:  \begin{array}{l}\tableau{{{ 1}}\\{{\textcolor{Red} 3}}} \\ \\ \tableau{{{\textcolor{Red} 4}}\\{ 2}}\end{array}\!\begin{array}{c} \\ \xrightarrow{2,3} \end{array}\!  \begin{array}{l}\tableau{{\ 1}\\{\textcolor{Red} 4}}\\ \\ \tableau{{3}\\{\textcolor{Red} 2}}\end{array} \begin{array}{c} \\ \xrightarrow{2,4} \end{array}\!  \begin{array}{l}\tableau{{\textcolor{Blue} 1}\\{\textcolor{Blue} 2}}\\ \\ \tableau{{3}\\{4}}\end{array} \:|\: 
\begin{array}{l}\tableau{{\textcolor{Red} 1}\\ \\ {{\textcolor{Red} 2}}\\{3}\\{4}} \end{array} 
\!\begin{array}{c} \\ \xrightarrow{1,2} \end{array}\!   \begin{array}{l} \tableau{{\textcolor{Blue} 2}\\ \\{1}\\{3}} \\ \tableau{{4}}\end{array} \! = end(J) \,.$$

\vspace{12pt}This gives us $fill(J) = $ \[\tableau{{1}&{1}&{2}\\{3}&{2}&\\{4}&&}\,.\]

}


\column{0.25}\block{The Reverse Map in Type $A_{n-1}$}{

 Consider the tableau in $\bigotimes_{i=1}^r B^{p_i,1}$ from the previous example \[f(T)=\tableau{{1}&{1}&{2}\\{3}&{2}&\\{4}&&}\,.\]
 
 Use entries of columns $i$ and $i-1$ viewed as sets to build the desired sub-list of $\Gamma^i$ where the zero column is the size $n$ column of strictly increasing entries.

  
 This is done with two algorithms: \textcolor{Blue}{Reorder and Greedy}
 
 The resulting bijection is a crystal isomorphism [LL,2015].


}


 
 \block{The Reorder Rule}{

First, let us consider the circular order
\[a\preceq_a a+1\preceq_a\ldots\preceq_a n\preceq_a 1\preceq_a\ldots\preceq_a a-1\,.\]

We write all chains in $\preceq_a$ starting with $a$, so the subscript $a$ can be dropped.

Let $C$ and $C'$ be two columns.
\medskip We fix the entries in $C$ and wish to reorder those in $C'$.


\medskip For each $1\leq i \leq \#C'$, we have $$a_i=C'(i) = min\{C'(l):i\leq l \leq \#C'\}$$
where the minimum is taken with respect to the circle order on [$n$] starting with $C(i)$.

\vspace{12pt}\textcolor{blue}{Example:} If $C = \tableau{{2}\\{1}\\{3}\\{4}}\,$ and $C'=\tableau{{1}\\ {3}\\{4}}$. Then $reorder_C(C') = \tableau{{3}\\ {1}\\{4}}$.
}

\block{The Greedy Algorithm}{

We now rebuild the desired sublist of $\Gamma_i$ by going through $\Gamma_i$ root by root.   

\vspace{12pt} For root $(j_1,j_2)$ if $C[j_1]\prec C[j_2] \prec \hat{C'}[j_1]$ and $C \xrightarrow{(j_1,j_2)} \hat{C'}$ is in the -corresponding QBG, then apply it. Otherwise skip.  Continue.

\medskip So for our example, we have $\Gamma_1=((3,4),(2,4),(1,4))$
\medskip and get $$C = \tableau{{1}\\{2}\\{3}\\{4}} \xrightarrow{(3,4)} \tableau{{1}\\{2}\\{4}\\{3}} \xrightarrow{(2,4)} \tableau{{1}\\{3}\\{4}\\{2}}$$


}




}

\column{.25}\block{The Type $C_n$ Map}{

\begin{itemize}
\item The filling map is similar.
\item The inverse map has one major change.  Many $KN$ columns have both $i$ and $\overline{\imath}$ in them, so we use the splitting algorithm [Lecouvey] to bijectively make two columns with no $i,\overline{\imath}$ pairs in either.
\item Then similar reorder and greedy algorithms work.
\item So now the reverse map is made up of a process of \textcolor{Blue}{Split, Reorder, and Greedy}.
\item \textcolor{blue}{Example:}
\end{itemize}


$$\begin{array}{l}\tableau{{4}\\{5}\\{\overline{5}}\\{\overline{4}}\\{\overline{3}}} \end{array} \!\begin{array}{c} \\ \xrightarrow{split} \end{array}\! \begin{array}{ll}\tableau{{ 4}&{1}\\{5}&{2}\\{ \overline{3}}&{\overline{5}}\\{\overline{2}}&{\overline{4}}\\{\overline{1}}&{\overline{3}}} \end{array}  \,$$


\medskip The $\Gamma(k)$ in type $C_n$ comes in two parts.
\medskip We use the first to get a chain from the left split to the reordered right split

\medskip and the second to get a chain from the right split to the next column's left split.

}

\block{The Type $B_n$ Map}{

\begin{itemize}
\pause\item There is a similar filling map
\pause\item For the reverse, similar to $C_n$, we need a splitting map.
\pause\item Recall that we now have columns of length $k-2l$, so we need to Extend back to length $k$ [Briggs].
\pause\medskip\item Further, the greedy algorithm and reorder algorithm no longer work.
\pause\medskip\item There is a configuration of two columns $CC'$ that we call being \textcolor{Blue}{blocked-off}.
\pause\medskip\item Modify greedy and reorder to avoid block-off pattern.
\end{itemize}


\textcolor{blue}{Definition:} We say that columns $L = (l_1,l_2,...,l_k), R' = (r_1,r_2,...,r_k)$ are \textit{blocked off  at $i$ by $b=r_i$} iff $0<b\geq |l_i|$ \pause and$$\{1,2,...,b\}\subset \{|l_1|,|l_2|,...,|l_i|\}$$ and $$\{1,2,...,b\}\subset \{|r_1|,|r_2|,...,|r_i|\}$$ \vspace{10pt} \pause and $|\{j : 1\leq j\leq i, l_j<0, r_j> 0\}|$ is odd.

}

\block{Further Work}{

\begin{itemize}
\item The map in type $D_n$  similar to type $B_n$, but there is a second pattern to be avoided in Reorder and Greedy.
\item The bijections for types $B_n$ and $D_n$ given here are actually crystal isomorphisms.
\end{itemize}

}

\block{Bibliography}{
\begin{itemize}

\item C. Lenart, A. Lubovsky,  \textit{J. Algebraic Combin.}, 2015

\item C. Lenart, S. Naito, D. Sagaki, A. Schilling, M. Shimozono,  \textit{Int. Math. Res. Not.}, 2016

\end{itemize}
}

\end{columns}

\end{document}

 

