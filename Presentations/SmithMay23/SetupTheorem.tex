\block{Setup and proof outline}{
  \begin{itemize}
  \item
    See the statement of the Theorem.
    $\text{rank}(\Nal)+\text{rank}(\NbePe)=|E|+|P|$.   %\\
    $P_{\alpha},P_{\beta}\cong P$, $P_{\alpha}\cap P_{\beta}=\emptyset$.
  \item
    Weight (parameter) matrices  %\\
    $G=\text{diag}\{g_e\}_{e\in E} $,
    $R=\text{diag}\{r_e\}_{e\in E} $.
  \item
    Matrix with columns $P_\alpha \dunion P_\beta \dunion E$
    \[
    L\left( \begin{array}{c} \Nal\\ \NbePe \end{array} \right)
    = \left[\begin{array}{c|c|c} \Nal(P)  &  0  &  \Nal(E)G \\  \hline
0  & \NbePe(P)  &  \NbePe(E)R \end{array}\right]
    \]
  \end{itemize}

  Define
  \[
  F(L)=((\binom{2p}{p})-\text{tuple of determinants\ } L[Q_\alpha\overline{Q_\beta}E])
  \]
  indexed by sequences $Q_\alpha \overline{Q_\beta} \subseteq P_\alpha P_\beta$ where
  $Q_\alpha\subseteq P_\alpha$, $\overline{Q_\beta}\subseteq P_\beta, |Q_\alpha \overline{Q_\beta}|=p=|P|$.


%\vspace{0.1in}
Translate into exterior algebra definitions:
\[
\begin{split}
  \ext{L}\left( \begin{array}{c} \eNal\\ \eNbePe \end{array} \right)
   & := (\Is(\eNal)(P_\alpha) + \Is_G(\eNal(E)))\wedge(\Vs(\eNbePe)(P_\beta) + \Vs_R(\eNbePe)(E)) \\
  &  = (\Is_G(\eNal)\wedge\Vs_R(\eNbePe))
\end{split}
\]

\[
\begin{split}
  \ext{F}_E(\ext{L})& := \ext{L}/E = \sum_{Q_\alpha,\overline{Q_\beta}}\ext{L}[Q_\alpha \overline{Q_\beta} E]\ext{Q}_\alpha\overline{\ext{Q}_\beta} \\
  & =   ((\Is(\eNal)\backslash   e\alert{(\text{no\ }\ext{e})} + g_e(\Is(\eNal)/e)\wedge\alert{\ext{e}})                   \\
  & \;\;\;\;  \wedge  (\Vs(\eNbePe)\backslash e\alert{(\text{no\ }\ext{e})} + r_e(\Vs(\eNbePe)/e)\wedge \alert{\ext{e}}) ) / E \\
\alert{\text{2 of 4 terms}}  & = \Big( r_e  \;\;\; \;\;\;\;\;\;\;\;\;\;\;\;\;\; \Is(\ext{\eNal})\backslash e\wedge (\Vs(\eNbePe)/e)      \wedge  \alert{\ext{e}} \\
\alert{\text{vanish}}  & \;\;+ g_e (-1)^{r(\eNbePe)}(\Is(\eNal)/e)\wedge(\Vs(\eNbePe)\backslash e)    \wedge  \alert{\ext{e}}\Big) / E 
  \end{split}
\]
 

  %((\binom{2p}{p})-\text{tuple of determinants\ } \ext{L}[Q_\alpha \overline{Q_\beta} E])


\[
\begin{split}
  \ext{F}_E(\ext{L}) = \ext{L}/E 
   = \Big( r_e  \;\;\; \;\;\;\;\;\;\;\;\;\;\;\;\;\; & \Is(\eNal\backslash e)  \wedge (\Vs(\eNbePe/e))      \wedge  \ext{e} \\
   \;\;+ g_e (-1)^{r(\eNbePe)} ( & \Is(\eNal/e))\wedge(\Vs(\eNbePe\backslash e))   \wedge  \ext{e}\Big) / E \\
%  =  \Big( r_e  \Is(\eNal\backslash e)\wedge & (\Vs(\eNbePe/e))      
%  + g_e (-1)^{r(\eNbePe)}(\Is(\eNal/e))\wedge(\Vs(\eNbePe\backslash e))   \Big) \\
   %  & \wedge  \ext{e} / E \\
\end{split}
\]
\[  = r_e\left(\ext{L}\left(\begin{array}{c} \eNal\backslash e \\
    \eNbePe/e  \end{array} \right)  \wedge \ext{e} /E \right) +
   g_e(-1)^{r(\eNbePe)}\left(\ext{L}\left(\begin{array}{c} \eNal /e \\
    \eNbePe \backslash e \end{array} \right) \wedge \ext{e} /E \right)
   \]
   
%\hspace*{-0.3in}

%\mbox{\begin{minipage}{6in}${(\ext{N}\backslash e)^\perp = \epsilon(S')\epsilon(S'e)(\ext{N}^\perp/e)}$ ;
%         ${(\ext{N}/e)^\perp = \epsilon(S')\epsilon(S'e)(-1)^{|\{e\}|r\ext{N}^\perp}(\ext{N}^\perp\backslash e)}$\end{minipage}}

\[{(\ext{N}\backslash e)^\perp = \epsilon(S')\epsilon(S'e)(\ext{N}^\perp/e)}\]


\[{(\ext{N}/e)^\perp = \epsilon(S')\epsilon(S'e)(-1)^{|\{e\}|r\ext{N}^\perp}(\ext{N}^\perp\backslash e)}\]
         

%\vspace{0.3in}


%%%\begin{block}{Result}
%\hspace*{-0.3in}

\[
= \epsilon(S)\epsilon(S'e)
\left[
        r_e\left(\ext{L}\left(
        \begin{array}{c} \eNal\backslash e \\
    (\eNbe\backslash e)^\perp
    \end{array}
    \right)  \wedge \ext{e} /E \right)
+
        g_e\left(\ext{L}\left(
        \begin{array}{c} \eNal / e \\
    (\eNbe / e)^\perp \end{array} \right) \wedge \ext{e} /E \right)
\right]
\]
%%%%\end{block}

%%%\end{frame}




%%%%\begin{frame}
  With $\ext{L}(\eNal\;\; \eNbe)=\ext{L}\left(\begin{array}{c}\eNal \\ \eNbePe \end{array} \right)$, and more sign calculations:
    \begin{definition}
      For $E$, $P$ sets written as ordered sequences,
      \[
      \ext{F}_E(\eNal\;\;\eNbe) = \ext{L}(\eNal\;\;\eNbe)/E
      \]
    \end{definition}

    %%%\end{frame}
}


