\documentclass[24pt, landscape, margin=0mm, innermargin=15mm, blockverticalspace=15mm, colspace=15mm, subcolspace=8mm]{tikzposter}
%default fontsize is 25pt

\geometry{paperwidth=36in, paperheight=47in}



%%%%%%%%%%%%%%%%%%%%%%%%%%%%%%%%%%%%%%%%%%%%%%%%%%%%%%%%%%%%%%%%%%%%%%
% \tableau{} sets up an alignment with zero spacing, boxing each entry
% with \tableaucell{}.
%%%%%%%%%%%%%%%%%%%%%%%%%%%%%%%%%%%%%%%%%%%%%%%%%%%%%%%%%%%%%%%%%%%%%%

\usepackage{amsthm, amsmath, amssymb, amsopn}
\usepackage{lipsum}
\usepackage{pdfpages}

\definecolor{Blue}{rgb}{.255,.41,.884} % RoyalBlue of svgnames
\definecolor{Red}{rgb}{1, 0, 0} % Red of svgnames
\definecolor{Green}{rgb}{.196,.804,.196} % LimeGreen of svgnames
\definecolor{Yellow}{rgb}{1,.648,0} % Orange of svgnames

\def\B{{\mathcal{B}}}
\newcommand{\calp}{{\cal P}}
\newcommand{\cals}{{\cal S}}
\newcommand{\calw}{{\cal W}}
\newcommand{\calh}{{\cal H}}
\newcommand{\bF}{{\mathbb F}}
\newcommand{\fs}{\mathfrak S}
\newcommand{\bari}{\overline{\imath}}
\newcommand{\barj}{\overline{\jmath}}
\newcommand{\fg}{\mathfrak G}
\newcommand{\arr}{\begin{picture}(30,5) \put(3,2.5){\vector(1,0){24}} \end{picture}}
\newcommand{\arrl}[1]{\stackrel{#1}{\arr}}

%Dave's macros for fixing running heads, xarrows...
%\addtolength\headheight{2pt}

%\setlength{\parskip}{5pt}
%%\setlength{\parindent}{0pt}
%\setlength{\textheight}{9in}
%\setlength{\topmargin}{-.50in}
%\setlength{\oddsidemargin}{.125in}
%\setlength{\evensidemargin}{.125in}
%\setlength{\textwidth}{6.25in}

\newcommand{\casetwo}[3]{\left\{ \begin{array}{ll} #1 &\mbox{if $#2$} \\ [0.04in] #3 &\mbox{otherwise}\,. \end{array} \right.}
\DeclareMathOperator{\arm}{arm}
\DeclareMathOperator{\leg}{leg}
\DeclareMathOperator{\comaj}{comaj}
\DeclareMathOperator{\inv}{inv}
\DeclareMathOperator{\Inv}{Inv}
\DeclareMathOperator{\des}{des}
\DeclareMathOperator{\Des}{Des}
\DeclareMathOperator{\Diff}{Diff}
\DeclareMathOperator{\maj}{maj}
\DeclareMathOperator{\cinv}{cinv}
\DeclareMathOperator{\mdes}{mdes}
\DeclareMathOperator{\mcinv}{mcinv}
\DeclareMathOperator{\south}{r}

\newlength{\cellsize}
\cellsize=2.5ex


% \tableau{} sets up an alignment with zero spacing, boxing each entry
% with \tableaucell{}.

\newcommand\tableau[1]{
\vcenter{
\let\\=\cr
\baselineskip=-16000pt
\lineskiplimit=16000pt
\lineskip=0pt
\halign{&\tableaucell{##}\cr#1\crcr}}}

% \tableaucell{} generates a square box of side length \cellsize.  If
% its argument is non-void, it is typeset in math mode, centered in the
% box, and a frame is drawn.

\newcommand{\tableaucell}[1]{{%
\def \arg{#1}\def \void{}%
\ifx \void \arg
\vbox to \cellsize{\vfil \hrule width \cellsize height 0pt}%
\else
\unitlength=\cellsize
\begin{picture}(1,1)
\put(0,0){\makebox(1,1){$#1$}}
\put(0,0){\line(1,0){1}}
\put(0,1){\line(1,0){1}}
\put(0,0){\line(0,1){1}}
\put(1,0){\line(0,1){1}}
\end{picture}%
\fi}}

\makeatletter
\let\cal\mathcal
\makeatother

\newcommand{\setp}[2]{\mbox{$\{#1\::\:#2\}$}}
\newcommand{\stackprod}[2]{\prod_{\begin{array}{c}\vspace{-6mm}\;\\ \vspace{-1mm}\scriptstyle{#1}\\ \scriptstyle{#2}\end{array}} }


\newtheorem{conj}[equation]{Conjecture}
\newtheorem{prop}[equation]{Proposition}
\newtheorem{dfn}[equation]{Definition}
\newtheorem{alg}[equation]{Algorithm}
\newtheorem{thm}[equation]{Theorem}
\newtheorem{rem}[equation]{Remarks}
\newtheorem{cor}[equation]{Corollary}
\newtheorem{lem}[equation]{Lemma}
%\newtheorem{exa}[theorem]{Example}
%\renewcommand{\theequation}{\thesection.\arabic{equation}}
%\makeatletter\@addtoreset{equation}{section}\makeatother

\newcommand{\bN}{{\mathbb N}}
\newcommand{\bC}{{\mathbb C}}
\newcommand{\spa}{\;\: }
\newcommand{\bQ}{{\mathbb Q}} 
\newcommand{\bZ}{{\mathbb Z}}
\newcommand{\case}[6]{#1#2 \left\{ \begin{array}{ll} #3 &\mbox{if $#4$} \\#5 & #6\,. \end{array} \right.}
\newcommand{\cased}[5]{#1#2 \left\{ \begin{array}{ll} #3 &\mbox{if $#4$} \\ \;\\#5 &\mbox{otherwise\,.} \end{array} \right.}
\newcommand{\map}[3]{\mbox{$#1 \colon #2 \rightarrow #3$}}
\newcommand{\vsp}{}
\newcommand{\vsph}{}
\newcommand{\vsphh}{}
\newcommand{\boldt}[1]{\vsph{\bf #1}\vsph }
%\newcommand{\subsection}[1]{\vsphh{\bf #1}\vsphh}
\newcommand{\newl}{\newline}

\newcommand{\oh}{\overline{h}}
\newcommand{\eg}{{\rm end}(\gamma)}
\newcommand{\lra}{\relbar\joinrel\longrightarrow}
\newcommand{\llra}{\relbar\joinrel\relbar\joinrel\longrightarrow}
\newcommand{\coe}{\mbox{coeff}}
\def\R{\mathbb{R}}
\def\Z{\mathbb{Z}}
\def\C{\mathbb{C}}
\def\Q{\mathbb{Q}}
\def\T{\mathbb{T}}
\def\O{\mathcal{O}}
\def\SL{\mathit{SL}}
\def\L{\mathcal{L}}
\def\weight{\mathrm{weight}}
\def\Waff{W_{\mathrm{aff}}}
\def\ds{\displaystyle}
\def\hvee{h^\vee}
\def\h{\mathfrak{h}}
\def\hR{\mathfrak{h}^*_\mathbb{R}}
\newcommand{\stacksum}[2]{\sum_{\begin{array}{c}\vspace{-6mm}\;\\ \vspace{-1mm}\scriptstyle{#1}\\ \scriptstyle{#2}\end{array}} }
\def\jdot{\overline{i}}
\newcommand{\jd}[1]{\overline{i_{#1}}}

%\newcommand{\smallskip}{\vspace{1mm}}
%\newcommand{\medskip}{\vspace{2.5mm}}
%\newcommand{\bigskip}{\vspace{4mm}}


\title{Combinatorial models in the representation theory of quantum affine Lie algebras}
\author{Carly Briggs (Bennington College), \ Cristian Lenart (State Univ. of New York at Albany) \  \\ Adam Schultze (State Univ. of New York at Albany)}
% Only used if using @institute in settitle below
%\institute{State University of New York at Albany$^1$; University of Ottawa$^2$}

%%%%%%%%%%%%%%%%%%%%%%%%%%%%%%%%%%%%%%%%%%%%%%%%%%%%%%%%%%%%%%%%%%%%%%%
%% NOTE: Custom settings (themes/colors/title)
%%%%%%%%%%%%%%%%%%%%%%%%%%%%%%%%%%%%%%%%%%%%%%%%%%%%%%%%%%%%%%%%%%%%%%%%{{{

%\hspace{-15cm}
%\includegraphics{fpsac2015logo}
%\hspace{36.5cm}

%\titlegraphic{
%\vspace{2cm}
%\includegraphics[scale=1.2]{banner}
%}

\makeatletter
\def\TP@titlegraphictotitledistance{-1cm}
\settitle{ 
    \centering
    \vspace{-2cm}
    \vbox{
    \color{titlefgcolor}
    \hspace*{-11em}
    {\bfseries \Huge \sc  \@title \par}
    \vspace*{1em}
    {\huge \@author} 
    %\hspace{10em}  
    %\vspace*{1em} 
    %{\LARGE \@institute}
    \\[\TP@titlegraphictotitledistance]
    {\@titlegraphic}

}
}
\makeatother

% define a user theme
\definecolorstyle{mystyle} {
    \definecolor{colorOne}{named}{purple}
    \definecolor{colorTwo}{named}{yellow}
    \definecolor{colorThree}{named}{white}
}{
    % Background Colors
    \colorlet{backgroundcolor}{white}
    \colorlet{framecolor}{yellow!}
    % Title Colors
    \colorlet{titlefgcolor}{black}
    \colorlet{titlebgcolor}{white}
    % Block Colors
    %\colorlet{blocktitlebgcolor}{yellow}
    \colorlet{blocktitlebgcolor}{white}
    \colorlet{blocktitlefgcolor}{purple}
    \colorlet{blockbodybgcolor}{white}
    \colorlet{blockbodyfgcolor}{black}
    % Innerblock Colors
    \colorlet{innerblocktitlebgcolor}{white}
    \colorlet{innerblocktitlefgcolor}{black}
    \colorlet{innerblockbodybgcolor}{colorThree!30!white}
    \colorlet{innerblockbodyfgcolor}{black}
    % Note colors
    \colorlet{notefgcolor}{black}
    \colorlet{notebgcolor}{colorTwo!50!white}
    \colorlet{noteframecolor}{colorOne}
}


%%%%%%%%%%%%%%%%%%%%%%%%%%%%%%%%%%%%%%%%%%%%%%%%%%%%%%%%%%%%%%%%%%%%%%
% Theme selection
%%%%%%%%%%%%%%%%%%%%%%%%%%%%%%%%%%%%%%%%%%%%%%%%%%%%%%%%%%%%%%%%%%%%%%

% Default, Basic, Envelope, Wave, VerticalShading, Filled, Empty.
\usetitlestyle{Basic}

% Default, Australia, Britain, Sweden, Spain, Russia, Denmark, Germany
\usecolorstyle{mystyle}
%\usecolorstyle{Australia}

% Default, Rays, VerticalGradation,  BottomVerticalGradation, Empty
\usebackgroundstyle{VerticalGradation}

% Default, Basic, Minimal, Envelope, Corner, Slide, TornOut
%\useblockstyle[titleleft ]{Slide}
%titleinnersep=0cm
\useblockstyle[titleleft, titleinnersep=0cm]{Envelope}
% Default, Table
\useinnerblockstyle[titleleft, titleinnersep=.5cm]{Default}

%%%%%%%%%%%%%%%%%%%%%%%%%%%%%%%%%%%%%%%%%%%%%%%%%%%%%%%%%%%%%%%%%%%%%%

%\usetheme{Autumn}\usecolorstyle[colorPalette=BrownBlueOrange]{Germany}

%%%%%%%%%%%%%%%%%%%%%%%%%%%%%%%%%%%%%%%%%%%%%%%%%%%%%%%%%%%%%%%%%%%%%%
% NOTE: end custom settings
%%%%%%%%%%%%%%%%%%%%%%%%%%%%%%%%%%%%%%%%%%%%%%%%%%%%%%%%%%%%%%%%%%%%%%

\begin{document}
\maketitle[titletotopverticalspace=5.5cm, titletoblockverticalspace=2.5cm]

\begin{columns} 
\column{0.25}


%\block{}

\block{Abstract}{


We give an explicit description of the unique crystal isomorphism between two combinatorial models for tensor products of Kirillov-Reshetikhin cyrstals: the tableau model and the quantum alcove model.


%\medskip


%\textcolor{Blue}{Note.} The hyperbolic f.g.l. is very natural: related to \textcolor{Red}{(2-parameter) virtual generalized Todd genus} of Hirzebruch.
}

\block{Crystal Bases}{

 \textit{Main idea:} use colored directed graphs to encode certain representations $V$ of the quantum group $U_q(\mathfrak{g})$ as $q\rightarrow 0$ ($\mathfrak{g}$ complex semisimple or affine Lie algebra).
 
 \textit{Kashiwara (crystal) operators} are modified versions of the Chevalley generators (indexed by the simple roots $\alpha_i)$: $\tilde{e_i}, \tilde{f_i}$.  V has a \textit{crystal basis \textbf{B}} $$\tilde{e_i}, \tilde{f_i}:\textbf{B}\rightarrow\textbf{B}\sqcup {0}, $$ $$ \tilde{f_i}(b) = b' \Leftrightarrow \tilde{e_i}(b') = b \Leftrightarrow b\xrightarrow{i} b'.$$

\textit{Crystal graph:} directed graph on $\text{B}$ with edges colored $i\leftrightarrow a_i$.


\vspace{12pt}\textbf{Kirillov-Reshetikhin (KR) crystals}

\vspace{12pt} Correspond to certain $finite$-dimensional representations (not highest weight) or affine Lie algebras $\hat{\mathfrak{g}}$.  Consider the untwisted affine types $\textbf{A}_{n-1}^{(1)} - \textbf{G}_2^{(1)}$.  The corresponding crystals have edges (associated to crystal operators) $\tilde{f_0},\tilde{f_1},\hdots$.

\vspace{12pt}
Labeled by $p \times q$ rectangles, and denoted $\textbf{B}^{p,q}$.


\vspace{12pt} \textcolor{blue}{Definition.} Given a composition $\textbf{p} = (p_1,p_2,\hdots)$, let $$\textbf{B}^{\textbf{p}} = \textbf{B}^{p_1,1}\otimes\textbf{B}^{p_2,1}\otimes\hdots .$$ The crystal operators are defined on $\textbf{B}^{\textbf{p}}$ by a tensor product rule.

}

\block{The Tableau Model}{


 With the removal of the $\tilde{f_0}$ arrows, in types $A_{n-1}$ and $C_n$, we have 
$\textbf{B}^{k,1}\cong \textbf{B}(\omega_k)$
and in types $C_n$ and $D_n$, we have 
$$\textbf{B}^{k,1}\cong \textbf{B}(\omega_k) \sqcup \textbf{B}(\omega_{k-2}) \sqcup \textbf{B}(\omega_{k-4})\sqcup\hdots$$
where each $B(\omega_k)$ is given by $KN$ columns of height $k$.  These are strictly increasing fillings of the columns with entries $1<2<\hdots <n$ in type $A_{n-1}$.  With some additional conditions, they are fillings with entries $1<\hdots <n<\overline{n} <\hdots <\overline{1}$ in type $C_n$. Types $B_n$ and $D_n$ are similar.

}

\block{Type $A_4$ Crystal Graph of $\textbf{B}^{3,1}\otimes\textbf{B}^{2,1}$}{


%\includepdf[pages={-}]{KR_example.pdf}
\includegraphics[scale=.65]{KR_example.pdf}

} 

\column{.25}
\block{The Quantum Alcove Model for $\textbf{B}^{\textbf{p}}$}{

The main ingredient is the Weyl group $\textbf{W} = \langle s_\alpha :\alpha\in\Phi\rangle$.

The \textit{quantum Bruhat graph} on \textbf{W} is the directed graph with labeled edges $w \xrightarrow{\alpha} ws_\alpha,$ where 

\hspace{24pt} $l(ws_\alpha) = l(w) + 1$ (Bruhat graph), or 

\hspace{24pt} $l(ws_\alpha) = l(w) + 1 - 2\langle\rho,\alpha^{\vee}\rangle.$

\vspace{12pt}\textcolor{Blue}{Definition.} Given a dominant weight $\lambda=\omega_{p_1}+\hdots + \omega_{p_r}$, we associate with it a sequence of roots, called a $\lambda - chain$ (many choices possible): $$\Gamma = (\beta_1,\beta_2,\hdots,\beta_m).$$
Let $r_i := s_{\beta_i}$.  We consider subsets of positions in $\Gamma$ $$J = (j_1<j_2<\hdots <j_s)\subseteq \{1,\hdots,m\}.$$

\textcolor{Blue}{Definition.} A subset $J = \{j_1<j_2<\hdots < j_s\}$ is \textit{admissible} if we have a path in the quantum Bruhat graph $$Id = w_0\xrightarrow{\beta_{j_1}} r_{j_1}\xrightarrow{\beta_{j_2}} r_{j_1}r_{j_2} \hdots \xrightarrow{\beta_{j_s}} r_{j_1}\hdots r_{j_s}.  $$

\textcolor{blue}{Theorem [LNSSS, 2016]:} The collection of all admissible subsets, $\cm{A}(\Gamma)$,is a combinatorial model for $\textbf{B}^{\textbf{p}}$.


}

\block{The Two Realizations}{

\begin{itemize}
\item The Tableaux model is simpler and has less structure.
\item The Quantum Alcove model has extra structure which makes it easier to do several computations (energy function, combinatorial R-Matrix, charge statistic$\hdots$)

\end{itemize}

}

\block{Relating the Two Models}{

 We build a forgetful map $fill:\mathcal{A}(\Gamma)\rightarrow Tableau(\lambda)$ where $\lambda=\omega_{p_1}+\hdots\omega_{p_r}$.  

\vspace{12pt}\textcolor{blue}{Definition:} For any $k = 1,\hdots,n-1$ we define $\Gamma(k)$ to be the following chain of roots: 
$$((k,k+1),(k,k+2),\hdots,(k,n)\hdots$$
$$(2,k+1),(2,k+2),\hdots,(2,n)$$
$$(1,k+1),(1,k+2),\hdots,(1,n))$$

\vspace{12pt}\textcolor{blue}{Definition:} We construct a \textit{$\lambda$-chain} as a concatenation

 $\Gamma :=\Gamma^{\mu_1}\hdots\Gamma^1$ where $\Gamma^j = \Gamma(p_j)$.

\vspace{12pt}\textcolor{blue}{Example} Consider $n=4$ and $\lambda = (3,2,1,0)$.  Then the associated $\lambda$-chain is $ \Gamma = \Gamma^3\Gamma^2\Gamma^1 =$ $$ ((3,4),(2,4),(1,4) | (2,3),(2,4),(1,3),(1,4) | (1,2),(1,3),(1,4)).$$


\vspace{12pt}\textcolor{blue}{Example}
$J=\{1,2,4,5,8\}\in\mathcal{A}(\Gamma)$.
$$({\underline{(3,4)}},\underline{(2,4)},(1,4)|\underline{(2,3)},\underline{{(2,4)}},(1,3),{(1,4)}|{\underline{(1,2)}},(1,3),(1,4))$$


We get the corresponding path in the Bruhat order/quantum Bruhat graph
$$id=\begin{array}{l}\tableau{{1}\\{2}\\{\textcolor{Red} 3}} \\ \\ \tableau{{\textcolor{Red} 4}} \end{array} \!\begin{array}{c} \\ \xrightarrow{3,4} \end{array}\! \begin{array}{l}\tableau{{ 1}\\{\textcolor{Red} 2}\\{ 4} \\ \\ {\textcolor{Red}  3}} \end{array} \begin{array}{c} \\ {\xrightarrow{2,4}} \end{array}\! \begin{array}{l}\tableau{{\textcolor{Blue} 1}\\{\textcolor{Blue}{ 3}}}  \\ \tableau{{\textcolor{Blue}{ 4}}\\ \\{ 2}} \end{array}\!  \:|\:  \begin{array}{l}\tableau{{{ 1}}\\{{\textcolor{Red} 3}}} \\ \\ \tableau{{{\textcolor{Red} 4}}\\{ 2}}\end{array}\!\begin{array}{c} \\ \xrightarrow{2,3} \end{array}\!  \begin{array}{l}\tableau{{\ 1}\\{\textcolor{Red} 4}}\\ \\ \tableau{{3}\\{\textcolor{Red} 2}}\end{array} \begin{array}{c} \\ \xrightarrow{2,4} \end{array}\!  \begin{array}{l}\tableau{{\textcolor{Blue} 1}\\{\textcolor{Blue} 2}}\\ \\ \tableau{{3}\\{4}}\end{array} \:|\: 
\begin{array}{l}\tableau{{\textcolor{Red} 1}\\ \\ {{\textcolor{Red} 2}}\\{3}\\{4}} \end{array} 
\!\begin{array}{c} \\ \xrightarrow{1,2} \end{array}\!   \begin{array}{l} \tableau{{\textcolor{Blue} 2}\\ \\{1}\\{3}} \\ \tableau{{4}}\end{array} \! = end(J) \,.$$

\vspace{12pt}This gives us $fill(J) = $ \[\tableau{{1}&{1}&{2}\\{3}&{2}&\\{4}&&}\,.\]

}


\column{0.25}\block{The Reverse Map in Type $A_{n-1}$}{

 Consider the tableau in $\bigotimes_{i=1}^r B^{p_i,1}$ from the previous example \[f(T)=\tableau{{1}&{1}&{2}\\{3}&{2}&\\{4}&&}\,.\]
 
 Use entries of columns $i$ and $i-1$ viewed as sets to build the desired sub-list of $\Gamma^i$ where the zero column is the size $n$ column of strictly increasing entries.

  
 This is done with two algorithms: \textcolor{Blue}{Reorder and Greedy}
 
 The resulting bijection is a crystal isomorphism [LL,2015].


}


 
 \block{The Reorder Rule}{

First, let us consider the circular order
\[a\preceq_a a+1\preceq_a\ldots\preceq_a n\preceq_a 1\preceq_a\ldots\preceq_a a-1\,.\]

We write all chains in $\preceq_a$ starting with $a$, so the subscript $a$ can be dropped.

Let $C$ and $C'$ be two columns.
\medskip We fix the entries in $C$ and wish to reorder those in $C'$.


\medskip For each $1\leq i \leq \#C'$, we have $$a_i=C'(i) = min\{C'(l):i\leq l \leq \#C'\}$$
where the minimum is taken with respect to the circle order on [$n$] starting with $C(i)$.

\vspace{12pt}\textcolor{blue}{Example:} If $C = \tableau{{2}\\{1}\\{3}\\{4}}\,$ and $C'=\tableau{{1}\\ {3}\\{4}}$. Then $reorder_C(C') = \tableau{{3}\\ {1}\\{4}}$.
}

\block{The Greedy Algorithm}{

We now rebuild the desired sublist of $\Gamma_i$ by going through $\Gamma_i$ root by root.   

\vspace{12pt} For root $(j_1,j_2)$ if $C[j_1]\prec C[j_2] \prec \hat{C'}[j_1]$ and $C \xrightarrow{(j_1,j_2)} \hat{C'}$ is in the -corresponding QBG, then apply it. Otherwise skip.  Continue.

\medskip So for our example, we have $\Gamma_1=((3,4),(2,4),(1,4))$
\medskip and get $$C = \tableau{{1}\\{2}\\{3}\\{4}} \xrightarrow{(3,4)} \tableau{{1}\\{2}\\{4}\\{3}} \xrightarrow{(2,4)} \tableau{{1}\\{3}\\{4}\\{2}}$$


}




}

\column{.25}\block{The Type $C_n$ Map}{

\begin{itemize}
\item The filling map is similar.
\item The inverse map has one major change.  Many $KN$ columns have both $i$ and $\overline{\imath}$ in them, so we use the splitting algorithm [Lecouvey] to bijectively make two columns with no $i,\overline{\imath}$ pairs in either.
\item Then similar reorder and greedy algorithms work.
\item So now the reverse map is made up of a process of \textcolor{Blue}{Split, Reorder, and Greedy}.
\item \textcolor{blue}{Example:}
\end{itemize}


$$\begin{array}{l}\tableau{{4}\\{5}\\{\overline{5}}\\{\overline{4}}\\{\overline{3}}} \end{array} \!\begin{array}{c} \\ \xrightarrow{split} \end{array}\! \begin{array}{ll}\tableau{{ 4}&{1}\\{5}&{2}\\{ \overline{3}}&{\overline{5}}\\{\overline{2}}&{\overline{4}}\\{\overline{1}}&{\overline{3}}} \end{array}  \,$$


\medskip The $\Gamma(k)$ in type $C_n$ comes in two parts.
\medskip We use the first to get a chain from the left split to the reordered right split

\medskip and the second to get a chain from the right split to the next column's left split.

}

\block{The Type $B_n$ Map}{

\begin{itemize}
\pause\item There is a similar filling map
\pause\item For the reverse, similar to $C_n$, we need a splitting map.
\pause\item Recall that we now have columns of length $k-2l$, so we need to Extend back to length $k$ [Briggs].
\pause\medskip\item Further, the greedy algorithm and reorder algorithm no longer work.
\pause\medskip\item There is a configuration of two columns $CC'$ that we call being \textcolor{Blue}{blocked-off}.
\pause\medskip\item Modify greedy and reorder to avoid block-off pattern.
\end{itemize}


\textcolor{blue}{Definition:} We say that columns $L = (l_1,l_2,...,l_k), R' = (r_1,r_2,...,r_k)$ are \textit{blocked off  at $i$ by $b=r_i$} iff $0<b\geq |l_i|$ \pause and$$\{1,2,...,b\}\subset \{|l_1|,|l_2|,...,|l_i|\}$$ and $$\{1,2,...,b\}\subset \{|r_1|,|r_2|,...,|r_i|\}$$ \vspace{10pt} \pause and $|\{j : 1\leq j\leq i, l_j<0, r_j> 0\}|$ is odd.

}

\block{Further Work}{

\begin{itemize}
\item The map in type $D_n$  similar to type $B_n$, but there is a second pattern to be avoided in Reorder and Greedy.
\item The bijections for types $B_n$ and $D_n$ given here are actually crystal isomorphisms.
\end{itemize}

}

\block{Bibliography}{
\begin{itemize}

\item C. Lenart, A. Lubovsky,  \textit{J. Algebraic Combin.}, 2015

\item C. Lenart, S. Naito, D. Sagaki, A. Schilling, M. Shimozono,  \textit{Int. Math. Res. Not.}, 2016

\end{itemize}
}

\end{columns}

\end{document}

 

