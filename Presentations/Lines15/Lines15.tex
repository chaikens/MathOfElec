\documentclass{beamer}
%\mode<presentation>
\usepackage{cite}
\usepackage{amsmath}
%\usepackage{mathtools}
\title{
Resistive Networks, Linear Spaces and Tutte Polynomials\\
for Systems of Lines: Applications of Algebraic Combinatorics Workshop\\
Worcester Polytechnic Institute}
\author{Seth Chaiken\\
Dept. of Computer Science\\
College of Engineering and Applied Science\\
Univ. at Albany\\
\url{schaiken@albany.edu}
}
\date{August 14, 2015}
%%%%%%%%%%%%%%%%%%%%%%%%%%%%%%%%%%%%%%%%%%%%%%%%%%%%%%%%%%%%%
% Specialized Symbols
%
%   Disjoint Union
%\newcommand{\dunion}{\uplus}
\newcommand{\dunion}
%{\mbox{\hbox{\hskip4pt$\cdot$\hskip-4.62pt$\cup$\hskip2pt}}}
%{\mbox{\hbox{\hskip6pt$\cdot$\hskip-5.50pt$\cup$\hskip2pt}}}
{\mbox{\hbox{\hskip0.45em$\cdot$\hskip-0.44em$\cup$\hskip0.2em}}}
%{\mbox{\hbox{\hskip0.45em$+$\hskip-0.70em$\cup$\hskip0.3em}}}
%
% Dot inside a cup.
% If there is a better, more Latex like way 
% (more invariant under font size changes) way,
% I'd like to know.

\newcommand{\Bases}[1]{\ensuremath{{\mathcal{B}}(#1)}}
\newcommand{\Reals}{\ensuremath{\mathbb{R}}}
\newcommand{\FieldK}{\ensuremath{K}}
\newcommand{\Perms}{\ensuremath{\mathfrak{S}}}
%\newcommand{\rank}{{\rho}}% {{\mbox{rank}}}
%\newcommand{\Rank}{{\rho}}% {{\mbox{rank}}}
\newcommand{\rank}{{\mbox{r}}}% {{\mbox{rank}}}
\newcommand{\Rank}{{\mbox{r}}}% {{\mbox{rank}}}
\newcommand{\Card}[1]{\ensuremath{{\left|#1\right|}}}
\newcommand{\ext}[1]{\ensuremath{\mathbf{#1}}}
%\newcommand{\extvee}{\ensuremath{\mathbf{\vee}}}
\newcommand{\extvee}{\;\;}
\newcommand{\Plucker}{Pl\"{u}cker\ }

% Set Complement
% command to mess with overline, bar or custom 
% alternatives for sequence or set complement
%
\newcommand{\scomp}[1]{\ensuremath{\overline{#1}}}
%\newcommand{\scomp}[1]{\ensuremath{\bar{#1}}}
%
%   Put a symbol for a matroid in a box, or brackets
%\newcommand{\MVAR}[1]{{\boxed{#1}\;}}
\newcommand{\MVAR}[1]{{[#1]\;}}
%
\newcommand{\UNION}{\cup} %try to make this bold.
%
%Emphasize in color!
\newcommand{\Remph}[1]{{\color{red}#1}}
%
%%%%%%%%End of specialized symbols%%%%%%%%%%%%%%%%%%%%%%


\newcommand\coolover[2]{\mathrlap{\smash{%
\overbrace{\phantom{\begin{matrix} #2 %
\end{matrix}}}^{\mbox{$#1$}}}}#2}


\begin{document}
\begin{frame}
 \titlepage
\end{frame}

\note{(here we go, geronimo)}



\begin{frame}
Old hat: A 1-dimensional linear subspace of $\Reals^n$ \textbf{is the line} 
$\Reals (a_1, a_2, \ldots a_n)$.

New hat: Explain exterior algebra and Grassmanians in 4 words?


Hint:  The rank $r$ grade of the exterior algebra over $\Reals^n$ has 
basis the $\binom{n}{r}$ (multilinear, alternating) products
of the $r$ coordinate vectors $e_{i_1}\wedge e_{i_2}\wedge \ldots \wedge e_{i_d}$
with $i_1<i_2<\cdots < i_d$.

\end{frame}

\begin{frame}
\frametitle{Linear subspaces are (certain exterior algebra) lines.}

\begin{block}{extensors code subspaces}
The subspace of $\Reals^n$ spanned by independent vectors 
$v_1, v_2, \ldots, v_d$ is coded by the \textbf{line}
$\Reals(v_{i_1}\wedge v_{i_2}\wedge \ldots \wedge v_{i_d})$.
\end{block}

The row space of an $r \times n$ full rank matrix $M$ is uniquely determined by
the \textbf{line} of $\Reals$ multiples of the 
$\binom{n}{r}$-tuple of determinants $M[T]$ 
of the $r\times r$ submatrices.\\
($T$ is a sequence of $r$ columns.)

Row operations multiply all by a common $\neq 0$ factor.

\textbf{SPARSENESS:}

The space of these lines which represent $d$-dim spaces is a
$n(r-n)$-dimensional non-linear manifold called the Grassmannian.

The $\binom{n}{d}$ coordinates are constrained by the (quadratic)
\textbf{Grassmann-Pl\"{u}cker relations}:
\[
[s_1 s_2 ... s_d][t_1 t_2 ... t_d] = 
\sum_{i=1}^d [t_i s_2 ... s_d][t_1 t_2 ... \hat{t_i} s_1 ... t_d]
\]
(swap column $s_1$ with column $t_i$ for $i=1, ..., n$)


\end{frame}
  

\begin{frame}
\frametitle{A basis exchange axiom for matroids}

\[
[s_1 s_2 ... s_d][t_1 t_2 ... t_d] = 
\sum_{i=1}^d [t_i s_2 ... s_d][t_1 t_2 ... \hat{t_i} s_1 ... t_d]
\]

\begin{block}{Axiom}
If $B_1= \{s_1, s_2, \ldots, s_d\}$ and 
$B_2= \{t_1, t_2, \ldots, t_d\}$ are bases, 
then
there exists
$i$, $1 \le i \le d$ for which both sets
$B_1\setminus s_i \cup t_1$ $=$ 
$\{t_1, s_2, \ldots,  s_d\}$ 
and 
$B_2 \setminus t_1 \cup s_i$ $=$ 
$\{t_1, t_2, \ldots, \hat{t_i}, s_i, \ldots, t_d\}$ 
are bases.
\end{block}

\begin{block}{A Matroid Definition (one out of dozens)}
A matroid coded by its bases is any 
finite collection of $d$-sized sets that satisfies the
above axiom.
\end{block}

\begin{block}{Exercise}
Discover (chirotope) axioms that define \textbf{oriented matroids} by 
deriving the necessary conditions on the signs of the 
brackets (determinants) in the Grassmann-Pl\"{u}cker
relation.
\end{block}

\end{frame}


\begin{frame}[fragile]
\frametitle{Electrical current is a network flow}

The $\pm 1$ vertex-edge incidence matrix $M$
of a graph $\mathcal{N}$ fixes 
an arbitrary direction of each edge.

\[
Mi=0 \Longleftrightarrow \{i_e\}_{e\in E} \text{ is a \textbf{flow} (of conserved
current)}
\]

Let's make $M$ full row rank by removing redundant rows.

Let $T\subset E$ be any spanning tree.  $T^c = E\setminus T$ is a 
\textbf{cotree}.

Each $\{i_e\}_{e\not\in T} \in \Reals^{T^c}$ extends to a unique flow 
$\{i_e\}_{e\in E}$. 

How?  Take the $\{i_e\}_{e\in T}$ to \textbf{balance
the excess at each vertex}.

That is: Row operations transform $M$ to 

\[
\left[
\begin{array}{c|c}
 I & 0,\pm 1 s \\
\end{array}
\right]
\]

\begin{center}
if and only if $T$ is a spanning tree.
\end{center}

$M$ \textbf{represents} by its \textbf{columns} the (graphic) 
matroid whose bases are the spanning trees.

\end{frame}


\begin{frame}
\frametitle{Electrical network problem: Solve for $i,v\in\Reals^n$}

\[
M_Vi = 0 \Leftarrow\text{equivalently}\Rightarrow i\in\text{Row space}{(M_I)},
\text{flows or 1-cycles}.
\]
\[
M_Iv = 0 \Leftarrow\text{equivalently}\Rightarrow v\in\text{Row space}{(M_V)},
\text{bonds or 1-cocycles}.
\]
\[
F(i,v) = 0 \;\;\text{locally rank}\;\;n.
\]

First two are Kirchhoff's two laws:  Combinatorial, assumed exact in electrical 
network applications (geometrical for $\ge 1$ dim. elastic networks).  
Total rank $= n$. 

Second are the \textbf{constitutive} constraints.

Linear one-port network:  For all but one edge, \textbf{Ohm's law} is written
\[r_ei_e - g_e v_e = 0\]
For the one \textbf{port edge} $p$, demarking a pair of 
\textbf{terminal vertices},\\
either
$i_p = 1$ then solve for $v_p=$ \textbf{equivalent resistance}
by say eliminating the $v_e$, $i_e$ for the $n-1$ resistors.\\
or $v_p = 1$ then solve for $i_p=$ \textbf{equivalent conductance} ... .

\end{frame}


%%%%%%%%%%%%%%%%%%%%%%%%%%%%%%%%%%%%%%%%%%%%%%%%%
\begin{frame}{Equiv. Resistance $ :\equiv -(v_p/i_p)$
observed at a port $p$ 
%by the environment
}

\begin{theorem}[Kirchhoff 1847, called ``Maxwell's rule'']
Let $g_T$ denote $\prod_{e\in T}g_e$, etc.\\
$G/p$ is $G$ with edge $p$ \textbf{contracted} (vertices identified).\\
$G\setminus p$ is $G$ with edge $p$ \textbf{deleted}.\\
\[
-(\frac{v_p}{i_p})=
%\frac{\mbox{WTS}(G/p)}
%{\mbox{WTS}(G\backslash p)} 
\frac{\sum_{T:\text{spanning tree of}\;(G/p)}g_Tr_{T^c}}
{\sum_{T:\text{spanning tree of}\;(G\backslash p)}g_Tr_{T^c}}
= \frac{\mbox{Matrix-Tree Det}(G/p)}
       {\mbox{Matrix-Tree Det}(G\backslash p)}
\]
\end{theorem}

It's usually proved 
via the Matrix Tree Thm. on 2 DIFFERENT GRAPHS
$G/p$ and $G\backslash p$.  

\begin{theorem}[Matrix Tree]
Every $n-1 \times n-1$ subdeterminant of a graph's Laplacian
matrix enumerates the graph's spanning trees.
\end{theorem}

\end{frame}

\begin{frame}
\frametitle{A liney consequence of Kirchhoff's solution}
Pick any (resistor) edge $e$; factor the sums:
($T^c = E\setminus e \setminus T$ below)
\[
\frac{\mbox{WTS}(G/p)}
{\mbox{WTS}(G\backslash p)} 
=
\]
\[
\frac{r_e\sum_{T:\text{spanning tree of}\;((G\setminus e)/p)}g_Tr_{T^c}   
+g_e\sum_{T:\text{spanning tree of}\;((G/e)/p)}g_Tr_{T^c}   }
{r_e\sum_{T:\text{spanning tree of}\;((G\setminus e)\setminus p)}g_Tr_{T^c}
+g_e\sum_{T:\text{spanning tree of}\;((G/e)\backslash p)}g_Tr_{T^c}
}
\]

So, when we express our ratio by \textbf{the line} of all non-zero $\Reals$ 
multiples of $(\mbox{WTS}(G/p),\mbox{WTS}(G\setminus p))$, carefully picked 
generators satisfy \textbf{Tutte decomposition}: For 
``ordinary'' $e\neq p$
\[
(\mbox{WTS}(G/p),\mbox{WTS}(G\setminus p))
=
\]
\[
r_e(\mbox{WTS}((G\setminus e)/p),\mbox{WTS}((G\setminus e)\setminus p))
+
\]
\[
g_e(\mbox{WTS}((G/e)/p),\mbox{WTS}((G/e)\setminus p))
\]

\end{frame}

\begin{frame}
\frametitle{Multiport Linear electrical network problem}


Edge set $S= E\dunion P$. $2|E|+2|P|$ variables $i,e\in \Reals^S$.  
Flow (current) and bond (voltage) eqs. have rank 
$|E|+|P|$:
\[
M_Vi = 0 \Leftarrow\text{equivalently}\Rightarrow i\in\text{Row space}{(M_I)},
\text{cycles or flows}\;\mathcal{C}.
\]
\[
M_Iv = 0 \Leftarrow\text{equivalently}\Rightarrow v\in\text{Row space}{(M_V)},
\text{cocycles}\;\mathcal{C}^{\perp}.
\]
For the $|E| $ non-port, resistor edges,
\[
g_ev_e = r_ei_e
\]
So, the linear solution space has dim. $|P|$.

Project solutions into port voltage and current coordinate space $\Reals^{2|P|}$.

\end{frame}


\begin{frame}{Our exterior algebra valued Tutte function}
We've seen how a $|P|$-dim linear space of $\Reals^{2|P|}$ \textbf{is a line}.

\begin{theorem}
(After careful definitions...) For fixed $P$,\\
\begin{center}
\hfill
\begin{minipage}{0.80in}
each Pl\"{u}cker coordinate
\end{minipage}
\hfill
and
\hfill
\begin{minipage}{1.7in}
this extensor in the exterior algebra, coding the 
row space of the $p \times 2p$ $p$-port network 
solution matrix
\end{minipage}
\hfill
\end{center}
satisfy weighed Tutte recursion, when $/e$ and $\backslash e$
are restricted to $e\not\in P$:
\[
\text{Sol}(G) = r_e \text{Sol}(G\setminus e) + g_e \text{Sol}(G/e).
\]
\end{theorem}

\begin{block}{This result is about \textbf{a line of lines}.}
As as one of the linear resistance values varies from $0$ to $\infty$, 
the whole network solution ranges over a suitably defined \textbf{line}.
\end{block}



\end{frame}




%%%%%%%%%%%%%%%%%%%%%%%%%%%%%%%%%%%%%%
\begin{frame}{Example}
\input{k5.pspdftex}
\end{frame}


%%%%%%%%%%%%%%%%%%%%%%%%%%%%%%%%%%%%%%
\begin{frame}{Example}
$|E|+|P|$ 
simplified electrical network equations $Nx=0$.   
Kirchhoff's laws
apply to all cycles and cocyles with $r_ix_{e_i}$ as voltage and 
and $g_ix_{e_i}$ as current of resistor (not port) edges.  TWO 
voltage $vp$ and
current $ip$ variables are used for each port edge.
\[
\begin{array}{|cc|cccccccc|cc|}
ip_1 & ip_2  & e_1 & e_2 & e_3 & e_4 & e_5 & e_6 & e_7 & e_8 & vp_1 & vp_2 \\ \hline
 1   &  0   &  0  &  0  &+g_3 &  0  &  0  &-g_6 &  0  &-g_8 &      &      \\
-1   &  0   &-g_1 &  0  &  0  &  0  &+g_5 &  0  &+g_7 &  0  &      &      \\
 0   & +1   &  0  &  0  &  0  &+g_4 &-g_5 &  0  &  0  &+g_8 &      &      \\
 0   & -1   &  0  &-g_2 &  0  &  0  &  0  & g_6 &+g_7 & g_8 &      &   \\ \hline
     &      &+r_1 &  0  &-r_3 &  0  &  0  &  0  &  0  &  0  &  1   &  0   \\
     &      &  0  &+r_2 &  0  &-r_4 &  0  &  0  &  0  &  0  &  0   &  1   \\
     &      &-r_1 &  0  &  0  &+r_4 &+r_5 &  0  &  0  &  0  &  0   &  0   \\
     &      &  0  &-r_2 &+r_3 &  0  &  0  &+r_6 &  0  &  0  &  0   &  0   \\
     &      &-r_1 &+r_2 &  0  &  0  &  0  &  0  &+r_7 &  0  &  0   &  0   \\
     &      &  0  &  0  &+r_3 &+r_4 &  0  &  0  &  0  &+r_8 &  0   &  0\\ \hline 
\end{array}
\]
Top 4 rows: Basis for cocycle space. Represents graphic matroid.\\
Bot 6 rows: Basis for cycle space. Represents cographic matroid.\\
\end{frame}

%%%%%%%%%%%%%%%%%%%%%%%%%%%%%%%%%%%%%%
\begin{frame}{How Tutte Decomposition Emerges}
For all choices denoted by ?? of the $\binom{2|P|}{|P|}$ size $|P|$ subsets of
the $2|P|$ columns $\{ip_k, vp_k\}$, the matrices in the equation below are
square.\\
So the elementary multilinearity of determinants
means Tutte decompostion holds for all $e_i\not\in P$:
  
\input{portmatrec.pspdftex}

(Technical detail:  Define the Tutte function on all graphs with 
distinguished or port subset $P$ so the det. signs are consistent 
with the decomposition.)
\end{frame}


\begin{frame}
\frametitle{Non-linear situations (Chua et. al.) }
Relaxation oscillator built with a piecewise linear resistor.
The $(v,i)$ locus is contained in two Thevenin theorem (affine) lines.

%\begin{center}
\input{relaxPrimative.pspdftex}
\begin{minipage}[t]{1.6in}
(After Chua, DeSoer and Kuh's 1987 textbook)
\end{minipage}
%\end{center}

\input{relaxPrimativePlot.pspdftex}
%($v=-C\int i dt$, circuit diagram)

%($v$ waveform,  dynamical system direction field) 
\end{frame}

\begin{frame}
\frametitle{Operational amplifier realization}

Positive feedback via $r_A$ and $r_B$ 
implements the sometimes \textbf{negative resistance} driving the capacitor.

\input{relaxOpAmp.pspdftex}



\begin{minipage}[b]{1.5in}
The inductor models parasitic inductance and/or amplifier delay needed to
resolve the dynamics at the \textbf{impasse} points.
\end{minipage}
\input{relaxOpPlot.pspdftex}

\end{frame}

\begin{frame}
\frametitle{Oriented matroid (= combinatorics of dependency signs) 
conditions}

Why must the solution of a network with monotone 
increasing non-linear resistors be unique (Duffin, 1947).

Immediately for linear resistors only: The Matrix Tree Theorem coefficients
are all $+1$.


Suppose there are two solutions.  On each resistor's curve there are two
points joined by a positive slope line.

\input{nonLinMono.pspdftex}
\begin{minipage}[b]{1.7in}
Vector $\Delta v$ and $\Delta i$ are separately constrained by 
Kirchhoff's (linear) laws.
\end{minipage}

%(figure with $\Delta v_e$ and $\Delta i_e$)
\end{frame}

\begin{frame}
\frametitle{A better reason:(1980s EE results abstracted to OM)}

\begin{itemize}
\item
$\Delta v \in \text{linear cycle space}\; \mathcal{C}$,
$\Delta i \in \mathcal{C}^{\perp}$, \\
so $\Delta v \cdot \Delta i = 0$.  
\item
Monotonicity $\Leftrightarrow$ 
$\text{sign}(\Delta i_e) = \text{sign}(\Delta v_e) \in \{0, +, -\}$\\
so
$\Delta v \cdot \Delta i > 0$ for non-zero $\Delta v$ or $\Delta i$,\\
which is a contradiction.
\end{itemize}


What if an amplifier structure makes $\mathcal{C}_I$ and $\mathcal{C}_V$ be
non-orthogonal?  Uniquess is still guaranteed for there are no
$\Delta i\in \mathcal{C}_I$ and $\Delta v \in \mathcal{C}_V$ with
$\text{signs}(\Delta i) = \text{signs}(\Delta v) \neq (0, ..., 0)$



\end{frame}

\begin{frame}
\frametitle{An amplifying subsystem simulates a negative resistance; 
how uniqueness fails 
(Fosseprez, Hasler, Marthy, Neirynck, Oberlin, de Werra 80-90s}

\input{opSingularBridge.pspdftex}


\input{singularBridge.pspdftex}
\begin{minipage}[b]{2.3in}
The ideal amplifier makes $\mathcal{C}_V$ and
$\mathcal{C}_I$ contain a common non-zero sign pattern.
Positive resistance values can be chosen (for example, all 1) so 
in all 4 resistor edges, $\Delta v$ $=$ $\Delta i$ $=$ $\Delta$.
\end{minipage}



\end{frame}


\begin{frame}
\frametitle{Time for the ball game!}
\end{frame}


\begin{frame}
Interesting combinatorial theory emerges when quantities or relations
of linear electrical network analysis are expressed as lines or more
generally affine linear subspaces.

Our starting point is Thevenin's and Norton's theorems.  They conclude that
the voltage $v_p$ and current $i_p$ at a pair of terminals are characterized
by the affine constraint $av_p + bi_p + c = 0$.  

How the load line is used...

\end{frame}



\begin{frame}{Outline}
\begin{enumerate}
\item Spanning trees and equivalent (linear) resistance.
\item An exterior algebra (extensor) Tutte function and\\
 a (linear) resistance network's
behavior projected on distinguished coordinates.
\item Rayleigh's inequalities.
\item Tutte polynomials on pairs and (linear) amplifier networks.
\item Distinguished graph vertices and splitting formulas.
\end{enumerate}
\end{frame}


%%%%%%%%%%%%%%%%%%%%%%%%%%%%%%%%%%%%%%
\begin{frame}{Next steps}
\begin{enumerate}
\item One (terminal-pair) port $\rightarrow$ set of ports $P$.
\item 1-dim subspace of homogeneous coordinates of solutions
$((v_p, i_p))$ $\rightarrow$ p-dim subspace of $k^{2|P|}$.
\item p-dim subspace $\rightarrow$ EXTENSOR (decomposible
exterior algebra, i.e., anti-symmetric tensor) with $\binom{2p}{p}$ Plucker
coordinates (determinants).
\end{enumerate}
\end{frame}



%%%%%%%%%%%%%%%%%%%%%%%%%%%%%%%%%%%%%%%%%%%%%%%%%
\begin{frame}
\frametitle{Rayleigh Identity which $\Rightarrow$ inequality, 
``Neg. Spanning Tree Correlation''}
\[
\Gamma_e(G)\text{ is equivalent conductance across }e.
\text{ Rayleigh: }0 \le \frac{\partial \Gamma_{p}}{\partial g_f} =
\frac{\partial \frac{T_G}{T_{G/e}}}{\partial g_f}
\]
is equivalent to 
\[
0 \le \frac{\partial T_G}{\partial g_f}T_{G/e} - 
       T_G\frac{\partial T_{G/e}}{\partial g_f} 
=
T_{G/f}T_{G/e} - T_GT_{G/e/f}
\]
\begin{theorem}
\[
T_{G/f}T_{G/e} - T_GT_{G/e/f} = \left( T^+_{G/e \text{ \& } G/f} - T^-_{G/e \text{ \& } G/f} \right)^2
\]
$T^{\pm}_{G/e \text{ \& } G/f}$ enumerate the $\pm$ common spanning trees.
\end{theorem}
Choe, Cibulka, Hladky, Lacroix and Wagner gave bijective proofs; 
we give det. based proofs and generalizations.
\end{frame}



\begin{frame}
\frametitle{Linear Alg./Oriented Matroid Proof of Rayleigh's Identity}
Let $R$ be the transfer resistance matrix for 2 ports across $e$ and $f$.
Our result implies that
\[
\det R = \left|\begin{array}{cc} R_{ee} & R_{ef} \\ R_{fe} & R_{ff} \end{array}\right|
= \alert{+} \frac{T_{G/e/f}}{T_G}
\]
It and better-known results tell us
\[
R_{ee} = \frac{T_{G/e}}{T_G};\;\;R_{ff} = \frac{T_{G/f}}{T_G};\;\;
R_{ef}=R_{fe}=\frac{ T^+_{G/e \text{ \& } G/f} - T^-_{G/e \text{ \& } G/f} }{T_G}
\]
%Rayleigh's identity 
$T_{G/f}T_{G/e} - T_GT_{G/e/f} = \left( T^+_{G/e \text{ \& } G/f} - T^-_{G/e \text{ \& } G/f} \right)^2$
is immediate after substituting these into
\[
\det R = R_{ee}R_{ff}-(R_{ef})^2
\]
\alert{The $+$ follows from physical grounds if the $g_e, r_e \geq 0$.  Our
characterization and proof are combinatorial.}
\end{frame}

\begin{frame}
\frametitle{New Rayleigh's Identities!}

The same method generates identities and inequalities from
\[
\left|
\begin{array}{ccc} R_{ee} & R_{ef} & R_{eg} \\ 
                   R_{fe} & R_{ff} & R_{fg} \\
                   R_{ge} & R_{gf} & R_{gg}
\end{array}\right|
= \alert{+} \frac{T_{G/e/f/g}}{T_G} \ge 0
\]
when all $r_{..}, g_{..} \ge 0$, ETC...

\vfill

(Applications???)

\vfill

\Remph{Might the same methods address a much harder problem:
The same inequality for } forests \Remph{ instead of 
spanning trees?}

\vfill
\end{frame}



%%%%%%%%%%%%%%%%%%%%%%%%%%%%%%%%%%%%%%%%%%%%%%%%%
\begin{frame}{Pairs: The Common Covector Model}
\input{commonCovectorModel.pspdftex}
\end{frame}


%%%%%%%%%%%%%%%%%%%%%%%%%%%%%%%%%%%%%%%%%%%%%%%%%
\begin{frame}
\frametitle{Voltage and Current graphs $G_V$, $G_I$}

\framebox{\begin{minipage}{0.52\textwidth}
``Voltage graph'' $G_V$ (EE\cite{HaslerNeirynck,PathologicalAct}, NOT Gross, ...) represents KVL
$\mathbf{v}\in \text{Cocycles}$ W/ SOME $v_e \equiv 0$
\end{minipage}}
\framebox{
\begin{minipage}{0.4\textwidth}
``Current graph'' $G_I$ represents KCL
$\mathbf{i}\in \text{Cycles}$ WITH SOME FLOWS $\equiv 0$
\end{minipage}}

\begin{itemize}
\item 
They are EQUAL GRAPHS for resistor networks.

\item
For networks with idealized amplifiers, they are not 
equal.  

\input{idealAmp.pspdftex}

The output voltage and current are whatever makes the input
voltage and current BOTH BE zero.

\item
(More) realistic amp. model $=$ idealized amp. $+$ 
resistors.

\end{itemize}


\framebox{
\begin{minipage}{0.2\textwidth}
\Remph{open}
\[
G_v = G\backslash e
\]
\[ 
G_I = G\backslash e
\]
\end{minipage}}
\framebox{\begin{minipage}{0.2\textwidth}
\Remph{short}
\[
G_v = G/e
\]
\[
G_I = G/e
\]
\end{minipage}}
\framebox{
\begin{minipage}{0.2\textwidth}
\Remph{nullator}
\input{nullator.pspdftex}
\[
G_v = G/e
\]
\[ 
G_I = G\backslash e
\]
\end{minipage}}
\framebox{\begin{minipage}{0.2\textwidth}
\Remph{norator}
\input{norator.pspdftex}
\[
G_v = G\backslash e
\]
\[
G_I = G/e
\]
\end{minipage}}

\end{frame}


\begin{frame}{Distinguished graph vertices and splitting formulas}
Let $Q$ be a set of distinguished, labelled graph VERTICES, analogous to the
distinguished port edges $P$

\begin{theorem}
Given graph $G(V\dunion Q, E\dunion P)$ let $T(G,P,Q)$ be the Tutte polynomial 
determined by restricting $/e$ and $\backslash e$ to $e\not\in P$ AND 
carrying along the partition of $Q$ defined by the components of the contracted edges.\\
Construct $G^{Q}(V\dunion Q, E\dunion P\dunion P_Q)$ by 
adding to $G$ a new vertex $Z$ and the $|Q|$ new port edges
from $Z$ to each vertex in $Q$.\\
Then $T(G,P,Q)$ and $T(G^{Q},P\dunion P_Q)$ (the ported Tutte polynomial) 
determine each other by substitutions.
\end{theorem}

So we can use ported Tutte polys to express 
splitting formulas for Tutte polynomials of graph, beginning with Crapo 
\cite{CrapoChromJoining} 
and continuing with Andrzejak\cite{andrzejak97splitting},
Bonin and de Meir \cite{BonindeMierTutteGenParConn}, 
and Narayanan \cite{NarayananDecompVS1986,NaraGenMinorENet}.
\end{frame}

\begin{frame}{etc}
\begin{center}
Extra slides...
\end{center}
\end{frame}



%%%%%%%%%%%%%%%%%%%%%%%%%%%%%%%%%%%%%%%
\begin{frame}{Why Electricity, EE?}
\begin{itemize}
\item Scholarly topic suggested by G.-C. Rota $\approx$ 1980?.
\item $\approx 100$ yrs. geometry-like intuition of
circuit configurations known by engineers, EE books:
\Remph{``Intuitive Analog Circuit Design (2013)''
\cite{intuitAna}}; ``Non-linear Circuits'' \cite{HaslerNeirynck}
translates to our Oriented Matroid pair model.

\item Geometry of linear spaces and oriented matroids;
Tutte decomp. w/
techniques from Barnabi, Brini and Rota's Exterior Calculus 
\cite{exteriorCalc})

\item
Real behavior $\approx$ ideal plus perturbations,
ideal constraints predict intended real behavior,
%???? RISK THIS?? \Remph{applied everyday geometry:}

\item 
Interesting, accessible, intuitively understandable
intentential designs, applicable,
easy to both simulate and build physically, dimension
$\approx$
12 or 24, depending on formulation
\item
Analogs to chemical (and real algebraic geometry 
\cite{signsInChemRAG}), biological, \Remph{elastic/tensegrity strs.} etc., 
random walks ...
\item
Merely one scalar non-linearity can cause chaos.
\end{itemize}
\end{frame}

\note{
(nonframe)Analogies:
\begin{enumerate}
\item Electric: Voltage, Resistance, Current, Capacitance, Inductance
\item Mech. Dyn: Force, Viscous friction, Velocity, Elasticity, Mass
\item Statics: Displacement, Elasticity, Force,  ???, ???
\item Random Walk: First hit probability, transition probability, flow, ???, ??? \\
Doyle and Snell \cite{DoyleSnellRandom},
Lyons and Peres \cite{ProbOnTreesNetworks}
(ref? Thanks, David G. Wagner's course notes).
\item Traffic: Time, Clogginess (travel time/flow ratio), flow, ???, ???\\
Condition: All vehicles reach their destinations in equal times regardless of 
route (ref? Thanks, Peter Shor).
\item Knot theory: ???, $\pm 1$ crossing sign, ???, ???, ???\\
Equivalent resistance ``Conductance Invariant'' of rational tangles is a 
Reinmeister move 
invariant\cite{RationalTangles} (Thanks, Jay Goldman).
\end{enumerate}

KCL+KVL+Ohms=(one of KCL)+Ohms+(min power)

Future: Memistor?
}



%%%%%%%%%%%%%%%%%%%%%%%%%%%%%%%%%%%%%%%%%%%%%%%%%%%%%%%
\begin{frame}{Kirchhoff (1847)
\cite{Kirchhoff}
 Maxwell (1891)
\cite{MaxR} 
The equivalent resistance PROBLEM IS SOLVED
by the Matrix Tree Theorem. (1) POSE! the \Remph{VARIABLES} or 
\Remph{COORDINATES}}

\framebox{\begin{minipage}{.47\textwidth}
\framebox{\Remph{$v_e, i_e : e\in E$}}
\input{resistor.pspdftex}
(Use \textbf{voltage drops along the flow}, 
not potentials $V_1$, $V_2$.)
\begin{center}\textbf{ordinary, resistor edge} $e\in E$
\end{center}

(IF linear Ohm's law use $|E|$ 
variables ($g_ev_e=r_ei_e$) ELSE
use $2|E|$ variables.)

\end{minipage}}
\framebox{\begin{minipage}{.47\textwidth}
\framebox{\Remph{$v_p, i_p : p\in P$}}
\input{portWithEnviron.pspdftex}
\begin{center}\textbf{DISTINGUISHED, PORT edge} $p\in P$
\end{center}
The interface to an environment is modelled with
$2|P|$ variables.
\begin{center}(math, not EE 
sign convention)
\end{center}
\end{minipage}}\\
\end{frame}

%%%%%%%%%%%%%%%%%%%%%%%%%%%%%%%%%%%%%%%%%%%%%%%%%
\begin{frame}{(2) POSE: EQUATIONS.  \Remph{Preview the consequences.}}
\begin{itemize}
\item 
(KCL) $(i_e)_{e\in S}$ is a cycle (a flow).
\item
(KVL) $(v_e)_{e\in S}$ is a cocycle
\item
(constituitive Law) $i_e=g_e(v_e)$
non-linear, usually monotonic increasing $R\rightarrow R$.\\ 
(Sometimes use Ohm's approximation $i_e=g_ev_e$)
\end{itemize}
\begin{block}{Combinatorics!}
The signs $\{ +, -, 0\}$ have a \Remph{DUAL-PAIR ORIENTED MATROID} structure
(combinatorial, geometric, topological).
\end{block}
\begin{block}{Engineering with amplifiers!}
There's good unique solvablility due to STRUCTURE,
when the \Remph{NON-DUAL PAIR} (for voltages and currents)
is ALMOST DUAL: No common covectors.
\end{block}
\end{frame}



%%%%%%%%%%%%%%%%%%%%%%%%%%%%%%%%%%%%%%%%%%%%%%%%%
\begin{frame}{Multiple Ports. 
(your stereo: \Remph{3=}power plug \& 2 speakers) }

\begin{itemize}
\item One formula expresses $\binom{2|P|}{|P|}$ different Matrix Tree 
Theorems...
\item
... long vertex-based proofs are shortened; Rayleigh inequalities too.
\item
Interesting \textbf{non-commutative ranges} of
new ORIENTED MATROID Tutte invariants with pattern:
\[
\text{TF}(N(P\dunion E)) = F(N(P\dunion E)/E)
\]
(They distinguish DIFFERENT ORIENTATIONS of the SAME MATROID.)
\item
Ported/Relative OM Tutte Poly. terms \Remph{embed SPECIFIC MINORS} as
\Remph{variables}, making
proofs just with $\partial T/\partial x_e$ easier. 
\item
Formalize composition of systems\cite{NarayananDecompVS1986}, 
Tutte poly. splitting formulas.
\item
Model practical devices (transistors, op amps);
Label variables to observe.
\item 
Align EE applications with knots\cite{RelTuttePoly} (Ported = ``Relative'')
and combinatorial geometry\cite{SetPointedLV} (Ported = ``Set Pointed'').  
\end{itemize}
\end{frame}

%%%%%%%%%%%%%%%%%%%%%%%%%%%%%%%%%%%%%%%%%%%%%%%%%%%%
\begin{frame}{Constraint/Generator Duals and 2 Results.}
\begin{minipage}{0.48\textwidth}

\begin{itemize}
\item
(Part 1) Technique:
\begin{center}$\text{Solution Space}$\end{center}
\begin{center}$=$\end{center}
\begin{center}$\bigcap \text{Constraint Subspaces}$\end{center}
\item
\Remph{Result:} An exterior algebraic
algebraic Tutte function: Each of its
$\binom{2|P|}{|P|}$
Pl\"{u}cker coordinates
satisfies a Matrix
Tree Theorem.\\
 
This and det. formulas
easily prove Rayleigh inequalities.
\end{itemize}

\end{minipage}
\begin{minipage}{0.48\textwidth}

\begin{itemize}
\item
(Part 2) Combine with:
\begin{tabular}{c}
$\text{Solution Space}$\\
$=$\\
$\text{Closure}(\text{Set of Generators})$\\
\end{tabular}
\item
To apply: An oriented matroid's
COVECTOR SET encodes ALL POSSIBLE
 $(+,-,0)$ coordinate behaviors or
$\delta$s.
\item
\Remph{Result:} 
An oriented matroid pair model
for some non-linear problem
\Remph{(AMPLIFIER!)} well-posedness. 
(How? Sign contradictions $\Rightarrow$
a KERNEL=$\{(0)\}$.)
\end{itemize}
\end{minipage}
\end{frame}

\begin{frame}
{Part 1) Use Matrix $M$ in CONSTRAINTS $MX=0$ to get...}
%\begin{center}
The Tutte-like function $\mathbf{M}_E():\text{Extensor }
\mathbf{N}\rightarrow\text{Extensor }\mathbf{M_E(N)}$.
%\end{center}

{\small
(\Remph{STUDENT NOTE:} An EXTENSOR represents the row-space of an $r\times s$
$r-$rank matrix $M$ by the $\binom{s}{r}$-TUPLE of the DETERMINANTS of 
$M$'s $r\times r$ submatrices. \Remph{Pl\"{u}cker coords}.)
}

Given $N$ (matrix), construct $N^\perp$ 
with orthog. comp. row space.

Construct:  ($G=\mbox{diag}(g_e)$, $R=\mbox{diag}(r_e)$)
\[
M = \left[\begin{array}{c|c|c} N(P)  &  0  &  N(E)G \\  \hline
0  & N^{\perp}(P)  &  N^{\perp}(E)R \end{array}\right]
\]
with columns labelled by $P_I\dunion P_V\dunion E$.

Extensor $\mathbf{M}$ over $k[g_e, r_e](P_V\dunion P_I \dunion E)$
is the \Remph{$\wedge$-product} of $M$'s \textbf{row vectors}. The contraction result
$\mathbf{M}_E(\mathbf{N}) = \mathbf{M}/E$ appears:
\[
\mathbf{M} = \mathbf{M}_E(\mathbf{N})\mathbf{e_1}\mathbf{e_2}\cdots\mathbf{e}_{|E|} + (\cdots) 
\]

\Remph{
$\mathbf{M}_E(\mathbf{N})$ is our Tutte function $\mathbf{N}\rightarrow \text{Ext. Alg.}$}
\end{frame}


\begin{frame}{Contracting means ``Eliminate variables''}
ELIMINATE the variables indexed by $E$, leaving $2|P|$ variables
labelled by $P_I$ and $P_V$.  ie, CONTRACT $E$. \textbf{Answer} 
$\mathbf{M_E}$ IS:

\[
\mathbf{M}_E = \bigwedge^{\text{Exterior}}_{\text{JOIN over rows}} \left[\begin{array}{c|c} A_{I,I}  &  A_{I,V}   \\  \hline
    A_{V_I}  & A_{V,V} \end{array}\right] 
[\mathbf{p_{I_1}, \cdots, p_{I_p}; p_{V_1}, \cdots, p_{V_p}}]^{\mathbf{t}} 
\]

\[
 = \ldots + C_i\mbox{\bf XXX} + \ldots; \text{Equiv. Resistance} = 
\text{certain}\;\; C_i/C_j
\]

All the other $C_k$'s have similar interpretations.

{\bf $\binom{2|P|}{|P|}$ Matr. Tree Theorems:}
Each $C_k(N)$ (a PRINCIPAL MINOR of MATRIX \textbf{\Remph{$A$}} ABOVE!)
$= 
g_e C_k(N/e) + r_e C_k(N\backslash e)$ ($e\not\in P$, $e$ not (co)loop).

Each $C_k$ is a signed weighted enumerator of
forests satisfying \textbf{conditions ...}
\end{frame}


%%%%%%%%%%%%%%%%%%%%%%%%%%%%%%%%%%%%%%%%%%%%%%%%%
\begin{frame}{Conditions (what sets $F$ are enumerated by one det. $C_i$)
}
The \textbf{conditions ...}
are on the rank, nullity of $F$ and, WHAT ORIENTED MINOR is \
$G/F\setminus (E\setminus F)$, the minor
with ONLY PORT EDGES from contracting $F$
and deleting the other resistor edges, leaving the
ports.

The conditions for a given $C_k$ \textit{sometimes}
make all the signs the same (eg: $C_i$ and 
$C_j$ in 1-port equivalent resistance $R=C_i/C_j$)

\textit{Othertimes}, the oriented \textbf{P-minors}
in the completed Tutte decomposition of $C_k$ determine
some + and some - signs.

\begin{center}
\begin{minipage}{0.3\textwidth}
\begin{tabular}{c}
When $[G/F|P]$ is \\
\input{c2plus.pspdftex} \\
the term is \\
\Remph{{\LARGE\bf +}}$g_Fr_{E\setminus F}$ \\
\end{tabular}
\end{minipage}
\begin{minipage}{0.3\textwidth}
\begin{tabular}{c}
When $[G/F|P]$ is\\
\input{c2minus.pspdftex}\\
the term is\\
\Remph{{\LARGE\bf -}}$g_Fr_{E\setminus F}$\\
\end{tabular}
\end{minipage}
\end{center}

\end{frame}

%%%%%%%%%%%%%%%%%%%%%%%%%%%%%%%%%%%%%%%%%%%%%%%%%
\begin{frame}
\frametitle{Application: Rayleigh Identity, ``Neg. Spanning Tree Correlation''}
\[
\Gamma_e(G)\text{ is equivalent conductance across }e.
\text{ Rayleigh: }0 \le \frac{\partial \Gamma_{p}}{\partial g_f} =
\frac{\partial \frac{T_G}{T_{G/e}}}{\partial g_f}
\]
is equivalent to 
\[
0 \le \frac{\partial T_G}{\partial g_f}T_{G/e} - 
       T_G\frac{\partial T_{G/e}}{\partial g_f} 
=
T_{G/f}T_{G/e} - T_GT_{G/e/f}
\]
In fact,
\[
T_{G/f}T_{G/e} - T_GT_{G/e/f} = \left( T^+_{G/e \text{ \& } G/f} - T^-_{G/e \text{ \& } G/f} \right)^2
\]
$T^{\pm}_{G/e \text{ \& } G/f}$ enumerate the $\pm$ common spanning trees.
\end{frame}

\begin{frame}
\frametitle{Known Partial and Full Combinatorial Proofs}
\[
T_{G/f}T_{G/e} - T_GT_{G/e/f} = \left( T^+_{G/e \text{ \& } G/f} - T^-_{G/e \text{ \& } G/f} \right)^2
\]
$T^{\pm}_{G/e \text{ \& } G/f}$ enumerate the $\pm$ common spanning trees.

\vfill
Choe (2004) 
proved essentially this using the vertex-based all-minors matrix tree theorem,
combinatorial cases and Jacobi's theorem relating the minors of a matrix to
the minors of its inverse..

\vfill
Cibulka, Hladky, Lacroix and Wagner (2008) gave a completely bijective proof
that utilizes some natural 2:2 and 2:1 correspondances.

\vfill
\Remph{Difficulty:} Some terms on the left \Remph{cancel} and some
reduce to terms with coefficients $\pm 2$.
\vfill
\end{frame}



%%%%%%%%%%%%%%%%%%%%%%%%%%%%%%%%%%%%%%%%%%%%%%%%%
\begin{frame}{``Colors'' are parameters on every Tutte decomposition step}

The 
Bollobos/Riordan/Zaslavsky\cite{BollobasRiordanTuttePolyColored,MR93a:05047}, 
Traldi-Ellis-Monaghan\cite{DichWeighedGraphsTraldi89,Ellis-Monaghan-Traldi}, 
(sdc unpub)
BRZ theory for well-definedness
of ``Relative Tutte Polynomials for Colored Graphs'' ALL GOES THROUGH 
(Diao and Hetyei \cite{RelTuttePoly}):
The 3 BRZ conditions on (colors,initial values) GENERALIZE TO 5;
activity theory WORKS TOO, when
based on linear orders on the non-port-elements.

\begin{block}{In a nutshell}
The 5 conditions $\Longrightarrow$ activities define an 
unambiguous Tutte function 
from the deletion/contraction and initial value formulas.\\

Additional conditions $\Longrightarrow$ the Tutte function has a rank-nullity
expansion.\\

\Remph{
(The rank-nullity conditions are satisfied in our application.)}
\end{block}

\begin{block}{To specify the activity/deletion-contraction linear 
order GLOBALLY is 
UNNECESSARY.}
The Gordon/McMahon computation-tree-based 
activity theory also generalizes. (sdc).
\end{block}
\end{frame}


%%%%%%%%%%%%%%%%%%%%%%%%%%%%%%%%%%%%%%%%%%%%%%%%%
\begin{frame}[allowframebreaks]{References}
\bibliographystyle{plain}
\bibliography{../../bib/MathOfElec}{}
\end{frame}


%%%%%%%%%%%%%%%%%%%%%%%%%%%%%%%%%%%%%%%%%%%%%%%%%%%%
%spare slides

\end{document}
%%%%%%%%%%%%%%%%%%%%%%%%%%%%%%%%%%%%%%%%%%%%%%%%%
\begin{frame}{(15)}

\end{frame}

%%%%%%%%%%%%%%%%%%%%%%%%%%%%%%%%%%%%%%%%%%%%%%%%%
\begin{frame}{(16)}
\end{frame}

%%%%%%%%%%%%%%%%%%%%%%%%%%%%%%%%%%%%%%%%%%%%%%%%%
\begin{frame}{(17)}


Our Tutte-like function $\mathbf{M}_E(\mathbf{N}):\text{Extensors}\rightarrow\text{Extensors}$.

Given $N$ (matrix), construct $N^\perp$ 
with orthog. comp. row space.

Construct:  ($G=\mbox{diag}(g_e)$, $R=\mbox{diag}(r_e)$)
\[
M = \left[\begin{array}{c|c|c} N(P)  &  0  &  N(E)G \\  \hline
0  & N^{\perp}(P)  &  N^{\perp}(E)R \end{array}\right]
\]
with columns labelled by $P_I\dunion P_V\dunion E$.

Extensor $\mathbf{M}$ over $k[g_e, r_e](P_V\dunion P_I \dunion E)$
is the product of $M$'s \textbf{row vectors}. The contraction result
$\mathbf{M}_E(\mathbf{N}) = \mathbf{M}/E$ appears:
\[
\mathbf{M} = \mathbf{M}_E(\mathbf{N})\mathbf{e_1}\mathbf{e_2}\cdots\mathbf{e}_{|E|} + (\cdots) 
\]

$\mathbf{M}_E(\mathbf{N})$ is our Tutte function $\mathbf{N}\rightarrow \text{Ext. Alg.}$

\end{frame}



%%%%%%%%%%%%%%%%%%%%%%%%%%%%%%%%%%%%%%%%%%%%%%%%%
\begin{frame}{What are the conditions like?}

\[
\text{Extensor\ }= \sigma(C)(a_1\vee a_2\vee \ldots \vee a_d)
\]
\[
C = \ldots + t + \ldots
\]
$t \leftrightarrow \text{\ sets of (``contractible'' non-port edges} 
E_1, E_2$ for which
\[
N_V/E_1\backslash(E\setminus E_1) = \text{\ certain OMs on\ } P
\]
AND
\[
N_I/E_2\backslash(E\setminus E_2) = \text{\ certain OMs on\ } P
\]

Transfer resistance might be 0 and might be $\neq 0$ iff
\[
\exists E_1, E_2 \text{so (diag) and (diag)}
\]

($K_4$ Wheatstone bridge diagram)

\[
\frac{R_1}{R_2} > \frac{R_3}{R_4} 
\text{\ neg\ }
\frac{R_1}{R_2} < \frac{R_3}{R_4} 
\text{\ pos\ }
\]

\end{frame}


\begin{frame}
\frametitle{Equivalent resistance is a coefficient ratio in an 
implicitly defined linear function}

(diagram)

In other words
\[
R_pi_p + v_p = 0
\]
or dually,
\[
\mathcal{B} = \{(i_p,v_p)\}
= \{ t(-1, R_p) | t\in R\}
\]

\end{frame}

\note{nonframe(linear subspace over the $R^{\mathbf{P}}$)}

%%%%%%%%%%%%%%%%%%%%%%%%%%%%%%%%%%%%%%%%%%%%%%%%%
\begin{frame}
\frametitle{The $2\times d$ port variable constraint space, 
and its solution spaces, are $d$-dimensional.}

We represent these spaces by carefully defined
\textbf{extensors}, as Barnabei, Brini and Rota \cite{exteriorCalc}
term ``decomposible antisymmetric tensors''

The solution extensor (not a ray) satisfies:

\[
E(N) = \text{sign}(...)(g_e E(N/e) + r_e E(N\backslash e)
\text{\ for\ }e\not\in P
\]

\end{frame}


\note{
(nonframe) 
The port elements represent the basis chosen so coordinates correspond
to voltages and currents belonging to network edges.  Geometric aspects
of extensors are surveyed by 
Barnabei and Brini \cite{exteriorCalc}.

Cordovil\cite{CommAlgOMs} 
presented a \textbf{commutative algebraic} function of 
oriented matroids in which monomial signs encode orientation information.
The analogy to our \textbf{exterior algebraic} functions must be explored. 
}

%%%%%%%%%%%%%%%%%%%%%%%%%%%%%%%%%%%%%%%%%%%%%%%%%
\begin{frame}{Coefs}
\[
E(N) = \ldots + C_i\mbox{\bf XXX} + \ldots
\]
\[
R = C_i/C_j
\]
All the other $C_k$'s have similar interpretations.

Each $C_k$ is a determinant.

Each $C_k$ is a signed weighted enumerator of
forests satisfying \textbf{conditions ...}

Each $C_k$ satisfies
\[
C_k(N) = g_e C_k(N/e) + r_e C_k(N\backslash e)
\text{\ for\ }e\not\in {P}
\]

\end{frame}

%%%%%%%%%%%%%%%%%%%%%%%%%%%%%%%%%%%%%%%%%%%%%%%%%
\begin{frame}{Conditions}
What is the nature of the conditions?  We state them using the 
network's graphic oriented matroid.

(diagram--glob w/ ports)

\end{frame}





\note{
nonframe \textbf{port} (sdc, engineering literature), 
\textbf{distinguished element}, 
element of the restriction's set (Diao and Hetyei), 
a element in the set of points (Las Vergnas).
}










\end{document}


\begin{theorem}
Weighted Tree Sum (WTS) is a colored Tutte function:
\[
\mbox{WTS}(G') =
g_e \mbox{WTS}(G'/e) + r_e \mbox{WTS} (G' \backslash e)
\text{\ for all\ }e \not\in P
\]
\[
\mbox{WTS}(\text{coloop}(e)) = g_e
\]
\[
\mbox{WTS}(\text{loop}(e)) = r_e
\]
\end{theorem}
