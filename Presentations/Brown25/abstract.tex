\documentclass{article}
\begin{document}
Let $N$ be a linear representation (i.e, matrix) of a matroid whose ground
set $S$ includes a finite, distinguished subset $P$.  We give
function $L(N)$ that, unlike what we know of other Tutte functions and
work like the Hopf algebra variants
of
Krajewski, Moffatt and Tanasa,
has values
in an \emph{anti-commutative} algebra. Let deletion and contraction be limited
to $e\not\in P$.  Then, the values are in the exterior algebra generated
by $P_\alpha \coprod P_\beta$.  The construction relies on concrete minor operations
to establish consistent signs of the constituent terms so that, with suitable
accounting for sequential orderings of set elements, $L(N)=L(N\setminus e)+L(N/e)$
in the exterior algebra. Our construction is derived from the structure of
the equilibrium equations for linear electrical networks, and of their generalization
to multi-dimensional elastic frameworks.  Further, the construction does not
require orthogonality for the spaces that generalize spaces of feasible currents and voltages,
or of forces and displacements.  Hence $L$ will be defined on equal rank pairs $(N_\alpha, N_\beta)$
(where originally, $N_\alpha=N_\beta=N$).  We take the Tutte identities those for Welsh and Kayibi's
linking polynomial of matroid pairs. With $N_\alpha\neq N_\beta$, we can derive the digraph
all-minors matrix tree theorem by taking $P$ to be the set of vertices.  We so
get ratio of common basis expansion solutions for
linear electrical and other linear systems with multi-terminal amplifiers (where a voltage or force
at one place is a multiple of current or displacement at a different place).  To incorporate
resistance ($r_e$), conductance ($g_e$), elasticity coefficient, etc. parameters, we use parametrized
Tutte function theory
for which $L(N)=r_e L(N\setminus e)+ g_e L(N/e)$; the
term for common basis $B$ includes
$\prod _{e\in B} g_e \prod _{e\not\in B\cup P} r_e$.

\end{document}
