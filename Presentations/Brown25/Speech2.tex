\documentclass[14pt]{extarticle}
\usepackage{amsmath}
\setlength{\textwidth}{7.5in}
\setlength{\hoffset}{-1.0in}
\setlength{\voffset}{-1.2in}
\setlength{\textheight}{9.8in}

\usepackage{enumitem}
\setlist{itemsep=3pt}
\renewcommand{\baselinestretch}{0.95}

\begin{document}



{\bf
  \begin{itemize}

  \item The math at the bottom reviews how the Cauchy-Binet theorem proves a graph's Laplacian
    determinant counts spanning trees.  We present a broad linear algebra framework for this. 

  \item I see in Connelly and Guest's new book that
    in first order stiffness theory, ... there are matrices like the Laplacian, ...
      and Poisson and Dirichlet type problems.

  \item I'd be very happy to meet collaborators to help go from my electrical network
    thinking with its one-dimensional voltages and currents to the higher dimensional
    world of this workshop, and see what problems are analogous to electrical ones.

  \item Briefly,  (1) solution matrices can be encoded in pure exterior algebra elements, (2) As such,
    they package Plucker coordinate values, and (3) the solution is a Tutte function of a class
    of problems.

  \item
    Entries in solution matrices for nice electrical network analyses ... and also larger minors
    in those matrices ... turn out to be ratios of those Plucker coordinates. 

    \item I conjecture that's true and might be interesting for stiffness problems.
    
  \item Now some details.
      
  \item We start with two matrices, $N$-Alpha and Beta.  Their matroids
    have for their elements a mixture of hatted and not-hatted, ... and $\mathbf{p}$ \& $\mathbf{e}$ type
    symbols.
   For now, think of the row spaces.

  \item We make two pure exterior algebra elements,  boldface
    $\mathbf{N}$-Alpha and Beta to represent the row spaces.  From now on, every pure, that is,
    indecomposible element, 
    a product of vectors and or dual vectors, we will call an ``extensor''.  


    \item Only extensors represent subspaces, that is, points in Grassmannians.
      They have Plucker coodinates which satisfy the Grassmann-Plucker relations.

      \item
        Plucker coordinates are proportional to the maximal minors of
        full-row rank matrices whose rows generate the subspace.  So ... non-zero Plucker coordinates encode
    which subsets are matroid bases.  

  \item We construct function $L$ of those
    $\mathbf{N}$-Alpha and Beta so it has extensor values.

\item Result 1 is that $L$ obeys Tutte's deletion and contraction identity:  BUT
  only for $e$ type elements, not the $p$'s.  

\item There is also a direct sum identity ... $L$ of a direct sum is an exterior product, not
  a commutative ring product.

\item We have TO CAREFULLY DEFINE the extensor operations for deletion \&
  contraction \& direct sum so the signs in the $L$--s we combine are consistent.

\item
  The boldface $\mathbf{N}$--s are the exterior products of the row sums
  after plugging and multiplying next to each entry the label of the column label it belongs to.
  
\item
  Those exterior products
   represent linear subspaces of a big space generated
  by matroid ground set elements and their hatted versions.  

\item Since the subspaces are given with a special basis of either a hatted or unhatted version
  of each matroid element, we have well-defined vector and dual vector basis elements
  of the big linear space.

\item
  Finally $L$ equals the bilinear pairing or composition on these two.  The status
  of vector versus dual vector is used in defining this.

\item
  I find it interesting that, ... in order to get a Tutte function out
  of this, it seems require two special things.

\item
One is to make the Tutte function relative, ... to the set of ports $P$.

\item
  We need this because the $\mathbf{e}$---s  disappear when they are contracted or deleted.

\item
  I like to call the distinguised elements ports. ... We never delete or contract them.

%\item
%  Those port elements in the exterior algebra are vectors, and the hatted
%  ones are covectors, ie., 1-forms or dual vectors.

\item In applications, ports relate to variables used to specify inputs
  or parameters, ... like how much you push on nodes with forces or electric currents,
  and responses, ... like how the nodes move, or change their voltages.

\item But this setup, you can interchange inputs and responses ANY WAY that the
  relevant matroids encode is feasible and well-posed.  Electrical engineers would
  call $L$ a MULTIPORT LINEAR DEVICE model.

  %The cascade form of two ports has at the first port both the voltage and current for independent, input
  %variables and at the second port, both the voltage and current are response variables. (Not for a
  %lighting talk, just thinking.)

\item
  Two, it seems this extensor Tutte function needs to be constructed on 
  two arguments, labelled $\alpha$ \& $\beta$.

%\item
%  We recover the basis enumerator when the two are equal and $P$ is empty.  It
  %  is the sum of squared determinants though, not always ones.
  % (This idea is subsumed by the new introduction)

\item Now for the final step.

\item
  We define the bilinear pairing function 
  after distinguishing  
  four kinds of generators:\\
  vector $\mathbf{e}$'s, vector $\mathbf{p}$'s ... and
dual vector $\widehat{\mathbf{e}}$--hats and $\widehat{\mathbf{p}}$--hats.  

\item
  The pairing function is defined \emph{WITH} rules on the slide for basis monomials.  Dual vectors
  in the left evaluate on vectors in the right.   But ... when you reverse that ... 
  they behave like anticommutative coefficients.

\item
  The $\mathbf{N}$--s mix vectors and dual vectors ... they represent linear mappings.
  Between Alpha and Bets, the hatted and unhatted status of $p$--s and $e$--s are interchanged.
  $\mathbf{N}_\beta$ is like the adjoint of $\mathbf{N_\alpha}$.  So ... the
  bilinear pairing is composition of mappings  ... like matrix multiplication.

\item
  Result 2 is a Cauchy-Binet kind of expansion. The bracketed minors denote sums of terms
  because they are extensor representations of minors without any non-port elements.
  The notation hides common basis expansions of many different minors our two matroids.

%\item
%  In my old writeup, I start with $\mathbf{N}$-Alpha, exterior multiply it by the DUAL of
%  $\mathbf{N}$-Beta,
%  and finally contract all the $e$--s.  A non-zero exterior product is disjoint subspace join, and
%  contraction is projection. I call that the Laplace form.  

\item
  I finish with some take-homes and morals, and my name, Seth Chaiken
  of Albany, NY.

  \item
  Three punchlines:  One, ports are IM-PORT-ANT.
  Two, let's do matroid recursion on matrices in exterior algebra.
  Three, we find a Tutte function there.
  Thank you!
  \end{itemize}
}

\end{document}


