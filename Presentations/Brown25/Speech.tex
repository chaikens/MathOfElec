\documentclass[14pt]{extarticle}
\usepackage{amsmath}
\setlength{\textwidth}{7.5in}
\setlength{\hoffset}{-1.0in}
\setlength{\voffset}{-1.2in}
\setlength{\textheight}{9.8in}

\usepackage{enumitem}
\setlist{itemsep=3pt}
\renewcommand{\baselinestretch}{0.95}

\begin{document}

{\bf
  \begin{itemize}
  \item We start with two matrices, $N$-Alpha and Beta.  The matroids
    they represent have the $\mathbf{p}$ \& $\mathbf{e}$ symbols, each
    exclusively hatted or
    not hatted, for their ground set elements. ...  For now, think of the
    row spaces.

  \item We make two exterior algebra elements,  boldface
    $\mathbf{N}$-Alpha and Beta to represent the row spaces.

  \item We construct function $L$ of those
    $\mathbf{N}$-Alpha and Beta, ... by a bilinear pairing 
    that performs composition ... 
  $\mathbf{N}\text{-Alpha} \text{\;\;bar\;\;} \mathbf{N}\text{-Beta}$.

  \item This bilinear operation has exterior algebra,
    not field or commutative ring values. 

\item Result 1 is that $L$ obeys Tutte's deletion and contraction identity:  BUT
  only for $e$ type elements, not the $p$'s.  

\item There is also a direct sum identity ... $L$ of a direct sum is an exterior product, not
  a commutative ring product.

\item We have TO CAREFULLY DEFINE the exterior algebra operations for deletion \&
  contraction \& direct sum so the signs in the $L$--s we combine are consistent.

\item
  As pure, ... or indecomposable anti-symmetric tensors, ... that is 
  products of vectors, the boldface
  $\mathbf{N}$--s represent linear subspaces of a big space generated
  by matroid ground set elements and their hatted versions.  

\item  So with these designated bases, related to ground sets, we get duals of basis vectors.
  I'll say how the duals are used after relating these things to matroids.

\item
  $\mathbf{N}$-Alpha \& Beta represent points in the
  Grassmannian, and have Plucker coordinates.
  The Plucker coordinates are the maximal minors of the matrices.
  So, the matroid bases are encoded by which Plucker coordinates are non-zero.

\item
To construct an extensor from a matrix, I multiply each column's boldface symbol with its entries.
Boldface $\mathbf{N}$ is the exterior product of the row sums.

\item
  Finally $L$ equals the bilinear pairing which expresses an linear endomorphism
  of the exterior algebra generated by the p--s.

\item
  I find it interesting that, ... in order to get a Tutte function out
  of this, it seems require two special things.

\item
One is to make the Tutte function relative, ... to the set of ports $P$.

\item
  We need them because the $\mathbf{e}$---s  disappear when they are contracted or deleted.

\item
  I like to call the distinguised elements ports. ... We never delete or contract them.

%\item
%  Those port elements in the exterior algebra are vectors, and the hatted
%  ones are covectors, ie., 1-forms or dual vectors.

\item In applications, ports relate to variables used to specify inputs
  or parameters, ... like how much you push on nodes with forces or electric currents,
  and responses, ... like how the nodes move, or change their voltages.

\item But this setup, you can interchange inputs and responses ANY WAY that the
  relevant matroids encode is feasible and well-posed.  Electrical engineers would
  call $L$ a MULTIPORT LINEAR DEVICE model.

  %The cascade form of two ports has at the first port both the voltage and current for independent, input
  %variables and at the second port, both the voltage and current are response variables. (Not for a
  %lighting talk, just thinking.)

\item
  Two, it seems this exterior algebra Tutte function needs to be constructed on 
  two arguments, labelled $\alpha$ \& $\beta$.

\item
  We recover the basis enumerator when the two are equal and $P$ is empty.  It
  is the sum of squared determinants though, not always ones.

\item Now for the final step.

\item
  To define the final bilinear pairing function 
  we distinguish 
  four kinds of generators:\\
  vector $\mathbf{e}$'s, vector $\mathbf{p}$'s and
dual vector $\widehat{\mathbf{e}}$ hats and $\widehat{\mathbf{p}}$ hats.  

\item
  It's defined here \emph{WITH} these rules for algebra basis monomials.  Dual vectors
  in the left evaluate on vectors in the right, but reverse that ... and
  they behave like anticommutative coefficients.

\item
  The $\mathbf{N}$--s mix vectors and dual vectors ... they represent linear mappings.
  The hatted and unhatted status of $p$--s and $e$--s are interchanged.
  $\mathbf{N}_\beta$ is like the adjoint of $\mathbf{N_\alpha}$.  So ... the
  bilinear pairing is composition of mappings  ... like matrix multiplication.

\item
  Therefore ... I call this the Cauchy-Binet form.  We get result 2.  The common
  independent set expansion here hides many common basis expansions.  One of them
  is the famous matrix tree theorem.

\item
  I finish with some take-homes and morals, and my name, Seth Chaiken
  of Albany, NY.
  Three punchlines:  One, ports are IM-PORT-ANT.
  Two, let's do matroid recursion on matrices in exterior algebra.
  Three, we find a Tutte function there.
  Thank you!
  \end{itemize}
}

\end{document}


