\documentclass[14pt]{extarticle}
\usepackage{amsmath}
\setlength{\textwidth}{7.0in}
\setlength{\hoffset}{-0.6in}
\setlength{\voffset}{-1.2in}
\setlength{\textheight}{9.8in}

\usepackage{enumitem}
\setlist{itemsep=3pt}
\renewcommand{\baselinestretch}{0.95}

\begin{document}

{\bf
  \begin{itemize}
  \item Here are $N_\alpha$ and $\beta$, the 2 matrices we start with.
    They represent two matroids.  The \textbf{p}---s \&
    \textbf{e}---s, unhatted and hatted forms, are ground set elements.
    But for now, think of the row spaces.

  \item From each matrix we make an exterior algebra element, denoted in boldface,
    to represent its row space.

  \item We construct a Tutte Function $L$ of those two elements by a
    kind of bilinear pairing ... 
  $\mathbf{N_\alpha} \text{\;\;bar\;\;} \mathbf{N_\beta}$.

\item But this bilinear function has exterior algebra, not field or commutative ring values. 

\item Result 1 is that $L$, as a function of $\mathbf{N_\alpha}$ \&
  $\mathbf{N_\beta}$ obeys Tutte's deletion and contraction identity:  BUT
  only for $e$ type elements, not the $p$'s.

\item I omit the direct sum identity, but its product is exterior, not
  commutative ring product.

\item We have TO CAREFULLY DEFINE the exterior algebra operations for deletion \&
  contraction \& direct sum so the signs in $L$'s expansion are consistant.

\item Here's some more background.
  
\item
  As pure, ... or indecomposable anti-symmetric tensors, ... that is 
  products of vectors, the boldface
  $\mathbf{N}$--s represent linear subspaces of the big space generated
  by matroid ground set elements and their hatted copies.  

\item  Having a basis related to ground set elements, we will in a moment
  consider the unhatted and hatted versions to be vector duals.

\item
  So, $\mathbf{N_\alpha}$ \& $\mathbf{N_\beta}$ represent points in the
  Grassmannian, and have Plucker coordinates.

\item
  The Plucker coordinates are the maximal minors of the matrices.

\item
So, the matroid bases are encoded by which Plucker coordinates are non-zero.

\item
I multiply each column's boldface symbol with its entries.

\item
  What are the boldface $\mathbf{N}$s?
  They are the exterior products of the row sums.

\item
  Finally $L$ equals the bilinear pairing on these two.

\item
  I find it interesting that, ... in order to get a Tutte function out
  of this, it seems require two special things.

\item
One is to make the Tutte function relative, ... to the set of ports $P$.

\item
  We need that in order to get a non-trivial exterior algebra
  value because the $\mathbf{e}$---s  disappear when they are contracted or deleted.

\item
  I like to call the distinguised elements ports. ... We never delete or contract them.

%\item
%  Those port elements in the exterior algebra are vectors, and the hatted
%  ones are covectors, ie., 1-forms or dual vectors.

\item In applications, ports relate to variables used to specify inputs
  or parameters, ... like how much you push on a node with force or an electric current,
  and responses, ... like how much every node moves, or changes its voltage.

\item
  Two, it seems this exterior algebra Tutte function needs to be constructed using
  two arguments, labelled $\alpha$ \& $\beta$.

\item
  We recover the basis enumerator when the two are equal and $P$ is empty.  It
  is the sum of squared determinants though, not always ones.

\item Now for the final step.

\item
  To define the final bilinear pairing function 
  we distinguish 
  four kinds of generators:\\
  vector $\mathbf{e}$'s, vector $\mathbf{p}$'s and
dual vector $\widehat{\mathbf{e}}$ hats and $\widehat{\mathbf{p}}$ hats.  

\item
  It's defined here \emph{WITH} these rules for algebra basis monomials.  Covectors
  in the left evaluate on vectors in the right, but reverse that ... and
  they behave like anticommutative scalars.

\item
  I found this very recently, ... an algebra valued, not scalar valued bilinear
  function on tensors or their relatives.  I'd appreciate any pointers.
  I also must ask some physicists and try the old subscipts approach.

\item
I call $L$-'-s construction the Cauchy-Binet form because of the expansion in result 2.

\item
I used to define $L$ from $\mathbf{N_\alpha}$, and the DUAL of $\mathbf{N_\beta}$.

\item
I took their exterior product and then contracted all of $E$...I call it the
Laplace form.

\item
  I finish with some take-homes and morals, and my name, Seth Chaiken
  of Albany, NY, which
  I'll leave for you to read at your own pace.  Thank you!
  \end{itemize}
}

\end{document}


