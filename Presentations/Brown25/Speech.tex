\documentclass{article}
\setlength{\textwidth}{7in}
\setlength{\hoffset}{-1in}

\begin{document}
\begin{itemize}
\item
To talk about what kind of Tutte function $L$ is, I define the
objects boldface $\mathbf{N_\alpha}$ and $\mathbf{N_\beta}$.

\item
They are exterior algebra elements.

\item
They represent linear subspaces over the basis comprised of a matroid's ground set.

\item
So, they represent points in the Grassmannian, and have Plucker coordinates.

\item
The matroid bases are encoded by which Plucker coordinates are non-zero.

\item
Here's how I get boldface $\mathbf{N_\alpha}$ and $\mathbf{N_\beta}$ from matrices.

\item
For each column, there's a $\mathbf{p}$ or $\mathbf{e}$ symbol or a hatted one.

\item
I multiply the column's symbol with each numeric entry.

\item
Each $\mathbf{N_\alpha}$ and $\mathbf{N_\beta}$ is the exterior product of the row sums.

\item
Finally, to get $L$, I evaluate a kind of bilinear pairing on $\mathbf{N_\alpha}$ and
$\mathbf{N_\beta}$.

\item
What's interesting is that in order to get a non-scalar exterior algebra value, we
seem to have to do two special things.

\item
One is to make the Tutte function relative to set $P$.

\item
We fix distinguished set $P$ which I call ports, and we never delete or contract ports.

\item
$P$ seems to be needed because when we do the full Tutte decomposition, each irreducible
is in the exterior algebra generated by the $\mathbf{p_\alpha}$s and the
$\widehat{\mathbf{P_\beta}}$ hats.

\item
Two, it seems the construction needs a pair, $\mathbf{N_\alpha}$ and $\mathbf{N_\beta}$.

\item
We recover to basis enumerating Tutte function by making them equal, and $P$ empty.

\item
To construct $L$, I use a kind of bilinear pairing of exterior algebras with
four kinds of generators, vector $\mathbf{e}$'s and $\mathbf{p}$'s and
1-form or covector $\widehat{\mathbf{e}}$ and $\widehat{\mathbf{p}}$ hats.

\item
The pairing is defined \emph{WITH} the rules for basis monomials on the slide.

\item
I found this very recently, so I'd appreciate any pointers if anyone recognizes something
like it.

\item
I call the above construction of $L$ the Cauchy-Binet form because of the
expansion in result 2.

\item
A while ago I defined $L$ from $\mathbf{N_\alpha}$ and the dual of $\mathbf{N_\beta}$.

\item
I took their exterior product and then contracted all of $E$.  I call this the
Laplace expansion form.
\end{itemize}
\end{document}

