\documentclass[14pt]{extarticle}
\usepackage{amsmath}
\setlength{\textwidth}{7.0in}
\setlength{\hoffset}{-0.6in}
\setlength{\voffset}{-1.2in}
\setlength{\textheight}{9.5in}

\usepackage{enumitem}
\setlist{itemsep=3pt}
%\renewcommand{\baselinestretch}{0.90}

\begin{document}

{\bf
  \begin{itemize}
  \item Here $N_\alpha$ and $\beta$, the 2 matrices we start with.
    They represent two matroids.  The \textbf{p}---s \&
    \textbf{e}---s are ground set elements.  But think of the row spaces for now.

\item We make two exterior algebra elements to represent those row spaces:
   boldface $\mathbf{N_\alpha}$ \& $\mathbf{N_\beta}$.

\item We construct a Tutte Function $L$ of those two things by a kind of bilinear pairing
  $\mathbf{N_\alpha} \text{\;\;bar\;\;} \mathbf{N_\beta}$:

\item But this bilinear function has exterior algebraic, not field or ring values. 

\item Result 1 is that $L$, as a function of pair $\mathbf{N_\alpha}$ \&
  $\mathbf{N_\beta}$ obeys result ONE, Tutte's deletion and contraction identity:  BUT
  only for $e$ type elements, not the $p$'s.

\item I omit the direct sum identity, but its product is exterior, not
  commutative ring product.

\item We have TO CAREFULLY DEFINE the exterior algebra operations for deletion \&
  contraction \& direct sum so the signs in $L$'s expansion are consistant.

\item Here's some more background.
  
\item
  As pure, ... or indecomposabale anti-symmetric tensors, ... that is 
  products of vectors, the boldface
  $\mathbf{N}$s represent linear subspaces over the
  basis comprised of the matroids' ground sets.

\item The ground elements correspond to the boldface $\mathbf{p}$ \&
  $\mathbf{e}$ vectors, ... or dual vectors when they have hats.

\item
  So, $\mathbf{N_\alpha}$ \& $\mathbf{N_\beta}$ represent points in the
  Grassmannian, and have Plucker coordinates.

\item
  The Plucker coordinates are the maximal minors of the matrices.

\item
So, the matroid bases are encoded by which Plucker coordinates are non-zero.

\item
I multiply a column's boldface symbol with every numeric entry.

\item
  What are the boldface $\mathbf{N}$s?
  They are the exterior products of the row sums.

\item
  Finally to get $L$ I use the bilinear pairing on the two $\mathbf{N}$s.

\item
  I find it interesting that, ... in order to get a Tutte function out
  of this, it seems require two special things.

\item
One is to make the Tutte function relative, ... to the set of ports $P$.

\item
  We need that in order to get a non-trivial exterior algebra
  for the Tutte function values to live in.

\item The $\mathbf{e}$---s  disappear when they are contracted or deleted.

\item
  I like to call the distinguised elements ports. ... We never delete or contract them.

\item
  Those port elements in the exterior algebra are vectors, and the hatted
  ones are covectors, ie., 1-forms or dual vectors.

\item In applications, ports relate to variables used to specify inputs
  or parameters, ... like how much you push on a node with force or an electric current,
  and responses, ... like how much every node moves, or change its voltage.

\item
  Two, it seems the Tutte function needs pairs,
  $\mathbf{N_\alpha}$ ... $\mathbf{N_\beta}$.

\item
  We recover the basis enumerator with $P$ empty and
 $\mathbf{N_\alpha}$ ... $\mathbf{N_\beta}$.

\item
  The bilinear pairing in the final step to construct $L$ ... 
  is defined after distinguishing 
  four kinds of generators:\\
  vector $\mathbf{e}$'s, vector $\mathbf{p}$'s and
dual vector $\widehat{\mathbf{e}}$ hats and $\widehat{\mathbf{p}}$ hats.  

\item
It's defined on the slide \emph{WITH} these rules for algebra basis monomials.

\item
  I found this very recently, ... a pairing that should return algebra values
  instead of ring values.  I'd appreciate pointers if anyone recognizes
  something like it.

\item
I call $L$'---s construction the Cauchy-Binet form because of the expansion in result 2.

\item
A while ago I defined $L$ from $\mathbf{N_\alpha}$, and the DUAL of $\mathbf{N_\beta}$.

\item
I took their exterior product and then contracted all of $E$...and called it the
Laplace form.

\item
  I'll finish with some take-homes and morals, and my name, Seth Chaiken, which
  I'll leave for you to read at your own pace.  Thank you!
  \end{itemize}
}

\end{document}


