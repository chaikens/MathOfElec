\documentclass[paper=8.5in:11in,fontsize=10pt,DIV=16]{scrartcl}
\usepackage{lmodern}
\pagestyle{empty}
\renewcommand{\familydefault}{\sfdefault}
\newenvironment{slide}{\clearpage
}{}
\usepackage{enumitem}
\setlist{noitemsep}
%\title{XeTaL}
%\author{Carl Capybara}
\begin{document}


\begin{slide}
  \begin{enumerate}
  \item
    Distinguish matroid element ``ports'' associated with 
    electric or elastic system parameter and solution variables of interest.
    (All vars are paired: (voltage, current), (force, displacement), etc.
    One gets a pair of submodels with dual matroids in elementary
    situations; not duals otherwise.)
  \item
    Exterior algebra forms of deletion and contraction of a non-port
    yield a pair of simpler systems.
  \item
    Cancelling non-port elements with a kind of bilinear pairing
    yields the parameter/solution variables of interest relation,
    in the form of an
    \textbf{exterior algebra valued function} of systems,
    that \textbf{is a Tutte function} (when the minor and
    direct sum operations are sign-consistent).
  \end{enumerate}

  With the suitable incidence matrix form, we get the all-minors matrix tree
  theorem; but all the minors are packed into \textbf{one exterior algebra
    object} that is a Tutte function of graphs.

  Seth Chaiken, Assoc. Prof. Emeritus, University at Albany.

%\end{frame}
\end{slide}
\end{document}

