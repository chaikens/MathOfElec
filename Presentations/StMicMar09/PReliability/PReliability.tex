\documentclass[12pt,leqno]{amsart}




\usepackage{amsmath,amssymb,amsfonts,amsthm}
\usepackage{eucal,graphicx}
\usepackage{color}
\usepackage[pdftex]{hyperref}


\setlength{\textwidth}{6.5in}
\setlength{\oddsidemargin}{0.0in}
\setlength{\evensidemargin}{0.0in}
\setlength{\textheight}{9in}
\setlength{\topmargin}{-.4in}

%\renewcommand{\baselinestretch}{2}     % Activate for double spacing.
%\renewcommand{\baselinestretch}{1.6}   % Activate for 1-1/2 spacing.
%\renewcommand{\baselinestretch}{1.3}   % Activate for 1-1/3 spacing.


\newcommand \comment[1]{}			%  Silent version.
%\renewcommand \comment[1]{\emph{#1}}		%  Comment revealed.
\newcommand \dateadded[1]{\comment{[Date added: #1.]}}
\newcommand \mylabel[1]{\label{#1}\comment{{\rm \{#1\} }}}
\newcommand \myref[1]{\ref{#1}\comment{{\{#1\}}}}

\newtheorem{lem}{Lemma}
\newtheorem{cor}[lem]{Corollary}
\newtheorem{prop}[lem]{Proposition}
\newtheorem{thm}[lem]{Theorem}
\newtheorem{definition}[lem]{Definition}


\theoremstyle{remark}
\newtheorem{exam}{Example}%[section]
\newcommand \myexam[1]{\smallskip\begin{exam}[\emph{#1}]}

\renewcommand{\phi}{\varphi}
\newcommand\eset{\varnothing}
\newcommand\inv{^{-1}}
\newcommand\setm{\setminus}
\newcommand\chiz{\chi^\bbZ}
\newcommand\bbR{\mathbb{R}}
\newcommand\bbZ{\mathbb{Z}}
\newcommand\cH{\mathcal{H}}


\newcommand\Ueloop{\ensuremath{U^e_{0}}}
\newcommand\Uecoloop{\ensuremath{U^e_{1}}}
\newcommand\Uefdyad{\ensuremath{U^{ef}_{1}}}
\newcommand\Uefgtriad{\ensuremath{U^{efg}_{1}}}
\newcommand\Uefgtriangle{\ensuremath{U^{efg}_{2}}}

%   Disjoint Union
%\newcommand{\dunion}{\uplus}
\newcommand{\dunion}
%{\mbox{\hbox{\hskip4pt$\cdot$\hskip-4.62pt$\cup$\hskip2pt}}}
{\mbox{\hbox{\hskip6pt$\cdot$\hskip-5.50pt$\cup$\hskip2pt}}}
%
% Dot inside a cup.
% If there is a better, more Latex like way 
% (more invariant under font size changes) way,
% I'd like to know.


\newcommand{\Bases}[1]{\ensuremath{{\mathcal{B}}(#1)}}
\newcommand{\Reals}{\ensuremath{\mathbb{R}}}
\newcommand{\FieldK}{\ensuremath{K}}
\newcommand{\Perms}{\ensuremath{\mathfrak{S}}}
\newcommand{\rank}{{\rho}}% {{\mbox{rank}}}
\newcommand{\Rank}{{\rho}}% {{\mbox{rank}}}
\newcommand{\Card}[1]{\ensuremath{{\left|#1\right|}}}
\newcommand{\ext}[1]{\ensuremath{\mathbf{#1}}}

% Set Complement
% command to mess with overline, bar or custom 
% alternatives for sequence or set complement
%
%\newcommand{\scomp}[1]{\ensuremath{\;\overline{#1}\;}}
%\newcommand{\scomp}[1]{\ensuremath{\bar{#1}}}
%\newcommand{\scomp}[1]{\ensuremath{\genfrac{}{}{}{}{}{#1}}}
\newcommand{\scomp}[1]{\ensuremath{\overline{#1}}}



\newcommand{\pn}{\;\raisebox{-0.2\height}{\input{p1b.pdf_t}}}
\newcommand{\qn}{\;\raisebox{-0.2\height}{\input{q1b.pdf_t}}}
\newcommand{\pz}{\;\;\raisebox{-0.2\height}{\input{p0b.pdf_t}}}
\newcommand{\qz}{\;\;\raisebox{-0.2\height}{\input{q0b.pdf_t}}}
\newcommand{\pzqz}{\;\raisebox{-0.2\height}{\input{p0q0b.pdf_t}}}
\newcommand{\pzqn}{\;\raisebox{-0.2\height}{\input{p0q1b.pdf_t}}}
\newcommand{\pnqz}{\;\raisebox{-0.2\height}{\input{p1q0b.pdf_t}}}
\newcommand{\pnqn}{\;\raisebox{-0.2\height}{\input{p1q1b.pdf_t}}}
\newcommand{\pqegg}{\;\raisebox{-0.2\height}{\input{pqeggb.pdf_t}}}

\allowdisplaybreaks


%%%%%%%%%%%%%%%%%%%%%%%%%%%%%%%%%%%%%%%%%

\begin{document}

\title[Reliability in Ported Graphs]
{Reliability in Ported Graphs}

\author{Seth Chaiken}
\address{Computer Science Department\\
The University at Albany (SUNY)\\
Albany, NY 12222}
\email{\tt sdc@cs.albany.edu}

\author{Joanna Ellis-Monaghan}
\address{Department of Mathematics\\
St. Michaels College\\
Colchester, VT 05439}
\email{\tt jellis-monaghan@uvm.edu}


\begin{abstract}
\end{abstract}

\subjclass[2000]{Primary 05B35; Secondary 05C99, 05C15, 57M25, 94C05}



\keywords{Tutte function, Tutte polynomial}

\thanks{Version of \today.}

\maketitle
\pagestyle{headings}


%%%%%%%%%%%%%%%%%%%%%%%%
\section{Introduction}

In this note we propose an extension to the graph and matroid theoretic
model for network reliability studied by Colbourn and others.
The classical model characterizes 
$R(G)$, the reliability polynomial in $p$ of graph $G$, by
\begin{equation}
\label{pAdditiveTutteEq}
R(G)=pR(G/e)+(1-p)R(G\setminus e)\;\;\text{for }e\in E(G),
\end{equation}
and
\begin{equation}
\label{unportedReliabilityIVs}
R(G)=\left\{
 \begin{array}{ll}
  1 & \text{ if } |E(G)|=0 \text{ and } |V(G)|=1    \\
  0 & \text{ if } |E(G)|=0 \text{ and } |V(G)|> 1. 
 \end{array}
\right.
\end{equation}
Here, $E=E(G)$ is the edge set, 
$V(G)$ is the vertex set,
and $G/e = G\setminus e$ if $e\in E$ is a loop.
It is immediate that $R(G)$ equals the probability that the
random subgraph $(V(G),A)$ is path-connected 
when the edges $A\subseteq E$ are chosen
each independently with probability $p$, $0\le p \le 1$.
($A$ is often called the \emph{state}.) Indeed,
\begin{equation}
\label{unportedSetExpansion}
R(G)=\sum_{A\subseteq E}p^{|A|}(1-p)^{|E|-|A|}\chi(G,A)
\end{equation}
where $\chi(G,A)$ is 1 if $(V(G),A)$ is a path-connected graph and 
$0$ otherwise.  $\chi(G,A)=1$ when 
$G/A\setminus (E\setminus A)$ is a single vertex graph
with no edges and 
$\chi(G,A)=0$ when 
$G/A\setminus (E\setminus A)$ is a multiple vertex graph
with no edges.

Our extension is to introduce a set $P$ of distinguished edges
(perhaps more precisely, edge names) so that 
equation \eqref{pAdditiveTutteEq} is restricted to apply only
to $e \not\in P$.  We call $P$ the \emph{set of ports}.  
For convenience, we use $E(G)$ for the edges of $G$ other than
ports, and $S(G)$ for all the edges of $G$.  Then,
to unambigously define $R(G)$, we must combine 
\eqref{pAdditiveTutteEq} with \emph{initial values}
$I(Q)$, where each $Q$ is a minor of $G$ obtained by
contracting some $A\subseteq E$ and deleting
$E\setminus A$, leaving a multigraph whose edges all
belong to $P$.  The vertices of $G$ are unlabelled, as are the 
the vertices of $Q$.  The extension of 
\eqref{unportedSetExpansion} is
\begin{equation}
\label{pSetExpansion}
R(G)=\sum_{A\subseteq E}p^{|A|}(1-p)^{|E|-|A|}I(G/A|P),
\end{equation}
where $G/A|P$ abbreviates $G/A\setminus (E\setminus A)$.
When $0\le I(G/A|P)\le 1$ is interpreted as a probability, we
can discuss a probabilistic significance to our ported
$R(G)$ value.  However, after we give motivations, we
will discuss another probablistic interpretation in addition
to this one.

\section{Enter Matroid Theory}

Assume for now $P=\emptyset$.
It is impractical to compute $R(G)$, either as a polynomial or 
number for given $p$, even for small graphs of interest
in practical applications.  Upper and lower bounds on the
coefficients are therefore desired because, for example,
\[
\sum L_k p^k \le R(G) \le \sum U_k p^k
\]
when
\[
L_k\le c_k \le U_k\text{ and } R(G)=\sum c_k p^k.
\]
For the same reason, bounds on the coefficients when
$R(G)$ is expanded about numbers besides $0$ are useful,
as will as other information about coefficient behavior.
Matroid theory has provided such information.

When $G$ is a (non-empty) disconnected graph, $R(G)=0$.  
Otherwise, $R(G)$ is related to $G's$ graphic matroid, specificially
through its Tutte polynomial $T(M(G);X,Y)$ via
\begin{equation}
R(G)=p^{n-1}(1-p)^{m-n+1}T(M(G);1,\frac{1}{1-p}).
\end{equation}
where $m=|E(G)|$ and $n=|V(G)|$.  Combinatorial theory of
Tutte polynomial of matroids, and specifically graphic matroids,
has provided information useful for approximating the 
coefficients in several expansions and other analyses.

The basics of Tutte polynomial theory has been extended to
$P$-ported and $P$-ported parametrized Tutte functions, where
restrictions like we gave already on reducing elements in $P$ 
are applied to Tutte equations for matroids.  

\emph{The goal of our project is to formulate ported extensions of
the theory that has been successful in helping to
approximate the non-ported reliability polynomial.}

\subsection{Probability Variables}

Suppose a separate probability variable $0\le p_e \le 1$ is given
for each $e\in E$.  Take $q_e = 1 - p_e$.  Then \eqref{pAdditiveTutteEq} 
becomes
\begin{equation}
\label{ParampAdditiveTutteEq}
R(G)=p_eR(G/e)+q_eR(G\setminus e)\;\;\text{for }e\in E(G),
\end{equation}
and so, when $e$ is a loop, $R(G)=1\cdot R(G\setminus e)$ and
when $e$ is a coloop, $R(G)=p_e R(G/ e)$.  In the notation of
[sdc,emt], $x_e=p_e$, $y_e=q_e$, $X_e=p_e$ and $Y_e=1$.  The
identities 
\[
x_eY_f+y_eX_f = x_fY_e+ y_fX_e = p_e+p_f-p_fq_e,
\]
\[
x_ey_f+y_eX_f = x_fy_e+ y_fX_e = p_eq_f+p_fq_e
\]
and 
\[
x_eY_f+y_ex_f = x_fY_e+y_fx_e = p_e + p_f - p_ep_f
\]
are satisfied.  These imply the satisfaction of the 
identities shown in [sdc] to be necessary and sufficient
for a ported parametrized Tutte polynomial to be well-defined
by the equations
\[
R(G) = x_e R(G/e) + y_e R(G\setminus e)\text{ if }e\not\in P
\text{ is a non-separator,}
\]
\[
R(G) = X_e R(G/e)\text{ if }e\not\in P\text{ is a coloop, and}
\]
\[
R(G) = Y_e R(G\setminus e)\text{ if }e\not\in P\text{ is a loop.}
\]
One can also prove the well-definedness of the ported
reliability polynomial with probability variables from the
corank-nullity expansion
\[
R(G)=\sum_{A\subseteq E}R(G/A|P)
x_Au^{r(G)-r(G/A|P)-r(A)}
y_{\scomp{A}}1^{|A|-r(A)}
\]
with the values $u=0$ and $v=1$, i.e.,
\[
\label{PortedVarProb}
R(G)\;\;\;=\sum_{A\subseteq E, A\cup P \text{ spans }G}
R(G/A|P)
x_A
y_{\scomp{A}}
\;\;\;=
\sum_{A\subseteq E, A\cup P \text{ spans }G}
R(G/A|P)
p_A
q_{\scomp{A}}
\]
since 
$r(G)-r(G/A|P)-r(A)=r(G)-r(P\cup A)$.  This expansion is also expresses
probabilistic interpretations of the ported reliability polynomial
with variable probabilities.

\section{Applications}

We know two disparate applications of our ported reliability 
function extension.

The first is to introduce ports for the purpose of expressing
variations of basic (all-terminal) reliability, beginning
with $k$-terminal reliability for $k=2$ and for higher $k$.
The second is to use ports to to model some edges whose joint
state probability has a different distribution than that
of the non-port edges $E$, possibly more complex than
the product of independent Bernoulli distributions.  Indeed, we
can express the idea that if one of several edges fails, then
the others are more likely to fail also, perhaps because the
other ones are forced to carry more traffic.  It is clear from
\eqref{PortedVarProb} that when the $R(G/A|P)=R(Q)$ are assigned
values each signifying the probability that at least one 
``functioning'' 
subset $T\subseteq P$ is edge connected in graph $Q$ (whose edge set is $P$),
then $R(G)$ expresses the all-terminal reliablility.

\begin{prop}
Let $R(G)$ be the ported reliability function with 
$R(Q)=I(Q)$ for all $P$-minors $Q$ of $G$.  Let
the $p_e$ be probability values, $0\le p_e \le 1$
and $q_e = 1-p_e$ for all $e\in E$.
\begin{enumerate}
\item If $I(Q)=1$ for all $P$-minors $Q$, then
$R(G)=R(G/P)$, the non-ported reliability of $G/P$.
\item If $0\le I(Q)\le 1$ for all $P$-minors $Q$, then
$0\le R(G) \le R(G/P) \le 1$.
\end{enumerate}
\end{prop}

We see that $R(G)$ is a probability value no matter what
probabilities are assigned the $I(Q)$.

Example: $I(\pzqz) = I(\pzqn) = I(\pnqz) = I(\pqegg) = 1$ and  
$I(\pnqn) = 0$ models a situation that whenever both $p$ and $q$
in $P=\{p,q\}$ are needed to maintain connectivity, one
or both will fail.

(???) Although the $I(Q)$ with $Q$ ranging over $P$-minors are each 
probability values, $I$ is \emph{not} a probability distribution
on $P$-minors.  Rather, it is the \emph{likelyhood function}
\[
I(Q) = Pr(\text{ edges in }Q\text{ maintain connectivity }|\;\; G/A|P = Q\;\;).
\]
This suggests writing Bayes' formula:
\[
Pr(\;\; G/A|P = Q\;\;| \text{ edges in }Q\text{ maintain connectivity })=
\frac{I(Q)\sum_{G/A|P=Q}p_Aq_{\scomp{A}}}{R(G)}.
\]





\section{Activities and Boolean Interval Expansions}

Ported Tutte polynomials, including those with parameters,
have activities and boolean interval expansions in which the
the terms indexed by matroid bases are generalized to
terms indexed by so-called $P$-subbases.  
In this section, we extend known expressions for $R(G)$ 
in terms of activities and interval partitions of facial
simplicial complexes to the ported case.  
We thus extend the known $h$-vector expression for $R(G)$.
It is convenient
to work with variable probabilities and then to specialize
to $p_e=p$.

A $P$-subbasis $F$
is an independent subset of $E$ for which $F\cup P$ is a spanning
subset[sdc].  

From [sdc],
\[
T^{\mathcal{C}}(M)=
\sum_{F\in \mathcal{B}_P}[M/F|P]
%\left(
\Big(
\sum_{\substack{
       IP(F)\subseteq K \subseteq F\\
       EP(F)\subseteq L \subseteq E\setminus F
      }}
 x_{K\cup (E\setminus F\setminus L)}\;
 v^{\Card{E\setminus F\setminus L}}\;
 y_{L\cup (F\setminus K)}\;
 u^{\Card{F\setminus K}}\;\;
%\right)
\Big)
\]

Substituting $u=0$, $v=1$, $x_e=p_e$, $y_e=q_e$, and $[Q]=R(Q)$, we get:
\[
R(G)\;\;=
\sum_{F\in \mathcal{B}_P}R(G/F|P)
%\left(
\Big(
\sum_{\substack{
       EP(F)\subseteq L \subseteq E\setminus F
      }}
 p_{E\setminus L}\;
 q_{L}
%\right)
\Big).
\]





Let's write
\[
R(G)=\sum_{A\subseteq E, A\cup P \text{ spans }G}
R(G/A|P)
x_A
y_{\scomp{A}} =
\sum_{A\subseteq E, A\cup P \text{ spans }G}
R(G/A|P)
p_A
q_{\scomp{A}}
\]
as
\[
R(G)=\sum_{\text{coindependent }F\subseteq E}
R(G/\scomp{F}|P)
p_{\scomp{F}}
q_F.
\]


\section{Beyond and Beyond}
The lesson of the Tutte tree methodology applied to ZBR-type 
theorems for matroids is that the parametrized Tutte identities determine a 
class of computations and the identities have a solution
when all computations in the class that start with
the same input object have the same outcomes.  The lesson
of the methodology, underscored by its $P$-ported 
extension, applied to Tutte functions of oriented matroids and to
graphs is that some useful functions have computations where
the recursions are constrained by matroid structure but
the initial values vary with additional structure.
In the oriented matroid case, the additional structure is
the orientation.  In the non-ported graph case, the additional
structure is the number of path-connected components.  

This observation suggests we investigate values associated
with a Tutte tree (equivalently, a choice of applications
of Tutte equations to objects obtained recursively) in 
ways more general than the polynomial of monomials determined
by the activity status of the elements along each root-to-leaf path.
For example, each path may contribute a product (monomial)
where the factor corresponding to a separator (loop or coloop)
\emph{depends on structure of the object on which
a deletion/contraction reduction produces a separator
in the resulting minor}.   
Here, the coloop becomes  internally active 
along the path in which it occurs, or the loop becomes 
externally active along the path in which it occurs.

As before, the value would be well-defined
if \emph{all computation trees} produce the same
value under the given set of rules.  Are there 
rules leading to well-defined function
for which the factor due to an active element
varies with the object in which it becomes
active?

Does the computation tree represent a sample
in a random process?  Might it correspond to
a process in which network elements are 
chosen to be utilized (or reserved for use)
and then found to be functioning or not?

Is there a notion of a random Tutte computation
tree?  I.E., a tree that evolved (grew) from
a process in which one of the non-separators
is picked for reduction at each node on the
tree-growth boundary?


\end{document}
