\documentclass{beamer}
\usepackage{amsmath}
\usepackage{url}




%\usetheme{Berlin}
\title{Ported Parametrized Tutte Functions}
\author{Seth Chaiken\\
CS Department\\
Univ. at Albany\\
State Univ. of New York\\
\url{sdc@cs.albany.edu}\\
\url{http://www.cs.albany.edu/~sdc}
%\verb|sdc@cs.albany.edu|\\
%\verb|http://www.cs.albany.edu/~sdc|
}
%\address{Computer Science Department\\
%The University at Albany (SUNY)\\
%Albany, NY 12222, U.S.A.}

%\email{\tt sdc@cs.albany.edu}
\date{Version of \today}
%March 26, 2009\\
%St. Michael's College and Univ. Vermont\\
%Combinatorics Seminar}
\begin{document}

\newcommand{\Remph}[1]{{\color{red}#1}}

%   Disjoint Union
%\newcommand{\dunion}{\uplus}
\newcommand{\dunion}
%{\mbox{\hbox{\hskip4pt$\cdot$\hskip-4.62pt$\cup$\hskip2pt}}}
{\mbox{\hbox{\hskip6pt$\cdot$\hskip-5.50pt$\cup$\hskip2pt}}}
%
% Dot inside a cup.
% If there is a better, more Latex like way 
% (more invariant under font size changes) way,
% I'd like to know.

\newcommand{\en}{\;\raisebox{-0.2\height}{\input{e1b.pdf_t}}}
\newcommand{\ez}{\;\;\raisebox{-0.2\height}{\input{e0b.pdf_t}}}

\newcommand{\pn}{\;\raisebox{-0.2\height}{\input{p1b.pdf_t}}}
\newcommand{\qn}{\;\raisebox{-0.2\height}{\input{q1b.pdf_t}}}
\newcommand{\pz}{\;\;\raisebox{-0.2\height}{\input{p0b.pdf_t}}}
\newcommand{\qz}{\;\;\raisebox{-0.2\height}{\input{q0b.pdf_t}}}
\newcommand{\pzqz}{\;\raisebox{-0.2\height}{\input{p0q0b.pdf_t}}}
\newcommand{\pzqn}{\;\raisebox{-0.2\height}{\input{p0q1b.pdf_t}}}
\newcommand{\pnqz}{\;\raisebox{-0.2\height}{\input{p1q0b.pdf_t}}}
\newcommand{\pnqn}{\;\raisebox{-0.2\height}{\input{p1q1b.pdf_t}}}
\newcommand{\pqegg}{\;\raisebox{-0.2\height}{\input{pqeggb.pdf_t}}}

\newcommand{\pnsub}{\input{p1.pdf_t}}
\newcommand{\qnsub}{\input{q1.pdf_t}}
\newcommand{\pzsub}{\input{p0.pdf_t}}
\newcommand{\qzsub}{\input{q0.pdf_t}}
\newcommand{\pzqzsub}{\input{p0q0.pdf_t}}
\newcommand{\pzqnsub}{\input{p0q1.pdf_t}}
\newcommand{\pnqzsub}{\input{p1q0.pdf_t}}
\newcommand{\pnqnsub}{\input{p1q1.pdf_t}}
\newcommand{\pqeggsub}{\input{pqegg.pdf_t}}




\begin{frame}
\titlepage
\end{frame}
\section*{Outline}
\begin{frame}
\tableofcontents
\end{frame}

\section{Generalizing Tutte Functions}

\begin{frame}
\frametitle{Our Ported Parametrized {\small separator-strong} Tutte Equations}
\begin{itemize}
\item
$T(G)=x_eT(G/e)+y_eT(G\setminus e)$\\  
if  $e$ is a non-separator and  $e\not\in P$.
\item
$T(G)=X_eT(G/e) \text{ if } e 
\text{ is a coloop (isthmus) and } e\not\in P.$
\item
$T(G)=Y_eT(G\setminus e) \text{ if } e
\text{ is a loop and } e\not\in P.$
\end{itemize}
\begin{block}{They express two generalizations of Famous Tutte Polynomial properties}
\begin{enumerate}
\item
We include four generally different parameters for each $e$\\
($x_e$, $y_e$ in additive equation; $X_e$, $Y_e$ in
separator multiplicative equations).\\
Zaslavsky, Bollobas-Riordan, Ellis-Monaghan-Traldi:\\
\Remph{NO SOLUTION} unless conditions hold on the
parameters and $T(\emptyset)$.

\item
Deletion ($G\setminus e$), Contraction ($G/e$) and
reduction of separators ($T(G'\oplus\ez)=Y_eT(G')$,
$T(G'\oplus\en)=X_eT(G')$) is \Remph{restricted to
$e\not\in P$}.\\
Las Vergnas.  Diao-Hetyei and sdc combined (1) with (2).
\end{enumerate}

\end{block}
\end{frame}

\begin{frame}
\frametitle{The Famous Tutte Polynomial}

\begin{block}{Recursive charactization}
Take $P=\emptyset$, $x_e=y_e=1$, $X_e=X$ and $Y_e=Y$ for
all $e$,\\
define $T(\emptyset)=1$:

$T(G)(X,Y)$ is then a well-defined polynomial in $X,Y$.
\end{block}

\begin{block}{Interesting graph or matroid invariants
are evaluations of $T(G)$ for values of $X,Y$.}
\begin{itemize}
\item $T(G;1,1)=$ number of spanning trees (in a connected
graphs) or number of bases (in a matroid).
\item $T(G;\text{???}(-\lambda))=$ number of graph colorings over
$\lambda$ colors (\Remph{chromatic polynomial}).
\end{itemize}
\end{block}
\end{frame}


\begin{frame}
\frametitle{Tutte Equations}
\begin{block}{Classical Tutte Equations}
$T(G)$ is a matroid \emph{invariant} that satisfies:
\[
T(G)=T(G/e)+T(G\setminus e)\text{ if $e$ is a non-separator}
\]
\[
T(G_1\oplus G_2)=T(G_1)T(G_2)
\]
and so it is a polynomial in $X$ and $Y$ where
\[
X=T(\text{coloop, i.e. one element isthmus matroid})
\]
\[
Y=T(\text{(one element) loop matroid}).
\]
\Remph{provided we know that } the Tutte equations
uniquely determine $T(G)$.
\end{block}
\end{frame}

\begin{frame}
\frametitle{Well-definedness of Tutte Eq. Solutions---Newer, Easy Way}

\begin{theorem}
The following polynomial in $u$, $w$ defined below for all
matroids $G$ satisfies the additive and multiplicative
Tutte Equations:
\[
R(G)=\sum_{A\subseteq E}u^{\text{rank}(G)-\text{rank}(A)}
                        w^{|A|-\text{rank}(A)}
\]
\end{theorem}
\begin{corollary}
$T(G;X,Y)$ is well-defined by $R(G)(X-1,Y-1)$.
\end{corollary}
\begin{proof}
Verify that $R(\en)=u+1$, $R(\ez)=v+1$;
use $T(\en)=X$, $T(\ez)=Y$; and apply
induction on $|E|$.
\end{proof}
\end{frame}


\begin{frame}
\frametitle{Well-definedness of Tutte Eq. Solutions---Orig, Hard Way}
\begin{theorem}[Tutte, Brylawski]
\[
T(X,Y)=\sum_{\text{Bases }B\subseteq E}
X^{\text{Internal Activity}(B)}
Y^{\text{External Activity}(B)}
\]
\Remph{independently} of $E$'s order used to define
the activities.
\end{theorem}
\end{frame}

\begin{frame}{Reminder about Activities}

Given a linear order on $E$,\\
\hspace{0.5in}Given a basis $B$ (spanning tree if $G$ is connected):
\begin{itemize}
\item $e\not\in B$ is \Remph{externally active} if $e$ is the 
\Remph{smallest} element of the (unique) circuit in 
$B\cup\{e\}$.
\item $e\in B$ is \Remph{internally active} if $e$ is the 
\Remph{smallest} element of the (unique) cocircuit in 
$E\setminus B\cup\{e\}$.
\item $\text{Internal (External) Activity}(B)$ is the 
\Remph{number} of internally (externally) active elements.
\end{itemize}

\Remph{Huh??} 
We will get intuition for this and extend it with $P\neq\emptyset$
with a Tutte (Computation) Tree (Gordon-MacMahon) view.

\vfill
H. Crapo also proved the well-definedness of the
Tutte polynomial from its corank-nullity polynomial 
expression.  But that doesn't fully generalize to
parametrized Tutte functions (Zaslavsky).

\end{frame}

\subsection{Tutte (Computation) Trees}
\begin{frame}
\frametitle{Recursive Computations and Trees}
\begin{itemize}
\item Every process of applying 
subset of the Tutte equations applied left-to-right
to calculate some $T(G)$ is a \Remph{recursive computation}.
\item Recursive computations (ignoring dataflow independent 
orderings) correspond to \Remph{computation trees}.
\end{itemize}
\end{frame}

\begin{frame}
\frametitle{The result from a Tutte Tree}
bla bla
\vfill
It is natural, but not mandatory, to use a somehow-defined
\Remph{next} non-separator to determine the two recursions
used to compute $T(G')$ for an intermediate minor $G'$.
\vfill
Tutte (computation) trees were defined formally and
used by Gordon-MacMahon to study Tutte polynomials of
\Remph{greedoids}, where sometimes, the same element 
priority order
cannot be used under each branch.
\end{frame}

\subsection{Ambiguities among Tutte Equations}

\begin{frame}
{First Ambiguity among Tutte Equations}
\[
x_eY_f+y_eX_f=x_fY_e+y_fX_e
\]
\begin{center}\input{DyadProblem.pdf_t}\end{center}

\begin{block}{A Detail}
$T(\text{loop matroid on }e) = Y_eT(\emptyset\text{(empty matroid)})$, etc.
so the real ZBR condition is 
\[
T(\emptyset)(x_eY_f+y_eX_f)=T(\emptyset)(x_fY_e+y_fX_e)
\]
\end{block}


\end{frame}

\begin{frame}{Two More: One for separated $P$ and another like it.}
\input{TriadProblems.pdf_t}
\end{frame}

\begin{frame}{Another Two More: With all 5, are we done?}
\input{TriangleProblems.pdf_t}
\end{frame}

\subsection{Solution}
\begin{frame}
\frametitle{Solution---Setup}
\begin{block}{When do recursive equations have a solution?}
``Have a solution'' here means ``\Remph{Every calculation of $T(G)$
using the Tutte equations and initial values
on members of $\mathcal{F}$ gives the same answer.}
\end{block}

\begin{definition}[Sep. Strong Ported Parametrized Tutte Function]
Let $P$ be a set and $\mathcal{F}$ be a family of graphs, oriented
matroids or matroids that is closed under deletion and 
contraction of elements
\Remph{not in} $P$.  Deletion of loops and contraction of coloops is 
allowed.

Let ring $R$ elements $X_e, Y_e, x_e$ and $y_e$ (for each $e\not\in P$)
and $R-$module elements $I(Q)$ for every $Q\in \mathcal{F}$ 
with $Q$ \Remph{over elements of $P$ only} also be given.

This structure \Remph{has a Tutte function} if and only if
the Ported Parametrized Tutte Equations have (a necessarilly
unique) solution over all of $\mathcal{F}$.
\end{definition}

The $X_e, Y_e, x_e, y_e$ and $I(Q)$ are called parameters and
initial values.
\end{frame}


\begin{frame}{Solution---Theorem}
\begin{theorem}[After Zaslavsky, Bollobas-Riordan, Ellis-Monaghan-Traldi]
$\mathcal{F}$ and values as above
\Remph{has a Tutte function} iff the following equations
are satisfied whenever they arise from a member $G\in\mathcal{F}$:
\begin{enumerate}
\item Suppose $G=Q\oplus G'$ where $S(Q)\subseteq P$.
\begin{enumerate}
\item With $G'$ a 2-circuit $\{e,f\}$ (and so 2-cocircuit too),
$I(Q)(x_eY_f+y_eX_f)=I(Q)(x_fY_e+y_fX_e)$.
\item
With $G'$ a 3-circuit $\{e,f,g\}$,
$I(Q)X_g(x_ey_f+y_eX_f)=
I(Q)X_g(x_fy_e+y_fX_e)$. 
\item 
With $G'$ a 3-cocircuit $\{e,f,g\}$,
$I(Q)Y_g(x_eY_f+y_ex_f)=
I(Q)Y_g(x_fY_e+y_fx_e)$. 
\end{enumerate}
These generalize the 3 ZBR equations merely by replacing
$I(\emptyset)$ with $I(Q)$.
\item With $\{e,f\}=E$ in series and not isolated (from $P$),
$I(G/e\setminus f)(x_ey_f+y_eX_f)=
I(G/e\setminus f)(x_fy_e+y_fX_e)$.
\item With $\{e,f\}=E$ in parallel and not isolated,
$I(G/e\setminus f)(x_eY_f+y_ex_f)=
I(G/e\setminus f)(x_fY_e+y_fx_e)$.
\end{enumerate}
\end{theorem}
\end{frame}

\subsection{Proof Ideas}
\begin{frame}
\frametitle{Proof Outline}
\begin{block}{Ported ZBR equations are necessary}
Consider the $1+4$ matroid/graph classes 
with $E(G)=\{e,f\}$ or $E(G)=\{e,f,g\}$,
where $E(G)=S(G)\setminus P$, corresponding
to the 5 ZBR conditions.

For each, show (as I illustrated before)
that assuming certain pairs of computations of $T(G)$
give equal results implies the condition.
\end{block}
\begin{block}{Ported ZBR equations are sufficient}
Induction: Assume $G$ is a minimum $|E(G)|$ counter example, 
where $E(G)=S(G)\setminus P$.  So: $T(G/e)$ and $T(G\setminus e)$
are well-defined from the Tutte Equations for every $e\in E(G)$.

Lemma (Zaslavsky) shows \Remph{all of} $E(G)$ is a series class
or a parallel class.

The relevent Tutte equations\\
 (Is $E$ isolated? Or is $E$ connected to some of $P$?)\\
show there's a smaller $E$ counterexample.
\end{block}
\end{frame}


\subsection{Proof Details}
\begin{frame}
\frametitle{Some Details}
\begin{itemize}
\item $|E|\geq 2$.
\item
No $e\in E$ is a separator in $G$.
\item
For \Remph{no} $e,f\in E(G)$ is this a Tutte tree:
\input{Tree2Ordinary.pdf_t}
\raisebox{0.25in}{\hspace{0.2in}\framebox{\begin{minipage}[b]{2in}
The Tutte Tree formalism here 
\Remph{means} $e$ is a non-separator in $G$
and $f$ is a non-separator in both $G/e$
and $G\setminus e$.
	       \end{minipage}}}
\item
Lemmas: Each $e\in E(G),f\in E(G)$, $e\neq f$, 
is series pair or a parallel pair.\\
$e,f$ parallel and $f,g$ series is impossible.\\
So \Remph{all} of $E$ is a series class or is a parallel class.
\end{itemize}
\end{frame}

\begin{frame}
\frametitle{One of 5 cases}
\begin{center}...\end{center}
\end{frame}




\section{Tutte (Computation) Trees and Internal/External Activities}

\begin{frame}
\frametitle{Root-Leaf Paths in Tutte (Computation) Trees}
A {$P$-subbasis $T\subseteq E(G)$ 
(``contracting set'' [Diao-Hetyei])}
is an independent set (forest) for which 
$T\cup P$ is spanning.

%So, in $G/T$, $P$ is spanning.
\input{TutteTree.pdf_t}
\raisebox{0.4in}{\begin{minipage}[b]{2in}
Path $\pi$ contributes\\
$[G'|P]x^{IP(T)}y^{EP(T)}X^{IA(T)}Y^{EA(T)}$ to 
our Tutte Poly.
\end{minipage}}

\vfill
\Remph{All is determined by the Tutte tree, NOT an element order!}

\end{frame}


\begin{frame}
\frametitle{Internal/External Activities and Tutte Trees}


$E$ is partitioned: $T=IP(T)\cup IA(T)$, 
$E\setminus T=EP(T)\cup EA(T)$.  $IP(T)=$ $\{$elements contracted 
along$\;\pi\}$.  

\vfill
$EP(T)=$ $\{$elements deleted along$\;\pi$\}.  

\vfill
In $G'$, $IA(T)$ is all coloops, $EA(T)$ is all loops.

\vfill
$2^E$ is partitioned into intervals \\
$\{[X_T,Y_T]|P-\text{subbasis} T\}$,\\
$X_T=IP(T)\subseteq (T=IP(T)\cup IA(T))\subseteq (T\cup EA(T))=Y_T$.

\end{frame}


\begin{frame}
\frametitle{Tutte Polynomials and Activities}
\begin{enumerate}
\item
When the conditions in our $P$-ported ZBR theorem are satisfied,
\Remph{all} Tutte trees yield the same \Remph{value in the $R$-module},
called \Remph{THE} Tutte polynomial 
(because trees$\leftrightarrow$computations.)
This value has multiple \Remph{polynomial expressions}.  
\item
The $P$-quotient $[G/IP(T)|P]$ in the term contributed by
$P$-subbasis $T$ is \Remph{determined by} the \Remph{internally
passive} elements of $T$.
\end{enumerate}
\end{frame}

\section{Normal Ported Param. Tutte Functions}
\begin{frame}
\frametitle{Ported Rank-Nullity Polynomial}
\[
R(G)=\sum_{A\subseteq E(G)}[G/A|P]x_A
                 y_{E\setminus A}
                 u^{\text{rank}(G)-\text{rank}[G/A|P]-\text{rank(A)}}
                 w^{|A|-\text{rank}(A)}
\]
\begin{block}{$R(G)$ is a Tutte function.}
\begin{itemize}
\item
\Remph{$R(G)$ is well-defined} as the above generating function.
\item
Exercise: $R(G)$ satisfies the $P$-ported parametrized Tutte Equations.
\item
The Tutte equations plus
\[
X_e=x_e+y_eu;\;\;\;Y_e=x_ew+y_e;\;\;\;I(Q)=[Q]
\]
have the unique solution $R(G)$, a polynomial
in the (many) $x_e, y_e and [Q]$ and (two) $u, w$.
\item
Any Tutte function writable this way is called 
\Remph{normal}, because Zaslavsky coined this
name for them when $P=\emptyset$.
\item
The most popular Tutte functions are normal;
only the ``Bla Bla the matroid structure''
(Zaslavsky).
\end{itemize}
\end{block}
\end{frame}

\subsection{Tree and Forest Enumerators}
\begin{frame}
\frametitle{Ported Tree Enumerator}
\end{frame}

\begin{frame}
\frametitle{Ported Forest Enumerator}
\end{frame}

\subsection{Determinantal and Extensor Tutte Functions}
\begin{frame}
\frametitle{Determinantal and Extensor Tutte Functions}
\end{frame}

\section{Tutte Functions of Ported and Labelled Graphs}
\begin{frame}
\begin{itemize}
\item The matroid structure determines the Tutte Trees.
\item Other structures, on which matroid deletion and
contraction operations act, determine the initial values.
\item ZBR analyzed parametrized Tutte functions of 
non-ported graphs with unlabelled vertices.  The
indecomposibles are then $E_n$, the $n$-vertex graphs
with no edges.  
\end{itemize}
\end{frame}

\begin{frame}
\frametitle{ZBR Conditions for edge-Parametrized unlabelled Graphs}
\end{frame}


\end{document}

