\documentclass{beamer}
\usepackage{amsmath}
\usepackage{url}




%\usetheme{Berlin}
\title{Ported Parametrized Tutte Functions: Old and New Applications}
\author{Seth Chaiken\\
CS Department\\
Univ. at Albany\\
State Univ. of New York\footnote{Governor David Paterson}, USA\\
\url{sdc@cs.albany.edu}\\
\url{http://www.cs.albany.edu/~sdc}
%\verb|sdc@cs.albany.edu|\\
%\verb|http://www.cs.albany.edu/~sdc|
}
%\address{Computer Science Department\\
%The University at Albany (SUNY)\\
%Albany, NY 12222, U.S.A.}

%\email{\tt sdc@cs.albany.edu}
\date{March 26, 2009\\
St. Michael's College and Univ. Vermont\\
Combinatorics Seminar}
\begin{document}

\newcommand{\Remph}[1]{{\color{red}#1}}

%   Disjoint Union
%\newcommand{\dunion}{\uplus}
\newcommand{\dunion}
%{\mbox{\hbox{\hskip4pt$\cdot$\hskip-4.62pt$\cup$\hskip2pt}}}
{\mbox{\hbox{\hskip6pt$\cdot$\hskip-5.50pt$\cup$\hskip2pt}}}
%
% Dot inside a cup.
% If there is a better, more Latex like way 
% (more invariant under font size changes) way,
% I'd like to know.


\newcommand{\pn}{\;\raisebox{-0.2\height}{\input{p1b.pdf_t}}}
\newcommand{\qn}{\;\raisebox{-0.2\height}{\input{q1b.pdf_t}}}
\newcommand{\pz}{\;\;\raisebox{-0.2\height}{\input{p0b.pdf_t}}}
\newcommand{\qz}{\;\;\raisebox{-0.2\height}{\input{q0b.pdf_t}}}
\newcommand{\pzqz}{\;\raisebox{-0.2\height}{\input{p0q0b.pdf_t}}}
\newcommand{\pzqn}{\;\raisebox{-0.2\height}{\input{p0q1b.pdf_t}}}
\newcommand{\pnqz}{\;\raisebox{-0.2\height}{\input{p1q0b.pdf_t}}}
\newcommand{\pnqn}{\;\raisebox{-0.2\height}{\input{p1q1b.pdf_t}}}
\newcommand{\pqegg}{\;\raisebox{-0.2\height}{\input{pqeggb.pdf_t}}}

\newcommand{\pnsub}{\input{p1.pdf_t}}
\newcommand{\qnsub}{\input{q1.pdf_t}}
\newcommand{\pzsub}{\input{p0.pdf_t}}
\newcommand{\qzsub}{\input{q0.pdf_t}}
\newcommand{\pzqzsub}{\input{p0q0.pdf_t}}
\newcommand{\pzqnsub}{\input{p0q1.pdf_t}}
\newcommand{\pnqzsub}{\input{p1q0.pdf_t}}
\newcommand{\pnqnsub}{\input{p1q1.pdf_t}}
\newcommand{\pqeggsub}{\input{pqegg.pdf_t}}




\begin{frame}
\titlepage
\end{frame}
\section*{Outline}
\begin{frame}
\tableofcontents
\end{frame}

\section{Generalizing Tutte Functions}

\begin{frame}
\frametitle{Our Ported Parametrized {\small separator-strong} Tutte Equations}
\begin{itemize}
\item
$T(G)=x_eT(G/e)+y_eT(G\setminus e)$\\  
if  $e$ is a non-separator and  $e\not\in P$.
\item
$T(G)=X_eT(G/e) \text{ if } e 
\text{ is a coloop (isthmus) and } e\not\in P.$
\item
$T(G)=Y_eT(G\setminus e) \text{ if } e
\text{ is a loop and } e\not\in P.$
\end{itemize}
Zaslavsky, Bollobas-Riordan, Ellis-Monaghan-Traldi;
$P\neq \emptyset$ (sdc).

\begin{block}{The Famous Tutte Polynomial}
Take $P=\emptyset$, $x_e=y_e=1$, $X_e=X$ and $Y_e=Y$ for
all $e$,\\
define $T(\emptyset)=1$:

$T(G)(X,Y)$ is then a well-defined polynomial in $X,Y$.
\end{block}

\begin{theorem}[Tutte, Brylawski]
\[
T(X,Y)=\sum_{\text{Bases }B\subseteq E}
X^{\text{Internal Activity}(B)}
Y^{\text{External Activity}(B)}
\]
\Remph{independently} of $E$'s order used to define
the activities.
\end{theorem}
\end{frame}

\begin{frame}{Reminder about Activities}

Given a linear order on $E$,\\
\hspace{0.5in}Given a basis $B$ (spanning tree if $G$ is connected):
\begin{itemize}
\item $e\not\in B$ is \Remph{externally active} if $e$ is the 
\Remph{smallest} element of the (unique) circuit in 
$B\cup\{e\}$.
\item $e\in B$ is \Remph{internally active} if $e$ is the 
\Remph{smallest} element of the (unique) cocircuit in 
$E\setminus B\cup\{e\}$.
\item $\text{Internal (External) Activity}(B)$ is the 
\Remph{number} of internally (externally) active elements.
\end{itemize}

\Remph{Huh??} 
We will get intuition for this and extend it with $P\neq\emptyset$
with a Tutte (Computation) Tree (Gordon-MacMahon) view.

\vfill
H. Crapo also proved the well-definedness of the
Tutte polynomial from its corank-nullity polynomial 
expression.  But that doesn't fully generalize to
parametrized Tutte functions (Zaslavsky).

\end{frame}






\subsection{Problems...}

\begin{frame}{$T(2-\text{circuit})$ is \Remph{not} defined unless...}
\[
x_eY_f+y_eX_f=x_fY_e+y_fX_e
\]
\begin{center}\input{DyadProblem.pdf_t}\end{center}

These are Tutte (computation) trees, defined formally and
used by Gordon-MacMahon to study Tutte polynomials of
\Remph{greedoids}, where sometimes, the same element order
cannot be used under each branch.

\begin{block}{A Detail}
$T(\text{loop matroid on }e) = Y_eT(\emptyset\text{(empty matroid)})$, etc.
so the real ZBR condition is 
\[
T(\emptyset)(x_eY_f+y_eX_f)=T(\emptyset)(x_fY_e+y_fX_e)
\]
\end{block}


\end{frame}

\begin{frame}{Problems: 2nd ZBR-type and 1st for $P\neq\emptyset$}
\input{TriadProblems.pdf_t}
\end{frame}

\begin{frame}{More Problems..., 2nd for $P\neq\emptyset$}
\input{TriangleProblems.pdf_t}
\end{frame}

\subsection{Solution}
\begin{frame}
\frametitle{Solution---Setup}
\begin{block}{When do recursive equations have a solution?}
``Have a solution'' here means ``\Remph{Every calculation of $T(G)$
using the Tutte equations and initial values
on members of $\mathcal{F}$ gives the same answer.}
\end{block}

\begin{definition}[Sep. Strong Ported Parametrized Tutte Function]
Let $P$ be a set and $\mathcal{F}$ be a family of graphs, oriented
matroids or matroids that is closed under deletion and 
contraction of elements
\Remph{not in} $P$.  Deletion of loops and contraction of coloops is 
allowed.

Let ring $R$ elements $X_e, Y_e, x_e$ and $y_e$ (for each $e\not\in P$)
and $R-$module elements $I(Q)$ for every $Q\in \mathcal{F}$ 
with $Q$ \Remph{over elements of $P$ only} also be given.

This structure \Remph{has a Tutte function} if and only if
the Ported Parametrized Tutte Equations have (a necessarilly
unique) solution over all of $\mathcal{F}$.
\end{definition}

The $X_e, Y_e, x_e, y_e$ and $I(Q)$ are called parameters and
initial values.
\end{frame}


\begin{frame}{Solution---Theorem}
\begin{theorem}[After Zaslavsky, Bollobas-Riordan, Ellis-Monaghan-Traldi]
$\mathcal{F}$ and values as above
\Remph{has a Tutte function} iff the following equations
are satisfied whenever they arise from a member $G\in\mathcal{F}$:
\begin{enumerate}
\item Suppose $G=Q\oplus G'$ where $S(Q)\subseteq P$.
\begin{enumerate}
\item With $G'$ a 2-circuit $\{e,f\}$ (and so 2-cocircuit too),
$I(Q)(x_eY_f+y_eX_f)=I(Q)(x_fY_e+y_fX_e)$.
\item
With $G'$ a 3-circuit $\{e,f,g\}$,
$I(Q)X_g(x_ey_f+y_eX_f)=
I(Q)X_g(x_fy_e+y_fX_e)$. 
\item 
With $G'$ a 3-cocircuit $\{e,f,g\}$,
$I(Q)Y_g(x_eY_f+y_ex_f)=
I(Q)Y_g(x_fY_e+y_fx_e)$. 
\end{enumerate}
These generalize the 3 ZBR equations merely by replacing
$I(\emptyset)$ with $I(Q)$.
\item With $\{e,f\}=E$ in series and not isolated (from $P$),
$I(G/e\setminus f)(x_ey_f+y_eX_f)=
I(G/e\setminus f)(x_fy_e+y_fX_e)$.
\item With $\{e,f\}=E$ in parallel and not isolated,
$I(G/e\setminus f)(x_eY_f+y_ex_f)=
I(G/e\setminus f)(x_fY_e+y_fx_e)$.
\end{enumerate}
\end{theorem}
\end{frame}

\subsection{Proof Ideas}
\begin{frame}
\frametitle{Proof Outline}
\begin{block}{Ported ZBR equations are necessary}
Consider the $1+4$ matroid/graph classes 
with $E(G)=\{e,f\}$ or $E(G)=\{e,f,g\}$,
where $E(G)=S(G)\setminus P$, corresponding
to the 5 ZBR conditions.

For each, show (as I illustrated before)
that assuming certain pairs of computations of $T(G)$
give equal results implies the condition.
\end{block}
\begin{block}{Ported ZBR equations are sufficient}
Induction: Assume $G$ is a minimum $|E(G)|$ counter example, 
where $E(G)=S(G)\setminus P$.  So: $T(G/e)$ and $T(G\setminus e)$
are well-defined from the Tutte Equations for every $e\in E(G)$.

Lemma (Zaslavsky) shows \Remph{all of} $E(G)$ is a series class
or a parallel class.

The relevent Tutte equations\\
 (Is $E$ isolated? Or is $E$ connected to some of $P$?)\\
show there's a smaller $E$ counterexample.
\end{block}
\end{frame}

\begin{frame}
\frametitle{Some Details}
\begin{itemize}
\item $|E|\geq 2$.
\item
No $e\in E$ is a separator in $G$.
\item
For \Remph{no} $e,f\in E(G)$ is this a Tutte tree:
\input{Tree2Ordinary.pdf_t}
\raisebox{0.25in}{\hspace{0.2in}\framebox{\begin{minipage}[b]{2in}
The Tutte Tree formalism here 
\Remph{means} $e$ is a non-separator in $G$
and $f$ is a non-separator in both $G/e$
and $G\setminus e$.
	       \end{minipage}}}
\item
Lemmas: Each $e\in E(G),f\in E(G)$, $e\neq f$, 
is series pair or a parallel pair.\\
$e,f$ parallel and $f,g$ series is impossible.\\
So \Remph{all} of $E$ is a series class or is a parallel class.
\end{itemize}
\end{frame}

%\begin{frame}
%\frametitle{One of 5 cases}
%\begin{center}...\end{center}
%\end{frame}

\section{Tutte (Computation) Trees and Internal/External Activities}
\begin{frame}
\frametitle{Tutte (Computation) Trees and Internal/External Activities}
A {$P$-subbasis $T\subseteq E(G)$ 
(``contracting set'' [Diao-Hetyei])}
is an independent set (forest) for which 
$T\cup P$ is spanning.

%So, in $G/T$, $P$ is spanning.
\input{TutteTree.pdf_t}
\raisebox{0.4in}{\begin{minipage}[b]{2in}
Path $\pi$ contributes\\
$[G'|P]x^{IP(T)}y^{EP(T)}X^{IA(T)}Y^{EA(T)}$ to 
our Tutte Poly.
\end{minipage}}

$E$ is partitioned: $T=IP(T)\cup IA(T)$, 
$E\setminus T=EP(T)\cup EA(T)$.  $IP(T)=$ $\{$elements contracted 
along$\;\pi\}$.  

$EP(T)=$ $\{$elements deleted along$\;\pi$\}.  

In $G'$, $IA(T)$ is all coloops, $EA(T)$ is all loops.

$2^E$ is partitioned into intervals 
$\{[X_T,Y_T]|P-\text{subbasis} T\}$,
$X_T=IP(T)\subseteq (T=IP(T)\cup IA(T))\subseteq (T\cup EA(T))=Y_T$.
\vfill
\Remph{All is determined by the Tutte tree, NOT an element order!}
\end{frame}

\begin{frame}
\frametitle{Tutte Polynomials and Activities}
\begin{enumerate}
\item
When the conditions in our $P$-ported ZBR theorem are satisfied,
\Remph{all} Tutte trees yield the same \Remph{value in the $R$-module},
called \Remph{THE} Tutte polynomial 
(because trees$\leftrightarrow$computations.)
This value has multiple \Remph{polynomial expressions}.  
\item
The $P$-quotient $[G/IP(T)|P]$ in the term contributed by
$P$-subbasis $T$ is \Remph{determined by} the \Remph{internally
passive} elements of $T$.
\end{enumerate}
\end{frame}


\begin{frame}
\frametitle{$P$-subbasis (Spanning Tree) Polynomial}
\[
\mathcal{T}_P(G)=\sum_{T:P-\text{subbasis}}
[G/T|P]x^Ty^{E\setminus T}
=
\sum_{T:P-\text{subbasis}}
[G/IP(T)|P]x^Ty^{E\setminus T}
\]
\[
\text{with notation: } z^S=\prod_{e\in S}z_e (z = x \text{ or } y).
\]
Remark: If $G$ is a tree, then $E$ is the one 
$P$-subbasis, $IP(E)=\emptyset$ and  $IA(E)=E$.
\[
\mathcal{T}(G)=x_e\mathcal{T}(G/e)+y_e\mathcal{T}(G\setminus e)
\;\;\text{if $e\not\in P$ is a non-separator.}
\]
\[
\mathcal{T}(G)=y_e\mathcal{T}(G\setminus e)
\;\;\text{if $e\not\in P$ is a loop.}
\]
\[
\mathcal{T}(G)=x_e\mathcal{T}(G/ e)\;\;\text{if $e$ is an isthmus.}
\]
To enumerate the spanning trees in $E$, set
\[
[\text{all loops}]\leftarrow 1\;\;\;\text{ and }\;\;\;
[\text{other $P$-quotients}]\leftarrow 0
\]
\end{frame}


\begin{frame}
\frametitle{$P$-ported Forest Polynomial}

\[
\mathcal{F}(G)=\sum_{F\subseteq E:F\text{ is a forest}}
[G/F|P]x^F
\]
The $\sum x^F$ coefficient of each $[Q]$ variable enumerates
a class of forests.  Examples: $[\pzqz]\rightarrow$ $F$ spans
both $p$ and $q$; $[\pnqn]\rightarrow$ $F$ spans neither
$p$ nor $q$.
\[
\mathcal{F}(G)=\sum_{T:P-\text{subbasis}}
[G/T|P]x^{IP(T)}(1+x)^{IA(T)}
\]
\[=
\sum_{T:P-\text{subbasis}}
[G/IP(T)|P]x^{IP(T)}\sum_{F\subseteq IA(T)}x^F
\]
\[\text{where  }(1+x)^S=\prod_{e\in S}(1+x_e)\]
\[
\mathcal{F}(G)=x_e\mathcal{F}(G/e)+\mathcal{F}(G\setminus e)
\;\;\text{if $e\not\in P$ is a non-separator.}
\]
\[
\mathcal{F}(G)=\mathcal{F}(G\setminus e)
\;\;\text{if $e\not\in P$ is a loop.}
\]
\[
\mathcal{F}(G)=(1+x_e)\mathcal{F}(G/ e)
\;\;\text{if $e\not\in P$ is an isthmus.}
\]
\end{frame}

\section{Electricity in Graphs}
\begin{frame}
\frametitle{Spanning Tree polys solve Equations of Kirchhoff and Ohm}
\begin{block}{Variables}
\begin{minipage}{1.25in}
For each $e\in E(G)$ or $p\in P$,
\end{minipage}
\begin{minipage}{2.7in}
$v_e$ or $v_p$ is the \Remph{voltage drop} across $e$ or $p$;\\
$i_e$ or $i_p$ is the \Remph{current flow} through $e$ or $p$.
\end{minipage}
\end{block}
\begin{block}{Equations}
For some (unimodular) basis for the cocycle space $(k^j_{e \text{ or } p})$,
$j=1,\ldots,\text{rank}(G)$, 0 net flow across cuts:
\[
\sum_{e\in E}k^j_e i_e + \sum_{p\in P} k^j_p i_p = 0
\text{ for }j=1,\ldots,\text{rank}(G).
\]
For some (unimodular) basis for the cycle space $(c^h_{e \text{ or } p})$,
$h=1,\ldots,\text{nullity}(G)$, 0 sum of diffs. of potential around cycles:
\[
\sum_{e\in E}c^h_e v_e + \sum_{p\in P} c^h_p v_p = 0
\text{ for }h=1,\ldots,\text{nullity}(G).
\]
For each $e\in E:\;\;\;$ $x_ev_e - y_ei_e = 0$\\
{\small Physical Conductance = $x_e:y_e$, Resistance = $y_e:x_e$}
\end{block}
\end{frame}


\begin{frame}
\frametitle{Matrix $M$:Flow, Potential Eqs. (Kirchhoff's) and Ohm's}
\input{Tableau.pdf_t}
\end{frame}

\begin{frame}
\frametitle{Solution using Determinants}
Let us apply current $i_{p_{\beta}}$ through port edge $p_{\beta}$,
leave the other ports ``open'' (means $i_{p_{\gamma}}=0$, $\gamma\neq\beta$)
and determine what is voltage drop $v_{p_{\alpha}}$ across port $p_{\alpha}$.

\[
v_{p_{\alpha}}=-\frac{M[I_{e_1},...,I_{e_n},V_{e_1},...,V_{e_n},
                V_{p_1}...,\Remph{I_{p_{\beta}}},...,V_{p_k}]}
                   {M[I_{e_1},...,I_{e_n},V_{e_1},...,V_{e_n},
                V_{p_1}...,\Remph{V_{p_{\alpha}}},...,V_{p_k}]}
                  \cdot i_{p_{\beta}}
\]
In short: Coefficients in the linear relationships among port voltage
and current variables are \Remph{ratios of full-rowed minors of $M$},
all with
\begin{itemize}
\item $k=|P|$ of the $2k$ column labels 
$I_{p_1},\ldots,I_{p_k},V_{p_1},\ldots,V_{p_k}$.
\item
\Remph{All} $2n=2|E|$ column labels
$I_{e_1},\ldots,I_{e_n},V_{e_1},\ldots,V_{e_n}$.
\end{itemize}

\begin{theorem}
Each of these minors (with carefully defined sign)
satisfies the $P$-ported parametrized
Tutte equations.
\end{theorem}

Remark: Ratios of these minors with numerator gotten by replacing 
\Remph{more than one} denominator label are \Remph{higher order minors}
of a square matrix relating some $k$ port variables with some
$k$ port variables.

\end{frame}


\begin{frame}
\frametitle{Why the Tutte equations? Sketch}

The last row is $(0 \cdots 0 y_{e_n} 0 \cdots -x_{e_n} 0 \cdots 0)$.

So, each minor can be written 
\[
-x_{e_n}(\text{a }(2n+k-1)\text{ by }(2n+k-1)\text{ minor})
\]
\[
+y_{e_n}(\text{a }(2n+2k-1)\text{ by }(2n+k-1)\text{ minor})
\]

In the first minor, the column for $e_n$ of the cycle matrix was deleted.

Do row operations so the column for $e_n$ of the cocycle matrix becomes
$(1,0,\cdots,0)^t$.

The resulting matrix corresponds to $G/e_n$.  The other minor
corresponds to $G\!\setminus\! e_n$.
\end{frame}

\begin{frame}
\frametitle{Application: Rayleigh Identity, ``Neg. Spanning Tree Correlation''}
\[
\Gamma_e(G)\text{ is equivalent conductance across }e.
\text{ Rayleigh: }0 \le \frac{\partial \Gamma_{p}}{\partial g_f} =
\frac{\partial \frac{T_G}{T_{G/e}}}{\partial g_f}
\]
is equivalent to 
\[
0 \le \frac{\partial T_G}{\partial g_f}T_{G/e} - 
       T_G\frac{\partial T_{G/e}}{\partial g_f} 
=
T_{G/f}T_{G/e} - T_GT_{G/e/f}
\]
In fact,
\[
T_{G/f}T_{G/e} - T_GT_{G/e/f} = \left( T^+_{G/e \text{ \& } G/f} - T^-_{G/e \text{ \& } G/f} \right)^2
\]
$T^{\pm}_{G/e \text{ \& } G/f}$ enumerate the $\pm$ common spanning trees.


\end{frame}

\begin{frame}
\frametitle{Known Partial and Full Combinatorial Proofs}
\[
T_{G/f}T_{G/e} - T_GT_{G/e/f} = \left( T^+_{G/e \text{ \& } G/f} - T^-_{G/e \text{ \& } G/f} \right)^2
\]
$T^{\pm}_{G/e \text{ \& } G/f}$ enumerate the $\pm$ common spanning trees.

\vfill
Choe (2004) 
proved essentially this using the vertex-based all-minors matrix tree theorem,
combinatorial cases and Jacobi's theorem relating the minors of a matrix to
the minors of its inverse..

\vfill
Cibulka, Hladky, Lacroix and Wagner (2008) gave a completely bijective proof
that utilizes some natural 2:2 and 2:1 correspondances.

\vfill
\Remph{Difficulty:} Some terms on the left \Remph{cancel} and some
reduce to terms with coefficients $\pm 2$.
\vfill
\end{frame}



\begin{frame}
\frametitle{Linear Alg./Oriented Matroid Proof of Rayleigh's Identity}
Let $R$ be the transfer resistance matrix for 2 ports across $e$ and $f$.
Our result implies that
\[
\det R = \left|\begin{array}{cc} R_{ee} & R_{ef} \\ R_{fe} & R_{ff} \end{array}\right|
= \alert{+} \frac{T_{G/e/f}}{T_G}
\]
It and better-known results tell us
\[
R_{ee} = \frac{T_{G/e}}{T_G};\;\;R_{ff} = \frac{T_{G/f}}{T_G};\;\;
R_{ef}=R_{fe}=\frac{ T^+_{G/e \text{ \& } G/f} - T^-_{G/e \text{ \& } G/f} }{T_G}
\]
%Rayleigh's identity 
$T_{G/f}T_{G/e} - T_GT_{G/e/f} = \left( T^+_{G/e \text{ \& } G/f} - T^-_{G/e \text{ \& } G/f} \right)^2$
is immediate after substituting these into
\[
\det R = R_{ee}R_{ff}-(R_{ef})^2
\]
\alert{The $+$ follows from physical grounds if the $g_e, r_e \geq 0$.  Our
characterization and proof are combinatorial.}
\end{frame}

\begin{frame}
\frametitle{New Rayleigh's Identities!}

The same method generates identities from
\[
\left|
\begin{array}{ccc} R_{ee} & R_{ef} & R_{eg} \\ 
                   R_{fe} & R_{ff} & R_{fg} \\
                   R_{ge} & R_{gf} & R_{gg}
\end{array}\right|
= \alert{+} \frac{T_{G/e/f/g}}{T_G}
\]
ETC...

\vfill

(Applications???)

\vfill

\Remph{Might the same methods address a much harder problem:
The same inequality for } forests \Remph{ instead of 
spanning trees?}

\vfill
\end{frame}

\section{Correlation in Forests}

\begin{frame}
\frametitle{Negative Correlation in Forests Conjecture}
Conjecture[Grimmett-Winker, Kahn and Pemantle]:
For every pair of edges $p,q$ in a graph, let
$F_p$ ($F_q$) enumerate all forests with $p$ ($q$ resp.),
$F_{pq}$ enumerate those with both $p$ and $q$, and
$F$ enumerate all forests.
\[
F_pF_q-FF_{pq}\geq 0 
\text{ if }x_e,x_p,x_q\geq 0\text{ for all }e\in E\text{ and }p,q.
\]
Consider random forests, take for $e, p, q$: $x_e=Pr(e)/(1-Pr(e))$.
\[
\text{Corr}(p\in RF,q\in RF) =
\frac{-1}{\sigma^2}\left(\frac{F_p}{F}\frac{F_q}{F}-
\frac{F_{pq}}{F}\right).
\]
\[\text{By calculation: }
W(G)=\mathcal{F}_{\pnsub}\mathcal{F}_{\qnsub}-
\mathcal{F}_{\text{all}}\mathcal{F}_{\pnqnsub}=
\frac{F_pF_q-FF_{pq}}{x_px_q}
\]
Example where the conjecture is true: If $p$ and $q$ are in series, then
$\mathcal{F}_{\pnsub}=\mathcal{F}_{\qnsub}=\mathcal{F}_{\text{all}}$
and 
$\mathcal{F}_{\pnqnsub}=\mathcal{F}_{\text{all}}-\mathcal{F}_{\pqeggsub}$,
so 
\[
W(G)=\mathcal{F}_{\text{all}}\mathcal{F}_{\text{all}}-
\mathcal{F}_{\text{all}}(\mathcal{F}_{\text{all}}-\mathcal{F}_{\pqeggsub})
= \mathcal{F}_{\text{all}}\mathcal{F}_{\pqeggsub} \geq 0
\]
\end{frame}

\begin{frame}
\frametitle{Wagner's Conjectured Formula}
\[
W(G)=\mathcal{F}_{\pnsub}\mathcal{F}_{\qnsub}-
\mathcal{F}_{\text{all}}\mathcal{F}_{\pnqnsub}=?
\sum_{A\subseteq E}x^A\left((\sum \pm x^L)^2\right)
\]
For each $A\subseteq E$, the sum is over (some?) forests $L$,
$L\cap A=\emptyset$, for which there is some $B\subseteq L$,
$A\cup B\cup\{p,q\}$ \Remph{is a circuit}.

We call $A\cup B\cup\{p,q\}$ a \Remph{linking circuit}.
The signs are related to the \Remph{relative orientations} of 
$p$ and $q$ in the linking circuit.

$W^?(G)$ denotes Wagner's formula.  Since $L\cap A=\emptyset$,
any $x_e^2$ can only come from one or more $(\sum\cdots)^2$ expressions.

We sketch a $P$-ported Tutte decomposition approach
to the conjecture.
(It remains unproven.)
\end{frame}

\begin{frame}
\frametitle{Towards Tutte Decompositions for $p,q$-in-Forest Correlation}
\[
W(G)=\mathcal{F}_{\pnsub}\mathcal{F}_{\qnsub}-\mathcal{F}_{\text{all}}\mathcal{F}_{\pnqnsub}
\]
Each of the four $\mathcal{F}_R$ satisfy separator-strong $P$-ported
Tutte equations with $P=\{p,q\}$:
\[
\mathcal{F}_R(G)=x_e\mathcal{F}_R(G/e)+\mathcal{F}_R(G\setminus e)
\text{ for non-sep. }e\in E.
\]
\[
\mathcal{F}_R(G)=x_e\mathcal{F}_R(G/e)
\text{ for isthmus }e\;\;\;
\mathcal{F}_R(G)=\mathcal{F}_R(G\setminus e)
\text{ for loop }e.
\]
Therefore;
\[
W(G) = W(G\setminus e) + x_e^2 W(G/e) + x_e B(G/e,G\setminus e)
\]
Where:
\[
\begin{split}
B(G_1,G_2) =\;\;
&\mathcal{F}_{\pnsub}(G_1)\mathcal{F}_{\qnsub}(G_2)+
\mathcal{F}_{\qnsub}(G_1)\mathcal{F}_{\pnsub}(G_2)\\
-&\mathcal{F}_{\text{all}}(G_1)\mathcal{F}_{\pnqnsub}(G_2)-
\mathcal{F}_{\pnqnsub}(G_1)\mathcal{F}_{\text{all}}(G_2)
\end{split}
\]
\end{frame}

\begin{frame}
\frametitle{Towards Tutte Decompositions...}
\[
W(G) = W(G\setminus e) + x_e^2 W(G/e) + x_e B(G/e,G\setminus e)
\]
\[
\begin{split}
\text{Where }B(G_1,G_2) =\;\;
&\mathcal{F}_{\pnsub}(G_1)\mathcal{F}_{\qnsub}(G_2)+
\mathcal{F}_{\qnsub}(G_1)\mathcal{F}_{\pnsub}(G_2)\\
-&\mathcal{F}_{\text{all}}(G_1)\mathcal{F}_{\pnqnsub}(G_2)-
\mathcal{F}_{\pnqnsub}(G_1)\mathcal{F}_{\text{all}}(G_2)
\end{split}
\]
\[
B(G_1,G_2) = x_eB(G_1/e,G_2)+B(G_1\setminus e,G_2)
\text{ if }e\not\in \{p, q\}\text{ is non-sep. in }G_1.
\]
\[
B(G_1,G_2) = (x_e+1)B(G_1/e,G_2)
\text{ if }e\not\in \{p, q\}\text{ is an isthmus in }G_1.
\]
\[
B(G_1,G_2) = B(G_1\setminus e,G_2)
\text{ if }e\not\in \{p, q\}\text{ is a loop in }G_1.
\]
\Remph{And similarly for $G_2$}.

Values on indecomposibles: $W(\pqegg)=1$, $W(\text{ 4 others})=0$.

Values of $B$ on \Remph{pairs} of indecomposibles are
expressed in a symmetric 5x5 table.
\end{frame}


\begin{frame}
\frametitle{An Approach to an Inductive Proof}
Verify $W(G)=W^?(G)$ for small cases.  Use induction
to verify $W(G)=W^?(G)$ for separable $G$.
\[
W(G)=x_e^2W(G/e)+W(G\setminus e)+x_eB(G/e,G\setminus e)
\]
\[
=\text{ (by induction) }
x_e^2W^?(G/e)+W^?(G\setminus e)+x_eB(G/e,G\setminus e)
\]
From a combinatorial definition (of $\sigma(A,L_1,L_2)=\sigma_G$)
\[
W^?(G)=\sum_{A\subseteq E}x^A(\sum_L{\pm x^L})^2=
\sum_{A\subseteq E}x^A(\sum_{L_1,L_2\subseteq E\setminus A}
(\sigma(A,L_1,L_2) x^{L_1}x^{L_2}))
\]
extract a combinatorial description of the terms with degree 1 in $x_e$.
There are 3 kinds: $x_e$ in $x^A$, $x_e$ in $x^{L_1}$, $x_e$ in $x^{L_2}$.
This will be completed to a proof if we verify (combinatorially)
that all non-separable $G$:
\[
B(G/e,G\setminus e)=\frac{\partial W^?(G)}{\partial x_e}|_{x_e=0}
\]

%Example: Suppose $W(G')=W^?(G')$ is true for $G'$, and
%$G=G'\oplus G_0$ where $\{p,q\}\cap G'=\emptyset$.

\end{frame}

\begin{frame}{Bi-Tutte Decomposition for $B$}
Remember, $B$ is well defined and 
\[
B(G/e,G\!\setminus\! e)=\sum_{F_1,F_2:\text{ forests in }E} x^{F_1}x^{F_2}
B(G/e/F_1|P,G\!\setminus\! e/F_2|P).
\]

Since $B(G/e/F_1|P,G\!\setminus\! e/F_2|P)=
B(G/F_1/e|P,G/F_2\!\setminus\! e|P)=
B((G/F_1|P\cup e)/e,(G/F_2|P\cup e)\!\setminus\!e)$,
we can use \Remph{ONE} $\{p,q,e\}$-ported Tutte tree,
and then at each leaf labelled by $Q$, append at most
two branches, ``Left'' for $Q/e$ and ``Right'' for 
$Q\setminus e$.

The monomials have the form $c_{A,B}x^{2A}x^B$.
There are at most $2^{|B|}$ Left/Right pairs of 
new leaves ($Q_L/e$, $Q_R\setminus e$); the pair
contributes $B(Q_L/e,Q_R\setminus e)$ to $c_{A,B}$.

This at least helps us extract the coefficients of
particular terms, and how various partitions of
$B$ contribute....
\end{frame}

\end{document}


\begin{frame}
\[
\mathcal{F}_{\{p,q\}}=
[\pzqz]\mathcal{F}_{\pzqzsub} +
[\pzqn]\mathcal{F}_{\pzqnsub} +
[\pnqz]\mathcal{F}_{\pnqzsub} +
[\pnqn]\mathcal{F}_{\pnqnsub} +
[\pqegg]\mathcal{F}_{\pqeggsub}
\]
\[
\mathcal{F}_{\pnsub}=\mathcal{F}_{\pnqnsub}+\mathcal{F}_{\pnqzsub}+
\mathcal{F}_{\pqeggsub}
\]
\[
\mathcal{F}_{\qnsub}=\mathcal{F}_{\pnqnsub}+\mathcal{F}_{\pzqnsub}+
\mathcal{F}_{\pqeggsub}
\]
\[
W(G)=\mathcal{F}_{\pnsub}\mathcal{F}_{\qnsub}-\mathcal{F}\mathcal{F}_{\pnqnsub}
\]
\[
W(G)\geq 0\text{ whenever }x_e\geq 0\text{ for all }e\in E(G)
\]
\end{frame} 
































%%%%%%%%% REMOVED %%%%%%%%%%%%%%%%%%%%%%%%%%%%%%%%
\begin{frame}{Solution---Theorem}
\begin{theorem}[After Zaslavsky, Bollobas-Riordan, Ellis-Monaghan-Traldi]
$\mathcal{F}$ given with parameters and initial values in an $R$-module $M$
\Remph{has a Tutte function} if and only if the following equations
are satisfied whenever they arise from a member $G\in\mathcal{F}$ with
two or three $e, f, g$ elements not in $P$:
\begin{enumerate}
\item With $e,f$ an isolated 2-circuit in $G$, 
$I(G\setminus\{e,f\})(x_eY_f+y_eX_f)=
I(G\setminus\{e,f\})(x_fY_e+y_fX_e)$.
\item
With $e,f,g$ an isolated 3-circuit in $G$, 
$I(G\setminus\{e,f,g\})X_g(x_ey_f+y_eX_f)=
I(G\setminus\{e,f,g\})X_g(x_fy_e+y_fX_e)$. 
\item 
With $e,f,g$ an isolated 3-cocircuit in $G$, 
$I(G\setminus\{e,f,g\})Y_g(x_eY_f+y_ex_f)=
I(G\setminus\{e,f,g\})Y_g(x_fY_e+y_fx_e)$. 
\item With $e,f$ in series and not isolated (from elements 
in $P$) in $G$,
$I(G\setminus\{e,f\})(x_eY_f+y_eX_f)=
I(G\setminus\{e,f\})(x_fY_e+y_fX_e)$.
\item With $e,f$ in parallel and not isolated 
in $G$,
$I(G/e\setminus f)(x_eY_f+y_ex_f)=
I(G/e\setminus f)(x_fY_e+y_fx_e)$.
\end{enumerate}
\end{theorem}
\end{frame}

This $\mathcal{F}_{\input{p1.pdf_t}}$ 
$\mathcal{F}_{\input{q1.pdf_t}}$
$\mathcal{F}_{\input{p0.pdf_t}}$
$\mathcal{F}_{\input{q0.pdf_t}}$
$\mathcal{F}_{\input{p1q1.pdf_t}}$
$\mathcal{F}_{\input{p0q1.pdf_t}}$
$\mathcal{F}_{\input{p1q0.pdf_t}}$
$\mathcal{F}_{\input{pqegg.pdf_t}}$
 notation is s(l)ick. Even $\pqegg$ inside text.

\begin{frame}
\frametitle{A Tutte Tree with Bi-Tutte Trees at Each Node}
\begin{block}{Tree for $W(G)$}
values on indecomposibles
\end{block}
\begin{block}{Bi-Tree for $R(G_1,G_2)$}
\end{block}
\begin{block}{One Tree for $R(G/e,G\setminus e)$}
\end{block}
\end{frame}

