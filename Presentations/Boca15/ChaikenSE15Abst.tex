\documentclass[12pt]{article}
  \oddsidemargin 0em
  \evensidemargin 0em
  \topmargin 0em
  \textwidth 40em
  \parindent 0pt

\begin{document}
 
  \subsubsection*{Ported or Relative Oriented Matroids and Electric Circuits}

  Seth Chaiken, CS Dept., University at Albany
  \medskip

The idea of holding back from deletion/contraction of every element
in a distinguished set during a Tutte function decomposition has a long 
history of multiple origins and applications.  
What we call a ``ported'' Tutte decomposition or matroid has also been 
termed ``pointed'' (for a single distinguished element), ``set-pointed'', 
%``restricted'',
and ``relative''; it occurs in the representation of 
``a matroid perspective.''  
It is now known that the entire elementary theory (activities, rank-nullity
generating functions, etc) of 
Tutte decomposition, even when the element decomposition operations
are individually parametrized or ``colored'', 
goes through into the ported generalization.

We have observed that the analogous generalizations apply to 
\emph{oriented} matroids.  Ported parametrized 
oriented graphic matroids are essential 
for our 
motivating application: Electrical networks analyzed with combinatorial 
methods stemming from their solution by tree enumeration due to Kirchhoff and 
Maxwell, commonly formulated with the Matrix Tree Theorem.  Ports 
enable us to pose the relevent generalizations.  The talk will
briefly survey our results such as an exterior algebraic Tutte function 
to generalize tree count and how to derive a family of 
Rayleigh-like inequalities.   


  \medskip

Keywords:  relative, ported, set-pointed, oriented matroid, matroid perspective,
parametrized, colored, Tutte polynomial, exterior algebra, Rayleigh inequality, 
electrical network, matrix tree theorem
\end{document}
