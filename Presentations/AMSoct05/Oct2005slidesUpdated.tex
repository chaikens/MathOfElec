%% BEGIN semsamp2.tex
% This is a sample document for seminar.sty, v0.93 (and maybe later).
%
% This file contains both landscape and portrait mode slides.
% Choose one of the following to print them out:
%  - If using PSTricks, try the semcolor style option.
%  - If using Rokicki's dvips, try the semrot style option.
%  - To print the landscape slides, put \landscapeonly in the preamble.
%    To print the portrait slides, include the portrait style option and
%    put \portraitonly in the preamble.
%
%
\documentclass[%
  slidesonly,%  Try notes or notesonly instead.
  %notes,%      Use instead of slidesonly to typeset the notes and slides.
  %notesonly,%  Use instead of slidesonly to typeset notes.
  %semcolor,%   Try me if using PSTricks.
  %semrot,%     Try me if using Rokicki's dvips.
  %semhelv,%    Try me if using a PostScript printer.
  %article,%    Try me.
  %portrait,%   Try me.
  %sem-a4,%     Try me if using A4 paper.
  semlayer,%     This must be included, but you need the semcolor option to
  amsmath
  ]{seminar}                                  % actually see the overlays.

\usepackage{epsfig}
\usepackage{amsmath}
\usepackage{graphics}

%\slidesmag{5}
%default slidesmag is 4
\articlemag{1}


%\extraslideheight{1in}

%\twoup                     % Try me for twoup printing.

%\portraitonly              % To print only portrait slides
%\landscapeonly             % To print only landscape slides

%\notslides{\ref{questions}-7,1}   %Try me: The slides are omitted.
%\onlyslides{\ref{questions}-7,1}  %Try me: Only these slides are included.
%\onlynotestoo                     %Try me: For selecting notes as well.

\colorlayers{red,blue}      % Try deleting this if using the semcolor option,
                            % to get \blue and \red to use PostScript color.

%\overlaysfalse             % Suppress overlays with semcolor option.
%\layersfalse               % Suppress color layers with semcolor option.

\rotateheaderstrue          % Try this out if using rotation macros.


\title{}
\author{}
\date{\today}

\newcommand{\sref}[1]{SLIDE \ref{#1}}
\newcommand{\heading}[1]{\begin{center}\large\bf #1\end{center}}

\newcommand{\dunder}[1]{\underline{\underline{#1}}}

%   Disjoint Union
%\newcommand{\dunion}{\uplus}
\newcommand{\dunion}
%{\mbox{\hbox{\hskip4pt$\cdot$\hskip-4.62pt$\cup$\hskip2pt}}}
%{\mbox{\hbox{\hskip6pt$\cdot$\hskip-5.50pt$\cup$\hskip2pt}}}
{\mbox{\hbox{\hskip0.45em$\cdot$\hskip-0.44em$\cup$\hskip0.2em}}}
%{\mbox{\hbox{\hskip0.45em$+$\hskip-0.70em$\cup$\hskip0.3em}}}
%
% Dot inside a cup.
% If there is a better, more Latex like way 
% (more invariant under font size changes) way,
% I'd like to know.

\newcommand{\Bases}[1]{\ensuremath{{\mathcal{B}}(#1)}}
\newcommand{\Reals}{\ensuremath{\mathbb{R}}}
\newcommand{\FieldK}{\ensuremath{K}}
\newcommand{\Perms}{\ensuremath{\mathfrak{S}}}
%\newcommand{\rank}{{\rho}}% {{\mbox{rank}}}
%\newcommand{\Rank}{{\rho}}% {{\mbox{rank}}}
\newcommand{\rank}{{\mbox{r}}}% {{\mbox{rank}}}
\newcommand{\Rank}{{\mbox{r}}}% {{\mbox{rank}}}
\newcommand{\Card}[1]{\ensuremath{{\left|#1\right|}}}
\newcommand{\ext}[1]{\ensuremath{\mathbf{#1}}}
%\newcommand{\extvee}{\ensuremath{\mathbf{\vee}}}
\newcommand{\extvee}{\;\;}
\newcommand{\Plucker}{Pl\"{u}cker\ }

% Set Complement
% command to mess with overline, bar or custom 
% alternatives for sequence or set complement
%
\newcommand{\scomp}[1]{\ensuremath{\overline{#1}}}
%\newcommand{\scomp}[1]{\ensuremath{\bar{#1}}}


%
%   Put a symbol for a matroid in a box, or brackets
%\newcommand{\MVAR}[1]{{\boxed{#1}\;}}
\newcommand{\MVAR}[1]{{[#1]\;}}

\newcommand{\UNION}{\cup} %try to make this bold.

\setlength{\slidewidth}{9.5in}
\setlength{\slideheight}{7.5in}

%\newpagestyle{MH}%
%  {University at Albany Computer Science Dept.\hfil\thepage}{}
%\pagestyle{MH}
\pagestyle{empty}
\slideframe{none}

\begin{document}

%\maketitle          % This won't show up when \onlynotestoo is in effect.

%\begin{slide}
%  \ifslidesonly              % Title slide only for slidesonly selection.
%    \maketitle
%    \addtocounter{slide}{-1}
%    \slidepagestyle{empty}
%  \fi
%\end{slide}

\begin{slide}
\begin{center}
Ported alias Set-Pointed and Non-scalar Tutte Functions \\

Seth Chaiken\\
Computer Science Department\\[-0.8ex]
University at Albany, State Univ. of New York, USA\\[-0.8ex]
\texttt{sdc@cs.albany.edu}\\
\verb|http://www.cs.albany.edu/~sdc/Matroids|\\

Presented October 9, 2005

Corrected and extended November 1, 2005

\end{center}
\end{slide}

\begin{slide}
Traditionally a \textbf{Tutte Function} $F$
\[
F:\left\{\begin{array}{l} \mbox{Matroids } \\ \mbox{(or Graphs)} \end{array}\right.
\rightarrow \mbox{Commutative Ring\ \ } (R,+,\cdot)
\]
\begin{equation}
F(\mathcal{N}) = g_eF(\mathcal{N}/e) + r_eF(\mathcal{N}\setminus e)
\tag{A}
\end{equation}
for all $e$ not a loop or inthmus.  $g_e$, $r_e$ are parameters or 1.
\begin{equation}
F(\mathcal{N}_1\oplus\mathcal{N}_2)
=F(\mathcal{N}_1)\cdot F(\mathcal{N}_2)
\tag{M}
\end{equation}
We survey results where

\noindent (1) $(R,+,\cdot)$ is replaced by discrete or
other algebraic structures (Matroids!), or ``Matroids'' is replaced by 
matroid presentations;

\noindent \textbf{and} (2) given a distinguished subset $P$ (\textbf{ports} or
\textbf{set of points}), (A) is restricted to $e\not\in P$.
\end{slide}

\begin{slide}
\begin{enumerate}
\item $F:\mbox{Matroids}\rightarrow\mbox{Matroids}$.
\item $F:\mbox{Extensors}\rightarrow\mbox{Extensors}$.\\
The multiplication of exterior (aka Cayley, Grassmann) algebra
is anticommutative: $\mathbf{N}_1\mathbf{N}_2=(\mathbf{N}_1\wedge\mathbf{N}_2)=
\mathbf{N}_2\mathbf{N}_1(-1)^{r(\mathbf{N}_1)r(\mathbf{N}_2)}$
\item $F:\mbox{(Oriented) Matroids}\rightarrow\mbox{Commutative Ring}$ where 
a substitution expresses (2) in the unimodular (regular) oriented case.
\item Algebraic expressions of (3) for $P$-unions and generalized parallel
connections over $P$.
\end{enumerate}
\vspace{1em}
In all cases, ``ports'', ``set of points'' $P$, for which deletion/contraction is
forbidden, must satisfy $P\neq\emptyset$ for interesting results.

\end{slide}

\begin{slide}
Result 1: (Construction of Recski, Weinberg 70's; new observation here)
\begin{center}
Given: Matroid $\mathcal{N}(P,E)$ has ground set $P\dunion E$.\\
$P_V$ and $P_I$ are two disjoint copies of $P$:\\
$P_V=\{p_V: p\in P\}$; $P_I=\{p_I: p\in P\}$.
\end{center}
\[
\text{(The matroid!)\ }\mathcal{M}_E(\mathcal{N})(P_I\dunion P_V)
:\equiv \big(\mathcal{N}(P_I,E))\cup\mathcal{N}^{\perp}(P_V,E)\big)/E
\]
satisfies
\[
\mathcal{M}_E(\mathcal{N}) = \mathcal{M}_{E'}(\mathcal{N}/ e) \cup_{\mathcal{B}}
\mathcal{M}_{E'}(\mathcal{N}\setminus e)\text{ if }e\not\in P\text{ and }E'=E\setminus e,
\]
where $\cup_{\mathcal{B}}$ denotes union of matroid basis collections; and
\[
\mathcal{M}_{E}\big(\mathcal{N}_1(E_1,P_1)\oplus\mathcal{N}_2(E_2,P_2)\big)=
\mathcal{M}_{E_1}\big(\mathcal{N}_1(E_1,P_1)\big)\oplus
\mathcal{M}_{E_2}\big(\mathcal{N}_2(E_2,P_2)\big).
\]
where $\oplus$ denotes matroid direct sum.

Proof:  $E$ is independent in $\mathcal{N}(P_I,E)\cup\mathcal{N}^{\perp}(P_V,E)$, so
$B$ is a basis of $\mathcal{M}_E(\mathcal{N})$ iff $A\subseteq E$, 
$B\dunion A \dunion (E\setminus A)$ is a basis of the union; thence either $e\in A$ or $e\in (E\setminus A)$.
\end{slide}

\begin{slide}
Result 2. applies to a given
\textbf{decomposible}  
%element 
($\mathbf{N}(P,E)=\mathbf{v}_1\mathbf{v}_2\cdots\mathbf{v}_r$ of
vectors)
in the \textbf{exterior algebra} $\mathcal{E}$ over vector space
$k(P\dunion E)$ with distinguished basis $(P\dunion E)$.

We say an \textbf{extensor} is a decomposible element in $\mathcal{E}$.

Let $P\dunion E$ label the columns of matrix $N$ (full row rank $r$).

%Think that $N$ presents a rank $r$ (oriented) matroid by (signed) linear dependencies
%among the columns.

The (oriented) matroid of column dependencies (rank $r$) is determined by 
the \textit{row space} of $N$ \hspace{2em}$\leftrightarrow$ $k$-multiples of one extensor
$\mathbf{N}$, such as\\
$\mathbf{N}=\wedge$(row vectors of $N$)
$\in$ $\mathcal{E}$.

$\mathcal{E}$ is quotient of the assoc. algebra 
generated by $k(P\dunion E)$
modulo the ideal generated by $\mathbf{v}^2$, $\mathbf{v}\in k(P\dunion E)$.
$\mathcal{E}$ has dimension $2^{|P\dunion E|}=2^n$. 

$\mathcal{E}$ is graded: At rank $r$, $\mathcal{E}_r$ has 
dim. $\binom{n}{r}$, $r=0,1,\ldots,n$.

Natural coordinates of extensor $\textbf{N}(P,E)$ are the $r\times r$ 
\textit{determinants} in matrix $N$.

\end{slide}

\begin{slide}
Exterior \textit{sum} can be expressed by addition of the expansion coefficients
under the basis of all $2^n$ subsets of $P\dunion E$, (each subset with an element order fixed).

But, the sum of two extensors is not necessarilly \textit{decomposible}, not necessarilly
an \textit{extensor}.

The exterior \textit{product} of extensors for disjoint subspaces represents the
\textbf{subspace join}.

When a ground set $S$ is given (like $S=E\dunion P$, distinguished basis for 
$k(E\dunion P)$), \textbf{deletion} and  \textbf{contraction} of $e\in S$, 
and \textbf{dualization}, known from multilinear algebra, represent
the corresponding (oriented) matroid operations.

We define deletion/contraction so 
\[
\mathbf{N} = \mathbf{N}\setminus e + (\mathbf{N}/e)\mathbf{e}
\]

Dualization: Copy the oriented matroid chirotope dualization formula
(also known as Hodge star): Coefficients (determinants!)
$\mathbf{N}^{\perp}[X]=\mathbf{N}[\overline{X}]\epsilon(\overline{X},X)$.

\end{slide}
\begin{slide}

Our Tutte-like function $\mathbf{M}_E(\mathbf{N}):\text{Extensors}\rightarrow\text{Extensors}$.

Given $N$ (matrix), construct $N^\perp$ so their row spaces are orthogonal
complements ($N^\perp$ presents the (oriented) dual of the matroid from $N$).

Form the matrix:  ($G=\mbox{diag}(g_e)$, $R=\mbox{diag}(r_e)$)
\[
M = \left[\begin{array}{c|c|c} N(P)  &  0  &  N(E)G \\  \hline
0  & N^{\perp}(P)  &  N^{\perp}(E)R \end{array}\right]
\]
with columns labelled by $P_I\dunion P_V\dunion E$.

Extensor $\mathbf{M}$ over $k[g_e, r_e](P_V\dunion P_I \dunion E)$
is the product of $M$'s \textbf{row vectors}, and define $\mathbf{M}_E(\mathbf{N})$ by:
\[
\mathbf{M} = \mathbf{M}_E(\mathbf{N})\mathbf{e_1}\mathbf{e_2}\cdots\mathbf{e}_{|E|} + (\cdots) 
\]


\end{slide}

\begin{slide}
Result 2: (2003)
\begin{align*}
   \epsilon(PE)\;\mathbf{M}_E(\mathbf{N}(P,E))&=\\
&\epsilon(PE')\;\big(g_e\mathbf{M}_{E'}(\mathbf{N}/e) 
                + r_e\mathbf{M}_{E'}(\mathbf{N}\setminus e)\big)
\end{align*}
\begin{align*}
   \epsilon(P_1P_2E)&\;\mathbf{M}_E(\;\mathbf{N}_1(P_1,E_1)\;\;\mathbf{N}_2(P_2,E_2)\;)
   =\\
&\epsilon(P_1E_1)\epsilon(P_2E_2)\;
   \mathbf{M}_{E_1}(\mathbf{N}_1(P_1,E_1))\;\mathbf{M}_{E_2}({N}_2(P_2,E_2))
\end{align*}
Corollory:
\begin{equation}
\label{MESubsetSum}
\epsilon(PE)\ext{M}_E(\ext{N})=\epsilon(P)\sum_
             {\begin{array}{c}
                A\subseteq E:
	\rank_{\ext{N}}{A}=\Card{A},\\
	\rank{\ext{N}}-\rank{(\ext{N}/A|P)}-
        \rank_{\ext{N}}{A}=0
               \end{array}}
    \ext{M}_\emptyset(\ext{N}/A|P) g_Ar_{\scomp{A}}.
\end{equation}
Compare: (with $u=0$ and $v=0$)
\[
R(\mathcal{N}(P,E))=\sum_{A\subseteq E}
	\MVAR{\mathcal{N}/A|P}g_Ar_{\scomp{A}}
	u^{\Rank{\mathcal{N}}-\Rank{[\mathcal{N}/A|P]}-\Rank{A}}
	v^{\Card{A}-\Rank{A}}.
\]

\end{slide}

\begin{slide}
Result 3:  The ported, parametrized corank-nullity polynomials
of oriented and non-oriented matroids 
\[
R_P(\mathcal{N}(P,E))=\sum_{A\subseteq E}
	\MVAR{\mathcal{N}/A|P}g_Ar_{\scomp{A}}
	u^{\Rank{\mathcal{N}}-\Rank{[\mathcal{N}/A|P]}-\Rank{A}}
	v^{\Card{A}-\Rank{A}}.
\]
satisfy the ported Tutte equations.

(In the invariant case, this $R$ is universal for Tutte $P$-invariants
of oriented and non-oriented matroids)

$\MVAR{\mathcal{N}/A|P}$ is a (commutative) \textit{monomial} whose factors 
\textit{are} \textbf{connected (oriented) matroids} over subsets of $P$.

(Las Vergnas ``Big Tutte Polynomial'' ('75,'99), oriented/parametrized by sdc.)

$R_P$ can distinguish some orientations of the same matroid when
$|P|\ge 2$.

\end{slide}

\begin{slide}
Result 4 pertains to three matroid combination operations:

Given $\mathcal{N}_1(P,E_1)$ and $\mathcal{N}_2(P,E_2)$ with
only elements $P$ in common.

(1.) Ported matroid union: $\mathcal{B}\big(\mathcal{N}_1\UNION\mathcal{N}_2\big) =
\{B_1\cup B_2 : B_i\in\mathcal{B}(\mathcal{N}_i)\}$

(2.) Duality conjugate $\UNION^*$ of $\UNION$.

(3.) Given that $P$ is a modular flat and a common submatroid
in $\mathcal{N}_1$ and $\mathcal{N}_2$, the generalized parallel connection.

When $|P|=1$, both dual $\UNION$ and generalized parallel connection REDUCE
to (one base point) parallel connection.


\end{slide}


\begin{slide}
$K$=polynomial ring containing $u$ and $v$.  
$K_P$=commutative $K$-module generated by monomials
$[q_i]$ signifying matroids $q_i$ over subsets of $P$, and $[\emptyset]=1$.

Recall ported corank-nullity polynomial $R_P:\text{Matroids}\rightarrow K_P$.

Result 4:  For each combination operation ($*_i,i=1,2,3$) there is a bilinear 
$M_i:K_P\times K_P\rightarrow K_P$ map and an $u^jv^k$-valued function $f_i$ such 
that 
\[
R_P(\mathcal{N}_1 *_i \mathcal{N}_2) =
f_i(\mathcal{N}_1,\mathcal{N}_2,\mathcal{N}_1*_i\mathcal{N}_2)
M_i\big(R_P(\mathcal{N}_1),R_P(\mathcal{N}_2)\big)
\]

In other words, once $*_i$ is determined on pairs of matroids over subsets of $P$,
the ported $R_P$ for $\mathcal{N}_1*_i\mathcal{N}_2$ can be calculated by
formal multiplication
\begin{center}
$f_i(..)R_P(\mathcal{N}_1)*_iR_P(\mathcal{N}_2)=f_i(..)M_i\big(R_P(\mathcal{N}_1),R_P(\mathcal{N}_2)\big)$
\\
and substitutions of 
$\big(1/f_i(q_a,q_b,q_a*_iq_b)\big)[q_a*_iq_b]\leftarrow[q_a]*_i[q_b]$.\\
\end{center}

\end{slide}

\begin{slide}
\textbf{Splitting Formulas} 
generalize $F(\mathcal{N}_1\oplus\mathcal{N}_2)=F(\mathcal{N}_1)F(\mathcal{N}_2)$\\
and solve this type of problem:

Given $\mathcal{N}_1(P\dunion E_1)$, $\mathcal{N}_2(P\dunion E_2)$, can we calculate 
the Tutte polynomial of 
$\mathcal{N}_1*_i\mathcal{N}_2(P\dunion{E_1}\dunion{E_2})$ from
Tutte polynomials of minors of 
$\mathcal{N}_1$ and $\mathcal{N}_2$ gotten by deletion/contraction of subsets of
$P$?

For one-point series and parallel connections, we can re-derive Brylawski's
(1971) formulas by the above bilinear method (1989).

For generalized parallel connection, with $P$ a modular flat and common restriction
in both $\mathcal{N}_1$ and $\mathcal{N}_2$, a splitting formula was obtained
by Bonin and de Mier (2004).  

Their formula, for the Tutte polynomial, 
is in terms of:
\begin{enumerate}
\item The lattice of flats $F\le P$, 
\item characteristic polys. of $P/F$, and 
\item Tutte polynomials of $\mathcal{N}_1/F$ and 
$\mathcal{N}_2/F$.
\end{enumerate}
\end{slide}

\begin{slide}
Our formulas are (1) for $R_P$ (Big Tutte polynomial)
and (2) are in terms of $R_P(\mathcal{N}_1)*_iR_P(\mathcal{N}_2)$.
Our bilinear forms' coefficients and $f$ in 
$R_P(\mathcal{N}_1*_i\mathcal{N}_2)=f_i(...)R_P(\mathcal{N}_1)*_iR_P(\mathcal{N}_2)$:
\begin{enumerate}
\item $*=\cup$ (1989)
\[
f(\mathcal{N}_1,\mathcal{N}_2,\mathcal{N}_1\cup\mathcal{N}_2) = 
u^{\Rank\mathcal{N}_1\cup\mathcal{N}_2-\Rank\mathcal{N}_1-\Rank\mathcal{N}_2}.
\]
\item $*=\cup^*$ (1989) (nullity $n\mathcal{N}=\Card{\mathcal{N}}-\Rank\mathcal{N}$)
\[
f(\mathcal{N}_1,\mathcal{N}_2,\mathcal{N}_1\cup^*\mathcal{N}_2) = 
v^{n(\mathcal{N}_1\cup^*\mathcal{N}_2)-n\mathcal{N}_1-n\mathcal{N}_2}
\]
\item Generalized Parallel $P$-Connection $*$ (1991)
Since $P$ is a modular flat, quotients 
$Q_i=\mathcal{N}_i/A_i|P$ correspond to flats in $P$,
and $[Q_i]*[Q_j]=(1/f)[Q']$ where $Q'$ is the quotient
corresponding to the join of those flats in $P$.
\[
f(\mathcal{N}_1,\mathcal{N}_2,\mathcal{N}_1*\mathcal{N}_2) = 
v^{n(\mathcal{N}_1*\mathcal{N}_2)-n\mathcal{N}_1-n\mathcal{N}_2}
\]
\end{enumerate}
\end{slide}

\begin{slide}
Brylawski's splitting formulas for $|P|=1$ (series/union and parallel/co-union conn.)
can be derived from our 
$R_P(\mathcal{N}_1)*R_P(\mathcal{N}_2)$ formula.

Can Bonin and de Meir's generalized parallel connection
splitting formula for all $|P|$ also be 
derived that way?  

Are there other expressions for the splitting function?
\end{slide}

\begin{slide}
Example $P=\{p\}$: The 2 matroids over $P$ are $[\mathcal{P}_i]$, rank $i=0,1$.

Problem: express $R_P(\mathcal{N}_1*\mathcal{N}_2)$ in terms of 
4 polynomials $R(\mathcal{N}_j\setminus p)$, $R(\mathcal{N}_j/p)$, $j=1,2$.

Tool 1: for various $\mathcal{N}$, write
$R_P(\mathcal{N}) = [\mathcal{P}_1]R^{(1)}(\mathcal{N}) + 
                   [\mathcal{P}_0]R^{(0)}(\mathcal{N})$
and try to express $R^{(0)}(\mathcal{N})$, $R^{(1)}(\mathcal{N})$ in terms
of $R(\mathcal{N}\setminus p)$, $R(\mathcal{N}/ p)$.

Solve the following equations:

\begin{minipage}{2.1in}
\begin{align*}
R(\mathcal{N}/ p) &= R_P(\mathcal{N})|_{ \begin{array}{c}
					\left[\mathcal{P}_0\right]\leftarrow u^0v^1\\
                                        \left[\mathcal{P}_1\right]\leftarrow u^0v^0
					\end{array}} \\
                  &= vR^{(0)} + R^{(1)}
\end{align*}
\end{minipage}
\begin{minipage}{2in}
\begin{align*}
R(\mathcal{N}\setminus p) =& R_P(\mathcal{N})|_{ \begin{array}{c}
					\left[\mathcal{P}_0\right]\leftarrow u^0v^0\\
                                        \left[\mathcal{P}_1\right]\leftarrow u^1v^0
					\end{array}}\\
                          =& R^{(0)} + uR^{(1)}
\end{align*}
\end{minipage}

Solution: (Brylawski 1971, Cor. 6.14)
\begin{align*}
R^{(0)}(\mathcal{N}) &= \frac{1}{1-uv}\big(R(\mathcal{N}\setminus p) - uR(\mathcal{N}/ p)\big)\\
R^{(1)}(\mathcal{N}) &= \frac{1}{1-uv}\big(R(\mathcal{N}/ p) - vR(\mathcal{N}\setminus p)\big)
\end{align*}
\end{slide}

\begin{slide}
Tool 2: $R(\mathcal{N}_1*\mathcal{N}_2)=
R_P(\mathcal{N}_1*\mathcal{N}_2)|_{\left[\mathcal{P}_0\right]\leftarrow 1+v;
				   \left[\mathcal{P}_1\right]\leftarrow 1+u}$
before substitution\begin{multline*}
=v^{n\mathcal{N}_1*\mathcal{N}_2-n\mathcal{N}_1-n\mathcal{N}_2}\\
\left(
   \left[\mathcal{P}_0\right]R^{(0)}(\mathcal{N}_1)
  +\left[\mathcal{P}_1\right]R^{(1)}(\mathcal{N}_1)
\right)
*
\left(
   \left[\mathcal{P}_0\right]R^{(0)}(\mathcal{N}_2)
  +\left[\mathcal{P}_1\right]R^{(1)}(\mathcal{N}_2)
\right)
\end{multline*}
\begin{multline*}
=v^{-1}\Big(
    \left[\mathcal{P}_0\right]*\left[\mathcal{P}_0\right]R^{(0)}(\mathcal{N}_1)R^{(0)}(\mathcal{N}_2)\\
+\left[\mathcal{P}_0\right]*\left[\mathcal{P}_1\right]
\big(R^{(0)}(\mathcal{N}_1)R^{(1)}(\mathcal{N}_2)+
      R^{(1)}(\mathcal{N}_1)R^{(0)}(\mathcal{N}_2)\big)\\
+\left[\mathcal{P}_1\right]*\left[\mathcal{P}_1\right]
      R^{(1)}(\mathcal{N}_1)R^{(1)}(\mathcal{N}_2)\Big)
\end{multline*}

\begin{multline*}
\left[\mathcal{P}_0\right]*\left[\mathcal{P}_0\right]\leftarrow v\left[\mathcal{P}_0\right]
\leftarrow v(v+1)\\
\left[\mathcal{P}_0\right]*\left[\mathcal{P}_1\right]\leftarrow 1\cdot\left[\mathcal{P}_0\right]
\leftarrow (v+1)\\
\left[\mathcal{P}_1\right]*\left[\mathcal{P}_1\right]\leftarrow 1\cdot\left[\mathcal{P}_1\right]
\leftarrow (u+1)
\end{multline*}
now rederives splitting formula of Brylawski (1971, Thm. 6.15), using
\[
R^{(0)}(\mathcal{N}_2)=\frac{1}{1-uv}\left(R(\mathcal{N}_2\setminus p) - uR(\mathcal{N}_2/p)\right)
\text{  etc.}
\]
\end{slide}



\end{document}


\begin{slide}
The row space of $N$, (within $k(P\dunion E)$) 
uniquely corresponds to the set of non-zero $k$-multiples of one 
decomposible (extensor) $\mathbf{N}$ $\in$ $\mathcal{E}_r$.

We can use $\mathbf{N}=\mathbf{v}_1\mathbf{v}_2\cdots\mathbf{v}_r$ where
$\mathbf{v}_i$ is the $i$-th row of matrix $N$:
\[
\mathbf{v}_i = \sum_{p\in P}N_{i,p}\mathbf{p} + \sum_{e\in E}N_{i,e}\mathbf{e}
\]
We then get
\[
\mathbf{N} = \sum_{B\in P\dunion E, |B|=r}N[B]\mathbf{b}_1\mathbf{b}_2\cdots\mathbf{b}_r,
\]
in which $N[B]$ is the $r\times r$ \textit{determinant} in $N$ with columns 
$b_1b_2\cdots b_r$ in that order.

These $\binom{n}{r}$ exterior products of vectors drawn from
$E\dunion P$, comprise one basis for $\mathcal{E}_r$.

\end{slide}










