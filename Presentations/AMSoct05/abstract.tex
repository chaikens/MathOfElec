\documentclass[12pt]{article}

\usepackage{amsmath}


\title{Ported alias Set-Pointed and Non-scalar Tutte Functions}

\author{Seth Chaiken\\
\small Computer Science Department\\[-0.8ex]
\small University at Albany, State Univ. of New York, USA\\[-0.8ex]
\small \texttt{sdc@cs.albany.edu}
}

\date{
Formatted:\today\\
\small Mathematics Subject Classifications: 52C40, 15A75, 05C50}

\begin{document}
\maketitle

\begin{abstract}
Natural generalizations of the Tutte polynomial and deletion/contraction
decomposition expressions are obtained when we restrict deletion/contraction
to elements not in a distinguished subset we call ports.  These expressions
include symbols that signify connected unoriented or oriented matroids on 
subsets of ports.  We survey new and old algebraic identities, product
operations and transformations on such expressions that correspond to
various operations on, and combinations of, unoriented or oriented matroids.
Some of the identities correspond to identities on particular kinds of
representations of the matroids.  For example, the ported Tutte equations 
interpreted for a certain ported matroid-to-matroid function correspond 
to a non-commutative variant of the ported Tutte equations 
interpreted for matroid realizations expressed in exterior algebra.  
Other examples involve ported matroid union,
its dual, and generalized parallel connection.  The topic is related
to resistive electrical network or random walk models, and to splitting 
formulas for Tutte polynomial computations on decomposed graphs.
\end{abstract}

\end{document}


