%% BEGIN semsamp2.tex
% This is a sample document for seminar.sty, v0.93 (and maybe later).
%
% This file contains both landscape and portrait mode slides.
% Choose one of the following to print them out:
%  - If using PSTricks, try the semcolor style option.
%  - If using Rokicki's dvips, try the semrot style option.
%  - To print the landscape slides, put \landscapeonly in the preamble.
%    To print the portrait slides, include the portrait style option and
%    put \portraitonly in the preamble.
%
%
\documentclass[%
  slidesonly,%  Try notes or notesonly instead.
  %notes,%      Use instead of slidesonly to typeset the notes and slides.
  %notesonly,%  Use instead of slidesonly to typeset notes.
  %semcolor,%   Try me if using PSTricks.
  %semrot,%     Try me if using Rokicki's dvips.
  %semhelv,%    Try me if using a PostScript printer.
  %article,%    Try me.
  %portrait,%   Try me.
  %sem-a4,%     Try me if using A4 paper.
  semlayer,%     This must be included, but you need the semcolor option to
  amsmath
  ]{seminar}                                  % actually see the overlays.

\usepackage{epsfig}
\usepackage{amsmath}
\usepackage{graphics}

%\slidesmag{5}
%default slidesmag is 4
\articlemag{1}


%\extraslideheight{1in}

%\twoup                     % Try me for twoup printing.

%\portraitonly              % To print only portrait slides
%\landscapeonly             % To print only landscape slides

%\notslides{\ref{questions}-7,1}   %Try me: The slides are omitted.
%\onlyslides{\ref{questions}-7,1}  %Try me: Only these slides are included.
%\onlynotestoo                     %Try me: For selecting notes as well.

\colorlayers{red,blue}      % Try deleting this if using the semcolor option,
                            % to get \blue and \red to use PostScript color.

%\overlaysfalse             % Suppress overlays with semcolor option.
%\layersfalse               % Suppress color layers with semcolor option.

\rotateheaderstrue          % Try this out if using rotation macros.


\title{}
\author{}
\date{\today}

\newcommand{\sref}[1]{SLIDE \ref{#1}}
\newcommand{\heading}[1]{\begin{center}\large\bf #1\end{center}}

\newcommand{\dunder}[1]{\underline{\underline{#1}}}

%   Disjoint Union
%\newcommand{\dunion}{\uplus}
\newcommand{\dunion}
%{\mbox{\hbox{\hskip4pt$\cdot$\hskip-4.62pt$\cup$\hskip2pt}}}
{\mbox{\hbox{\hskip6pt$\cdot$\hskip-5.50pt$\cup$\hskip2pt}}}
%
% Dot inside a cup.
% If there is a better, more Latex like way 
% (more invariant under font size changes) way,
% I'd like to know.

\newcommand{\Bases}[1]{\ensuremath{{\mathcal{B}}(#1)}}
\newcommand{\Reals}{\ensuremath{\mathbb{R}}}
\newcommand{\FieldK}{\ensuremath{K}}
\newcommand{\Perms}{\ensuremath{\mathfrak{S}}}
\newcommand{\rank}{{\rho}}% {{\mbox{rank}}}
\newcommand{\Rank}{{\rho}}% {{\mbox{rank}}}
\newcommand{\Card}[1]{\ensuremath{{\left|#1\right|}}}
\newcommand{\ext}[1]{\ensuremath{\mathbf{#1}}}
%\newcommand{\extvee}{\ensuremath{\mathbf{\vee}}}
\newcommand{\extvee}{\;\;}
\newcommand{\Plucker}{Pl\"{u}cker\ }

% Set Complement
% command to mess with overline, bar or custom 
% alternatives for sequence or set complement
%
\newcommand{\scomp}[1]{\ensuremath{\overline{#1}}}
%\newcommand{\scomp}[1]{\ensuremath{\bar{#1}}}


%
%   Put a symbol for a matroid in a box, or brackets
%\newcommand{\MVAR}[1]{{\boxed{#1}\;}}
\newcommand{\MVAR}[1]{{[#1]\;}}

\newcommand{\UNION}{\cup} %try to make this bold.

\setlength{\slidewidth}{9.5in}
\setlength{\slideheight}{7.5in}

%\newpagestyle{MH}%
%  {University at Albany Computer Science Dept.\hfil\thepage}{}
%\pagestyle{MH}
\pagestyle{empty}
\slideframe{none}

\begin{document}

%\maketitle          % This won't show up when \onlynotestoo is in effect.

%\begin{slide}
%  \ifslidesonly              % Title slide only for slidesonly selection.
%    \maketitle
%    \addtocounter{slide}{-1}
%    \slidepagestyle{empty}
%  \fi
%\end{slide}

\begin{slide}
\begin{center}
Ported alias Set-Pointed and Non-scalar Tutte Functions \\

Seth Chaiken\\
Computer Science Department\\[-0.8ex]
University at Albany, State Univ. of New York, USA\\[-0.8ex]
\texttt{sdc@cs.albany.edu}\\
\texttt{http://www.cs.albany.edu/~sdc/Matroids}\\

October 9, 2005

\end{center}
\end{slide}

\begin{slide}
Traditionally a \textbf{Tutte Function} $F$
\[
F:\left\{\begin{array}{l} \mbox{Matroids } \\ \mbox{(or Graphs)} \end{array}\right.
\rightarrow \mbox{Commutative Ring\ \ } R
\]
\begin{equation}
F(\mathcal{N}) = g_eF(\mathcal{N}/e) + r_eF(\mathcal{N}\setminus e)
\tag{A}
\end{equation}
for all $e$ not a loop or inthmus.  $g_e$, $r_e$ are parameters or 1.
\begin{equation}
F(\mathcal{N}_1\oplus\mathcal{N}_2)
=F(\mathcal{N}_1)\cdot F(\mathcal{N}_2)
\tag{M}
\end{equation}
We survey results where

\noindent (1) $(R,+,\cdot)$ is replaced by discrete (Matroids!) or
other algebraic structures; or ``Matroids'' is replaced by 
matroid presentations.

\noindent \textbf{and} (2) Given a distinguished subset $P$ (``ports'' or
``set of points''), (A) is restricted to $e\not\in P$.
\end{slide}

\begin{slide}
\begin{enumerate}
\item $F:\mbox{Matroids}\rightarrow\mbox{Matroids}$.
\item $F:\mbox{Extensors}\rightarrow\mbox{Extensors}$.\\
Anticommutative multiplication: 
$\mathbf{E}_1\mathbf{E}_2=\mathbf{E}_2\mathbf{E}_1(-1)^{r(\mathbf{E}_1)r(\mathbf{E}_2)}$
\item $F:\mbox{(Oriented) Matroids}\rightarrow\mbox{Commutative Ring}$ where 
a substitution expresses (2) in the unimodular (regular) oriented case.
\item Algebraic expressions of (3) for $P$-unions and generalized parallel
connections over $P$.
\end{enumerate}
\end{slide}

\begin{slide}
Result 1:

Matroid $\mathcal{N}(P,E)$ has ground set $P\dunion E$.

$P_V$ and $P_I$ are two disjoint copies of $P$:\\
$P_V=\{p_V: p\in P\}$; $P_I=\{p_I: p\in P\}$.
\[
\text{(The matroid!)\ }\mathcal{M}_E(\mathcal{N})(P_I\dunion P_V)
:\equiv \big(\mathcal{N}(P_I,E))\cup\mathcal{N}^{\perp}(P_V,E)\big)/E
\]
satisfies
\begin{equation}
\mathcal{M}_E(\mathcal{N}) = \mathcal{M}_{E'}(\mathcal{N}/ e) \cup_{\mathcal{B}}
\mathcal{M}_{E'}(\mathcal{N}\setminus e)\text{ where }e\not\in P\text{ and }E'=E\setminus e.
\end{equation}
\begin{equation}
\mathcal{M}_{E}\big(\mathcal{N}_1(E_1,P_1)\oplus\mathcal{N}_2(E_2,P_2)\big)=
\mathcal{M}_{E_1}\big(\mathcal{N}_1(E_1,P_1)\big)\oplus
\mathcal{M}_{E_2}\big(\mathcal{N}_2(E_2,P_2)\big).
\end{equation}
where $\cup_{\mathcal{B}}$ denotes union of matroid basis collections and \\
$\oplus$ denotes matriod direct sum.

Proof:  $E$ is independent in $\mathcal{N}(P_I,E)\cup\mathcal{N}^{\perp}(P_V,E)$, so
$B$ is a basis of $\mathcal{M}_E(\mathcal{N})$ iff $A\subseteq E$, 
$B\dunion A \dunion (E\setminus A)$ is a basis of the union; thence either $e\in A$ or $e\in (E\setminus A)$.
\end{slide}

\begin{slide}
Result 2. applies to a given
\textbf{Decomposible}  
%element 
($=\mathbf{v}_1\mathbf{v}_2\cdots\mathbf{v}_r$ of
vectors)
in the exterior algebra over vector space
$\mathbf{k}(P\dunion E)$ with distinguished basis $(P\dunion E)$.
\[
\mathbf{N}(P,E): \text{\textbf{Extensor} over }\mathbf{k}(P\dunion E).
\]

Let $P\dunion E$ label the columns of matrix $N$ (full row rank $r$).

%Think that $N$ presents a rank $r$ (oriented) matroid by (signed) linear dependencies
%among the columns.

The (oriented) matroid of column dependencies (rank $r$) is determined by 
the \textit{row space} of $N$ $\leftrightarrow$ $k$-multiples of one extensor
$\mathbf{N}$
$\in$ the \textbf{exterior algebra} $\mathcal{E}$.

$\mathcal{E}$ is quotient of the algebra 
generated by $\mathbf{k}(P\dunion E)$
modulo the ideal generated by $\mathbf{v}^2$, $\mathbf{v}\in \mathbf{k}(P\dunion E)$.

$\mathcal{E}$ has dimension $2^{|P\dunion E|}=2^n$. 

$\mathcal{E}$ is graded: At rank $r$, $\mathcal{E}_r$ has 
dim. $\binom{n}{r}$, $r=0,1,\ldots,n$.

\end{slide}

\begin{slide}
The row space of $N$, (within $\mathbf{k}(P\dunion E)$) 
uniquely corresponds to the set of non-zero $\mathbf{k}$-multiples of one 
decomposible (extensor) $\mathbf{N}$ $\in$ $\mathcal{E}_r$.

We can use $\mathbf{N}=\mathbf{v}_1\mathbf{v}_2\cdots\mathbf{v}_r$ where
$\mathbf{v}_i$ is the $i$-th row of matrix $N$:
\[
\mathbf{v}_i = \sum_{p\in P}N_{i,p}\mathbf{p} + \sum_{e\in E}N_{i,e}\mathbf{e}
\]
We then get
\[
\mathbf{N} = \sum_{B\in P\dunion E, |B|=r}N[B]\mathbf{b}_1\mathbf{b}_2\cdots\mathbf{b}_r,
\]
in which $N[B]$ is the $r\times r$ \textit{determinant} in $N$ with columns 
$b_1b_2\cdots b_r$ in that order.

These $\binom{n}{r}$ exterior products of vectors drawn from
$E\dunion P$, comprise one basis for $\mathcal{E}_r$.

\end{slide}

\begin{slide}
Exterior \textit{sum} can be expressed by addition of the expansion coefficients
under the basis of all $2^n$ subsets of $P\dunion E$, (each subset with an element order fixed).

But, the sum of two extensors is not necessarilly \textit{decomposible}, not necessarilly
an \textit{extensor}

The exterior \textit{product} of extensors for disjoint subspaces represents the
\textbf{subspace join}.

When a ground set $S$ is given (like $S=E\dunion P$, distinguished basis for 
$\mathbf{k}(E\dunion P)$), \textbf{deletion} and  \textbf{contraction} of $e\in S$, 
and \textbf{dualization}, known from multilinear algebra, represent
the corresponding (oriented) matroid operations.

We define deletion/contraction so 
\[
\mathbf{N} = \mathbf{N}\setminus e + (\mathbf{N}/e)\mathbf{e}
\]

Dualization: Copy the oriented matroid chirotope dualization formula
(also known as Hodge star): Coefficients (determinants!)
$\mathbf{N}^{\perp}[X]=\mathbf{N}[\overline{X}]\epsilon(X,\overline{X})$.

\end{slide}
\begin{slide}

Our Tutte-like function $\mathbf{M}_E(\mathbf{N}):\text{Extensors}\rightarrow\text{Extensors}$.

Given $N$ (matrix), construct $N^\perp$ so their row spaces are orthogonal
complements ($N^\perp$ presents the (oriented) dual of the matroid from $N$).

Form the matrix:  ($G=\mbox{diag}(g_e)$, $R=\mbox{diag}(r_e)$)
\[
M = \left[\begin{array}{c|c|c} N(P)  &  0  &  N(E)G \\  \hline
0  & N^{\perp}(P)  &  N^{\perp}(E)R \end{array}\right]
\]
with columns labelled by $P_I\dunion P_V\dunion E$.

Extensor $\mathbf{M}$ over $\mathbf{k}[g_e, r_e](P_V\dunion P_I \dunion E)$
is the product of $M$'s \textbf{row vectors}, and define $\mathbf{M}_E(\mathbf{N})$ by:
\[
\mathbf{M} = \mathbf{M}_E(\mathbf{N})\mathbf{e_1}\mathbf{e_2}\cdots\mathbf{e}_{|E|} + (\cdots) 
\]


\end{slide}

\begin{slide}
Result 2:
\[
   \epsilon(PE)\mathbf{M}_E(\mathbf{N}(P,E))=
\]
\[
   \epsilon(PE')\big(g_e\mathbf{M}_{E'}(\mathbf{N}/e) 
                + r_e\mathbf{M}_{E'}(\mathbf{N}\setminus e)\big)
\]
\[
   \epsilon(P_1P_2E)\mathbf{M}_E(\mathbf{N}_1(P_1,E_1)\mathbf{N}_2(P_2,E_2))
   =
\]
\[
   \epsilon(P_1E_1)\epsilon(P_2E_2)
   \big(\mathbf{M}_{E_1}(\mathbf{N}_1(P_1,E_1))\mathbf{M}_{E_2}({N}_2(P_2,E_2))\big)  
\]
Corollory:
\begin{equation}
\label{MESubsetSum}
\epsilon(PE)\ext{M}_E(\ext{N})=\epsilon(P)\sum_
             {\begin{array}{c}
                A\subseteq E:
	\rank_{\ext{N}}{A}=\Card{A},\\
	\rank{\ext{N}}-\rank{(\ext{N}/A|P)}-
        \rank_{\ext{N}}{A}=0
               \end{array}}
    \ext{M}_\emptyset(\ext{N}/A|P) g_Ar_{\scomp{A}}.
\end{equation}
Compare: (with $u=0$ and $v=0$)
\[
R(\mathcal{N}(P,E))=\sum_{A\subseteq E}
	\MVAR{\mathcal{N}/A|P}g_Ar_{\scomp{A}}
	u^{\Rank{\mathcal{N}}-\Rank{[\mathcal{N}/A|P]}-\Rank{A}}
	v^{\Card{A}-\Rank{A}}.
\]

\end{slide}

\begin{slide}
Result 3:  The ported, parametrized corank-nullity polynomials
of oriented and non-oriented matroids 
\[
R(\mathcal{N}(P,E))=\sum_{A\subseteq E}
	\MVAR{\mathcal{N}/A|P}g_Ar_{\scomp{A}}
	u^{\Rank{\mathcal{N}}-\Rank{[\mathcal{N}/A|P]}-\Rank{A}}
	v^{\Card{A}-\Rank{A}}.
\]
satisfy the ported Tutte equations.

(In the invariant case, this $R$ is universal for Tutte $P$-invariants
of oriented and non-oriented matroids)

\end{slide}

\begin{slide}
Result 4 pertains to three matroid combination operations:

Given $\mathcal{N}_1(W,E_1)$ and $\mathcal{N}_1(W,E_2)$ with
only elements $P$ in common.

(1.) Ported matroid union: $\mathcal{B}\big(\mathcal{N}_1\UNION\mathcal{N}_2\big) =
\{B_1\cup B_2 : B_i\in\mathcal{B}(\mathcal{N}_i)\}$

(2. Duality conjugate of $\UNION$.)

(3.) Given that $W$ is a modular flat and a common submatroid
in $\mathcal{N}_1$ and $\mathcal{N}_2$, the generalized parallel connection.

When $|W|=1$, both dual $\UNION$ and generalized parallel connection REDUCE
to (one base point) parallel connection.


\end{slide}


\begin{slide}
(Switch $P$ to $W$.)
$K$=polynomial ring containing $u$ and $w$.  
$K_W$=commutative $K$-module generated by monomials
$[q_i]$ signifying matroids $q_i$ over subsets of $W$, and $[\emptyset]=1$.

Recall ported corank-nullity polynomial $R_W:\text{Matroids}\rightarrow K_W$.

Result 4:  For each combination operation ($*_i,i=1,2,3$) there is a bilinear 
$M_i:K_W\times K_W\rightarrow K_W$ map and an $u^jw^k$-valued function $f_i$ such 
that 
\[
R_W(\mathcal{N}_1 *_i \mathcal{N}_2) =
f_i(\mathcal{N}_1,\mathcal{N}_2,\mathcal{N}_1*_i\mathcal{N}_2)
M_i\big(R_W(\mathcal{N}_1),R_W(\mathcal{N}_2)\big)
\]

In other words, once $*_i$ is determined on pairs of matroids over subsets of $W$,
the ported $R_W$ for $\mathcal{N}_1*_i\mathcal{N}_2$ can be calculated by
a formal multiplication of $R_W(\mathcal{N}_1)*_iR_W(\mathcal{N}_2)$ followed
by a substitution of $[q_j]*_i[q_k]$ products.

\end{slide}


\end{document}











