\documentclass[12pt,leqno]{amsart}
\usepackage{color,graphics}

\allowdisplaybreaks

\newtheorem{theorem}{Theorem}


%%%%%%%%%%%%%%%%%%%%%%%%%%%%%%%%%%%%%%%%%

\begin{document}

\title{A Path in Forest Exchange Property of Series-Parallel Graphs}

\author{Seth Chaiken}
\address{Computer Science Department\\
The University at Albany (SUNY)\\
Albany, NY 12222, U.S.A.}
\email{\tt sdc@cs.albany.edu}

\subjclass[2000]{...}



\keywords{Series-Parallel Graphs, Forests, Negative Association, Negative Correlation}

\thanks{Version of \today.}

\maketitle
\pagestyle{headings}


%%%%%%%%%%%%%%%%%%%%%%%%
\section{An Exchange Propery}

It is well-known that a series-parallel subgraph, defined as 
a graph with
no subgraph homeomorphic to $K_4$, is equivalent to a subgraph of an
$st$-series-parallel network.  Here an $st$-series-parallel network
has two distinguished vertices $s$, $t$
constructable from 
single edge $st$-series-parallel networks using the series and
parallel connection operations.  
All graphs can have multiple edges unless otherwise indicated.

\begin{theorem}
Let $G$ be a series-parallel subgraph 
with two designated vertices $s$ and $t$.

\begin{enumerate}
\item
Whenever 
$F_0$ and $F_1$ are forest subgraphs in $G$ for which
$F_1$ contains an $st$ path $P$, there is a unique 
subset $A\subseteq P\setminus F_0$ for which 
$F_0\cup A$ is a forest that contains an $st$ path $P'$
for which $A\subseteq P'$.

\item In this situation, let $F_1'=F_0\cup A$, which contains path $P'$,
and $F_0'=F_1\setminus A$.   $A$ is also the unique subset 
$A\subseteq P'\setminus F_0'$ for which 
$F_0'\cup A$ is a forest that contains an $st$ path $P$
for which $A\subseteq P$.
\end{enumerate}
\end{theorem}


{\bf Remark:} When $F_0=\emptyset$, the property is satisfied with
$A=P$.  When $F_0$ contains an $st$ path, the property is
satisfied with $A=\emptyset$.

\begin{proof}

The theorem claims that there is a unique subset of
zero or more edges $A\subseteq P\subseteq F_1$ from $st$ path $P$ which 
when added to $F_0$ comprises or completes an $st$ path $P'$ within
a forest; and
when the same set is removed from $F_1$ it is the unique 
subset of a path in $F_0\cup A$ that comprises or completes an $st$ path
when added to $F_1\setminus A$.  Let us call $A$ an ``exchange set.''

If $F_1$ doesn't contain an $st$ path,
there is nothing to prove.  So,  assume $F_1$ contains $st$ path $P$.
Note that if a forest contains an $st$-path then
that $st$ path is unique.

If $F_0$ already contains an $st$ path, then $A=\emptyset$ is the
only exchange set because additional paths cannot be produced by
adding edges to $F_0$ without producing circuits.  $F_1\setminus A$
$=$ $F_1$ contains an $st$ path, so $\emptyset$ is the only 
subset of a path that can be added to $F_1$ without producing circuits.

So, let us assume that $F_0$ does not contain an $st$ path.  

The following structural characterization
of an $st$ series-parallel subgraph
facilitates the proof:

An $st$ series-parallel subgraph
is either the edgeless graph with vertices $s,t$, a one-edge 
graph with vertices $s,t$, or is graph composed from
already constructed $st$ series-parallel subgraphs by series or
by parallel connection along the $st$ terminals.  See the figure.

\begin{center}
\input{induction.pdf_t}
\end{center}


We proceed
by induction on the number of the above series or 
parallel construction steps for 
$G$.

The theorem has been verified for the graph $G$ with vertices $s,t$ and no
edges.  For the one-edge graph $e=st$, we need check only $F_0=\emptyset$
and $F_1=\{e\}$.  In this case, it is immediate that only $A=\{e\}$
satisfies the conclusion.

So, we assume $G$ is the parallel connection along $s,t$ of two graphs
constructed with fewer steps, or is the series connection along $s,t$ of such 
graphs.  Let the subgraphs be $G^1$ and $G^2$.
Let $F_i^j=F_i\cap G^j$  for $i=0,1$
and $j=1,2$.  Recall that we can assume that $F_0$ does not
contain an $st$ path.

\begin{center}
\input{CaseParallel.pdf_t}
\end{center}

Consider the case of parallel connection.  
Assume the $st$ path $P$ in $F_1$ is contained
in $F_1^1\subseteq G^1$, otherwise reverse the roles of $G^1$, $G^2$.
Let $A$ be the exchange set for forests $F_0^1,F_1^1$ in 
$G^1$ that results from induction.  $F_0\cup A$ 
has as $st$ path (since $F_0\cup A \supset F_0^1\cup A$) 
but no circuit, since otherwise, $F_0^2$ would
have an $st$ path, contradicting our assumption.  Also, by
induction, $A$ is unique.

The path $P'\subseteq F_0^1\subseteq G^1$, so none of 
its edges can comprise or complete an $st$ path in 
$F_0^2\setminus A = F_0^2 \subseteq G^2$.  Therefore,
$A$ is the unique exchange set for $F_0,F_1$.

\begin{center}
\input{CaseSeries.pdf_t}
\end{center}


Now for series connection.  
The $st$ path $P$ in $F_1$ is uniquely
partitioned into two interterminal paths $P^1$ and 
$P^2$ in $F_1^1$ and $F_1^2$ respectively.  By induction,
there are unique $A^1$, $A^2$ and $P'^1, P'^2$ satisfying the theorem
for $F_0^1,F_1^1$ and $F_0^2,F_1^2$ respectively.
$A^1\cap A^2$ $=$ $P'^1\cap P'^2$ $=$ $\emptyset$.
Therefore $A=A^1\cup A^2\subset P'$ is the unique set
for which 
$P'=P'^1\cup P'^2$ is an $st$ path in $G$ with
$A\subseteq P'$.

$P'^1$ and any subset of it must belong to $G^1$ and
$P'^2$ and any subset of it must belong to $G^2$.  Therefore,
the properties and uniqueness of $A^1$, $A^2$ and
$P^1$, $P^2$ following by induction from part (2) of the
theorem imply the same properties for $A=A^1\cup A^2$ and
$P=P^1=P^2$.
\end{proof}


\section{Converses}

Clearly, if any subgraph $F_1$ contains an $st$ path $P$, then zero or more
edges $A\subset P$ can be added to any forest $F_0$ to comprise or
complete an $st$ path within forest $F_0$.  The conditions that
$F_1$ is a forest and $F_1\cup F_0$ is a series-parallel subgraph
are sufficient to guarantee the uniqueness of $A$ and the
property (2) in our theorem.


The only obstacle to the uniqueness seems to be 
two $st$ paths with at least one common edge
$e$ where $e$ is traversed in opposite directions
when traversing the two $st$ paths.  Let $G^+$ be $G$ with
a new edge $st$ added. In this situation,
it is known that there is in $G+$ a subgraph homeomorphic to $K_4$.
The $K_4$ homoeomorph is comprised of 6 edge-disjoint paths, and
$e$ and $st$ are in two of the pairs of paths that are 
also vertex disjoint.

\input{Obstruction.pdf_t}


\section{Perhaps Excessive Arguments}

If $F_0$ also contains an $st$ path, we claim the theorem is satisfied by
$A=\emptyset$.  $F_0\cup\emptyset$ is a forest with 
$st$ path $P'$, and clearly $A=\emptyset\subseteq P'$.  
To prove $A=\emptyset$ is unique, suppose $A\neq \emptyset$ 
also satisfies the theorem, so
$F_0\cup A$ is a forest.  Since $F_0\cup A$ is a forest, 
$P'$ is the unique $st$ path in $F_0\cup A$.  Therefore, 
$A\subseteq P'\setminus F_0$.  But $P'\subseteq F_0$, so
$A\neq\emptyset$ is contradicted.

We therefore verified that when $F_0$ contains an $st$ path,
$A=\emptyset$ uniquely satisfies (1) of the theorem.
In this situation, $F_0'=F_1\setminus A=F_1$ contains the 
unique path $P$, so $\emptyset=A$ is the only set for 
which $F_0'\cup A$ is a forest that contains an $st$ path $P$
for which $A\subseteq P$.



{\bf ------- Parallel case:}
There cannot be an $st$ path in $F_1^2$, because
that path together with $P$ would be a circuit, which
contradicts $F^1$ being a forest.

By induction, $A$ is the unique subset of $P$ for which
$F_0^1\cup A$ is a forest in $G^1$ that 
contains $st$ path $P'$ with $A\subseteq P'$. 
Since $F_0^2$ is a forest without an $st$ path,
$F_0\cup A=F_0^1\cup A\cup F_0^2$ is a forest 
in $G$ that contains $st$ path $P'$.  Therefore, (1) is confirmed.





\end{document}
