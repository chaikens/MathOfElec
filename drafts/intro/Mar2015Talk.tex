\documentclass{beamer}
%\mode<presentation>
\usepackage{cite}
\usepackage{amsmath}
\title{Introduction}
\author{Seth Chaiken}

%%%%%%%%%%%%%%%%%%%%%%%%%%%%%%%%%%%%%%%%%%%%%%%%%%%%%%%%%%%%%
% Specialized Symbols
%
%   Disjoint Union
%\newcommand{\dunion}{\uplus}
\newcommand{\dunion}
%{\mbox{\hbox{\hskip4pt$\cdot$\hskip-4.62pt$\cup$\hskip2pt}}}
%{\mbox{\hbox{\hskip6pt$\cdot$\hskip-5.50pt$\cup$\hskip2pt}}}
{\mbox{\hbox{\hskip0.45em$\cdot$\hskip-0.44em$\cup$\hskip0.2em}}}
%{\mbox{\hbox{\hskip0.45em$+$\hskip-0.70em$\cup$\hskip0.3em}}}
%
% Dot inside a cup.
% If there is a better, more Latex like way 
% (more invariant under font size changes) way,
% I'd like to know.

\newcommand{\Bases}[1]{\ensuremath{{\mathcal{B}}(#1)}}
\newcommand{\Reals}{\ensuremath{\mathbb{R}}}
\newcommand{\FieldK}{\ensuremath{K}}
\newcommand{\Perms}{\ensuremath{\mathfrak{S}}}
%\newcommand{\rank}{{\rho}}% {{\mbox{rank}}}
%\newcommand{\Rank}{{\rho}}% {{\mbox{rank}}}
\newcommand{\rank}{{\mbox{r}}}% {{\mbox{rank}}}
\newcommand{\Rank}{{\mbox{r}}}% {{\mbox{rank}}}
\newcommand{\Card}[1]{\ensuremath{{\left|#1\right|}}}
\newcommand{\ext}[1]{\ensuremath{\mathbf{#1}}}
%\newcommand{\extvee}{\ensuremath{\mathbf{\vee}}}
\newcommand{\extvee}{\;\;}
\newcommand{\Plucker}{Pl\"{u}cker\ }

% Set Complement
% command to mess with overline, bar or custom 
% alternatives for sequence or set complement
%
\newcommand{\scomp}[1]{\ensuremath{\overline{#1}}}
%\newcommand{\scomp}[1]{\ensuremath{\bar{#1}}}
%
%   Put a symbol for a matroid in a box, or brackets
%\newcommand{\MVAR}[1]{{\boxed{#1}\;}}
\newcommand{\MVAR}[1]{{[#1]\;}}
%
\newcommand{\UNION}{\cup} %try to make this bold.
%%%%%%%%End of specialized symbols%%%%%%%%%%%%%%%%%%%%%%



\begin{document}
\begin{frame}
 \titlepage
\end{frame}

\note{(here we go, geronimo)}

%%%%%%%%%%%%%%%%%%%%%%%%%%%%%%%%%%%%%%%
\begin{frame}{Why Electricity, EE?}
\begin{itemize}
\item $>$ Some 100 years geometry-like intuition

(Engineering books: ``Intuitive Analog Circuit Design''
\cite{intuitAna}; ``Non-linear Circuits'' \cite{HaslerNeirynck}
translates to our Oriented Matroid pair model.)

\item Geometry of linear spaces and oriented matroids.

(Our Tutte decomposed exterior algebraic
function of oriented matroids.)

(Techniques from Barnabi, Brini and Rota's Exterior Calculus 
\cite{exteriorCalc})

\item
Real behavior $\approx$ ideal plus perturbations

\item
Ideal constraints predict intended real behavior
(how geometry is applied everyday)

\item 
Interesting, accessible, intuitively understandable
intentential designs), applicable,
easy to both simulate and build physically, structure of moderate
dimension (about 12 or 24, depending on what they may count / formulation)
\item
Analogs to chemical (and real algebraic geometry 
\cite{signsInChemRAG}), biological, mechanical, etc., random walks ...
\item
Merely one scalar non-linearity can cause chaos.
\end{itemize}
\end{frame}

\note{
(nonframe)Analogies:
\begin{enumerate}
\item Electric: Voltage, Resistance, Current, Capacitance, Inductance
\item Mech. Dyn: Force, Viscous friction, Velocity, Elasticity, Mass
\item Statics: Displacement, Elasticity, Force,  ???, ???
\item Random Walk: First hit probability, transition probability, flow, ???, ??? \\
Doyle and Snell \cite{DoyleSnellRandom},
Lyons and Peres \cite{ProbOnTreesNetworks}
(ref? Thanks, David G. Wagner's course notes).
\item Traffic: Time, Clogginess (travel time/flow ratio), flow, ???, ???\\
Condition: All vehicles reach their destinations in equal times regardless of 
route (ref? Thanks, Peter Shor).
\item Knot theory: ???, $\pm 1$ crossing sign, ???, ???, ???\\
Equivalent resistance ``Conductance Invariant'' of rational tangles is a 
Reinmeister move 
invariant\cite{RationalTangles} (Thanks, Jay Goldman).
\end{enumerate}

KCL+KVL+Ohms=(one of KCL)+Ohms+(min power)

Future: Memistor?
}



%%%%%%%%%%%%%%%%%%%%%%%%%%%%%%%%%%%%%%%%%%%%%%%%%%%%%%%
\begin{frame}{Kirchhoff (1847)
\cite{Kirchhoff}
 Maxwell (1891)
\cite{MaxR} 
The equivalent resistance problem
solved by the Matrix Tree Theorem.}

\begin{minipage}{.4\textwidth}
(resistor edge diagram)\\
$i_e$\\
$v_e=v_1-v_2$\\
$g_ev_e=r_ei_e$
\end{minipage}
\begin{minipage}{.4\textwidth}
(port edge diagram, environment)\\
$i_p$\\
$v_p=v_1-v_2$\\
mathematics sign convention.\\
$g_ev_e=r_ei_e$
\end{minipage}\\
\end{frame}

%%%%%%%%%%%%%%%%%%%%%%%%%%%%%%%%%%%%%%%%%%%%%%%%%
\begin{frame}{Equations}
(KCL) $(i_e)_{e\in S}$ is a cycle (a flow)\\
(KVL) $(v_e)_{e\in S}$ is a cocycle\\
(constituitive Law) $i_e=g_e(v_e)$
non-linear, usually monotonic increasing $R\rightarrow R$ 
(Ohm's approximation $i_e=g_ev_e$\\
signs ($\pm$) have oriented matroid structure
(combinatorial, geometric, topological)\\
\end{frame}



%%%%%%%%%%%%%%%%%%%%%%%%%%%%%%%%%%%%%%%%%%%%%%%%%
\begin{frame}{Equiv. Resistance is a Ratio}

\[
\frac{\mbox{WTS}(G/p)}
{\mbox{WTS}(G\backslash p)}
\]
\[
\mbox{(WeightedTreeSum)}\mbox{WTS}(G') =
g_e \mbox{WTS}(G'/e) + r_e \mbox{WTS} (G \backslash e)
\text{\ for all\ }e \not\in P
\]
\end{frame}

%%%%%%%%%%%%%%%%%%%%%%%%%%%%%%%%%%%%%%%%%%%%%%%%%
\begin{frame}{Benefits of Multiple ports}

\begin{itemize}
\item
Formalize composition of systems\cite{NarayananDecompVS1986}
\item
Label variables to observe
\item
Model practical devices (transistors, op amps)
\item
Interesting \textbf{non-commutative ranges} of
new Tutte functions with pattern:
\[
\text{TF}(N(P\dunion E)) = F(N(P\dunion E)/E)
\]
\item
Systematize ad-hoc vertex-based proofs.
\end{itemize}
\end{frame}


%%%%%%%%%%%%%%%%%%%%%%%%%%%%%%%%%%%%%%%%%%%%%%%%%
\begin{frame}
\frametitle{Equivalent resistance is a coefficient ratio in an 
implicitly defined linear function}

(diagram)

In other words
\[
R_pi_p + v_p = 0
\]
or dually,
\[
\mathcal{B} = \{(i_p,v_p)\}
= \{ t(-1, R_p) | t\in R\}
\]

\end{frame}

\note{nonframe(linear subspace over the $R^{\mathbf{P}}$)}

%%%%%%%%%%%%%%%%%%%%%%%%%%%%%%%%%%%%%%%%%%%%%%%%%
\begin{frame}
\frametitle{The $2\times d$ port variable constraint space, 
and its solution spaces, are $d$-dimensional.}

We represent these spaces by carefully defined
\textbf{extensors}, as Barnabei, Brini and Rota \cite{exteriorCalc}
term ``decomposible antisymmetric tensors''

The solution extensor (not a ray) satisfies:

\[
E(N) = \text{sign}(...)(g_e E(N/e) + r_e E(N\backslash e)
\text{\ for\ }e\not\in P
\]

\end{frame}


\note{
(nonframe) 
The port elements represent the basis chosen so coordinates correspond
to voltages and currents belonging to network edges.  Geometric aspects
of extensors are surveyed by 
Barnabei and Brini \cite{exteriorCalc}.

Cordovil\cite{CommAlgOMs} 
presented a \textbf{commutative algebraic} function of 
oriented matroids in which monomial signs encode orientation information.
The analogy to our \textbf{exterior algebraic} functions must be explored. 
}

%%%%%%%%%%%%%%%%%%%%%%%%%%%%%%%%%%%%%%%%%%%%%%%%%
\begin{frame}{Coefs}
\[
E(N) = \ldots + C_i\mbox{\bf XXX} + \ldots
\]
\[
R = C_i/C_j
\]
All the other $C_k$'s have similar interpretations.

Each $C_k$ is a determinant.

Each $C_k$ is a signed weighted enumerator of
forests satisfying \textbf{conditions ...}

Each $C_k$ satisfies
\[
C_k(N) = g_e C_k(N/e) + r_e C_k(N\backslash e)
\text{\ for\ }e\not\in {P}
\]

\end{frame}

%%%%%%%%%%%%%%%%%%%%%%%%%%%%%%%%%%%%%%%%%%%%%%%%%
\begin{frame}{Conditions}
The \textbf{conditions ...} are best described with 
a \textbf{pair of ported oriented matroids}
(aka \textbf{relative or set-pointed}

The conditions for a given $C_k$ \textit{sometimes}
make all the signs the same (example: $C_i$ and 
$C_j$ in 1-port equivalent resistance $R=C_i/C_j$)

\textit{Othertimes}, the oriented \textbf{P-minors}
in the completed Tutte decomposition of $C_k$ determine
the sign.

(diagram of the two orientations of $C_2$)

\end{frame}

%%%%%%%%%%%%%%%%%%%%%%%%%%%%%%%%%%%%%%%%%%%%%%%%%
\begin{frame}{Conditions}
What is the nature of the conditions?  We state them using the 
network's graphic oriented matroid.

(diagram--glob w/ ports)

\end{frame}




\note{
nonframe \textbf{port} (sdc, engineering literature), 
\textbf{distinguished element}, 
element of the restriction's set (Diao and Hetyei), 
a element in the set of points (Las Vergnas).
}

%%%%%%%%%%%%%%%%%%%%%%%%%%%%%%%%%%%%%%%%%%%%%%%%%
\begin{frame}
\frametitle{Voltage and Current graphs}

\begin{minipage}{0.4\textwidth}
``Voltage graph'' (EE, NOT Gross, et. al.) represents KVL
\[\mathbf{v}\in \text{Cocycle space}\]
\end{minipage}
\begin{minipage}{0.4\textwidth}
``Current graph'' represents KCL
\[\mathbf{i}\in \text{Cycle space}\]
\end{minipage}
They are equal graphs for resistor networks.

For networks with idealized amplifiers, they are not 
equal.  

(more) Realistic amplifier model $=$ idealized amplifiers $+$ 
resistors.

(ideal op amp nullator/norator T diagram)

The output voltage and current are whatever makes the input
voltage and current BOTH BE zero.

\begin{minipage}{0.4\textwidth}
\[
\text{(nullator)\ }N_v = N/e// 
N_I = N\backslash e
\]
\end{minipage}
\begin{minipage}{0.4\textwidth}
\[
\text{(norator)\ }N_v = N\backslash e// 
N_I = N/e
\]
\end{minipage}

\end{frame}


%%%%%%%%%%%%%%%%%%%%%%%%%%%%%%%%%%%%%%%%%%%%%%%%%
\begin{frame}{What are the conditions like?}

\[
\text{Extensor\ }= \sigma(C)(a_1\vee a_2\vee \ldots \vee a_d)
\]
\[
C = \ldots + t + \ldots
\]
$t \leftrightarrow \text{\ sets of (``contractible'' non-port edges} 
E_1, E_2$ for which
\[
N_V/E_1\backslash(E\setminus E_1) = \text{\ certain OMs on\ } P
\]
AND
\[
N_I/E_2\backslash(E\setminus E_2) = \text{\ certain OMs on\ } P
\]

Transfer resistance might be 0 and might be $\neq 0$ iff
\[
\exists E_1, E_2 \text{so (diag) and (diag)}
\]

($K_4$ Wheatstone bridge diagram)

\[
\frac{R_1}{R_2} > \frac{R_3}{R_4} 
\text{\ neg\ }
\frac{R_1}{R_2} < \frac{R_3}{R_4} 
\text{\ pos\ }
\]

\end{frame}

%%%%%%%%%%%%%%%%%%%%%%%%%%%%%%%%%%%%%%%%%%%%%%%%%
\begin{frame}{(Rayleigh)}
\end{frame}

%%%%%%%%%%%%%%%%%%%%%%%%%%%%%%%%%%%%%%%%%%%%%%%%%
\begin{frame}{(15)}
\end{frame}

%%%%%%%%%%%%%%%%%%%%%%%%%%%%%%%%%%%%%%%%%%%%%%%%%
\begin{frame}{(16)}
\end{frame}

%%%%%%%%%%%%%%%%%%%%%%%%%%%%%%%%%%%%%%%%%%%%%%%%%
\begin{frame}{(17)}
\end{frame}


%%%%%%%%%%%%%%%%%%%%%%%%%%%%%%%%%%%%%%%%%%%%%%%%%
\begin{frame}[allowframebreaks]{References}
\bibliographystyle{plain}
\bibliography{../../bib/MathOfElec}{}
\end{frame}

\end{document}
