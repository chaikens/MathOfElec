\documentclass{beamer}
%\mode<presentation>
\usepackage{cite}
\usepackage{amsmath}
\title{Ported or Relative Oriented Matroids and Electric Circuits}
\author{Seth Chaiken\\
Dept. of Computer Science\\
Univ. at Albany\\
\url{schaiken@albany.edu}
}
\date{March 2, 2015}
%%%%%%%%%%%%%%%%%%%%%%%%%%%%%%%%%%%%%%%%%%%%%%%%%%%%%%%%%%%%%
% Specialized Symbols
%
%   Disjoint Union
%\newcommand{\dunion}{\uplus}
\newcommand{\dunion}
%{\mbox{\hbox{\hskip4pt$\cdot$\hskip-4.62pt$\cup$\hskip2pt}}}
%{\mbox{\hbox{\hskip6pt$\cdot$\hskip-5.50pt$\cup$\hskip2pt}}}
{\mbox{\hbox{\hskip0.45em$\cdot$\hskip-0.44em$\cup$\hskip0.2em}}}
%{\mbox{\hbox{\hskip0.45em$+$\hskip-0.70em$\cup$\hskip0.3em}}}
%
% Dot inside a cup.
% If there is a better, more Latex like way 
% (more invariant under font size changes) way,
% I'd like to know.

\newcommand{\Bases}[1]{\ensuremath{{\mathcal{B}}(#1)}}
\newcommand{\Reals}{\ensuremath{\mathbb{R}}}
\newcommand{\FieldK}{\ensuremath{K}}
\newcommand{\Perms}{\ensuremath{\mathfrak{S}}}
%\newcommand{\rank}{{\rho}}% {{\mbox{rank}}}
%\newcommand{\Rank}{{\rho}}% {{\mbox{rank}}}
\newcommand{\rank}{{\mbox{r}}}% {{\mbox{rank}}}
\newcommand{\Rank}{{\mbox{r}}}% {{\mbox{rank}}}
\newcommand{\Card}[1]{\ensuremath{{\left|#1\right|}}}
\newcommand{\ext}[1]{\ensuremath{\mathbf{#1}}}
%\newcommand{\extvee}{\ensuremath{\mathbf{\vee}}}
\newcommand{\extvee}{\;\;}
\newcommand{\Plucker}{Pl\"{u}cker\ }

% Set Complement
% command to mess with overline, bar or custom 
% alternatives for sequence or set complement
%
\newcommand{\scomp}[1]{\ensuremath{\overline{#1}}}
%\newcommand{\scomp}[1]{\ensuremath{\bar{#1}}}
%
%   Put a symbol for a matroid in a box, or brackets
%\newcommand{\MVAR}[1]{{\boxed{#1}\;}}
\newcommand{\MVAR}[1]{{[#1]\;}}
%
\newcommand{\UNION}{\cup} %try to make this bold.
%%%%%%%%End of specialized symbols%%%%%%%%%%%%%%%%%%%%%%



\begin{document}
\begin{frame}
 \titlepage
\end{frame}

\note{(here we go, geronimo)}

%%%%%%%%%%%%%%%%%%%%%%%%%%%%%%%%%%%%%%%
\begin{frame}{Why Electricity, EE?}
\begin{itemize}
\item $\approx 100$ yrs. geometry-like intuition of
circuit configurations known by engineers, EE books:
``Intuitive Analog Circuit Design (2013)''
\cite{intuitAna}; ``Non-linear Circuits'' \cite{HaslerNeirynck}
translates to our Oriented Matroid pair model.

\item Geometry of linear spaces and oriented matroids;
Tutte decomp. w/
techniques from Barnabi, Brini and Rota's Exterior Calculus 
\cite{exteriorCalc})

\item
Real behavior $\approx$ ideal plus perturbations,
ideal constraints predict intended real behavior,
applied everyday geometry.

\item 
Interesting, accessible, intuitively understandable
intentential designs, applicable,
easy to both simulate and build physically, dimension
$\approx$
12 or 24, depending on formulation
\item
Analogs to chemical (and real algebraic geometry 
\cite{signsInChemRAG}), biological, mechanical, etc., random walks ...
\item
Merely one scalar non-linearity can cause chaos.
\end{itemize}
\end{frame}

\note{
(nonframe)Analogies:
\begin{enumerate}
\item Electric: Voltage, Resistance, Current, Capacitance, Inductance
\item Mech. Dyn: Force, Viscous friction, Velocity, Elasticity, Mass
\item Statics: Displacement, Elasticity, Force,  ???, ???
\item Random Walk: First hit probability, transition probability, flow, ???, ??? \\
Doyle and Snell \cite{DoyleSnellRandom},
Lyons and Peres \cite{ProbOnTreesNetworks}
(ref? Thanks, David G. Wagner's course notes).
\item Traffic: Time, Clogginess (travel time/flow ratio), flow, ???, ???\\
Condition: All vehicles reach their destinations in equal times regardless of 
route (ref? Thanks, Peter Shor).
\item Knot theory: ???, $\pm 1$ crossing sign, ???, ???, ???\\
Equivalent resistance ``Conductance Invariant'' of rational tangles is a 
Reinmeister move 
invariant\cite{RationalTangles} (Thanks, Jay Goldman).
\end{enumerate}

KCL+KVL+Ohms=(one of KCL)+Ohms+(min power)

Future: Memistor?
}



%%%%%%%%%%%%%%%%%%%%%%%%%%%%%%%%%%%%%%%%%%%%%%%%%%%%%%%
\begin{frame}{Kirchhoff (1847)
\cite{Kirchhoff}
 Maxwell (1891)
\cite{MaxR} 
The equivalent resistance problem IS SOLVED
by the Matrix Tree Theorem. (1) Let's POSE the problem: the VARIABLES}

\begin{minipage}{.48\textwidth}
\input{resistor.pspdftex}\\
(Use \textbf{voltage drops along the flow}, 
not potentials $V_1$, $V_2$.)\\
\begin{center}\textbf{ordinary, resistor edge} $e\in E$
\end{center}

(We'll soon see Ohm's\\
\textbf{EQUATION:}
\begin{center}$g_ev_e=r_ei_e$).
\end{center}

\end{minipage}
\begin{minipage}{.48\textwidth}
\input{portWithEnviron.pspdftex}\\
\begin{center}\textbf{DISTINGUISHED, PORT edge} $p\in P$
\end{center}
An interface to an environment.
\begin{center}(math, not EE 
sign convention)
\end{center}
\end{minipage}\\
\end{frame}

%%%%%%%%%%%%%%%%%%%%%%%%%%%%%%%%%%%%%%%%%%%%%%%%%
\begin{frame}{(2) Let's POSE the problem: EQUATIONS}
(KCL) $(i_e)_{e\in S}$ is a cycle (a flow)\\
(KVL) $(v_e)_{e\in S}$ is a cocycle\\
(constituitive Law) $i_e=g_e(v_e)$
non-linear, usually monotonic increasing $R\rightarrow R$ 
(Ohm's approximation $i_e=g_ev_e$\\
signs ($\pm$) have oriented matroid structure
(combinatorial, geometric, topological)\\
\end{frame}



%%%%%%%%%%%%%%%%%%%%%%%%%%%%%%%%%%%%%%%%%%%%%%%%%
\begin{frame}{SOLUTION: Equiv. Resistance $ :\equiv -(v_p/i_p)$
observed at a port $p$ by the environment
EQUALS a Ratio of Spanning Tree Enumerators!
(Port edge $p$ locates the \textit{2 terminals}.)}

\[
-(\frac{v_p}{i_p})=\frac{\mbox{WTS}(G/p)}
{\mbox{WTS}(G\backslash p)} 
= \frac{\mbox{Matrix-Tree Det}(G/p)}
       {\mbox{Matrix-Tree Det}(G\backslash p)}
\]
\begin{itemize}
\item
``Maxwell's rule'' uses MatrTreeT on 2 DIFFERENT GRAPHS\\
\begin{center}($G/p$ and $G\backslash p$)\end{center}
\item
Weighted Tree Sum (WTS) is a colored Tutte function:
\[
\mbox{WTS}(G') =
g_e \mbox{WTS}(G'/e) + r_e \mbox{WTS} (G \backslash e)
\text{\ for all\ }e \not\in P
\]
\[
\mbox{WTS}(\text{coloop}(e)) = g_e
\]
\[
\mbox{WTS}(\text{loop}(e)) = r_e
\]
\end{itemize}
\end{frame}

%%%%%%%%%%%%%%%%%%%%%%%%%%%%%%%%%%%%%%%%%%%%%%%%%
\begin{frame}{Benefits of Multiple Ports}

\begin{itemize}
\item One formula expresses $\binom{2|P|}{|P|}$ different Matrix Tree Theorems.
\item
Systematize their ad-hoc vertex-based proofs.
\item
Interesting \textbf{non-commutative ranges} of
new ORIENTED MATROID Tutte invariants with pattern:
\[
\text{TF}(N(P\dunion E)) = F(N(P\dunion E)/E)
\]
(They distinguish DIFFERENT ORIENTATIONS of the SAME MATROID.)
\item
Formalize composition of systems\cite{NarayananDecompVS1986}
\item
Label variables to observe
\item
Model practical devices (transistors, op amps)
\item 
Align EE applications with knots (Ported = ``Relative'')
and geometry (Ported = ``Set Pointed'').  
\end{itemize}
\end{frame}

%%%%%%%%%%%%%%%%%%%%%%%%%%%%%%%%%%%%%%%%%%%%%%%%%%%%
\begin{frame}{Constraint/Generator Duality}
\begin{minipage}{0.48\textwidth}

\begin{itemize}
\item
(Part 1) Technique:
\begin{center}$\text{Solution Space}$\end{center}
\begin{center}$=$\end{center}
\begin{center}$\cap \text{Constraint Subspaces}$\end{center}
\item
An
exterior algebraic Tutte function
and det. formulas that
easily prove Rayleigh inequalities.
\end{itemize}

\end{minipage}
\begin{minipage}{0.48\textwidth}

\begin{itemize}
\item
(Part 2) Combine with:
\begin{center}$\text{Solution Space}$ \end{center}
\begin{center}$=$ \end{center}
\begin{center}$\text{Closure}(\text{Set of Generators})$\end{center}
\item
An oriented matroid pair model
for some non-linear problem
well-posedness.
\end{itemize}
\end{minipage}
\end{frame}

\begin{frame}
{Part 1) Coef. Matrix $M$ in CONSTRAINTS $MX=0$}
\begin{center}
The Tutte-like function $\mathbf{M}_E(\mathbf{N}):\text{Extensors}\rightarrow\text{Extensors}$:
\end{center}

Given $N$ (matrix), construct $N^\perp$ 
with orthog. comp. row space.

Construct:  ($G=\mbox{diag}(g_e)$, $R=\mbox{diag}(r_e)$)
\[
M = \left[\begin{array}{c|c|c} N(P)  &  0  &  N(E)G \\  \hline
0  & N^{\perp}(P)  &  N^{\perp}(E)R \end{array}\right]
\]
with columns labelled by $P_I\dunion P_V\dunion E$.

Extensor $\mathbf{M}$ over $k[g_e, r_e](P_V\dunion P_I \dunion E)$
is the product of $M$'s \textbf{row vectors}. The contraction result
$\mathbf{M}_E(\mathbf{N}) = \mathbf{M}/E$ appears:
\[
\mathbf{M} = \mathbf{M}_E(\mathbf{N})\mathbf{e_1}\mathbf{e_2}\cdots\mathbf{e}_{|E|} + (\cdots) 
\]

$\mathbf{M}_E(\mathbf{N})$ is our Tutte function $\mathbf{N}\rightarrow \text{Ext. Alg.}$
\end{frame}

\begin{frame}{(Part 2) Common Covector Model}
\input{commonCovectorModel.pspdftex}
\end{frame}

\begin{frame}{Return to the Part 1 Equation $MX=0$}
\[
M = \left[\begin{array}{c|c|c} N(P)  &  0  &  N(E)G \\  \hline
0  & N^{\perp}(P)  &  N^{\perp}(E)R \end{array}\right]
;
\mathbf{M} = \mathbf{M}_E(\mathbf{N})\mathbf{e_1}\mathbf{e_2}\cdots\mathbf{e}_{|E|} + (\cdots) 
\]
ELIMINATE the variables indexed by $E$, leaving $2|P|$ variables
labelled by $P_I$ and $P_V$.  ie, CONTRACT $E$. \textbf{Answer} $A$ IS:

\[
\mathbf{M}_E = \bigwedge^{\text{Exterior}}_{\text{JOIN over rows}} \left[\begin{array}{c|c} A_{I,I}  &  A_{I,V}   \\  \hline
    A_{V_I}  & A_{V,V} \end{array}\right] 
[\mathbf{p_{I_1}, \cdots, p_{I_p}; p_{V_1}, \cdots, p_{V_p}}]^{\mathbf{t}} 
\]

\[
 = \ldots + C_i\mbox{\bf XXX} + \ldots; \text{Equiv. Resistance} = 
\text{certain}\;\; C_i/C_j
\]

All the other $C_k$'s have similar interpretations.

{\bf $\binom{2|P|}{|P|}$ Matr. Tree Theorems:}
Each $C_k(N)$ (a PRINCIPAL MINOR of $M$ ABOVE!)
$= 
g_e C_k(N/e) + r_e C_k(N\backslash e)$ ($e\not\in P$, $e$ not (co)loop).

Each $C_k$ is a signed weighted enumerator of
forests satisfying \textbf{conditions ...}
\end{frame}


%%%%%%%%%%%%%%%%%%%%%%%%%%%%%%%%%%%%%%%%%%%%%%%%%
\begin{frame}{Conditions}
The \textbf{conditions ...} are best described with 
a \textbf{pair of ported oriented matroids}

The conditions for a given $C_k$ \textit{sometimes}
make all the signs the same (example: $C_i$ and 
$C_j$ in 1-port equivalent resistance $R=C_i/C_j$)

\textit{Othertimes}, the oriented \textbf{P-minors}
in the completed Tutte decomposition of $C_k$ determine
the sign.

(diagram of the two orientations of $C_2$)

\end{frame}

%%%%%%%%%%%%%%%%%%%%%%%%%%%%%%%%%%%%%%%%%%%%%%%%%
\begin{frame}{(Rayleigh)}
\end{frame}



%%%%%%%%%%%%%%%%%%%%%%%%%%%%%%%%%%%%%%%%%%%%%%%%%
\begin{frame}{(Part 2) Common Covector Model}
\input{commonCovectorModel.pspdftex}
\end{frame}








\begin{frame}
\frametitle{Equivalent resistance is a coefficient ratio in an 
implicitly defined linear function}

(diagram)

In other words
\[
R_pi_p + v_p = 0
\]
or dually,
\[
\mathcal{B} = \{(i_p,v_p)\}
= \{ t(-1, R_p) | t\in R\}
\]

\end{frame}

\note{nonframe(linear subspace over the $R^{\mathbf{P}}$)}

%%%%%%%%%%%%%%%%%%%%%%%%%%%%%%%%%%%%%%%%%%%%%%%%%
\begin{frame}
\frametitle{The $2\times d$ port variable constraint space, 
and its solution spaces, are $d$-dimensional.}

We represent these spaces by carefully defined
\textbf{extensors}, as Barnabei, Brini and Rota \cite{exteriorCalc}
term ``decomposible antisymmetric tensors''

The solution extensor (not a ray) satisfies:

\[
E(N) = \text{sign}(...)(g_e E(N/e) + r_e E(N\backslash e)
\text{\ for\ }e\not\in P
\]

\end{frame}


\note{
(nonframe) 
The port elements represent the basis chosen so coordinates correspond
to voltages and currents belonging to network edges.  Geometric aspects
of extensors are surveyed by 
Barnabei and Brini \cite{exteriorCalc}.

Cordovil\cite{CommAlgOMs} 
presented a \textbf{commutative algebraic} function of 
oriented matroids in which monomial signs encode orientation information.
The analogy to our \textbf{exterior algebraic} functions must be explored. 
}

%%%%%%%%%%%%%%%%%%%%%%%%%%%%%%%%%%%%%%%%%%%%%%%%%
\begin{frame}{Coefs}
\[
E(N) = \ldots + C_i\mbox{\bf XXX} + \ldots
\]
\[
R = C_i/C_j
\]
All the other $C_k$'s have similar interpretations.

Each $C_k$ is a determinant.

Each $C_k$ is a signed weighted enumerator of
forests satisfying \textbf{conditions ...}

Each $C_k$ satisfies
\[
C_k(N) = g_e C_k(N/e) + r_e C_k(N\backslash e)
\text{\ for\ }e\not\in {P}
\]

\end{frame}

%%%%%%%%%%%%%%%%%%%%%%%%%%%%%%%%%%%%%%%%%%%%%%%%%
\begin{frame}{Conditions}
What is the nature of the conditions?  We state them using the 
network's graphic oriented matroid.

(diagram--glob w/ ports)

\end{frame}





\note{
nonframe \textbf{port} (sdc, engineering literature), 
\textbf{distinguished element}, 
element of the restriction's set (Diao and Hetyei), 
a element in the set of points (Las Vergnas).
}

%%%%%%%%%%%%%%%%%%%%%%%%%%%%%%%%%%%%%%%%%%%%%%%%%
\begin{frame}
\frametitle{Voltage and Current graphs}

\begin{minipage}{0.4\textwidth}
``Voltage graph'' (EE, NOT Gross, et. al.) represents KVL
\[\mathbf{v}\in \text{Cocycle space}\]
\end{minipage}
\begin{minipage}{0.4\textwidth}
``Current graph'' represents KCL
\[\mathbf{i}\in \text{Cycle space}\]
\end{minipage}
They are equal graphs for resistor networks.

For networks with idealized amplifiers, they are not 
equal.  

(more) Realistic amplifier model $=$ idealized amplifiers $+$ 
resistors.

(ideal op amp nullator/norator T diagram)

The output voltage and current are whatever makes the input
voltage and current BOTH BE zero.

\begin{minipage}{0.4\textwidth}
\[
\text{(nullator)\ }N_v = N/e// 
N_I = N\backslash e
\]
\end{minipage}
\begin{minipage}{0.4\textwidth}
\[
\text{(norator)\ }N_v = N\backslash e// 
N_I = N/e
\]
\end{minipage}

\end{frame}


%%%%%%%%%%%%%%%%%%%%%%%%%%%%%%%%%%%%%%%%%%%%%%%%%
\begin{frame}{What are the conditions like?}

\[
\text{Extensor\ }= \sigma(C)(a_1\vee a_2\vee \ldots \vee a_d)
\]
\[
C = \ldots + t + \ldots
\]
$t \leftrightarrow \text{\ sets of (``contractible'' non-port edges} 
E_1, E_2$ for which
\[
N_V/E_1\backslash(E\setminus E_1) = \text{\ certain OMs on\ } P
\]
AND
\[
N_I/E_2\backslash(E\setminus E_2) = \text{\ certain OMs on\ } P
\]

Transfer resistance might be 0 and might be $\neq 0$ iff
\[
\exists E_1, E_2 \text{so (diag) and (diag)}
\]

($K_4$ Wheatstone bridge diagram)

\[
\frac{R_1}{R_2} > \frac{R_3}{R_4} 
\text{\ neg\ }
\frac{R_1}{R_2} < \frac{R_3}{R_4} 
\text{\ pos\ }
\]

\end{frame}

%%%%%%%%%%%%%%%%%%%%%%%%%%%%%%%%%%%%%%%%%%%%%%%%%
\begin{frame}{(15)}

\end{frame}

%%%%%%%%%%%%%%%%%%%%%%%%%%%%%%%%%%%%%%%%%%%%%%%%%
\begin{frame}{(16)}
\end{frame}

%%%%%%%%%%%%%%%%%%%%%%%%%%%%%%%%%%%%%%%%%%%%%%%%%
\begin{frame}{(17)}


Our Tutte-like function $\mathbf{M}_E(\mathbf{N}):\text{Extensors}\rightarrow\text{Extensors}$.

Given $N$ (matrix), construct $N^\perp$ 
with orthog. comp. row space.

Construct:  ($G=\mbox{diag}(g_e)$, $R=\mbox{diag}(r_e)$)
\[
M = \left[\begin{array}{c|c|c} N(P)  &  0  &  N(E)G \\  \hline
0  & N^{\perp}(P)  &  N^{\perp}(E)R \end{array}\right]
\]
with columns labelled by $P_I\dunion P_V\dunion E$.

Extensor $\mathbf{M}$ over $k[g_e, r_e](P_V\dunion P_I \dunion E)$
is the product of $M$'s \textbf{row vectors}. The contraction result
$\mathbf{M}_E(\mathbf{N}) = \mathbf{M}/E$ appears:
\[
\mathbf{M} = \mathbf{M}_E(\mathbf{N})\mathbf{e_1}\mathbf{e_2}\cdots\mathbf{e}_{|E|} + (\cdots) 
\]

$\mathbf{M}_E(\mathbf{N})$ is our Tutte function $\mathbf{N}\rightarrow \text{Ext. Alg.}$

\end{frame}


%%%%%%%%%%%%%%%%%%%%%%%%%%%%%%%%%%%%%%%%%%%%%%%%%
\begin{frame}{``Colors'' are parameters on every Tutte decomposition step}

The Bollobos/Riordan/Zaslavsky BRZ theory for well-definedness
of ``Relative Tutte Polynomials for Colored Graphs'' all goes through 
(Diao and Hetyei \cite{RelTuttePoly}):
The 3 BRZ conditions on (colors,initial values) generalize to 5;
also the theory of activities
based on linear order on the non-port-elements.

\begin{block}{In a nutshell}
The 5 conditions $\Longrightarrow$ activities define an 
unambiguous Tutte function 
from the deletion/contraction and initial value formulas.\\

Additional conditions $\Longrightarrow$ the Tutte function has a rank-nullity
expansion.\\

(The rank-nullity conditions are satisfied in our application.)
\end{block}

\begin{block}{To specify the activity/deletion-contraction linear 
order GLOBALLY is 
UNNECESSARY.}
The Gordon/McMahon computation-tree-based 
activity theory also generalizes. (sdc).
\end{block}
\end{frame}









%%%%%%%%%%%%%%%%%%%%%%%%%%%%%%%%%%%%%%%%%%%%%%%%%
\begin{frame}[allowframebreaks]{References}
\bibliographystyle{plain}
\bibliography{../../bib/MathOfElec}{}
\end{frame}

\end{document}
