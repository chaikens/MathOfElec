\documentclass{beamer}
%\mode<presentation>
\usepackage{cite}
\usepackage{amsmath}
\title{Ported or Relative Oriented Matroids and Electric Circuits}
\author{Seth Chaiken\\
Dept. of Computer Science\\
Univ. at Albany\\
\url{schaiken@albany.edu}
}
\date{March 2, 2015}
%%%%%%%%%%%%%%%%%%%%%%%%%%%%%%%%%%%%%%%%%%%%%%%%%%%%%%%%%%%%%
% Specialized Symbols
%
%   Disjoint Union
%\newcommand{\dunion}{\uplus}
\newcommand{\dunion}
%{\mbox{\hbox{\hskip4pt$\cdot$\hskip-4.62pt$\cup$\hskip2pt}}}
%{\mbox{\hbox{\hskip6pt$\cdot$\hskip-5.50pt$\cup$\hskip2pt}}}
{\mbox{\hbox{\hskip0.45em$\cdot$\hskip-0.44em$\cup$\hskip0.2em}}}
%{\mbox{\hbox{\hskip0.45em$+$\hskip-0.70em$\cup$\hskip0.3em}}}
%
% Dot inside a cup.
% If there is a better, more Latex like way 
% (more invariant under font size changes) way,
% I'd like to know.

\newcommand{\Bases}[1]{\ensuremath{{\mathcal{B}}(#1)}}
\newcommand{\Reals}{\ensuremath{\mathbb{R}}}
\newcommand{\FieldK}{\ensuremath{K}}
\newcommand{\Perms}{\ensuremath{\mathfrak{S}}}
%\newcommand{\rank}{{\rho}}% {{\mbox{rank}}}
%\newcommand{\Rank}{{\rho}}% {{\mbox{rank}}}
\newcommand{\rank}{{\mbox{r}}}% {{\mbox{rank}}}
\newcommand{\Rank}{{\mbox{r}}}% {{\mbox{rank}}}
\newcommand{\Card}[1]{\ensuremath{{\left|#1\right|}}}
\newcommand{\ext}[1]{\ensuremath{\mathbf{#1}}}
%\newcommand{\extvee}{\ensuremath{\mathbf{\vee}}}
\newcommand{\extvee}{\;\;}
\newcommand{\Plucker}{Pl\"{u}cker\ }

% Set Complement
% command to mess with overline, bar or custom 
% alternatives for sequence or set complement
%
\newcommand{\scomp}[1]{\ensuremath{\overline{#1}}}
%\newcommand{\scomp}[1]{\ensuremath{\bar{#1}}}
%
%   Put a symbol for a matroid in a box, or brackets
%\newcommand{\MVAR}[1]{{\boxed{#1}\;}}
\newcommand{\MVAR}[1]{{[#1]\;}}
%
\newcommand{\UNION}{\cup} %try to make this bold.
%
%Emphasize in color!
\newcommand{\Remph}[1]{{\color{red}#1}}
%
%%%%%%%%End of specialized symbols%%%%%%%%%%%%%%%%%%%%%%



\begin{document}
\begin{frame}
 \titlepage
\end{frame}

\note{(here we go, geronimo)}

%%%%%%%%%%%%%%%%%%%%%%%%%%%%%%%%%%%%%%%
\begin{frame}{Why Electricity, EE?}
\begin{itemize}
\item Scholarly topic suggested by G.-C. Rota $\approx$ 1980?.
\item $\approx 100$ yrs. geometry-like intuition of
circuit configurations known by engineers, EE books:
\Remph{``Intuitive Analog Circuit Design (2013)''
\cite{intuitAna}}; ``Non-linear Circuits'' \cite{HaslerNeirynck}
translates to our Oriented Matroid pair model.

\item Geometry of linear spaces and oriented matroids;
Tutte decomp. w/
techniques from Barnabi, Brini and Rota's Exterior Calculus 
\cite{exteriorCalc})

\item
Real behavior $\approx$ ideal plus perturbations,
ideal constraints predict intended real behavior,
%???? RISK THIS?? \Remph{applied everyday geometry:}

\item 
Interesting, accessible, intuitively understandable
intentential designs, applicable,
easy to both simulate and build physically, dimension
$\approx$
12 or 24, depending on formulation
\item
Analogs to chemical (and real algebraic geometry 
\cite{signsInChemRAG}), biological, \Remph{elastic/tensegrity strs.} etc., 
random walks ...
\item
Merely one scalar non-linearity can cause chaos.
\end{itemize}
\end{frame}

\note{
(nonframe)Analogies:
\begin{enumerate}
\item Electric: Voltage, Resistance, Current, Capacitance, Inductance
\item Mech. Dyn: Force, Viscous friction, Velocity, Elasticity, Mass
\item Statics: Displacement, Elasticity, Force,  ???, ???
\item Random Walk: First hit probability, transition probability, flow, ???, ??? \\
Doyle and Snell \cite{DoyleSnellRandom},
Lyons and Peres \cite{ProbOnTreesNetworks}
(ref? Thanks, David G. Wagner's course notes).
\item Traffic: Time, Clogginess (travel time/flow ratio), flow, ???, ???\\
Condition: All vehicles reach their destinations in equal times regardless of 
route (ref? Thanks, Peter Shor).
\item Knot theory: ???, $\pm 1$ crossing sign, ???, ???, ???\\
Equivalent resistance ``Conductance Invariant'' of rational tangles is a 
Reinmeister move 
invariant\cite{RationalTangles} (Thanks, Jay Goldman).
\end{enumerate}

KCL+KVL+Ohms=(one of KCL)+Ohms+(min power)

Future: Memistor?
}



%%%%%%%%%%%%%%%%%%%%%%%%%%%%%%%%%%%%%%%%%%%%%%%%%%%%%%%
\begin{frame}{Kirchhoff (1847)
\cite{Kirchhoff}
 Maxwell (1891)
\cite{MaxR} 
The equivalent resistance PROBLEM IS SOLVED
by the Matrix Tree Theorem. (1) POSE! the \Remph{VARIABLES} or 
\Remph{COORDINATES}}

\framebox{\begin{minipage}{.47\textwidth}
\framebox{\Remph{$v_e, i_e : e\in E$}}
\input{resistor.pspdftex}
(Use \textbf{voltage drops along the flow}, 
not potentials $V_1$, $V_2$.)
\begin{center}\textbf{ordinary, resistor edge} $e\in E$
\end{center}

(IF linear Ohm's law use $|E|$ 
variables ($g_ev_e=r_ei_e$) ELSE
use $2|E|$ variables.)

\end{minipage}}
\framebox{\begin{minipage}{.47\textwidth}
\framebox{\Remph{$v_p, i_p : p\in P$}}
\input{portWithEnviron.pspdftex}
\begin{center}\textbf{DISTINGUISHED, PORT edge} $p\in P$
\end{center}
The interface to an environment is modelled with
$2|P|$ variables.
\begin{center}(math, not EE 
sign convention)
\end{center}
\end{minipage}}\\
\end{frame}

%%%%%%%%%%%%%%%%%%%%%%%%%%%%%%%%%%%%%%%%%%%%%%%%%
\begin{frame}{(2) POSE: EQUATIONS.  \Remph{Preview the consequences.}}
\begin{itemize}
\item 
(KCL) $(i_e)_{e\in S}$ is a cycle (a flow).
\item
(KVL) $(v_e)_{e\in S}$ is a cocycle
\item
(constituitive Law) $i_e=g_e(v_e)$
non-linear, usually monotonic increasing $R\rightarrow R$.\\ 
(Sometimes use Ohm's approximation $i_e=g_ev_e$)
\end{itemize}
\begin{block}{Combinatorics!}
The signs $\{ +, -, 0\}$ have a \Remph{DUAL-PAIR ORIENTED MATROID} structure
(combinatorial, geometric, topological).
\end{block}
\begin{block}{Engineering with amplifiers!}
There's good unique solvablility due to STRUCTURE,
when the \Remph{NON-DUAL PAIR} (for voltages and currents)
is ALMOST DUAL: No common covectors.
\end{block}
\end{frame}



%%%%%%%%%%%%%%%%%%%%%%%%%%%%%%%%%%%%%%%%%%%%%%%%%
\begin{frame}{SOLUTION: Equiv. Resistance $ :\equiv -(v_p/i_p)$
observed at a port $p$ by the environment
EQUALS a Ratio of Spanning Tree Enumerators!
(Port edge $p$ locates the \textit{2 terminals}.)}

\[
-(\frac{v_p}{i_p})=\frac{\mbox{WTS}(G/p)}
{\mbox{WTS}(G\backslash p)} 
= \frac{\mbox{Matrix-Tree Det}(G/p)}
       {\mbox{Matrix-Tree Det}(G\backslash p)}
\]
\begin{itemize}
\item
``Maxwell's rule'' uses MatrTreeT on 2 DIFFERENT GRAPHS\\
\begin{center}($G/p$ and $G\backslash p$) (Sorry, amplifiers come later.) \end{center}
\item
Weighted Tree Sum (WTS) is a colored Tutte function:
\[
\mbox{WTS}(G') =
g_e \mbox{WTS}(G'/e) + r_e \mbox{WTS} (G \backslash e)
\text{\ for all\ }e \not\in P
\]
\[
\mbox{WTS}(\text{coloop}(e)) = g_e
\]
\[
\mbox{WTS}(\text{loop}(e)) = r_e
\]
\end{itemize}
\end{frame}

%%%%%%%%%%%%%%%%%%%%%%%%%%%%%%%%%%%%%%%%%%%%%%%%%
\begin{frame}{Multiple Ports. 
(your stereo: \Remph{3=}power plug \& 2 speakers) }

\begin{itemize}
\item One formula expresses $\binom{2|P|}{|P|}$ different Matrix Tree 
Theorems...
\item
... long vertex-based proofs are shortened; Rayleigh inequalities too.
\item
Interesting \textbf{non-commutative ranges} of
new ORIENTED MATROID Tutte invariants with pattern:
\[
\text{TF}(N(P\dunion E)) = F(N(P\dunion E)/E)
\]
(They distinguish DIFFERENT ORIENTATIONS of the SAME MATROID.)
\item
Formalize composition of systems\cite{NarayananDecompVS1986}, 
Tutte poly. splitting formulas.
\item
Label variables to observe
\item
Model practical devices (transistors, op amps)
\item 
Align EE applications with knots (Ported = ``Relative'')
and combinatorial geometry (Ported = ``Set Pointed'').  
\end{itemize}
\end{frame}

%%%%%%%%%%%%%%%%%%%%%%%%%%%%%%%%%%%%%%%%%%%%%%%%%%%%
\begin{frame}{Constraint/Generator Duals and 2 Results.}
\begin{minipage}{0.48\textwidth}

\begin{itemize}
\item
(Part 1) Technique:
\begin{center}$\text{Solution Space}$\end{center}
\begin{center}$=$\end{center}
\begin{center}$\bigcap \text{Constraint Subspaces}$\end{center}
\item
\Remph{Result:} An exterior algebraic
algebraic Tutte function: Each of its
$\binom{2|P|}{|P|}$
Pl\"{u}cker coordinates
satisfies a Matrix
Tree Theorem.\\
 
This and det. formulas
easily prove Rayleigh inequalities.
\end{itemize}

\end{minipage}
\begin{minipage}{0.48\textwidth}

\begin{itemize}
\item
(Part 2) Combine with:
\begin{tabular}{c}
$\text{Solution Space}$\\
$=$\\
$\text{Closure}(\text{Set of Generators})$\\
\end{tabular}
\item
To apply: An oriented matroid's
COVECTOR SET encodes ALL POSSIBLE
 $(+,-,0)$ coordinate behaviors or
$\delta$s.
\item
\Remph{Result:} 
An oriented matroid pair model
for some non-linear problem
\Remph{(AMPLIFIER!)} well-posedness. 
(How? Sign contradictions $\Rightarrow$
a KERNEL=$\{(0)\}$.)
\end{itemize}
\end{minipage}
\end{frame}

\begin{frame}
{Part 1) Use Matrix $M$ in CONSTRAINTS $MX=0$ to get...}
%\begin{center}
The Tutte-like function $\mathbf{M}_E():\text{Extensor }
\mathbf{N}\rightarrow\text{Extensor }\mathbf{M_E(N)}$.
%\end{center}

{\small
(\Remph{STUDENT NOTE:} An EXTENSOR represents the row-space of an $r\times s$
$r-$rank matrix $M$ by the $\binom{s}{r}$-TUPLE of the DETERMINANTS of 
$M$'s $r\times r$ submatrices. \Remph{Pl\"{u}cker coords}.)
}

Given $N$ (matrix), construct $N^\perp$ 
with orthog. comp. row space.

Construct:  ($G=\mbox{diag}(g_e)$, $R=\mbox{diag}(r_e)$)
\[
M = \left[\begin{array}{c|c|c} N(P)  &  0  &  N(E)G \\  \hline
0  & N^{\perp}(P)  &  N^{\perp}(E)R \end{array}\right]
\]
with columns labelled by $P_I\dunion P_V\dunion E$.

Extensor $\mathbf{M}$ over $k[g_e, r_e](P_V\dunion P_I \dunion E)$
is the \Remph{$\wedge$-product} of $M$'s \textbf{row vectors}. The contraction result
$\mathbf{M}_E(\mathbf{N}) = \mathbf{M}/E$ appears:
\[
\mathbf{M} = \mathbf{M}_E(\mathbf{N})\mathbf{e_1}\mathbf{e_2}\cdots\mathbf{e}_{|E|} + (\cdots) 
\]

\Remph{
$\mathbf{M}_E(\mathbf{N})$ is our Tutte function $\mathbf{N}\rightarrow \text{Ext. Alg.}$}
\end{frame}


\begin{frame}{Contracting means ``Eliminate variables''}
ELIMINATE the variables indexed by $E$, leaving $2|P|$ variables
labelled by $P_I$ and $P_V$.  ie, CONTRACT $E$. \textbf{Answer} $A$ IS:

\[
\mathbf{M}_E = \bigwedge^{\text{Exterior}}_{\text{JOIN over rows}} \left[\begin{array}{c|c} A_{I,I}  &  A_{I,V}   \\  \hline
    A_{V_I}  & A_{V,V} \end{array}\right] 
[\mathbf{p_{I_1}, \cdots, p_{I_p}; p_{V_1}, \cdots, p_{V_p}}]^{\mathbf{t}} 
\]

\[
 = \ldots + C_i\mbox{\bf XXX} + \ldots; \text{Equiv. Resistance} = 
\text{certain}\;\; C_i/C_j
\]

All the other $C_k$'s have similar interpretations.

{\bf $\binom{2|P|}{|P|}$ Matr. Tree Theorems:}
Each $C_k(N)$ (a PRINCIPAL MINOR of $M_E$ ABOVE!)
$= 
g_e C_k(N/e) + r_e C_k(N\backslash e)$ ($e\not\in P$, $e$ not (co)loop).

Each $C_k$ is a signed weighted enumerator of
forests satisfying \textbf{conditions ...}
\end{frame}


%%%%%%%%%%%%%%%%%%%%%%%%%%%%%%%%%%%%%%%%%%%%%%%%%
\begin{frame}{Conditions (what sets $F$ are enumerated by one det. $C_i$)
}
The \textbf{conditions ...}
are on the rank, nullity of $F$ and, WHAT ORIENTED MINOR is \
$G/F\setminus (E\setminus F)$, the minor
with ONLY PORT EDGES from contracting $F$
and deleting the other resistor edges, leaving the
ports.

The conditions for a given $C_k$ \textit{sometimes}
make all the signs the same (eg: $C_i$ and 
$C_j$ in 1-port equivalent resistance $R=C_i/C_j$)

\textit{Othertimes}, the oriented \textbf{P-minors}
in the completed Tutte decomposition of $C_k$ determine
some + and some - signs.

\begin{center}
\begin{minipage}{0.3\textwidth}
\begin{tabular}{c}
When $[G/F|P]$ is \\
\input{c2plus.pspdftex} \\
the term is \\
\Remph{{\LARGE\bf +}}$g_Fr_{E\setminus F}$ \\
\end{tabular}
\end{minipage}
\begin{minipage}{0.3\textwidth}
\begin{tabular}{c}
When $[G/F|P]$ is\\
\input{c2minus.pspdftex}\\
the term is\\
\Remph{{\LARGE\bf -}}$g_Fr_{E\setminus F}$\\
\end{tabular}
\end{minipage}
\end{center}

\end{frame}

%%%%%%%%%%%%%%%%%%%%%%%%%%%%%%%%%%%%%%%%%%%%%%%%%
\begin{frame}
\frametitle{Application: Rayleigh Identity, ``Neg. Spanning Tree Correlation''}
\[
\Gamma_e(G)\text{ is equivalent conductance across }e.
\text{ Rayleigh: }0 \le \frac{\partial \Gamma_{p}}{\partial g_f} =
\frac{\partial \frac{T_G}{T_{G/e}}}{\partial g_f}
\]
is equivalent to 
\[
0 \le \frac{\partial T_G}{\partial g_f}T_{G/e} - 
       T_G\frac{\partial T_{G/e}}{\partial g_f} 
=
T_{G/f}T_{G/e} - T_GT_{G/e/f}
\]
In fact,
\[
T_{G/f}T_{G/e} - T_GT_{G/e/f} = \left( T^+_{G/e \text{ \& } G/f} - T^-_{G/e \text{ \& } G/f} \right)^2
\]
$T^{\pm}_{G/e \text{ \& } G/f}$ enumerate the $\pm$ common spanning trees.
\end{frame}

\begin{frame}{(Part 2) Common Covector Model}
\input{commonCovectorModel.pspdftex}
\end{frame}

\begin{frame}
\frametitle{Known Partial and Full Combinatorial Proofs}
\[
T_{G/f}T_{G/e} - T_GT_{G/e/f} = \left( T^+_{G/e \text{ \& } G/f} - T^-_{G/e \text{ \& } G/f} \right)^2
\]
$T^{\pm}_{G/e \text{ \& } G/f}$ enumerate the $\pm$ common spanning trees.

\vfill
Choe (2004) 
proved essentially this using the vertex-based all-minors matrix tree theorem,
combinatorial cases and Jacobi's theorem relating the minors of a matrix to
the minors of its inverse..

\vfill
Cibulka, Hladky, Lacroix and Wagner (2008) gave a completely bijective proof
that utilizes some natural 2:2 and 2:1 correspondances.

\vfill
\Remph{Difficulty:} Some terms on the left \Remph{cancel} and some
reduce to terms with coefficients $\pm 2$.
\vfill
\end{frame}



\begin{frame}
\frametitle{Linear Alg./Oriented Matroid Proof of Rayleigh's Identity}
Let $R$ be the transfer resistance matrix for 2 ports across $e$ and $f$.
Our result implies that
\[
\det R = \left|\begin{array}{cc} R_{ee} & R_{ef} \\ R_{fe} & R_{ff} \end{array}\right|
= \alert{+} \frac{T_{G/e/f}}{T_G}
\]
It and better-known results tell us
\[
R_{ee} = \frac{T_{G/e}}{T_G};\;\;R_{ff} = \frac{T_{G/f}}{T_G};\;\;
R_{ef}=R_{fe}=\frac{ T^+_{G/e \text{ \& } G/f} - T^-_{G/e \text{ \& } G/f} }{T_G}
\]
%Rayleigh's identity 
$T_{G/f}T_{G/e} - T_GT_{G/e/f} = \left( T^+_{G/e \text{ \& } G/f} - T^-_{G/e \text{ \& } G/f} \right)^2$
is immediate after substituting these into
\[
\det R = R_{ee}R_{ff}-(R_{ef})^2
\]
\alert{The $+$ follows from physical grounds if the $g_e, r_e \geq 0$.  Our
characterization and proof are combinatorial.}
\end{frame}

\begin{frame}
\frametitle{New Rayleigh's Identities!}

The same method generates identities and inequalities from
\[
\left|
\begin{array}{ccc} R_{ee} & R_{ef} & R_{eg} \\ 
                   R_{fe} & R_{ff} & R_{fg} \\
                   R_{ge} & R_{gf} & R_{gg}
\end{array}\right|
= \alert{+} \frac{T_{G/e/f/g}}{T_G} \ge 0
\]
when all $r_{..}, g_{..} \ge 0$, ETC...

\vfill

(Applications???)

\vfill

\Remph{Might the same methods address a much harder problem:
The same inequality for } forests \Remph{ instead of 
spanning trees?}

\vfill
\end{frame}





%%%%%%%%%%%%%%%%%%%%%%%%%%%%%%%%%%%%%%%%%%%%%%%%%
\begin{frame}{(Part 2) Common Covector Model}
\input{commonCovectorModel.pspdftex}
\end{frame}


%%%%%%%%%%%%%%%%%%%%%%%%%%%%%%%%%%%%%%%%%%%%%%%%%
\begin{frame}
\frametitle{Voltage and Current graphs $G_V$, $G_I$}

\framebox{\begin{minipage}{0.52\textwidth}
``Voltage graph'' $G_V$ (EE\cite{HaslerNeirynck,PathologicalAct}, NOT Gross, ...) represents KVL
$\mathbf{v}\in \text{Cocycles}$ W/ SOME $v_e \equiv 0$
\end{minipage}}
\framebox{
\begin{minipage}{0.4\textwidth}
``Current graph'' $G_I$ represents KCL
$\mathbf{i}\in \text{Cycles}$ WITH SOME FLOWS $\equiv 0$
\end{minipage}}

\begin{itemize}
\item 
They are EQUAL GRAPHS for resistor networks.

\item
For networks with idealized amplifiers, they are not 
equal.  

\input{idealAmp.pspdftex}

The output voltage and current are whatever makes the input
voltage and current BOTH BE zero.

\item
(More) realistic amp. model $=$ idealized amp. $+$ 
resistors.

\end{itemize}


\framebox{
\begin{minipage}{0.2\textwidth}
\Remph{open}
\[
G_v = G\backslash e
\]
\[ 
G_I = G\backslash e
\]
\end{minipage}}
\framebox{\begin{minipage}{0.2\textwidth}
\Remph{short}
\[
G_v = G/e
\]
\[
G_I = G/e
\]
\end{minipage}}
\framebox{
\begin{minipage}{0.2\textwidth}
\Remph{nullator}
\input{nullator.pspdftex}
\[
G_v = G/e
\]
\[ 
G_I = G\backslash e
\]
\end{minipage}}
\framebox{\begin{minipage}{0.2\textwidth}
\Remph{norator}
\input{norator.pspdftex}
\[
G_v = G\backslash e
\]
\[
G_I = G/e
\]
\end{minipage}}

\end{frame}

%%%%%%%%%%%%%%%%%%%%%%%%%%%%%%%%%%%%%%%%%%%%%%%%%
\begin{frame}{``Colors'' are parameters on every Tutte decomposition step}

The Bollobos/Riordan/Zaslavsky\cite{BollobasRiordanTuttePolyColored,MR93a:05047}, Traldi-Ellis-Monaghan\cite{Ellis-Monaghan-Traldi}, (sdc unpub)
BRZ theory for well-definedness
of ``Relative Tutte Polynomials for Colored Graphs'' ALL GOES THROUGH 
(Diao and Hetyei \cite{RelTuttePoly}):
The 3 BRZ conditions on (colors,initial values) GENERALIZE TO 5;
activity theory WORKS TOO, when
based on linear orders on the non-port-elements.

\begin{block}{In a nutshell}
The 5 conditions $\Longrightarrow$ activities define an 
unambiguous Tutte function 
from the deletion/contraction and initial value formulas.\\

Additional conditions $\Longrightarrow$ the Tutte function has a rank-nullity
expansion.\\

\Remph{
(The rank-nullity conditions are satisfied in our application.)}
\end{block}

\begin{block}{To specify the activity/deletion-contraction linear 
order GLOBALLY is 
UNNECESSARY.}
The Gordon/McMahon computation-tree-based 
activity theory also generalizes. (sdc).
\end{block}
\end{frame}


%%%%%%%%%%%%%%%%%%%%%%%%%%%%%%%%%%%%%%%%%%%%%%%%%
\begin{frame}[allowframebreaks]{References}
\bibliographystyle{plain}
\bibliography{../../bib/MathOfElec}{}
\end{frame}


%%%%%%%%%%%%%%%%%%%%%%%%%%%%%%%%%%%%%%%%%%%%%%%%%%%%
%spare slides


%%%%%%%%%%%%%%%%%%%%%%%%%%%%%%%%%%%%%%%%%%%%%%%%%
\begin{frame}{(15)}

\end{frame}

%%%%%%%%%%%%%%%%%%%%%%%%%%%%%%%%%%%%%%%%%%%%%%%%%
\begin{frame}{(16)}
\end{frame}

%%%%%%%%%%%%%%%%%%%%%%%%%%%%%%%%%%%%%%%%%%%%%%%%%
\begin{frame}{(17)}


Our Tutte-like function $\mathbf{M}_E(\mathbf{N}):\text{Extensors}\rightarrow\text{Extensors}$.

Given $N$ (matrix), construct $N^\perp$ 
with orthog. comp. row space.

Construct:  ($G=\mbox{diag}(g_e)$, $R=\mbox{diag}(r_e)$)
\[
M = \left[\begin{array}{c|c|c} N(P)  &  0  &  N(E)G \\  \hline
0  & N^{\perp}(P)  &  N^{\perp}(E)R \end{array}\right]
\]
with columns labelled by $P_I\dunion P_V\dunion E$.

Extensor $\mathbf{M}$ over $k[g_e, r_e](P_V\dunion P_I \dunion E)$
is the product of $M$'s \textbf{row vectors}. The contraction result
$\mathbf{M}_E(\mathbf{N}) = \mathbf{M}/E$ appears:
\[
\mathbf{M} = \mathbf{M}_E(\mathbf{N})\mathbf{e_1}\mathbf{e_2}\cdots\mathbf{e}_{|E|} + (\cdots) 
\]

$\mathbf{M}_E(\mathbf{N})$ is our Tutte function $\mathbf{N}\rightarrow \text{Ext. Alg.}$

\end{frame}



%%%%%%%%%%%%%%%%%%%%%%%%%%%%%%%%%%%%%%%%%%%%%%%%%
\begin{frame}{What are the conditions like?}

\[
\text{Extensor\ }= \sigma(C)(a_1\vee a_2\vee \ldots \vee a_d)
\]
\[
C = \ldots + t + \ldots
\]
$t \leftrightarrow \text{\ sets of (``contractible'' non-port edges} 
E_1, E_2$ for which
\[
N_V/E_1\backslash(E\setminus E_1) = \text{\ certain OMs on\ } P
\]
AND
\[
N_I/E_2\backslash(E\setminus E_2) = \text{\ certain OMs on\ } P
\]

Transfer resistance might be 0 and might be $\neq 0$ iff
\[
\exists E_1, E_2 \text{so (diag) and (diag)}
\]

($K_4$ Wheatstone bridge diagram)

\[
\frac{R_1}{R_2} > \frac{R_3}{R_4} 
\text{\ neg\ }
\frac{R_1}{R_2} < \frac{R_3}{R_4} 
\text{\ pos\ }
\]

\end{frame}


\begin{frame}
\frametitle{Equivalent resistance is a coefficient ratio in an 
implicitly defined linear function}

(diagram)

In other words
\[
R_pi_p + v_p = 0
\]
or dually,
\[
\mathcal{B} = \{(i_p,v_p)\}
= \{ t(-1, R_p) | t\in R\}
\]

\end{frame}

\note{nonframe(linear subspace over the $R^{\mathbf{P}}$)}

%%%%%%%%%%%%%%%%%%%%%%%%%%%%%%%%%%%%%%%%%%%%%%%%%
\begin{frame}
\frametitle{The $2\times d$ port variable constraint space, 
and its solution spaces, are $d$-dimensional.}

We represent these spaces by carefully defined
\textbf{extensors}, as Barnabei, Brini and Rota \cite{exteriorCalc}
term ``decomposible antisymmetric tensors''

The solution extensor (not a ray) satisfies:

\[
E(N) = \text{sign}(...)(g_e E(N/e) + r_e E(N\backslash e)
\text{\ for\ }e\not\in P
\]

\end{frame}


\note{
(nonframe) 
The port elements represent the basis chosen so coordinates correspond
to voltages and currents belonging to network edges.  Geometric aspects
of extensors are surveyed by 
Barnabei and Brini \cite{exteriorCalc}.

Cordovil\cite{CommAlgOMs} 
presented a \textbf{commutative algebraic} function of 
oriented matroids in which monomial signs encode orientation information.
The analogy to our \textbf{exterior algebraic} functions must be explored. 
}

%%%%%%%%%%%%%%%%%%%%%%%%%%%%%%%%%%%%%%%%%%%%%%%%%
\begin{frame}{Coefs}
\[
E(N) = \ldots + C_i\mbox{\bf XXX} + \ldots
\]
\[
R = C_i/C_j
\]
All the other $C_k$'s have similar interpretations.

Each $C_k$ is a determinant.

Each $C_k$ is a signed weighted enumerator of
forests satisfying \textbf{conditions ...}

Each $C_k$ satisfies
\[
C_k(N) = g_e C_k(N/e) + r_e C_k(N\backslash e)
\text{\ for\ }e\not\in {P}
\]

\end{frame}

%%%%%%%%%%%%%%%%%%%%%%%%%%%%%%%%%%%%%%%%%%%%%%%%%
\begin{frame}{Conditions}
What is the nature of the conditions?  We state them using the 
network's graphic oriented matroid.

(diagram--glob w/ ports)

\end{frame}





\note{
nonframe \textbf{port} (sdc, engineering literature), 
\textbf{distinguished element}, 
element of the restriction's set (Diao and Hetyei), 
a element in the set of points (Las Vergnas).
}










\end{document}
