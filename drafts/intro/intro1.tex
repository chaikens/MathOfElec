\documentclass{article}
\usepackage{cite}
\title{Introduction}
\author{Seth Chaiken}
\begin{document}
\maketitle

(here we go, geronimo)

Why Electricity, EE?
\begin{itemize}
\item $>$ Some 100 years geometry-like intuition

(Marc Thompson ``Intuitive Analog Circuit Design'' 2nd ed 2013
\cite{intuitAna}, Hasler, Neirynck translates pretty directly into
OM pair analysis \cite{HaslerNeirynck})



\item
Real behavior often equals ideal plus perturbations

\item
Ideal constraints result in intended real behavior

(Ideal predicts the real)

(how geometry is applied everyday)

\item 
Interesting, accessible, intuitively understandable
(because of (human, intentential) design), applicable,
easy to both simulate and build physically, structure of moderate
dimension (about 12 or 24, depending on what they may count / formulation)
\item
Analogs to chemical, biological, mechanical, etc., random walks.
\item
Merely one scalar non-linearity can cause chaos.
\end{itemize}


\bibliography{../../bib/MathOfElec}{}
\bibliographystyle{plain}

\end{document}
