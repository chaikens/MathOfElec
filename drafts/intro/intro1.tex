\documentclass{article}
\usepackage{cite}
\usepackage{amsmath}
\title{Introduction}
\author{Seth Chaiken}
\begin{document}
\maketitle

(here we go, geronimo)


\rule{\textwidth}{3pt}
(1)








Why Electricity, EE?
\begin{itemize}
\item $>$ Some 100 years geometry-like intuition

(Marc Thompson ``Intuitive Analog Circuit Design'' 2nd ed 2013
\cite{intuitAna}, Hasler, Neirynck translates pretty directly into
OM pair analysis \cite{HaslerNeirynck})


\item Geometry of linear spaces and oriented matroids.

(Our results include a Tutte decomposition of an exterior algebra
valued function of oriented matroids.)

(We apply techniquesfrom Barnabi, Brini and Rota's Exterior Calculus 
\cite{exteriorCalc})

\item
Real behavior often equals ideal plus perturbations

\item
Ideal constraints result in intended real behavior

(Ideal predicts the real)

(how geometry is applied everyday)

\item 
Interesting, accessible, intuitively understandable
(because of (human, intentential) design), applicable,
easy to both simulate and build physically, structure of moderate
dimension (about 12 or 24, depending on what they may count / formulation)
\item
Analogs to chemical (and real algebraic geometry 
\cite{signsInChemRAG}), biological, mechanical, etc., random walks ...



\item
Merely one scalar non-linearity can cause chaos.
\end{itemize}
\rule{\textwidth}{3pt}

Analogies:
\begin{enumerate}
\item Electric: Voltage, Resistance, Current, Capacitance, Inductance
\item Mech. Dyn: Force, Viscous friction, Velocity, Elasticity, Mass
\item Statics: Displacement, Elasticity, Force,  ???, ???
\item Random Walk: First hit probability, transition probability, flow, ???, ??? \\
Doyle and Snell \cite{DoyleSnellRandom},
Lyons and Peres \cite{ProbOnTreesNetworks}
(ref? Thanks, David G. Wagner's course notes).
\item Traffic: Time, Clogginess (travel time/flow ratio), flow, ???, ???\\
Condition: All vehicles reach their destinations in equal times regardless of 
route (ref? Thanks, Peter Shor).
\item Knot theory: ???, $\pm 1$ crossing sign, ???, ???, ???\\
Equivalent resistance ``Conductance Invariant'' of rational tangles is a 
Reinmeister move 
invariant\cite{RationalTangles} (Thanks, Jay Goldman).
\end{enumerate}

KCL+KVL+Ohms=(one of KCL)+Ohms+(min power)


Future: Memistor?


\subsection{Equivalent Resistance}




\pagebreak[3] \begin{quote}\rule{\textwidth}{3pt}
(2)\\
(Kirchhoff (1847), Maxwell (1891)) The equivalent resistance problem
is solved by
the Matrix Tree Theorem.
 
\rule{\textwidth}{3pt}
\end{quote}


Kirchhoff\cite{Kirchhoff} and 
Maxwell\cite{MaxR} solved the equivalent resistance problem
by the matrix tree theorem.


\rule{\textwidth}{3pt}\\
(3)\\
\begin{minipage}{.4\textwidth}
(resistor edge diagram)\\
$i_e$\\
$v_e=v_1-v_2$\\
$g_ev_e=r_ei_e$
\end{minipage}
\begin{minipage}{.4\textwidth}
(port edge diagram, environment)\\
$i_p$\\
$v_p=v_1-v_2$\\
mathematics sign convention.\\
$g_ev_e=r_ei_e$
\end{minipage}\\
\rule{\textwidth}{3pt}\\


\pagebreak[3] \begin{quote}\rule{\textwidth}{3pt}\\
(4)\\
(KCL) $(i_e)_{e\in S}$ is a cycle (a flow)\\
(KVL) $(v_e)_{e\in S}$ is a cocycle\\
(constituitive Law) $i_e=g_e(v_e)$
non-linear, usually monotonic increasing $R\rightarrow R$ 
(Ohm's approximation $i_e=g_ev_e$\\
signs ($\pm$) have oriented matroid structure
(combinatorial, geometric, topological)\\
\rule{\textwidth}{3pt}
\end{quote}

\pagebreak[3] \begin{quote}\rule{\textwidth}{3pt}\\
(5)\\
Equivalent resistance is a \textbf{ratio}
\[
\frac{\mbox{WTS}(G/p)}
{\mbox{WTS}(G\backslash p)}
\]
\[
\mbox{(WeightedTreeSum)}\mbox{WTS}(G') =
g_e \mbox{WTS}(G'/e) + r_e \mbox{WTS} (G \backslash e)
\text{\ for all\ }e \not\in P
\]
\rule{\textwidth}{3pt}
\end{quote}


\pagebreak[3] \begin{quote}\rule{\textwidth}{3pt}\\
(6)\\
Multiple ports

\begin{itemize}
\item
Formalize composition of systems
\item
Label variables to observe
\item
Model practical devices (transistors, op amps)
\end{itemize}

\rule{\textwidth}{3pt}
\end{quote}

\pagebreak[3] \begin{quote}\rule{\textwidth}{3pt}\\
(7)\\
Equivalent resistance is a coefficient ratio in an 
implicitly defined linear function.

(diagram)

In other words
\[
R_pi_p + v_p = 0
\]
or dually,
\[
\mathcal{B} = \{(i_p,v_p)\}
= \{ t(-1, R_p) | t\in R\}
\]

\rule{\textwidth}{3pt}
\end{quote}

(linear subspace over the $R^{\mathbf{P}}$)

\pagebreak[3] \begin{quote}\rule{\textwidth}{3pt}\\
(8)\\
The $2\times d$ port variable constraint space, 
and its solution spaces, are $d$-dimensional.

We represent these spaces by carefully defined
\textbf{extensors}, as Barnabei, Brini and Rota \cite{exteriorCalc}
term ``decomposible antisymmetric tensors''

The solution extensor (not a ray) satisfies:

\[
E(N) = \text{sign}(...)(g_e E(N/e) + r_e E(N\backslash e)
\text{\ for\ }e\not\in P
\]

\rule{\textwidth}{3pt}
\end{quote}

The port elements represent the basis chosen so coordinates correspond
to voltages and currents belonging to network edges.  Geometric aspects
of extensors are surveyed by 
Barnabei and Brini \cite{exteriorCalc}.

\pagebreak[3] \begin{quote}\rule{\textwidth}{3pt}\\
(9)\\
\[
E(N) = \ldots + C_i\mbox{\bf XXX} + \ldots
\]
\[
R = C_i/C_j
\]
All the other $C_k$'s have similar interpretations.

Each $C_k$ is a determinant.

Each $C_k$ is a signed weighted enumerator of
forests satisfying \textbf{conditions ...}

Each $C_k$ satisfies
\[
C_k(N) = g_e C_k(N/e) + r_e C_k(N\backslash e)
\text{\ for\ }e\not\in {P}
\]

\rule{\textwidth}{3pt}
\end{quote}


\pagebreak[3] \begin{quote}\rule{\textwidth}{3pt}\\
(10)\\
The \textbf{conditions ...} are best described with 
a \textbf{pair of ported oriented matroids}
(aka \textbf{relative or set-pointed}

The conditions for a given $C_k$ \textit{sometimes}
make all the signs the same (example: $C_i$ and 
$C_j$ in 1-port equivalent resistance $R=C_i/C_j$)

\textit{Othertimes}, the oriented \textbf{P-minors}
in the completed Tutte decomposition of $C_k$ determine
the sign.

(diagram of the two orientations of $C_2$)

\rule{\textwidth}{3pt}
\end{quote}

\pagebreak[3] \begin{quote}\rule{\textwidth}{3pt}\\
(11)\\
What is the nature of the conditions?  We state them using the 
network's graphic oriented matroid.

(diagram--glob w/ ports)

\rule{\textwidth}{3pt}
\end{quote}

\textbf{port} (sdc, engineering literature), 
\textbf{distinguished element}, 
element of the restriction's set (Diao and Hetyei), 
a element in the set of points (Las Vergnas).


\pagebreak[3] \begin{quote}\rule{\textwidth}{3pt}\\
(12)\\
Voltage and Current graphs



\begin{minipage}{0.4\textwidth}
``Voltage graph'' (EE, NOT Gross, et. al.) represents KVL
\[\mathbf{v}\in \text{Cocycle space}\]
\end{minipage}
\begin{minipage}{0.4\textwidth}
``Current graph'' represents KCL
\[\mathbf{i}\in \text{Cycle space}\]
\end{minipage}
They are equal graphs for resistor networks.

For networks with idealized amplifiers, they are not 
equal.  

(more) Realistic amplifier model $=$ idealized amplifiers $+$ 
resistors.

(ideal op amp nullator/norator T diagram)

The output voltage and current are whatever makes the input
voltage and current BOTH BE zero.

\begin{minipage}{0.4\textwidth}
\[
\text{(nullator)\ }N_v = N/e// 
N_I = N\backslash e
\]
\end{minipage}
\begin{minipage}{0.4\textwidth}
\[
\text{(norator)\ }N_v = N\backslash e// 
N_I = N/e
\]
\end{minipage}



\rule{\textwidth}{3pt}
\end{quote}

\pagebreak[3] \begin{quote}\rule{\textwidth}{3pt}\\
(13)\\
What are the conditions like?

\[
\text{Extensor\ }= \sigma(C)(a_1\vee a_2\vee \ldots \vee a_d)
\]
\[
C = \ldots + t + \ldots
\]
$t \leftrightarrow \text{\ sets of (``contractible'' non-port edges} 
E_1, E_2$ for which
\[
N_V/E_1\backslash(E\setminus E_1) = \text{\ certain OMs on\ } P
\]
AND
\[
N_I/E_2\backslash(E\setminus E_2) = \text{\ certain OMs on\ } P
\]

Transfer resistance might be 0 and might be $\neq 0$ iff
\[
\exists E_1, E_2 \text{so (diag) and (diag)}
\]

($K_4$ Wheatstone bridge diagram)

\[
\frac{R_1}{R_2} > \frac{R_3}{R_4} 
\text{\ neg\ }
\frac{R_1}{R_2} < \frac{R_3}{R_4} 
\text{\ pos\ }
\]

\rule{\textwidth}{3pt}
\end{quote}

\pagebreak[3] \begin{quote}\rule{\textwidth}{3pt}\\
(14)\\
(Rayleigh)



\rule{\textwidth}{3pt}
\end{quote}

\pagebreak[3] \begin{quote}\rule{\textwidth}{3pt}\\
(15)\\


\rule{\textwidth}{3pt}
\end{quote}

\pagebreak[3] \begin{quote}\rule{\textwidth}{3pt}\\
(16)\\


\rule{\textwidth}{3pt}
\end{quote}

\pagebreak[3] \begin{quote}\rule{\textwidth}{3pt}\\
(17)\\


\rule{\textwidth}{3pt}
\end{quote}




\bibliography{../../bib/MathOfElec}{}
\bibliographystyle{plain}

\end{document}
