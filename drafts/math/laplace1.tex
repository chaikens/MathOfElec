\documentclass{article}
\usepackage{cite}
\title{Laplace ... }
\author{Seth Chaiken}
\begin{document}
\maketitle

(a functional transformation, function $->$ function)

(a complex, more specifically, analytic function)

(a generating function, analog with $\int dt$ in place of $\sum$)

(approximation relations with discrete generating functions, 
aka ``$z$-transform''. Digital signal processing implements or
is approximately related to analog signal processing.)

(a formal ratio of an extensor's Plucker coordinates)

(eigenvalue of eigenvector $e^{s t}$ under a differential operator eg 
$(a\frac{d}{dx}^2 + b\frac{d}{dx} + c)(e^{st}) = (as^2 +bs + c)e^{st}$

(formal expression for impedances $R$, $1/Cs$, $Ls$)

(linear differential operator $s(f) = \frac{d}{dx}f$)

(deletion/contraction, 
``capacitor shorts out on high $s$, inductor opens on low $s$'' 
\cite{intuitAna})

(characterization of ``responses'' defined among port variables)

(?? Indicator of a step respone)

(asymptotic expansions as $s\rightarrow\infty$
and $s\rightarrow 0$)

\noindent\textbf{More and more}

(asymptotic behavior of a generating function's coefficients, which are
the generated function's values, determined by analytic properties of the
generated function.  Relate EE, circuits and signals literature to 
statistical mechanics and combinatorics.  
\cite{statMechForGraphers,multivarHalfPlane,AnalyticCombinatoricsBook})

in \cite{AnalyticCombinatoricsBook}: p.287 (Perspective of Ch. IV):
analytics: assign values to the generating function's variable.
``Singularities and growth. ... signularities provide essential information 
on the growth rate of a function's coefficients.  ... ``First Principle''
relates 
exponential growth to the location of sinularities.''
\quote{
within $[z^n]F(z) = A^n\Theta(n)$:
\begin{itemize}
\item First Principle: Location determines $A^n$ (exponential growth).
\item Second Principle: Nature determines the associate subexponential factor
$\Theta(n)$.
\end{itemize}
}

\noindent\textbf{must relate THIS to all the EE Laplace transform, 
Bode diagram, etc stuff.}

(random generation: \cite{BoltzmanSampRandCombGen})

(geometry of phase shifts)

(risetime related to basic topology, graph theory, topology of lumped
networks expressed by oriented matroids)

(FUTURE hopefully not too much: Complex oriented matroids of Anderson, et.
al. \cite{complexOM})

Fourier series as (1) elements of a group algebra, so group algebra product is
Fourier series convolution; (2) linear combination of group characters, which
are mutually orthogonal.


\bibliography{../../bib/MathOfElec}{}
\bibliographystyle{plain}

\end{document}
