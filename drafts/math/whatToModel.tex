\documentclass{article}
\usepackage{cite}
\title{What to Model ... }
\author{Seth Chaiken}
\begin{document}
\maketitle

Of a circuit:
\begin{itemize}
\item Resistance and other port quantities within the theory of 
\textbf{random walks}\cite{DoyleSnellRandom,ProbOnTreesNetworks}.
\item Electricity as statistics mechanincs of electron transfer.  Relate to 
random walk models.
\item Resistive port behavior; OM characterizes of 
forest term sign.
\item Monotone non-linear well-posedness.
\item OM of solution states.
\item Predictions and constraints derived from OM properties:
\begin{enumerate}
\item How orthogonality with an OM vector predicts one sign from others.
\item How OM covector elimination predicts a zero value is possible.
\end{enumerate}
\item OM of transients.
\item Signs of transients that might not be OM modelled.
\item Laplace transform polynomials.
\begin{itemize}
\item Bode diagram approx.\\
\textbf{NEW THING: Tropical algebra approximates real algebra!}
\item Stability
\item Phase margin.
\item Discrete model for spanning tree count: Sandpile group order.
\end{itemize}

\item Imaginary frequency $s=i\omega$.
\item Anderson's complex OM (after learning it properly.)
\item Cones of possible phases.  Is that what Anderson did?
\item Orders or levels of resistance, etc. magnitudes; as
in symbolic simulation.
\item Laplacian eigenvalues (what are the good for in elec. circuit
study?
\item Effect or information transfer directionality due to impedance
differences, ``half-resistors'',  \textbf{Bond graphs}.
\item Hamiltonian generalized position and momentum; that Frankel stuff.
\item Voltages and currents are orthogonal under a bilinear form that 
represents power.  What's behind this?
\end{itemize}

\section{Benefits of Formal Expression}

Vectors: (1) time series (2) transform expansion.  Good to formalize with
generating functions IE? some general Fourier expansion.  (How the same??)

Vectors: Multidimensional quantities: System state, multidim. signal, complex
value:  Good to formalize with the GROUP ALGEBRA or some of its generalization.

Connection:  Generating functions are the 
SEMI-GROUP algebra (over the coefficient space)
generated by $\{ z^n \}$.


\section{Engineering Intuitions}

\begin{enumerate}
\item
\textbf{A Study of paragraph-sized qualitative circuit operation descriptions}
\begin{itemize}
\item In terms of causal tracing of increases or decreases.
\item In terms of sinusoidal frequency response.
\item In terms of exponential frequency response.
\end{itemize}

\item 
Impedance differences between parts of circuits; impedance in intuition.

\item
\textbf{Approximation}

\item
\textbf{Purposeful Design}

\begin{itemize}
\item Need to account for 
\textbf{parasitic elements}
(1) Sometimes have small effects.
(2) Sometimes they must be compensated for.
\item Need for ``robustness'', researched with 
\textbf{Monte-Carlo methods} and
\textbf{Sensitivity analysis}. \cite{HiFreqChaosMakerEE}
\end{itemize}

\item
\textbf{Effects and Strategies}
\begin{itemize}
\item Effect of adding (or displacing) poles and zeros to a transfer function.
\item Compensation.
\item Feedback.
\item (Results of feedback): Miller effect, bootstrapping.
\end{itemize}

\item
\textbf{Physical causes expressed in an electrical model}.  Specifically, 
physics of devices (diode junctions, junction transtors, 
field-effect transistors, even non-ideal resistors, capacitor, inductor).

\item
\textbf{Active vs. Passive: Bias distinguished from signal}
``\textbf{Active device:} A device that can convert energy from a dc bias
source to a signal at an RF frequency.  Active devices are required for
oscillators and amplifiers.'' Microwave Devices in 
\textit{The Electrical Engineering 
Handbook}.\cite[ch.37, Streer and Trew]{EEHandbook}
(Find other 
definitions in the circuit theory literature?)

\item
\textbf{Thermal Runaway} Old: BJT power transistor latchup.  New: Hotspots 
in systems on a chip (SoC). There is a thermal network interacting with the 
electical network. Paper on it in VLSI \cite{ThermalVLSI}.

\item (Non-thermal?) Latchup (parasitic transistors) 
CMOS\cite{CMOSLatchUpTINote}

\end{enumerate}

\section{Topology}

EEs mean ``circuit shape'', the network graph decorated with 
element types.

Each network graph vertex models a conducting region of physical space, 
ignoring electromagnetics...

Idea: When we consider the system (circuit?) to be a physical object, the 
vertices are \textbf{topological contractions} of the conducting regions.
\textbf{The homology group} of the conductor surface is trivial (since the
conductor itself is simply connected.)   The same level homology group for 
the whole system is the cycle space which is the subject for Kirchhoff's 
current law.

Magnetic phonomena make non-graphic topologies (recalled from one of 
Duffin's papers, where a cycle is assumed to have zero flux).

(Refer to homotopy methods for analyzing non-differential non-linear 
equations.  Persistent homology is a new theme here.  This may bridge
statistical mechanical microscopic models to our macroscopic models.)

\section{OTHER}

The Boltzmann factor in interplay between combinatorics and statistical
mechanics (after reading Feynman's Lectures on Physics account).

Multiple interacting transport phenomena in transistors and other 
engineered systems (natural too!).
 
\section{Laplace etc from other document}


(a functional transformation, function $->$ function)

(a complex, more specifically, analytic function)

(a generating function, analog with $\int dt$ in place of $\sum$)

(approximation relations with dicrete generating functions, 
aka ``$z$-transform''. Digital signal processing implements or
is approximately related to analog signal processing.)

(a formal ratio of an extensor's Plucker coordinates)

(eigenvalue of eigenvector $e^{s t}$ under a differential operator eg 
$(a\frac{d}{dx}^2 + b\frac{d}{dx} + c)(e^{st}) = (as^2 +bs + c)e^{st}$

(formal expression for impedances $R$, $1/Cs$, $Ls$)

(linear differential operator $s(f) = \frac{d}{dx}f$)

(deletion/contraction, 
``capacitor shorts out on high $s$, inductor opens on low $s$'' 
\cite{intuitAna})

(characterization of ``responses'' defined among port variables)

(?? Indicator of a step respone)

(asymptotic expansions as $s\rightarrow\infty$
and $s\rightarrow 0$)

\noindent\textbf{More and more}

(asymptotic behavior of a generating function's coefficients, which are
the generated function's values, determined by analytic properties of the
generated function.  Relate EE, circuits and signals literature to 
statistical mechanics and combinatorics.  
\cite{statMechForGraphers,multivarHalfPlane,AnalyticCombinatoricsBook})

(random generation: \cite{BoltzmanSampRandCombGen})

(geometry of phase shifts)

(risetime related to basic topology, graph theory, topology of lumped
networks expressed by oriented matroids)

(FUTURE hopefully not too much: Complex oriented matroids of Anderson, et.
al. \cite{complexOM})




\bibliography{../../bib/MathOfElec}{}
\bibliographystyle{plain}

\end{document}
