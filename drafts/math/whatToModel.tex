\documentclass{article}
\usepackage{cite}
\usepackage{amsmath}
\usepackage{hyperref}
\title{What to Model ... }
\author{Seth Chaiken}
\begin{document}
\maketitle

\section{Of a circuit:}
\begin{itemize}
\item KVL and KCL defined spaces and dependencies of edge voltages and currents,
in a (1) graph-based network and more generally an orthogonal complemetary
decomposition of a linear space.
\item Equivalances to avoid circuits of voltage sources and capacitors
and cocircuits of current sources and inductors, in order to better (how?)
choose state variables (generalized coordinates?); existance of tree containing
all the voltage sources and capacitors whose cotree contains all the 
current sources and inductors.    (Starting point
of \cite{ChuaMcPhersonLagrange}; find early references ??)
Benefits of this concretely and for Lagrangian formulations.
Generalizes of course.
\item Basic Lagrange and Hamiltonian formulation: 
\cite{ChuaMcPhersonLagrange}.
\item Capacitor and inductor flux ($\phi = \int v(t) dt$) and
charge ($q = \int i(t) dt$).
General nonlinear laws: $L(i) = \phi$ and $C(v) = q$.
\item Transforms of linear circuit functions.
\item Resistance and other port quantities within the theory of 
\textbf{random walks}\cite{DoyleSnellRandom,ProbOnTreesNetworks}.
Given a graph on states (as with Markov chains)
the discrete Greens function is defined
in \cite{ChungYaoDiscreteGreensFn}
to be the inverse of $\Delta|_S$  where
\[
\Delta f(x)=\sum_y(f(x)-f(y)p_{xy}
\]
is the discrete Laplace equation
when $p_{xy}$ is the transition probability from $x$ to $y$, and
$S$ is a connected subset with non-empty boundary
$\{y\not\in S | p_{xy}\not= 0 \ \text{for some }\ x\in S\}$.
\cite{ChungYaoDiscreteGreensFn} treats Dirichlet eigenvalues
determining Greens functions,
heat kernels, expected hitting times
and Greens functions for paths, lattices and distance regular graphs.  

  Spanning tree counting was done by these methods in
  \cite{ChungYaoCovHeatKSpTrees}.

  The normalized Laplacian is the symmetric matrix of weights.



\item Electricity as statistics mechanincs of electron transfer.  Relate to 
  random walk models.

\item Resistive port behavior; OM characterizes of 
forest term sign.
\item Monotone non-linear well-posedness.
\item OM of solution states.
\item Predictions and constraints derived from OM properties:
\begin{enumerate}
\item How orthogonality with an OM vector predicts one sign from others.
\item How OM covector elimination predicts a zero value is possible.
\item (Possible application of ``fourientation'') 
Suppose voltage and current source edges are specified with a source 
direction for each.  A source directions is specified by an orientation, that
may one direction to specify a non-zero source, or (perhaps) none to specify a
zero source.  The (1) resistor values and, in the case of two or more sources,
(2) the source values determine a solution; the solution determines another 
orientation on all the edges, called the solution orientation.  Rules for the
solution orientation (SO):
\begin{enumerate}
\item SO is defined on all the edges.
\item SO on one edge may to in one direction for non-zero current and voltage 
in a resistor edge, or non-zero current or voltage in a voltage or current edge
respectively, or none to indicate those solution quantities are zero.
\item SO on a source edge may be the same or different from the given source 
direction.
\end{enumerate}
Questions:
\begin{enumerate}
\item Characterize the solution orientations; maybe there is Tutte-like function
whose evaluation is the number of them.
\item Can a graph be derived so that a fourientation determines the set of 
solution orientations?  
\item How do these questions differ when the resistor values vary and the 
sources values are constant, versus constant resistances and variable sources.
Are the problems of both varying any good?
\end{enumerate}
Geometry questions:  What about the space of solutions with given sign pattern?
\end{enumerate}
\item OM of transients.
\item Signs of transients that might not be OM modelled.
\item Laplace transform polynomials.
\begin{itemize}
\item Bode diagram approx.\\
\textbf{NEW THING: Tropical algebra approximates real algebra!}
\item Stability
\item Phase margin.
\item Discrete model for spanning tree count: Sandpile group order.
\end{itemize}

\item Imaginary frequency $s=i\omega$.
\item Anderson's complex OM (after learning it properly.)
\item Cones of possible phases.  Is that what Anderson did?
\item Orders or levels of resistance, etc. magnitudes; as
in symbolic simulation.
\item Laplacian eigenvalues and eigenvectors  (what are the good for in elec. circuit
study?) 

Spectral analysis of a graph laplacian matrix has more applications today than
spectral analysis of the adjacency matrix; see \cite{ChungSpectralGraphTheory}.

Let $L$ be the (non-normalized) Laplacian matrix.  Its eigenvalues have the 
following electrical interpretation:  
Consider the graph to be a network of unit resistors with an added ground 
node $0$ and a unit capacitor between each original node and the ground node.  
Let the function of time $v_k(t)$, $k\neq 0$ be the voltage of node $k$ 
relative to 
the ground node.  Then $d\bf{v}/dt = -L\bf{v}$, so the eigenvalue $\lambda_i$
corresponds to the solution $\bf{v}_i(t) = \bf{v}_i(0)e^{-\lambda_i t}$.

Discrete Greens function \cite{ChungYaoDiscreteGreensFn}.

\item Effect or information transfer directionality due to impedance
differences, ``half-resistors'',  \textbf{Bond graphs}.
\item Hamiltonian generalized position and momentum; that Frankel stuff.
\item Voltages and currents are orthogonal under a bilinear form that 
represents power.  What's behind this?
\end{itemize}


\section{Recent Inverse Problem Results}



Short survey notice \cite{LamAMSNotice} cites
work by Curtis, Ingerman and Morrow \cite{CurtisIngermanMorrowCircPlanarRes}
and de Verdi{\`e}re, Gitler and Vertigan \cite{VertdiereGitlerVertigan}.
Starting with solutions to the electrical network inverse problems for planar
networks whose response vertices are on their bounding circle, research moved
into such for other surfaces.

See recent preprint by Lam \cite{LamElectroidVar2014}.

Add more refs.



\section{Ports}

Alternative vocabulary: 
Distinguished elements, pointed mathematical objects, set pointed objects, 
restrictioned objects

In matroid theory literature, a ``port'' is a clutter of subsets; it is the 
discrete structure in a matroid associated with a single element extension.
We call a port an element $p$; $M$ on $E$ is the single element extension
of $M\setminus p$.  From \cite{MatroidPortsSteinerShellChari}, 
$\mathcal{P}= \{ C - e; C\in\mathcal{C}(\mathcal{M}), e\in C\}$.
Chari attributes this to Lehman and cites Seymour, Brylawski and Huseby.
Recently, it was seen to be the structure of secret sharing matroids.

\section{Electric Analogies}

\begin{enumerate}
\item inductance=mass, capacitance=spring stiffness, ...
\item Traffic flow: voltage=trip time, current=flow, resistance=route-time 
per unit flow, KVL=each driver adjusts his route to minimize his time.

Braess paradox: Removing an edge reduces travel time. (Thanks, Michael 
Sattinger.)
Simplest example [on Wikipedia] occurs with a \textit{voltage source}, an edge
with constant travel time.  Is related to Nash equilibrium.

\url{http://homepage.ruhr-uni-bochum.de/Dietrich.Braess/#paradox}
is a bibliography including 
C.M. Penchina and L.J. Penchina,
   ``The Braess paradox in mechanical, traffic, and other networks.''
\cite{BraessAJPhy}
   Amer. J. Phys. 71, 479 - 482 (2003)

\textbf{non-linear monotonic constituitive functions} might be more applicable
here, like when the flow saturates, the time becomes $\infty$ when the capacity
is reached (or exceeded?).  Maybe try to write a note on necessary 
conditions for a Braess paradox, which is in contrast to Raleigh's 
inequalities.
 
\item Spider web: voltage=position, current=force, resistance=spring constant.
\end{enumerate}

\section{Benefits of Formal Expression}

Vectors: (1) time series (2) transform expansion.  Good to formalize with
generating functions IE? some general Fourier expansion.  (How the same??)

Vectors: Multidimensional quantities: System state, multidim. signal, complex
value:  Good to formalize with the GROUP ALGEBRA or some of its generalization.

Connection:  Generating functions are the 
SEMI-GROUP algebra (over the coefficient space)
generated by $\{ z^n \}$.


\section{Engineering Intuitions}

\begin{enumerate}
\item
\textbf{A Study of paragraph-sized qualitative circuit operation descriptions}
\begin{itemize}
\item In terms of causal tracing of increases or decreases.
\item In terms of sinusoidal frequency response.
\item In terms of exponential frequency response.
\end{itemize}

\item 
Impedance differences between parts of circuits; impedance in intuition.

\item
\textbf{Approximation}

\item
\textbf{Purposeful Design}

\begin{itemize}
\item Need to account for 
\textbf{parasitic elements}
(1) Sometimes have small effects.
(2) Sometimes they must be compensated for.
\item Need for ``robustness'', researched with 
\textbf{Monte-Carlo methods} and
\textbf{Sensitivity analysis}. \cite{HiFreqChaosMakerEE}
\end{itemize}

\item
\textbf{Effects and Strategies}
\begin{itemize}
\item Effect of adding (or displacing) poles and zeros to a transfer function.
\item Compensation.
\item Feedback.
\item (Results of feedback): Miller effect, bootstrapping.
\end{itemize}

\item
\textbf{Physical causes expressed in an electrical model}.  Specifically, 
physics of devices (diode junctions, junction transtors, 
field-effect transistors, even non-ideal resistors, capacitor, inductor).

\item
\textbf{Active vs. Passive: Bias distinguished from signal}
``\textbf{Active device:} A device that can convert energy from a dc bias
source to a signal at an RF frequency.  Active devices are required for
oscillators and amplifiers.'' Microwave Devices in 
\textit{The Electrical Engineering 
Handbook}.\cite[ch.37, Streer and Trew]{EEHandbook}
(Find other 
definitions in the circuit theory literature?)

\item
\textbf{Thermal Runaway} Old: BJT power transistor latchup.  New: Hotspots 
in systems on a chip (SoC). There is a thermal network interacting with the 
electical network. Paper on it in VLSI \cite{ThermalVLSI}.

\item (Non-thermal?) Latchup (parasitic transistors) 
CMOS\cite{CMOSLatchUpTINote}

\end{enumerate}

\section{Topology}

Engineers use the term topology to  mean ``circuit shape'', 
the network graph decorated with element types, which determines
its instance of Kirchhoff law equations with a choice of constituitive laws.
Topologists and mathematical physicists (\cite{Frankel} especially Appendix
B, citing 
Eckmann\footnote{
From MR: ``Let numerical values $f(x_i)$ be assigned on the vertices
$x_i$ of a subgraph R of a linear graph K. It is required to find f in
$Q=K−R$ so that $f(x_i)$ for vertices $x_i$ in Q equals the mean of
its values on the vertices joined to $x_i$ by an edge. It is shown
that f always exists, and is unique if and only if R has at least one
vertex in each component of K. This is a generalization of a finite
version of the Dirichlet problem. A similar theorem is given in n
dimensions. The proof is based on the unique representation of a chain
with real coefficients as a sum of a harmonic chain (which is both a
cycle and a cocycle), a boundary and a coboundary.  Reviewed by
H. Whitney''}
(\cite{Eckmann}--perhaps \cite{EckmannWorks}, 
Bott\cite{BottInducedRep} and a book by Bamberg and 
Sternberg\cite{BambergSternbergBookII})
recognize this first as a case for homology and cohomology theory of a 
1-dimensional chain complex, and then as a source of analogies for more 
general spaces.  Going beyond the formulated (generally) differential 
and transform space equations, the dynamical system solutions have been
studied by topologists in their relation to the network topology
\cite{MathFoundElecSmale}.
   
Some authors pursued the topological view in terms of 
Lagrangian and Hamiltonian formulations\cite{ChuaMcPhersonLagrange}.

\subsection{Algebraic Topology}
The Laplacian generalizes to chain complexes, matrix tree theorem
generalizations.
(among others,\cite{Frankel,DuvalKlivansMartin}).

Weighted Laplacian (apps in persistent homology applications)
\cite{WuWeightedCoHLaplacian}.

\subsection{Back to Point Set Topology}

Each network graph vertex models a conducting region of physical space, 
ignoring electromagnetics...

Idea: When we consider the system (circuit?) to be a physical object, the 
vertices are \textbf{topological contractions} of the conducting regions.
\textbf{The homology group} of the conductor surface is trivial (since the
conductor itself is simply connected.)   The same level homology group for 
the whole system is the cycle space which is the subject for Kirchhoff's 
current law.

Magnetic phonomena make non-graphic topologies (recalled from one of 
Duffin's papers, where a cycle is assumed to have zero flux).

Recall conversation with Dan Silver and Susan Williams about topology, knots and Maxwell's equations.  Work on knot invariants from Laplacian
\cite{SilverWilliamsKnotLaplacian}.

(Refer to homotopy methods for analyzing non-differential non-linear 
equations.  Persistent homology is a new theme here.  This may bridge
statistical mechanical microscopic models to our macroscopic models.)

\section{OTHER}

The Boltzmann factor in interplay between combinatorics and statistical
mechanics (after reading Feynman's Lectures on Physics account).

The Boltzmann factor is 
$p_{\beta}(x)= C\alpha^{-x}$ $=$ $Ce^{-\beta x}$ is the probability 
or probability density of the exponential distribution.
Its origin in statistical mechanics:  $E_\beta = x + (E_\beta-x)$ where the 
number of equally probable states of an environment (heat bath) 
is $Ce^{E_\beta - x}$ and so the probability of one state of a 
(system+environment) is proportional to $e^{-\beta x}$.  It is the probability
of one system state with energy $x$ when the systema and environment are
in \textbf{thermal equilibrium}.  Then, $\beta$ is defined as 
\textbf{temperature}.

Caticha: This is a special case of maximizing \textbf{Shannon entropy}
subject to constraints on the expectations of one or more functions, such as 
$x(s)$.  Then temperature like quantities emerge as Lagrange multipliers.

Partition and generating function:
\[
Z = \sum_{s\in\text{states}}e^{-\beta x(s)} = \sum_{x}N(x)e^{-\beta x} 
\]
where $N(x)= $ number of states $s$ with $x(s) = x$.
\[
\frac{\partial{\log Z}}{\partial\beta}
=
\text{Expected Value}(x)
\]

(Consequences of $\text{max}\Omega(x)\Omega_{\text{bath}}(E_\beta -x )$
\cite{CatichaEIFP})

\textbf{Caticha:} The dynamical system on the microstates is called
the ``subject matter'', which is an assumption part of what we must know or 
assume
in order to do science.  Particular equations or their form as
Hamilton's equations are written down.
The dynamical system of evolving an evolving 
microstate maps to a dynamical system of evolving probability 
distributions.  Our knowledge is represented by a probability distribution
on microstates.  

Equal probabilities of all heat bath states with (heat bath) energy 
$E_\beta -x$ is 
equivalent to the system states' probability distribution has maximum
entropy among those with expected (system) energy value $x$.  But under the
second interpretation, temperature emerges as the value of the Lagrange 
multiplier for particular energy value $x$.  (Connection: The Max Ent
distribution over all finite discrete probability distributions is the uniform
distribution.)  



Multiple interacting transport phenomena in transistors and other 
engineered systems (natural too!).
 
\textbf{Objects categorized by computational effectiveness:}
\begin{enumerate}
\item Statistical Mechanical Ensemble, through which a world traces
merely a path of measure zero.
\item Countable set; RE set, recursive set.
\item A state of a world, existed physically, an approximation is 
impossible to store in computers.
\item A object that can possibly be stored using all the world's 
computers.
\item A object feasible to be a state in a computational simulation.
\item An object feasible to present to a human.
\end{enumerate}

 



\section{Laplace etc from other document}


(a functional transformation, function $->$ function)

(a complex, more specifically, analytic function)

(a generating function, analog with $\int dt$ in place of $\sum$)

(approximation relations with dicrete generating functions, 
aka ``$z$-transform''. Digital signal processing implements or
is approximately related to analog signal processing.)

(a formal ratio of an extensor's Plucker coordinates)

(eigenvalue of eigenvector $e^{s t}$ under a differential operator eg 
$(a\frac{d}{dx}^2 + b\frac{d}{dx} + c)(e^{st}) = (as^2 +bs + c)e^{st}$

(formal expression for impedances $R$, $1/Cs$, $Ls$)

(linear differential operator $s(f) = \frac{d}{dx}f$)

(deletion/contraction, 
``capacitor shorts out on high $s$, inductor opens on low $s$'' 
\cite{intuitAna})

(characterization of ``responses'' defined among port variables)

(?? Indicator of a step respone)

(asymptotic expansions as $s\rightarrow\infty$
and $s\rightarrow 0$)

\noindent\textbf{More and more}

(asymptotic behavior of a generating function's coefficients, which are
the generated function's values, determined by analytic properties of the
generated function.  Relate EE, circuits and signals literature to 
statistical mechanics and combinatorics.  
\cite{statMechForGraphers,multivarHalfPlane,AnalyticCombinatoricsBook})

(random generation: \cite{BoltzmanSampRandCombGen})

(geometry of phase shifts)

(risetime related to basic topology, graph theory, topology of lumped
networks expressed by oriented matroids)

(FUTURE hopefully not too much: Complex oriented matroids of Anderson, et.
al. \cite{complexOM})




\bibliography{../../bib/MathOfElec}{}
\bibliographystyle{plain}

\end{document}
