\documentclass{article}
\usepackage{cite}
\title{What to Model ... }
\author{Seth Chaiken}
\begin{document}
\maketitle

Of a circuit:
\begin{itemize}
\item Resistive port behavior; OM characterizes of 
forest term sign.
\item Monotone non-linear well-posedness.
\item OM of solution states.
\item OM of transients.
\item Signs of transients that might not be OM modelled.
\item Laplace transform polynomials.
\begin{itemize}
\item Bode diagram approx.
\item Stability
\item Phase margin.
\end{itemize}

\item Imaginary frequency $s=i\omega$.
\item Anderson's complex OM (after learning it properly.)
\item Cones of possible phases.  Is that what Anderson did?
\item Orders or levels of resistance, etc. magnitudes; as
in symbolic simulation.
\item Laplacian eigenvalues (what are the good for in elec. circuit
study?
\item Effect or information transfer directionality due to impedance
differences, ``half-resistors'',  \textbf{Bond graphs}.
\item Hamiltonian generalized position and momentum; that Frankel stuff.
\end{itemize}

\section{From other document..}


(a functional transformation, function $->$ function)

(a complex, more specifically, analytic function)

(a generating function, analog with $\int dt$ in place of $\sum$)

(approximation relations with dicrete generating functions, 
aka ``$z$-transform''. Digital signal processing implements or
is approximately related to analog signal processing.)

(a formal ratio of an extensor's Plucker coordinates)

(eigenvalue of eigenvector $e^{s t}$ under a differential operator eg 
$(a\frac{d}{dx}^2 + b\frac{d}{dx} + c)(e^{st}) = (as^2 +bs + c)e^{st}$

(formal expression for impedances $R$, $1/Cs$, $Ls$)

(linear differential operator $s(f) = \frac{d}{dx}f$)

(deletion/contraction, 
``capacitor shorts out on high $s$, inductor opens on low $s$'' 
\cite{intuitAna})

(characterization of ``responses'' defined among port variables)

(?? Indicator of a step respone)

(asymptotic expansions as $s\rightarrow\infty$
and $s\rightarrow 0$)

\noindent\textbf{More and more}

(asymptotic behavior of a generating function's coefficients, which are
the generated function's values, determined by analytic properties of the
generated function.  Relate EE, circuits and signals literature to 
statistical mechanics and combinatorics.  
\cite{statMechForGraphers,multivarHalfPlane,AnalyticCombinatoricsBook})

(random generation: \cite{BoltzmanSampRandCombGen})

(geometry of phase shifts)

(risetime related to basic topology, graph theory, topology of lumped
networks expressed by oriented matroids)

(FUTURE hopefully not too much: Complex oriented matroids of Anderson, et.
al. \cite{complexOM})




\bibliography{../../bib/MathOfElec}{}
\bibliographystyle{plain}

\end{document}
