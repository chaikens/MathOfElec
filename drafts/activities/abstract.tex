\documentclass{article}
\usepackage{amsmath}
\usepackage{cite}
\title{Tutte Activities Based on More General Element Orders (draft)}
\author{Seth Chaiken\\
Gary Gordon\\
Elizabeth McMahon}
\begin{document}
\maketitle
In 1997 Gordon and McMahon\cite{GordonMcMachonGreedoid} 
extended from matroids to greedoids
the internal and external 
element activities and the resulting subset-interval 
partition analysis for the Tutte polynomial.  They found it necessary
to abandon the traditional definition of activities based on a 
single (but arbitrary) linear element order.  Instead, the internal and 
external, active versus inactive element classification was based on 
properties of each individual root-to-leaf path down any computation tree
for the Tutte polynomial.  Independence of element order was generalized to
independence of the computation tree.

We report that these results (for matroids) naturally generalize when both (1) 
the deletion/contraction and loop/coloop removal operations, for each element,
are \textbf{parametrized} or 
\textbf{colored}, and (2) a subset of \textbf{port} elements 
$P$ is given and the resulting \textbf{restricted} or \textbf{ported}
Tutte functions satisfy $T(M) = g_e T(M/e) + r_e T(M\backslash e)$ 
\textbf{only for} $e\not\in P$.

After they are extended to involve the initial values $T(M')$ for 
the \textbf{$P$-minors} $M'$, the Bollobas/Riordan/Zaslavsky 
conditions on the parameters are quite easily 
shown to be necessary and sufficient for the Tutte polynomial to be
well-defined.

(note in this abstract draft: I guess it will be easy to extend everything
to ported, parametrized greedoid Tutte functions, but I have not 
begun to work on that.)


\bibliographystyle{plain}
\bibliography{../../bib/MathOfElec}
\end{document}

