\documentclass{article}
\usepackage{amsmath}
\usepackage{cite}
\title{How Combinatorics Improves Tree Enumeration}
\begin{document}
\maketitle

\section{Graph Coloring Counting}

Definition

The chromatic function is a polynomial proved by deletion/contraction
induction.

\section{Tutte Polynomial}

\[
T(G,x,y) = \left\{ \begin{array}{ll}
  x^{\text{\# isthmuses}}y^{\text{\# loops}} & \text{if }G\text{ contains loops or isthmuses only,}\\
  T(G/e,x,y)+T(G\setminus e,x,y) & \text{if }e\in E(G)
     \text{ is neither a loop nor an isthmus.}
  \end{array}
  \right\}
\]

Chromatic specialization:

\[
\chi(G,n) = (-1)^{|V(G)|-k(G)}n^{k(G)}T(G,(1-n),0)
\]

Tutte's activities expansion:
\[
T(G,x,y) = \sum_{\text{Bases A}}x^{|IA(A)|}y^{|EA(A)|}
\]


\section{Inclusion/Exclusion and Whitney's Solution}

\[
\chi(G,n) = \sum_{k=0}^{k=|E|}(-1)^k \text{\# }n\text{ colorings violating at least }
k\text{ edges}
\]

\[
\chi(G,n) = \sum_{A\subseteq E}(-1)^{|A|} 
(\text{\# }n\text{ colorings violating at least the edges }A)
\]

\[
\chi(G,n) = \sum_{A\subseteq E}(-1)^{|A|}
n^{|V(G/A)|} = n^{|V(G)|} + (\text{polynomial in } n \text{ of degree } < |V(G)|)
\]
which establishes that $\chi$ is a polynomial function.  

Whitney\cite{WhitLogExpMath} observed that 
many pairs of terms, those for which $A$ contains at least 
one \textbf{broken circuit}, may be cancelled.

If $A$ contains circuit $C$ and $e\in C$, then $V(G/A) = V(G/(A\setminus e))$

So Whitney's idea was to impose a linear order on a subset $\mathcal{B}$
of the $A$'s occurring 
in the sum so that a pairing $\{A,A'\}$ of the members of $\mathcal{B}$ 
is defined for which $\{A,A'\}=\{A'\cup e, A'\}$, $e\not\in A'$ for some $e$
and $k(G/A) = k(G/(A\setminus e))$.  Two benefits result:
\begin{enumerate}
\item Many of the terms may be omitted.
\item The subsets $A$ for the remaining terms all have a simple form with the 
result $|V(G/A)| = |V(G)| - |A| $.

\section{Posets, Inclusion/Exclusion and the Mobius Function}

The M\"{o}bius function emerges when the principle of inclusion/exclusion is
applied to enumerations on elements of general posets.

(I'll try a tutorial that directly relates the M\"{o}bius defining recurrance to
solving $g(x) = \sum_{z\le x}f(z)$ for $f()$.)

(The resulting mathematical abstraction is Stanley's 
\textbf{Incidence Algebra} of a poset\cite{StanleyEC1}.)

Stanley: ``Hence M\"{o}bius inversion results in a simplification
of Inclusion-Exclusion under appropriate circumstances. However, we shall also see that
the applications of M\"{o}bius inversion are much 
further-reaching than as a generalization of
Inclusion-Exclusion.'' \cite[ch.~3]{StanleyEC1}.


\section{Broken Circuits in the Tutte Computation Tree}

Every GM computation tree partitions $2^{E(G)}$ into 
boolean intervals\cite{GordonMcMachonGreedoid}.  The boolean
intervals correspond to the leaves of the tree.  Each leaf is
characterized by subsets of externally active elements
EA and internally active elements IA.
(not explained yet here at all).
Each interval has cardinality $2^{|\text{IA}|+|\text{EA}|}$.
The broken-circuit-free subsets are those in the intervals for which
$\text{EA}=\emptyset$.  Thus the subsets with broken circuits are partitioned
among the intervals with $\text{EA}\neq\emptyset$.  

Consider an interval with $\text{EA}\neq\emptyset$. It has the
form $[A, A\cup\text{IA}\cup\text{EA}]$.  Let's further partition
it into the $2^{|\text{IA}|}$ subintervals $[A\cup I, A\cup I\cup \text{EA}]$, 
$I\subseteq\text{IA}$.    
%We now observe, similarly to what Whitney observed, that
We now observe, like Whitney observed, that
$v=V(G/S)$ is the same for all $S\in[A\cup I, A\cup I\cup \text{EA}]$.  The
contribution of these $S$ to $\chi(G,n)$ is $n^v\sum_S(-1)^{|S|}$ $=$
$n^v(-1)^{|A\cup I|}(1-1)^{|EA|}$.

\section{The Broken Circuit Complex}

\section{Activities}

(Goal: Get at the facts about activities being h-vectors of broken-circuit complexes or 
something.)

(Bigger Goal: Extend that to GM activities, based on a computation tree.  Try to find 
the more generalized broken circuit complex, if possible!)

\section{Applications}

Why are general computation tree orders better?  Suppose an electrical network has resistors
A, B, and C.  Properties of $N/A$ may depend more on $B$ than $C$, while properties of
$N\setminus A$ may depend more on $C$ than $B$. EG: $N=(A \text{ ser } B) \text{ par } C$.
$N/A = B \text{ par } C$, but $N\setminus A = C$.  If $g(B) >> g(C)$, 
$g(N/A) = g(B) + g(C) \approx g(B)$
but $g(N\setminus A) = g(C)$ 

\[
R(N) = \frac{g(A)+g(B)}{g(A)g(B)+g(A)g(C)+g(B)g(C)}
\]

\[
R(N/A) = \lim_{g(A)\rightarrow\infty}R(N) = \frac{1}{g(B)+g(C)}
\]

\[
R(N\setminus A) = \lim_{g(A)\rightarrow 0}R(N) = \frac{g(B)}{g(B)g(C)} = \frac{1}{g(C)}
\]







\section{Literature}

New! Doman and Trinks \cite{DohmenTrinksAbsWitBrok}

Maybe an application clue from Urschel (who's also into football) et. al.
because the abstract mentions a ``heavy edge coarsening scheme''
\cite{UrschelFiedlerGraphLaplac}.

\end{enumerate}
\bibliographystyle{plain}
\bibliography{../../bib/MathOfElec}
\end{document}

