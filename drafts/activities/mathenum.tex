\documentclass{article}
\usepackage{amsmath}
\usepackage{cite}
\title{How Combinatorics Improves Tree Enumeration}
\begin{document}
\maketitle

\section{Graph Coloring Counting}

Definition

The chromatic function is a polynomial proved by deletion/contraction
induction.

\section{Inclusion/Exclusion and Whitney's Solution}

\[
\chi(G,n) = \sum_{k=0}^{k=|E|}(-1)^k \text{\# }n\text{ colorings violating at least }
k\text{ edges}
\]

\[
\chi(G,n) = \sum_{A\subseteq E}(-1)^k 
(\text{\# }n\text{ colorings violating at least the edges }A)
\]

\[
\chi(G,n) = \sum_{A\subseteq E}(-1)^k 
n^{|V(G/A)|} = n^{|V(G)|} + (\text{polynomial in } n \text{ of degree } < |V(G)|)
\]
which establishes that $\chi$ is a polynomial function.  

Whitney\cite{WhitLogExpMath} observed that 
many pairs of terms, those for which $A$ contains at least 
one \textbf{broken circuit}, may be cancelled.

If $A$ contains circuit $C$ and $e\in C$, then $k(G/A) = k(G/(A\setminus e))$

So Whitney's idea was to impose a linear order on a subset $\mathcal{B}$
of the $A$'s occurring 
in the sum so that a pairing $\{A,A'\}$ of the members of $\mathcal{B}$ 
is defined for which $\{A,A'\}=\{A'\cup e, A'\}$, $e\not\in A'$ for some $e$
and $k(G/A) = k(G/(A\setminus e))$.  Two benefits result:
\begin{enumerate}
\item Many of the terms may be omitted.
\item The subsets $A$ for the remaining terms all have a simple form with the 
result $k(G/A) = k(G) - |A| $.

\section{Posets, Inclusion/Exclusion and the Mobius Function}

The Mobius function emerges when the principle of inclusion/exclusion is
applied to enumerations on elements of general posets.

(I'll try a tutorial that directly relates the M\"{o}bius defining recurrance to
solving $g(x) = \sum_{z\le x}f(z)$ for $f()$.)

(The resulting mathematical abstraction is Stanley's 
\textbf{Incidence Algebra} of a poset\cite{StanleyEC1}.)

Stanley: ``Hence M\"{o}bius inversion results in a simplification
of Inclusion-Exclusion under appropriate circumstances. However, we shall also see that
the applications of M\"{o}bius inversion are much 
further-reaching than as a generalization of
Inclusion-Exclusion.'' \cite[ch.~3]{StanleyEC1}.

\section{The Broken Circuit Complex}

\section{Activities}

(Goal: Get at the facts about activities being h-vectors of broken-circuit complexes or 
something.)

(Bigger Goal: Extend that to GM activities, based on a computation tree.  Try to find 
the more generalized broken circuit complex, if possible!)



\section{Literature}

New! Doman and Trinks \cite{DohmenTrinksAbsWitBrok}

\end{enumerate}
\bibliographystyle{plain}
\bibliography{../../bib/MathOfElec}
\end{document}

