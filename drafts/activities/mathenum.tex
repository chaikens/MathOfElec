\documentclass{article}
\usepackage{amsmath}
\usepackage{cite}
\title{Notes on How Combinatorics Improves Tree Enumeration}
\begin{document}
\maketitle

\section{Graph Coloring Counting}

\noindent Definition

\noindent The chromatic function is a polynomial proved by deletion/contraction
induction.

\noindent Combinatorics gives us counting interpretations of the \textit{coefficients}
of the chromatic polynomial.  So the chromatic polynomial is a case of 
a \textbf{generating function} whose \textit{values} $\chi(G,n)$ are interesting.  


\section{Tutte Polynomial--Delection/Contraction}

\[
T(G,x,y) = \left\{ \begin{array}{ll}
  x^{\text{\# isthmuses}}y^{\text{\# loops}} & \text{if }G\text{ contains loops or isthmuses only,}\\
  T(G/e,x,y)+T(G\setminus e,x,y) & \text{if }e\in E(G)
     \text{ is neither a loop nor an isthmus.}
  \end{array}
  \right\}
\]

Chromatic specialization:

\[
\chi(G,n) = (-1)^{|V(G)|-k(G)}n^{k(G)}T(G,(1-n),0)
\]

Tutte's activities expansion:
\[
T(G,x,y) = \sum_{\text{Bases A}}x^{|IA(A)|}y^{|EA(A)|}
\]


\section{Inclusion/Exclusion and Whitney's Solution}

\[
\chi(G,n) = \sum_{k=0}^{k=|E|}(-1)^k \text{\# }n\text{ colorings violating at least }
k\text{ edges}
\]

\[
\chi(G,n) = \sum_{A\subseteq E}(-1)^{|A|} 
(\text{\# }n\text{ colorings violating at least the edges }A)
\]

\[
\chi(G,n) = \sum_{A\subseteq E}(-1)^{|A|}
n^{|V(G/A)|} = n^{|V(G)|} + (\text{polynomial in } n \text{ of degree } < |V(G)|)
\]
which establishes that $\chi$ is a polynomial function.  

Whitney\cite{WhitLogExpMath} observed that 
many pairs of terms, those for which $A$ contains at least 
one \textbf{broken circuit}, may be cancelled.

If $A$ contains circuit $C$ and $e\in C$, then $V(G/A) = V(G/(A\setminus e))$

So Whitney's idea was to impose a linear order on a subset $\mathcal{B}$
of the $A$'s occurring 
in the sum so that a pairing $\{A,A'\}$ of the members of $\mathcal{B}$ 
is defined for which $\{A,A'\}=\{A'\cup e, A'\}$, $e\not\in A'$ for some $e$
and $k(G/A) = k(G/(A\setminus e))$.  Two benefits result:
\begin{enumerate}
\item Many of the terms may be omitted.
\item The subsets $A$ for the remaining terms all have a simple form with the 
result $|V(G/A)| = |V(G)| - |A| $.

\section{Posets, Inclusion/Exclusion and the Mobius Function}

The M\"{o}bius function emerges when the principle of inclusion/exclusion is
applied to enumerations on elements of general posets.

(I'll try a tutorial that directly relates the M\"{o}bius defining recurrance to
solving $g(x) = \sum_{z\le x}f(z)$ for $f()$.)

(The resulting mathematical abstraction is Stanley's 
\textbf{Incidence Algebra} of a poset\cite{StanleyEC1}.)

Stanley: ``Hence M\"{o}bius inversion results in a simplification
of Inclusion-Exclusion under appropriate circumstances. However, we shall also see that
the applications of M\"{o}bius inversion are much 
further-reaching than as a generalization of
Inclusion-Exclusion.'' \cite[ch.~3]{StanleyEC1}.


\subsection{Characteristic Polynomial}

\[
\chi(G,n) = \sum_{\textbf{0}\le z\le \textbf{1}}\mu(\textbf{0}, z)n^{|V(G)|-\text{rank}(z)}
\]
where the sum is over the geometric lattice of flats (closed subsets of edges)
in the graphic matroid of $G$.  

Proof.  Each flat $z$ corresponds to a graph $G/A$ obtained by 
contracting some
subset $A$ of edges and then removing the loops. 
(The edges in the flat are $A$ plus those removed loops.)
The number of unrestricted colorings of $G/A$ is $n^{V(G/A)-\text{rank}(z)}$.
The number of legal colorings of each $G/A'$ is of course $\chi(G/A',n)$.
Each unrestricted coloring of $G/A$ corresponds to a unique legal coloring of
the graph obtained from $G/A$ by contracting the violated edges of $G/A$.  
Therefore 
\[
n^{V(G/A)-\text{rank}(z)} = \sum_{z'\ge z}\chi(G/A',n).
\]
The formula for $\chi(G,n)$ is the result of M\"{o}bius inversion because
$G$ (with no edges contracted) corresponds to the bottom flat $\textbf{0}$.

Factoring out $n^{k(G)}$ gives us a polynomial function that generalizes 
to all finite lattices.
$\chi(G,n)=n^{k(G)}\text{char}(\mathcal{L},n)$ where

\[
\text{char}(\mathcal{L},n) = 
\sum_{z\in\mathcal{L}}\mu(z,\mathbf{1})n^{\text{rank}(\mathbf{1}) - \text{rank}(z)}.
\]

\section{Broken Circuits in the Tutte Computation Tree}

Every GM computation tree partitions $2^{E(G)}$ into 
boolean intervals\cite{GordonMcMachonGreedoid}.  The boolean
intervals correspond to the leaves of the tree.  Each leaf is
characterized by subsets of externally active elements
EA and internally active elements IA.
(not explained yet here at all).
Each interval has cardinality $2^{|\text{IA}|+|\text{EA}|}$.
The broken-circuit-free subsets are those in the intervals for which
$\text{EA}=\emptyset$.  Thus the subsets with broken circuits are partitioned
among the intervals with $\text{EA}\neq\emptyset$.  

Consider an interval with $\text{EA}\neq\emptyset$. It has the
form $[A, A\cup\text{IA}\cup\text{EA}]$.  Let's further partition
it into the $2^{|\text{IA}|}$ subintervals $[A\cup I, A\cup I\cup \text{EA}]$, 
$I\subseteq\text{IA}$.    
%We now observe, similarly to what Whitney observed, that
We now observe, like Whitney observed, that
$v=V(G/S)$ is the same for all $S\in[A\cup I, A\cup I\cup \text{EA}]$.  The
contribution of these $S$ to $\chi(G,n)$ is $n^v\sum_S(-1)^{|S|}$ $=$
$n^v(-1)^{|A\cup I|}(1-1)^{|EA|}$.

The bases also correspond to the leaves of the GM computation tree.
The interval $[A,A\cup \text{IA}\cup\text{EA}]$ corresponds to basis
$A\cup\text{IA}$.  $A$ is the set of elements contracted on the path from the 
root to the leaf.  Of course the cobasis is $\overline{A}\cup\text{EA}$ where
$\overline{A}$ is the set of elements deleted on the path from the root to the 
leaf.



\section{The Broken Circuit Complex}

\cite{BrokenCctComplexBryl}

The ``broken-circuit complex'' is the simplicial complex of sets that do not contain
any broken circuits.  The $f$-vector codes the number of faces of each dimension, ie., 
cardinality $-1$.
The coefficients of the chromatic polynomial (characteristic poly in general)
are the $f$-vector of the broken-circuit complex.

\section{Activities}

(Goal: Get at the facts about activities being h-vectors of broken-circuit complexes or 
something.)


\[
\sum_{i=0}^d f_i(x-1)^{d-i}
=
\sum_{i=0}^d h_i(x)^{d-i}
\]

??What expansion of $\chi$ in $n$ = expansion of $\chi$ in $n-1$?

 

Tutte's expansion:
\[
T(G,x,y) = \sum_{\text{Bases} A\subseteq E(G)}x^{|IA(A)|}y^{|EA(A)|}
\]

Set expansion:
\[
T(G,x,y) = \sum_{A\subseteq E(G)}(x-1)^{k(A)-k(E)}(y-1)^{k(A) + |A| - |V|}
\]

\[
\chi(G,n) = (-1)^{|V|-k(G)}n^{k(G)}T(G,1-n,0)
\]

\[
\chi(G,n) = \sum_{A \text{ contains no broken circuits}}
            (-1)^{|A|}n^{|V|-|A|}
\]


\[
\chi(G,n) = (-1)^{|V|-k(G)}n^{k(G)}\sum_{A\subseteq E(G)}(-n)^{k(A)-k(E)}(-1)^{k(A) + |A| - |V|}
\]

\[
\chi(G,n) = (-1)^{|V|}(-1)^{k(G)}n^{k(G)}\sum_{A\subseteq E(G)}(-n)^{k(A)}(-n)^{-k(E)}(-1)^{k(A)}(-1)^{|A|}(-1)^{|V|}
\]

We get from ... Whitney's starting point for inclusion/exclusion:
\[
\chi(G,n) = \sum_{A\subseteq E(G)}n^{k(A)}(-1)^{|A|}
\]



(Bigger Goal: Extend that to GM activities, based on a computation tree.  Try to find 
the more generalized broken circuit complex, if possible!)

\section{Applications}

Why are general computation tree orders better?  Suppose an electrical network has resistors
A, B, and C.  Properties of $N/A$ may depend more on $B$ than $C$, while properties of
$N\setminus A$ may depend more on $C$ than $B$. EG: $N=(A \text{ ser } B) \text{ par } C$.
$N/A = B \text{ par } C$, but $N\setminus A = C$.  If $g(B) >> g(C)$, 
$g(N/A) = g(B) + g(C) \approx g(B)$
but $g(N\setminus A) = g(C)$ 

\[
R(N) = \frac{g(A)+g(B)}{g(A)g(B)+g(A)g(C)+g(B)g(C)}
\]

\[
R(N/A) = \lim_{g(A)\rightarrow\infty}R(N) = \frac{1}{g(B)+g(C)}
\]

\[
R(N\setminus A) = \lim_{g(A)\rightarrow 0}R(N) = \frac{g(B)}{g(B)g(C)} = \frac{1}{g(C)}
\]


\section{Topology, $f$\&$h$ vectors and all that}

This is addressed by Ellis-Monaghan and Merino
\cite{JEM-MerinoGraphPolyAppI}. 







FROM 
``Monomial Bases for Broken Circuit Complexes''
by Jason Brown and  Bruce Sagan
\cite{MonBasesBrkCircCompBrownSagan}
\begin{quote}
Let E be a finite set and let $\Delta$
be an abstract simplicial complex on E, ...

Let $f_i = f_i(\Delta)$ be the number of $S \in \Delta$ with
$|S| = i$. Then $\Delta$ has f-vector
$f = f(\Delta) = (f_0, f_1, . . . , f_r)$
as well as f-polynomial
$f(x) = f_{\Delta}(x) = f_0 + f_1x + · · · + f_rx^r$
where $x$ is a variable. 

...

Another important invariant of $\Delta$ is its $h$-vector. Define a polynomial
\[
h(x) := (1 - x)^r f(\frac{x}{1-x})
\]
\[
= f_0(1 - x)^r + f_1x(1 - x)^{r-1} + f_2x^2(1-x)^{r-2} + ... + f_rx^r. 
\]

Let $h_i$ be the coefficient of $x_i$ in h(x) so that 

...
\end{quote}

Jason Brown and  Bruce Sagan go on to give an combinatorial interpretation
of $h_i$ when $f(-\lambda)=\pm$chromatic polynomial's Whitney's expansion.
It uses that Cohen-Macauly stuff.

Bruce and R. Bloka did a line of research following Las Vergnas' 
\cite{ActOrdersMatrBasesLasVergnas} on 
topology of complexes defined in terms of linearly-defined activities
\cite{ActTopPropBlokaSagan}.  In his paper, 
Las Vergnas did 3 different complexes? 
and a lattice ordering of matroid bases, 
and wondered why M\"{o}bius function values in the lattice were zero.







\section{Literature}

New! Doman and Trinks \cite{DohmenTrinksAbsWitBrok}, paper citing
Doman's proof \cite{DohIndProofMatroids}.

Maybe an application clue from Urschel (who's also into football) et. al.
because the abstract mentions a ``heavy edge coarsening scheme''
\cite{UrschelFiedlerGraphLaplac}.

\end{enumerate}
\bibliographystyle{plain}
\bibliography{../../bib/MathOfElec}
\end{document}

