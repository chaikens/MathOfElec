\documentclass{article}
\usepackage{amsmath}
\usepackage{cite}
\setlength{\textwidth}{7.0in}
\setlength{\oddsidemargin}{0.0in}
\setlength{\evensidemargin}{0.0in}

\title{Tutte Activities Based on More General Element Orders}
\author{Seth Chaiken\\
Gary Gordon\\
Elizabeth McMahon}
\begin{document}
\maketitle
In 1997 Gordon and McMahon\cite{GordonMcMachonGreedoid} 
extended from matroids to greedoids
the internal and external 
element activities and the resulting subset-interval 
partition analysis for the Tutte polynomial.  They found it necessary
to abandon the traditional definition of activities based on a 
single (but arbitrary) linear element order.  Instead, the internal and 
external, active versus inactive element classification was based on 
properties of each individual root-to-leaf path down any computation tree
for the Tutte polynomial.  Independence of element order was generalized to
independence of the computation tree.

We report that these results (for matroids) naturally generalize when both (1) 
the deletion/contraction and loop/coloop removal operations, for each element,
are \textbf{parametrized} or 
\textbf{colored}, and (2) a subset of \textbf{port} elements 
$P$ is given and the resulting \textbf{restricted} or \textbf{ported}
Tutte functions satisfy $T(M) = g_e T(M/e) + r_e T(M\backslash e)$ 
\textbf{only for} $e\not\in P$.

After they are extended to involve the initial values $T(M')$ for 
the \textbf{$P$-minors} $M'$, the Bollobas/Riordan/Zaslavsky 
conditions on the parameters are quite easily 
shown to be necessary and sufficient for the Tutte polynomial to be
well-defined.

\newpage

\LARGE

\section{Brief History of Tutte Functions}
\begin{enumerate}
\item Well-defined invariants like the chromatic (poly.) function 
and the count of spanning trees were shown to satisfy
\[
T(G) = T(G/e) \pm T(G\setminus e).
\]
\item
Alternative expansions for some Tutte functions, like Whitney's
\[
\chi(G,\lambda) = \sum_{A\subseteq E(G)}(-1)^{|A|}\lambda^{k(G|A)}\text{\ \ \ }2^{|E|}
\text{ terms}
\]
$\sum$ could be SHORTENED to ``broken-circut-free'' subsets $A$, based on
an \textit{arbitrary linear ordering of }$E(G)$
\item
Tutte and Brylawski's universal 2-variable solution to 
\[
T(G) = T(G/e) + T(G\setminus e).
\]
expressed by
\[
T(G) = \sum_{\text{Bases}B\subseteq E}x^{|IA(B)|}y^{|EA(B)|}
\]
\textit{ internally ($IA$) and externally ($EA$)
active elements based on a arbitrary linear ordering of $E$}
\end{enumerate}

Tutte's and others proved
\textbf{every computation tree} computes every Tutte function
value \textbf{correctly BECAUSE} the Tutte eq. have a unique solution.

But \textit{independence of the linear ordering defining the activities} 
HAD TO BE PROVED FIRST.

\newpage

\section{GM's Computation Tree Approach}

Sets $IA$ and $EA$ are assoc. to \textbf{leaves} of an 
\textbf{arbitrary computation tree} $\mathcal{T}_j$

\noindent
$IA(\text{leaf}) = $ isthmuses at leaf's graph or matroid.

\noindent
$EA(\text{leaf}) = $ loops at leaf's graph or matroid.

\noindent Routine induction (``Let $\mathcal{T}_1$, 
$\mathcal{T}_2$ be two trees of a smallest 
counterexample. ... Contradiction!\footnote{Technical lemmas 
about series and parallel elements are applied.}'') proves:

\[
\sum_{\text{leaf of }\mathcal{T}_1}
x^{IA_1(\text{leaf})}y^{EA_1(\text{leaf})}
=
\sum_{\text{leaf of }\mathcal{T}_2}
x^{IA_2(\text{leaf})}y^{EA_2(\text{leaf})}
\]



\newpage
\section{What generalizes to GM computation tree approach}

\noindent \textbf{Greedoids (GM 1997)}

\noindent Sometimes the \textbf{only computation trees
that exist} are NOT BASED ON A LINEAR element ORDER!

\vfill


\noindent 2 Interval partitions of the boolean lattice
$2^{E}$ defined using \textbf{one part for each computation tree leaf}.

\vfill


\noindent \textbf{Zaslavsky-Riordan-Bollobas, Traldi}
on weight (or color) pairs ($x_e,y_e$) on labelled edges
and initial values ($X_e,Y_e$) on loops and insthuses -- necessary and 
sufficient for the weighted or colored Tutte equations to have a 
solution.

\vfill


\noindent \textbf{$P$-ported or restricted or set-pointed} Tutte functions
$T$ with weights or colors.  
(Diao-Hetyei, computation tree proofs by sdc)  
TWO more ZBR-type conditions re. \textbf{initial values of $T$ on 
the $P$-quotients} are needed besides ZBR's original three.

\vfill


\noindent \textbf{ORIENTED matroids}
 INCLUDING DISTINCT $P$-quotients.(sdc)

\newpage
\section{What DOES NOT Go Through}

Given a linear element order, $B\subset E$ is a broken-circuit if there
exists $e\not\in E$ so $B\cup e$ is a circuit and $e$ is the (FOR US!) 
GREATEST element in $B\cup e$.

\textbf{Given a computation tree} (unless it's from a linear order) 
``$B\subseteq E$ is a broken-circuit'' \textbf{is not well-defined}.  



\vfill

\section{What does go through}


What's a broken circuit $B$, COMPUTATIONALLY?  ANSWER:  There is a 
root-to-leaf path down which which each element of $B$ is contracted, and
a \textbf{LOOP IS PRODUCED} when contracting the last element of $B$.



When the tree is from a linear order, 
``$A\subseteq E$ is broken-circuit free'' \textbf{if and only if}
``the leaf of $\mathcal{T}$ gotten to by searching for $A$ has no loops
(ie., $EA(\text{leaf})=\emptyset$).

AND Whitney's expansion generalizes to those sets $A$ leading to loop-free
leaves in an arbitrary computation tree.











%}%Large
\newpage
\bibliographystyle{plain}
\bibliography{../../bib/MathOfElec}
\end{document}

