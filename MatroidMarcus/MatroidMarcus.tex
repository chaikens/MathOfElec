\documentclass{amsart}

\title{Grassmann or Exterior Algebra Matroid Representation}
%\title{Representing Matroids with Grassmann or Exterior Algebras}

\author{Seth Chaiken}


\newtheorem{MRes}{From Marcus}


\begin{document}

\newcommand{\matroidconcept}[1]{\emph{#1}}
\newcommand{\Plucker}{Pl\"{u}cker\ }


\maketitle


Report on matroid interpretations of exterior or Grassmann
algebra developed in Marcus\cite{MarcusFDMuAlPt2}.

$V$ is a finite dimensional vector space over field $R$.

$E$ is a finite set of labelled vectors in $V$, specifically there
is an map $E\rightarrow V$.   $E$ will constitute a matroid
\matroidconcept{ground set}.

\section{Decomposability}

\begin{MRes}[Theorem 1.1, p 5]
  Let $0\neq z \in \wedge^m V$.  Then $z$ is decomposible iff there
  exists a l.i. set of vectors $u_1, \cdots, u_m$ in $V$ such that
  $u_i \wedge z = 0$, $i = 1, \cdots, m$.
\end{MRes}

Decomposibles $z$ of the form $a e_1 \wedge e_2 \wedge \cdots e_r$ where
$e_i\in E$ correspond to \matroidconcept{flats}.
Each $K$ multiple equivalent expression for $z$ corresponds to a
\matroidconcept{ordered basis for a flat}.

\begin{MRes}[Theorem 1.2, p 6]
  The element
  \[
  z = \sum_{\omega\in Q_{m,n}}a_\omega e^{\wedge}_\omega
  \]
  is decomposible iff there exists an $A\in M_{m,n}(R)$ such that
  \[
  a_\omega = det A[1,\dots,m|\omega], \omega\in Q_{m,n}.
  \]
\end{MRes}

\matroidconcept{Flats} are represented by matrices.

\begin{MRes}[Theorem 1.3, p 7]
  If $dim V = n$ then any element $\wedge^{n-1}V$ is decomposible.
  In other words, the canonical multilinear map
  $\wedge:\times_1^{n-1} V \rightarrow \wedge^{n-1} V$ is surjective.
\end{MRes}

(Long pf using determinants.  Don't see yet what to make of it.)

Marcus then defines the \Plucker relations and proves (Theorem 1.4, p 11)
that $z$'s components satify the \Plucker relations iff $z$ is decomposible.
(Goal--relate to \matroidconcept{basis exchange}.)

Marcus Theorem 1.5, p 14 is the Cauchy-Binet theorem??.

\begin{MRes}[Grassmann representitive, p 18]
  If $W$ is any non-zero $m$-dimensional subspace of $V$, then any
  nonzero decomposible element $x_1 \wedge \cdots \wedge x_m$, $x_i\in W$, is
  called a Grassmann representitive for $W$.
\end{MRes}

This is needed to set up meaningful Tutte identities in exterior algebra,
because they
apply to Grassmann representitives, not to Grassmann points or \Plucker
coordinates.


\subsection{Hodge Star and Matroid Duality}

\section{Marcus' Duality}


\section{Marcus' Transformation}

\bibliographystyle{amsplain}
\bibliography{../bib/MathOfElec}


\end{document}
