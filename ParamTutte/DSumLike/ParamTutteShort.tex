\documentclass[12pt,leqno]{amsart}




\usepackage{amsmath,amssymb,amsfonts,amsthm}
\usepackage{eucal,graphicx}
\usepackage{color}
\usepackage[pdftex]{hyperref}


\setlength{\textwidth}{6.5in}
\setlength{\oddsidemargin}{0.0in}
\setlength{\evensidemargin}{0.0in}
\setlength{\textheight}{9in}
\setlength{\topmargin}{-.4in}

%\renewcommand{\baselinestretch}{2}     % Activate for double spacing.
%\renewcommand{\baselinestretch}{1.6}   % Activate for 1-1/2 spacing.
%\renewcommand{\baselinestretch}{1.3}   % Activate for 1-1/3 spacing.


\newcommand \comment[1]{}			%  Silent version.
%\renewcommand \comment[1]{\emph{#1}}		%  Comment revealed.
\newcommand \dateadded[1]{\comment{[Date added: #1.]}}
\newcommand \mylabel[1]{\label{#1}\comment{{\rm \{#1\} }}}
\newcommand \myref[1]{\ref{#1}\comment{{\{#1\}}}}

\newtheorem{lem}{Lemma}
\newtheorem{cor}[lem]{Corollary}
\newtheorem{prop}[lem]{Proposition}
\newtheorem{thm}[lem]{Theorem}
\newtheorem{definition}[lem]{Definition}


\theoremstyle{remark}
\newtheorem{exam}{Example}%[section]
\newcommand \myexam[1]{\smallskip\begin{exam}[\emph{#1}]}

\renewcommand{\phi}{\varphi}
\newcommand\eset{\varnothing}
\newcommand\inv{^{-1}}
\newcommand\setm{\setminus}
\newcommand\chiz{\chi^\bbZ}
\newcommand\bbR{\mathbb{R}}
\newcommand\bbZ{\mathbb{Z}}
\newcommand\cH{\mathcal{H}}


\newcommand\Ueloop{\ensuremath{U^e_{0}}}
\newcommand\Uecoloop{\ensuremath{U^e_{1}}}
\newcommand\Uefdyad{\ensuremath{U^{ef}_{1}}}
\newcommand\Uefgtriad{\ensuremath{U^{efg}_{1}}}
\newcommand\Uefgtriangle{\ensuremath{U^{efg}_{2}}}

%   Disjoint Union
%\newcommand{\dunion}{\uplus}
\newcommand{\dunion}
%{\mbox{\hbox{\hskip4pt$\cdot$\hskip-4.62pt$\cup$\hskip2pt}}}
{\mbox{\hbox{\hskip6pt$\cdot$\hskip-5.50pt$\cup$\hskip2pt}}}
%
% Dot inside a cup.
% If there is a better, more Latex like way 
% (more invariant under font size changes) way,
% I'd like to know.


\newcommand{\Bases}[1]{\ensuremath{{\mathcal{B}}(#1)}}
\newcommand{\Reals}{\ensuremath{\mathbb{R}}}
\newcommand{\FieldK}{\ensuremath{K}}
\newcommand{\Perms}{\ensuremath{\mathfrak{S}}}
\newcommand{\rank}{{\rho}}% {{\mbox{rank}}}
\newcommand{\Rank}{{\rho}}% {{\mbox{rank}}}
\newcommand{\Card}[1]{\ensuremath{{\left|#1\right|}}}
\newcommand{\ext}[1]{\ensuremath{\mathbf{#1}}}

% Set Complement
% command to mess with overline, bar or custom 
% alternatives for sequence or set complement
%
%\newcommand{\scomp}[1]{\ensuremath{\;\overline{#1}\;}}
%\newcommand{\scomp}[1]{\ensuremath{\bar{#1}}}
%\newcommand{\scomp}[1]{\ensuremath{\genfrac{}{}{}{}{}{#1}}}
\newcommand{\scomp}[1]{\ensuremath{\overline{#1}}}

\allowdisplaybreaks


%%%%%%%%%%%%%%%%%%%%%%%%%%%%%%%%%%%%%%%%%

\begin{document}

\title[Ported Separator-Strong Parametrized Tutte Functions]
{Ported, Relative, or Set Pointed Separator-Strong 
Parametrized Tutte Functions}

\author{Seth Chaiken}
\address{Computer Science Department\\
The University at Albany (SUNY)\\
Albany, NY 12222, U.S.A.}
\email{\tt sdc@cs.albany.edu}



\begin{abstract}
Tutte decompositions with deletion and contraction not done of elements
in a fixed set of ports $P$, and the 
resulting polynomial expressions and functions
for matroids, oriented matroids, graphs and an abstraction of
labelled graphs are investigated.
We give conditions on parameters $x_e$, $y_e$, $X_e$, $Y_e$ for $e\not\in P$,
and on initial values $I(Q)$ for indecomposibles, that are necessary and 
sufficient for the following equations to have a well-defined solution:
$T(M)=X_e T(M/e)$ for coloops $e\not\in P$, 
$T(M)=Y_e T(M\setminus e)$ for loops $e\not\in P$, and
$T(M)=x_e T(M/e) + y_e T(M\setminus e)$ for other $e\not\in P$.   They
generalize similar conditions given by Bollob\'{a}s and Riordan, Zaslavsky,
and Ellis-Monaghan and Traldi for separator-strong
Tutte functions defined without the
$e\not\in P$ restriction.  We complete the generalization to matroids
given by Diao and Hetyei which was motivated
by invariants for the virtual knots studied by Kauffman.  
Our motivations include electrical network analysis, oriented matroids,
and negative correlation of edge appearances in spanning trees. The $P$-ported
Tutte polynomials of oriented matroids express orientation information
that ordinary Tutte polynomials cannot.
The computation tree formalism of Gordon and McMahon gives
activities expansions for $P$-ported parametrized Tutte polynomials
more general than just
those determined by the linear element orderings
which originated in Tutte's dichromate.

The polynomials expressing the conditions for matroids
each have one factor $I(Q)$.  Since the elements are 
labelled, the methodology applies
to objects such as graphs with ports
for which similar ZBR theorems can be proven.  
We abstract graphs to objects that have 
ported Tutte functions
because they have matroids, but might
have different Tutte function values
on two objects with the same matroid.
Two new ZBR-type theorems are given
and are used to generalize the ZBR
theorem to graphs with port edges.
In some cases, the condition becomes
$I(Q)r+I(Q')s=0$, where $r,s$ are 
polynomials in $x,y,X,Y$.
The abstraction is then used to
characterize ported Tutte functions
of an object or a combination of two
objects, whose matroid or oriented 
matroid is a direct matroid or
oriented matroid sum.  
This extends with ports some 
known strong Tutte 
function and multiplicative 
Tutte function results.
\end{abstract}

\subjclass[2000]{Primary 05B35; Secondary 05C99, 05C15, 57M25, 94C05}



\keywords{Tutte function, Tutte polynomial}

\thanks{Version of \today.}

\maketitle
\pagestyle{headings}


%%%%%%%%%%%%%%%%%%%%%%%%
\section{Introduction}



In his 1971 paper \cite{BrylawskiPointed}
``A Combinatorial Model for Series-Parallel Networks,''
Thomas H. Brylawski
addressed
series/parallel graphs, matroids,
and series/parallel connections of matroids
from the Tutte polynomial point of view.
The rules for 
combining graphs or matroids in series or parallel,
and thus for generating series/parallel graphs and matroids,
refer to
a basepoint edge or element in each.
To study 
Tutte polynomials of series or parallel connections
and so help
characterize
series/parallel matroids as having 
invariant $\beta=1$,
Brylawski developed
a Tutte polynomial for ``pregeometries with basepoint $p_0$'' 
with the four variables $z$, $x$, $z'$ and $x'$.   
His polynomial satisfies
the well-known Tutte equations with deletion and contraction
allowed only for $e\neq p_0$.  
Our first motivation was to generalize this to $e\not\in P$
for a distinguished set $P$ of elements which we call 
\emph{ports}\cite{sdcPorted,MR0419272,SetPointedLV}.  We will
now further generalize so the non-port elements carry parameters.

Brylawski developed this
polynomial from
a ``class of polynomials whose variables are pointed and
nonpointed pregeometries over the integers.''
Such variables, for matroids or oriented matroids
having only port elements, are present
in our universal solution for 
Tutte equations with (1) Brylawsky's restriction extended
from one $p_0$ to many distinguished elements, and (2)
with parameters attached to the other elements.
(See Corollary \ref{UniversalCor} and \cite{sdcPorted}.)
Additional references are given below and in
sec. \ref{BackgroundSec}.  Here is our main definition:

\begin{definition}
\label{SSTMDefinition}
Let $P$ be a set of \textbf{ports} and $\mathcal{C}$ be a class of matroids or
oriented matroids closed under taking minors or oriented minors
by, for $e\not\in P$, deleting $e$ if $e$ is not a coloop or
contracting $e$ if $e$ is not a loop.
Such a $\mathcal{C}$ is called a \textbf{$P$-family}.  
Suppose $\mathcal{C}$ is
given with four parameters $(x_e,y_e,X_e,Y_e)$
in a commutative ring $R$, for each $e\not\in P$ that is an element
in some $M\in\mathcal{C}$.

A \textbf{separator-strong parametrized $P$-ported Tutte function}
$T$ maps $\mathcal{C}$ to $R$ or to an $R$-module and satisfies 
conditions 
\eqref{TA} and \eqref{TSSM} below for all $M\in\mathcal{C}$ and
all $e$ in $S(M)$.


\begin{equation}
\label{TA}
\tag{TA}
\begin{gathered}
T(M) = x_e T(M/e) + y_e T(M\setminus e) \\
\text{ if $e\not\in P$ and $e$ is a non-separator, 
       i.e., neither a loop nor a coloop.}
\end{gathered}
\end{equation}


\begin{equation}
\label{TSSM}
\tag{TSSM}
\begin{gathered}
\text{If } e\not\in P\text{ is a coloop in }M\text{ then }
T(M)=X_e T(M/e).\\
\text{If } e\not\in P\text{ is a loop in }M\text{ then }
T(M)=Y_e T(M\setminus e).
\end{gathered}
\end{equation}

The ground set of $M\in\mathcal{C}$ is denoted by $S(M)$ and
$E(M) = S(M)\setminus P$, the ground set elements of $M$ that are 
not ports.
The $Q\in \mathcal{C}$ for which neither \eqref{TA} nor
\eqref{TSSM} apply, equivalently, $E(Q)=\emptyset$,
are called the \textbf{indecomposibles} or
\textbf{$P$-quotients} in $\mathcal{C}$. The analogous
definitions are used for graphs, directed graphs,
and any other objects on which deletion and contraction are 
defined and act on matroids or oriented matroids associated
to the objects.

\end{definition}

This definition specifies the kind of 
parametrization that we study.
The equations are not consistant
except if certain algebraic conditions on the parameters
and \emph{initial values} $I(Q)$ on indecomposibles $Q$ 
are satisfied (sec. \ref{Complications}.)  
Those conditions
for $P=\emptyset$ (so the one $Q=\emptyset$)
were developed by Zaslavsky\cite{MR93a:05047} and
Bollob\'{a}s and Riordan\cite{BollobasRiordanTuttePolyColored}, and 
then expressed for
a common generalization thereof by
Ellis-Monaghan and Traldi\cite{Ellis-Monaghan-Traldi}, 
now called \emph{separator-strong Tutte functions}\cite{JoAndTom}.
In Part 1, we extend the theory for matroids,
as formulated by the latter authors,
to nonempty $P$.  
We again find necessary and sufficient
algebraic conditions on the parameters and the initial values.
Now the indecomposibles are
matroids or oriented matroids whose ground sets
are contained in $P$.
Diao and Hetyei \cite{RelTuttePoly} give 
similar conditions that
characterize some 
solutions defined with the same $e\not\in P$ restriction,
but for which the initial
values obey a symmetry condition abstracted from the
specialization to graphs.
This led us to complete the generalization to all
separator-strong parametrized $P$-ported Tutte functions.

Evidently, \eqref{TA} and \eqref{TSSM} 
specify how $T(M)$ can be recursively calculated from
the initial values $T(Q)=I(Q)$.
The statement that $T$ is a function 
means that
all such calculations
yield the same result.
A simple induction on $|E(M)|$
shows that if a Tutte function
with specified initial values 
exists for the given $P$-family and parameters, then the function
is unique.  

We follow many lines from Ellis-Monaghan and Traldi's
paper\cite{Ellis-Monaghan-Traldi}.  First, we generalize
their expression of the 
Zaslavsky, Bollob\'{a}s and Riordan (ZBR) theorem
for matroids.  When $P\neq\emptyset$,
the original three families of algebraic conditions
therein become five;
and as before, each condition
derives from a configuration with two or three
non-port elements that are in series or in parallel.  The
first three families are just like those for $P=\emptyset$
except the factor $I(Q)$ replaces $I(\emptyset)$.
The two new configurations are two elements
in series and in parallel, and connected to a non-empty
subset of $P$; their conditions also have an $I(Q)$
factor.

One interesting consequence is the initial
values 
only appear as one $I(Q)$ factor in each condition,
in place of $I(\emptyset)$ in the original ZBR theorem
for matroids.
It answers the natural
question we raised in \cite{TutteEx}, where 
it was proven that $I(Q)$ could be arbitrary 
for $P$-ported Tutte functions with a corank-nullity
expansion.  Parametrized Tutte functions with a 
corank-nullity expansion are called \emph{normal} by
Zaslavsly\cite{MR93a:05047}; this notion extends immediately
to $P\neq\emptyset$.

A new phenomenon, reported and used in \cite{TutteEx}, is that 
some $P$-ported Tutte functions with $\Card{P}\geq 2$
give some new information about oriented matroids.
If $M$ is an oriented
matroid, each indecomposible is an oriented
matroid because the oriented minor $M/A\setminus B$
is well-defined when $A,B$ partitions $E$. (See \cite{OMBOOK}
for oriented matroid minors.)
Hence, when $\Card{P}\geq 2$, $P$-ported Tutte functions
can have different values on different orientations of the
same orientable matroid.  (The smallest matroid with
two distinct orientations is the dyad or 2-circuit.  Both
orientations occur as minors of the oriented circuit
matroid of $K_4$.)
Other Tutte functions with the same domain can be defined after
forgetting the orientations.
Many of our statements refer to ``matroids or
oriented matroids'' because there are different indecomposibles
and Tutte functions depending on whether or 
not the matroids carry an orientation.  Analogous
statements apply to graphs versus directed graphs
and to any other objects with oriented matroids.

Second, we
develop an activities expansion 
and interval partition of subsets of $E(M)$ 
for any recursive
computation using Tutte equations \eqref{TA}
and \eqref{TSSM}.
This generalizes Tutte and Crapo's
classical work 
(see \cite{TutteDich,CrapoTP}, sec. 7.3 of \cite{BjornerChap}) 
and is anticipated
by Diao and Hetyei \cite{RelTuttePoly} and our 
earlier work\cite{TutteEx}.
The classical role of bases is now played by
independent sets $F$ with $F\cap P=\emptyset$
for which $F\cup P$ is spanning.
The computations and expansions are nicely
formalized by Gordon and McMahon's (Tutte)
\emph{computation tree}\cite{GordonMcMachonGreedoid} which we
adopt.   It facilitates our enterprise
to study the effects of distinguishing
certain elements, those in $P$, so they are never deleted or 
contracted in the course of Tutte decompositions
that carry the $x_e$, $y_e$, $X_e$, $Y_e$ parameters
into the resulting polynomials.  The advantage of computation
trees is that the computations are not restricted to those
based on a fixed element order.

Third, this activities expansion helps to 
derive the 
corank-nullity polynomial for normal
$P$-ported Tutte functions.
We give expansions based on boolean lattice interval partitions
and on the matroid's geometric lattice.

In Part 2, we introduce an abstraction of graphs called
\emph{objects with matroids or oriented matroids}.
It helps us to understand $P$-ported Tutte functions
of graphs, identities satisfied 
when a matroid is a direct sum, and the 
$P$-ported generalizations of
strong Tutte functions.  

Strong Tutte functions
were introduced 
Zaslavsky\cite{MR93a:05047} for matroids and graphs.
Besides \eqref{TA}, classical Tutte polynomials and other 
strong Tutte
functions of matroids satisfy the multiplicative identity
\begin{equation}
\tag{TSM}
\label{TSM}
T(M_1\oplus M_2)=T(M_1)T(M_2).
\end{equation}
A Tutte function on graphs is \emph{strong} when
$T(G)=T(G^1)T(G^2)$ if the graphic (i.e., circuit)
matroids satisfy $M(G)=M(G^1)\oplus M(G^2)$.
Different combinations of variations of Tutte equations determine
different kinds of Tutte functions.  For example, the additive 
identity \eqref{TA} alone (with $P=\emptyset$) characterizes
\emph{weak Tutte functions} \cite{MR93a:05047,ZaslavskyOct18}.

The second defining equation \eqref{TSSM} for separator-strong
Tutte functions is the weakening of \eqref{TSM}
so it applies only for cases where one of $M_1$ or $M_2$ is a 
\emph{separator}, that is, a loop or coloop.  
We follow the originating authors' terminology when we extend these
kinds of Tutte functions to $P$-ported matroids or oriented matroids,
$P$-ported graphs,
and then to ported objects 
with  matroids or oriented matroids.
The conclusions for $P$-ported parametrized
strong Tutte functions of matroids
or oriented matroids easily follow from those for the separator-strong ones.
See \ref{StrongTheorem} in sec. \ref{DirectSec}.

Our abstraction and Tutte computation
trees seem to make it easier to handle
the $P\neq\emptyset$ generalizations.
Previous derivations \cite{Ellis-Monaghan-Traldi}
are based on different indecomposible graphs
$E_k$ with $k$ vertices,
and edge set and matroid $\emptyset$, being the
single indecomposibles in disjoint minor-closed
subfamilies.  We must generalize this to
multiple $P$-minors in one graph which might
or might not share matroids.
We include examples to illustrate the complications.
The results include two abstracted ZBR-type theorems
which are then applied to $P$-ported graphs.  
Two different additional hypotheses result in two
different generalizations of the ZBR theorems.
In one of them,
just one of the five ZBR-type conditions is essentially
modified by involving the initial values on two different
indecomposible graphs. 

\subsection{Ports} 
Equations \eqref{TA} and \eqref{TSSM}
reduce to the well-known identities for the 
two-variable Tutte polynomial
when the parameters are $x_e=y_e=1$, $X_e=x$ and $Y_e=y$ for all $e$ and 
the set $P=\emptyset$.
The resulting Tutte polynomials or functions
have been called \emph{set-pointed} \cite{SetPointedLV,SetPointedLV1}, 
\emph{ported} \cite{sdcPorted,TutteEx} and 
\emph{relative} \cite{RelTuttePoly}.  
We prefer the terms ``port'' and ``$P$-ported equations
or functions'' because of our applications
\cite{sdcOMP,TutteEx} and because 
the $P$ is explicit.

A ``port'' edge, ``two terminals with the restriction
that the terminal currents have the same magnitude but
opposite sign''\cite{CRCHandbookPorts}, as termed 
in the circuit theory 
of electrical engineering,
is crucial for carefully analyzing and generalizing the
electrical resistance of a network.
Brylawsky's
1977 work on ``A Determinantal Identity for Resistive Networks''
\cite{BryDetIdResistive} helped underpin this and
our earlier work (see for example \cite{sdcOMP} in addition to
\cite{sdcPorted} and \cite{TutteEx}) where one port is
generalized to many.  Kirchhoff first showed, 
essentially, that network resistance is 
the ratio of two parametrized spanning tree enumerating
Tutte polynomials, one for the network with the port
deleted and the other for the nework with the port contracted.
Motivated by electricity,
Duffin\cite{DuffinSP} characterized series/parallel networks
as those graphs with the confluence property: For each
pair of edges $\{p,q\}$, in every circuit that contains
$\{p,q\}$, they appear in the same relative orientation.
Brylawsky\cite{BrylawskiPointed} mentioned this as
the one known characterization 
that cannot be expressed with base pointed Tutte polynomials.  However,
a graph fails to have the confluence property
if and only if there are two edges $P=\{p,q\}$ for which
both orientations $Q_1$ and $Q_2$ of the 2-circuit $U^{pq}_1$ are 
minors.
Hence a graph is series-parallel if and only if
for every $P=\{p,q\}$ with $\Card{P}=2$, one or both
coefficients of $[Q_1]$ and $[Q_2]$ in our
universal $P$-ported Tutte polynomial $T^{\mathcal{C}}(M(G))$
(Corollary \ref{UniversalCor})
is zero.  

Besides network resistance, other
useful generalizations of Tutte polynomial
theory defined by attaching parameters 
to elements are 
known\cite{Ellis-Monaghan-Traldi,Ellis-Monaghan-Merino-2}.
In the context of invariants under Reidemeister
moves in virtual knot diagrams\cite{KauffmanVirtualKnots}, 
as studied by Diao and Hetyei \cite{RelTuttePoly}, ports play the
role of ``zero edges'' which represent virtual crossings.
The other crossings are represented by edges with 
parameters $\pm 1$, depending on whether say the
left strand goes over or under.  

\subsection{Complications from Parametrization}
\label{Complications}
It is known that,  even when $P=\emptyset$,
parametrized Tutte equations fail to have a solution
except if the certain algebraic relations are true 
about the parameters and
initial values.

For example, if $M=U_{1}^{ef}$ is  ${e,f}$ in parallel, then
applying \eqref{TA} for $e$ gives the polynomial
($U^e_0$, $U^e_1$ are the loop, coloop matroids on $\{e\}$, etc.)
\[x_e T(U_0^f) + y_e T(U_1^f)\]
 whence applying it for $f$ gives 
\[x_f T(U_0^e) + y_f T(U_1^e).\]  
Then, \eqref{TSSM} tells us $T(U_0^f)=Y_fI(\emptyset)$, 
$T(U_1^f)=X_fI(\emptyset)$, etc.
The above are different polynomials in the parameters and initial values.  
The equation that says they are equal is an example of a relation that 
is necessary for a solution to exist.  
When the Tutte identities are 
parametrized, it is important to carefully distinguish 
between a solution value $T(M)$, where a solution $T$ 
is a function  that
satisfies all the relevent identities, and a formal polynomial  that
results from using a subset of the identities to calculate
$T(M)$ for one $M$\cite{MR93a:05047}.  

We follow \cite{BollobasRiordanTuttePolyColored,Ellis-Monaghan-Traldi}
to say $T(M)$ is well-defined when the parameters are in 
ring $R$ and the initial values are in
$R$ or an $R$-module for which the polynomial expressions for $T(M)$ 
obtained by all the recursive applications of the relevant
Tutte equations are equal.
The conditions are conveniently expressed as generators 
for the ideal $I$ such that the universal Tutte function is into
a quotient ring or module modulo $I$.
See Corollary \ref{UniversalCor}.

\section{Preliminaries}

For a matroid or oriented matroid $M$, the
ground set of \emph{elements} is denoted by $S(M)$ and the rank function 
is denoted by $r$.  Given
\emph{port} set $P$, $S(M)\setminus P=\{e\in S(M) \mid e\not\in P\}$ 
is denoted by $E(M)$.  
A matroid, oriented matroid or other object with elements
given with a set of ports $P$ is be called
\emph{$P$-ported.}  In graphs and directed graphs, the elements 
are the edges.  Graphic matroids or oriented matroids always
refer to the circuit matroids.  Graph connectivity refers to
path connectivity whereas connectivity in the graphic matroid
refers to 2-edge-connectivity of the graph.

A $P$-\emph{family} is a collection $\mathcal{C}$ 
of matroids or oriented matroids such that
given $M\in \mathcal{C}$ and $e\in E(M)$, 
the contraction $M/e\in\mathcal{C}$ if $e$ is not a loop in
$M$ and the deletion $M\setminus e\in\mathcal{C}$ if 
$e$ is  not a coloop in $M$.  
The set of non-port elements 
is $E(\mathcal{C})=\{e\mid e\in E(M)$ for some $M\in\mathcal{C}\}$.
It is straightforward to extend these definitions to families of
objects, such as graphs, where each member object has an associated matroid
on appropriate elements of the object, such as edges, and those elements
can be deleted and/or contracted consistantly with the matroid.


The \emph{$P$-minors} of $M$ are
obtained
by deleting or contracting zero or more non-port elements.
Thus, a $P$-family is 
a $P$-minor closed collection of matroids or oriented matroids.
The $P$-minors $Q$ for which $S(Q)\subseteq P$,
i.e., those with no non-port elements, are called the \emph{$P$-quotients}
of $M$.  We say a $P$-quotient belongs to $\mathcal{C}$ if it is a
$P$-quotient of some $M\in\mathcal{C}$.  Note that if $P$ is finite,
(and the objects are just matroids or oriented matroids)
there are only a finite number of $P$-quotients because there are 
only a finite number of matroids or oriented matroids over subsets
of $P$.  In the context of separator-strong Tutte functions,
Definition \ref{SSTMDefinition}, $P$-quotient and \emph{indecomposible}
are synonymous.

As usual, a \emph{separator} is an element that is a loop,  or is
a coloop, i.e., an isthmus in a graph.
The uniform matroid with elements $\{e, f, \cdots \}$ and rank $r$ is denoted
by $U^{ef\cdots}_r$.

Let $R$ be a commutative ring.
We sometimes assume that a $P$-family $\mathcal{C}$ comes equipped
with four parameters $x_e,y_e,X_e,Y_e\in R$ for each 
$e\in E(\mathcal{C})$ and for each $P$-quotient $Q\in\mathcal{C}$,
one initial value $I(Q)$ either in $R$ 
or in an $R$-module.  (One can consider $R$ to be
the $R$-module generated by $R$.)
Note that whether or
not the empty matroid $\emptyset$ is a $P$-quotient depends on
$\mathcal{C}$.  If $M\in\mathcal{C}$ with
$S(M)\cap P=\emptyset$ then $\emptyset$ certainly is a $P$-quotient.
In that case, \eqref{TSSM} specifies that $T(M) = X_e I(\emptyset)$
or $T(M) =Y_e I(\emptyset)$ if $e\not\in P$ is a separator.
Therefore,
we consider $U^e_0$ and $U^e_1$ to be decomposible
and the $X_e$ and $Y_e$ to be parameters
because $X_e$ and $Y_e$ are values of the Tutte function
only if $\emptyset\in \mathcal{C}$ and $I(\emptyset)=1$.
(We differ slightly from \cite{MR93a:05047} where $X_e I(\emptyset)$ and 
$Y_e I(\emptyset)$ are called point values.)

In the remainder, we consider only elements $e, f, g$ none of 
which are in $P$.

Two distinct elements $e,f$ in matroid $M$
are \emph{parallel} when
every cocircuit that contains one of them also contains the other.  This
is equivalent to $\{e, f\}$ being a two-element circuit.
They are \emph{series} when every circuit that contains one of them
also contains the other.  This is equivalent to $\{e, f\}$ being a two-element
cocircuit.
They are called a \emph{dyad} when 
they are both parallel and series.  Note that every 
dyad is a matroid connected
component of $M$.

Two distinct elements $\{e,f\}$  are a \emph{parallel pair connected
to $P$} when they are parallel and there is a cocircuit of the form
$\{e,f\}\cup P'$ with $\emptyset\neq P'\subseteq P$.

Two distinct elements $e,f$ are a \emph{series pair connected
to $P$} when they are series and there is a circuit of the form
$\{e,f\}\cup P'$ with $\emptyset\neq P'\subseteq P$.

Three distinct elements $e,f,g$ are called a \emph{triangle}
when they comprise a 3 element circuit $U_2^{efg}$ that is a 
connected component of $M$.

Three distinct elements $e,f,g$ are called a \emph{triad}
when they comprise a 3 element cocircuit $U_1^{ef}$ that is a 
connected component of $M$.

The following is critical to proving
that the identities of the $P$-ported ZBR Theorem \ref{BigTheorem}
all have the form
$I(Q)\cdot r=0$ where $r$ is a polynomial in the $x_e,y_e,X_e,Y_e$
parameters and $Q$ is one $P$-quotient.

\begin{prop}
\label{SameMinorProp}
Suppose $e,f$ are in series, or are in parallel, in matroid or
oriented matroid $M$.
\begin{enumerate}
\item The minors $M/e\setminus f=M/f\setminus e$
are equal as matroids.
\item If $M$ is oriented, the oriented minors 
$M/e\setminus f=M/f\setminus e$ are equal as oriented matroids.
\end{enumerate}
\end{prop}

\begin{proof}
If $e,f$ are in series, note that $M/e\setminus f$ $=$
$M\setminus f/e$.  $e$ is a coloop in $M\setminus f$,
so $M\setminus f/e$ $=$ $M\setminus\{e,f\}$, which is
clearly the same matroid or oriented matroid if $e,f$
are interchanged.  The relevant theory of minors of oriented matroids
can be found in \cite{OMBOOK}.

If $e,f$ are in parallel, $e,f$ are in series in 
the matroid or oriented matroid dual $M^*$ of $M$.
By the first case, $M^*\setminus e/ f$ $=$
$M^*\setminus f/e$ as matroids or as oriented matroids.
Thus $M/e\setminus f$ $=$ $(M^*\setminus e/ f)^*$ $=$
$(M^*\setminus f/ e)^*$ $=$ $M/f\setminus e$ as matroids or
as oriented matroids.
\end{proof}


\part{Ported Tutte Functions of Matroids}

\section{Matroid Tutte Function Characterization}
\label{ParamTutteSec}
Let $P$ be a set and $\mathcal{C}$ be a $P$-family of matroids or oriented
matroids.  We state, discuss and prove this
generalization of Theorem \ref{ZBRmatroids} of \cite{Ellis-Monaghan-Traldi}:

\begin{thm}
\label{BigTheorem}
The following two statements are equivalent.
\begin{enumerate}
\item $T$ from $\mathcal{C}$ to $R$ or an $R$-module is a $P$-ported 
separator-strong parametrized
Tutte function with $R$-parameters $(x, y, X, Y)$ whose values 
$T(Q)$ on $P$-quotients $Q\in\mathcal{C}$ are the initial
values $I(Q)$.
\item
\begin{enumerate}
\item For every $M=U^{ef}_1\oplus Q\in\mathcal{C}$ with 
$P$-quotient $Q$ ($U^{ef}_1$ is a dyad), 
\[
I(Q)(x_e Y_f + y_e X_f) = 
I(Q)(x_f Y_e + y_f X_e).
\]
\item
For every $M=U^{efg}_2\oplus Q\in\mathcal{C}$ with 
$P$-quotient $Q$ ($U^{efg}_2$ is a triangle), 
\[
I(Q)X_g(x_e y_f + y_e X_f) = 
I(Q)X_g(x_f y_e + y_f X_e).
\]
\item
For every $M=U^{efg}_1\oplus Q\in\mathcal{C}$ with 
$P$-quotient $Q$  ($U^{efg}_1$ is a triad), 
\[
I(Q)Y_g(x_e Y_f + y_e x_f) = 
I(Q)Y_g(x_f Y_e + y_f x_e).
\]
\item
If $\{e,f\}=E(M)$ is a parallel pair connected to $P$, 
\[
I(Q)(x_e Y_f + y_e x_f) = 
I(Q)(x_f Y_e + y_f x_e)
\]
where $P$-quotient $Q=M/e\setminus f=M/f\setminus e$.
\item
If $\{e,f\}=E(M)$ is a series pair connected to $P$, 
\[
I(Q)(x_e y_f + y_e X_f) = 
I(Q)(x_f y_e + y_f X_e)
\]
where $P$-quotient $Q=M/e\setminus f=M/f\setminus e$.
\end{enumerate}
\end{enumerate}
\end{thm}

\subsection{Remarks}
Proposition \ref{SameMinorProp} assures that the different 
expressions for $P$-quotients $Q$ in Theorem \ref{BigTheorem} 
are equal as matroid or as oriented matroids, depending
on how $M$ was given.  
The first three cases are trivial extensions of 
the conditions in the original ZBR theorem.
The only difference
is that our conditions have the factor $I(Q)$ 
in place of $\alpha=T(\emptyset)$.
Just two new conditions are required by $P\neq\emptyset$.  They
are vacuous when $P=\emptyset$.  

\begin{cor}[Zaslavsky-Bollob\'{a}s-Riordan theorem for matroids\cite{Ellis-Monaghan-Traldi}]
\label{ZBRmatroids}
Let R
be a commutative ring, let $\mathcal{C}$ be a minor-closed class of matroids 
defined on subsets of an $R$-parametrized class $U$, and let 
$\alpha\in R$. 
Then there is a parametrized Tutte polynomial on
$\mathcal{C}$
with $T(\emptyset)=\alpha$ if and only if the following identities are 
satisfied.
\begin{enumerate}
\item[a]
Whenever $e$ and $f$ are dyadic in $M\in\mathcal{C}$ 
(i.e., they constitute a two element circuit),
\[
\alpha\cdot(x_e Y_f + y_e X_f) = \alpha\cdot (x_f Y_e + y_f X_e ).
\]
\item[b]
Whenever $e$, $f$ and $g$ are triangular in $M\in\mathcal{C}$ 
(i.e. they constitute a three element circuit),
\[
\alpha\cdot  X_g\cdot (x_e Y_f + y_e x_f) = 
\alpha\cdot  X_g\cdot (Y_e x_f + x_e y_f ).
\]
\item[c]
Whenever $e$, $f$ and $g$ are triadic in $M\in\mathcal{C}$ 
(i.e. they constitute a three element cocircuit),
\[
\alpha\cdot Y_g\cdot  (x_e Y_f + y_e x_f) = 
\alpha\cdot Y_g \cdot (Y_e x_f + x_e y_g ).
\]
\end{enumerate}
\end{cor}

The terminology ``$R$-parametrized class $U$'' (of matroid elements)
was used in \cite{Ellis-Monaghan-Traldi} to emphasize that the parameters
$(x_e,y_e,X_e,Y_e)$ are attached to elements $e$, not equations.  
The assumption that $\mathcal{C}$ is minor-closed implies that the 
three conditions can be restricted to the pair or triple being the
only elements in $M$.  Theorem \ref{BigTheorem} 
is expressed that way.  Note how the one indecomposible $\emptyset$
in the original ZBR Theorem is generalized to $P$-quotients $Q$.
The initial values $I(Q)$ generalizing
$\alpha=I(\emptyset)$, together with the parameters, 
must be given to define a particular Tutte function.

We build upon \cite{Ellis-Monaghan-Traldi} which reconciles
the results of \cite{MR93a:05047} and 
\cite{BollobasRiordanTuttePolyColored} with a common generalization.  
It generalizes the fields and strong Tutte functions of \cite{MR93a:05047}
to the commutative rings and separator-strong Tutte functions of 
\cite{BollobasRiordanTuttePolyColored}, and the definedness on 
all matroids in \cite{BollobasRiordanTuttePolyColored} to definedness
on a minor-closed class in \cite{MR93a:05047}.  Further, the
Tutte functions need not to be $0$ or $1$ on $\emptyset$.

Our proof is the immediate result of adding considerations of ports to the
proof in \cite{Ellis-Monaghan-Traldi}, there described
as ``a straightforward adaption of the proof of Theorem 3.3 of 
\cite{MR93a:05047}.''

When $P=\emptyset$, the empty matroid
$\emptyset$ is the only indecomposible
for \eqref{TA} and \eqref{TSSM}, and
$\emptyset\in\mathcal{C}$ is required,
provided $\mathcal{C}\neq \emptyset$.
When we generalize to 
$P\neq\emptyset$
the indecomposibles
depend on $\mathcal{C}$ and  $\emptyset\not\in\mathcal{C}$ if
and only if every $M\in\mathcal{C}$ contains at least one
port element.

\subsection{Proof}
We sketch the proof with a few details, pointing out differences from
\cite{Ellis-Monaghan-Traldi}.

As in \cite{Ellis-Monaghan-Traldi}, 
the necessary relations are easy to deduce by 
applying \eqref{TA} and \eqref{TSSM} in two different
orders to a general configuration in each of the five families.

Now on to the converse.
As in \cite{Ellis-Monaghan-Traldi}, the strategy
is to first verify
that the conditions imply $T(M)$ is well-defined for 
$n = \Card{E(M)} = 0, 1$ and $2$.
Second, we rely on the hypothesis the $\mathcal{C}$ is closed under
$P$-minors in order to verify
that in a larger minimum $n$ counterexample, the elements of
$E(M)$ are either all in series or all in parallel, 
and then the conditions
imply that all calculation orders give the same result.  All the cases involve
two different combinations of deleting and contracting of several elements
in $E(M)$ where both combinations produce the same $P$-quotient matroids
or oriented matroids.  (In the modifications needed to prove
Theorem \ref{ZBRWellBehaved}, 
$M$ is replaced by object $N$ everywhere. Two possibly
different $P$-quotient objects can occur among some of the expressions 
for equal matroids or oriented matroids.)

Let $M\in\mathcal{C}$ be a counterexample with minimum $n=|E(M)|$,
noting that $n$ does not count $|P\cap S(M)|$.
Therefore, whenever $M'$ is a proper $P$-minor of $M$,
$T(M')$ is well-defined.  The Tutte conditions \eqref{TA} and \eqref{TSSM}
have the property that given $M$ and $e\in E(M)$, exactly one equation
applies.  Therefore, the induction hypothesis entails that
calculations that yield different values for $T(M)$ must start with
reducing by different elements of $E(M)$.  Since $T(M)$ is given 
unambiguously by the initial value $I(M)$ when $n=0$, we can assume
$n\geq 2$.

$M$ cannot contain a separator $e\in E(M)$.  This is a consequence of the
fact, applied to $P$-minors, that 
this $e$ is a separator in every minor of $M$ containing $e$.
Therefore, as observed in \cite{Ellis-Monaghan-Traldi}, every
computation has the same result $X_e T(M/e)$ or $Y_e T(M\setminus e)$ 
depending
on whether $e$ is a coloop or a loop.

Let $e$ be one element in $E(M)$.  Since no element in $E(M)$ is a 
separator, $V=x_{e} T(M/{e}) + y_{e} T(M\setminus {e})$ is well-defined, 
and so is $x_{e'} T(M/{e'}) + y_{e'} T(M\setminus {e'})$ 
for each other $e'\in E$.
We follow \cite{Ellis-Monaghan-Traldi} and define
$D=\{e'\in E(M) \mid V=x_{e'} T(M/{e'}) + y_{e'} T(M\setminus {e'})\}$.  The 
induction
hypothesis then tells us 
that there is at least 
one  $f\in E(M)\setminus D$.  
(Recall $e, e', f\not\in P$.)

Suppose that $e$ is a separator in both $M\setminus f$ and 
$M/f$ and $f$ is a separator in both 
$M\setminus e$ and $M/e$.  Then, 
$T$ would
be well-defined for all four of these $P$-minors and so
we can write 
\[
T(M)=x_e x_f T(M/\{e,f\}) + x_e y_f T(M/e\setminus f)
+ y_e x_f T(M\setminus e/f)
+ y_e y_f T(M\setminus\{e,f\}).
\]
Both computations give the same value because in this situation
the reductions by $e$ and $f$ commute. (The hypothesis $\mathcal{N}$
is well-behaved is used here when 
proving Theorem \ref{ZBRWellBehaved}.)
So, for $M$ to be a counterexample, there must be $e\in D$
and $f\not\in D$ ($e, f\not\in P$)
to which one case of the following lemma applies:


\begin{lem}
\cite{MR93a:05047}
Let $e , f$ be nonseparators in a matroid $M$. Within each column the
statements are equivalent:

\begin{minipage}{0.5\textwidth}
\begin{enumerate}
\item
$e$ is a separator in $M\setminus f$.
\item
$e$ is a coloop in $M\setminus f$.
\item
$e$ and $f$ are in series in $M$.
\item
$f$ is a separator in $M\setminus e$.
\end{enumerate}
\end{minipage}
\begin{minipage}{0.5\textwidth}
\begin{enumerate}
\item
$e$ is a separator in $M/ f$.
\item
$e$ is a loop in $M/ f$.
\item
$e$ and $f$ are in parallel in $M$.
\item
$f$  is a separator in $M/e$.
\end{enumerate}
\end{minipage}
\end{lem}

We claim that one of the following five cases must be satisfied:
\begin{enumerate}
\item 
$n=2$ and $E(M)=\{e,f\}$ is a dyad.
\item
$n\geq 3$ and $E(M)$ is a circuit not connected to $P$.
\item
$n\geq 3$ and $E(M)$ is a cocircut not connected to $P$.
\item
$n\geq 2$ and for some $\emptyset\neq P'\subseteq P$,
$P'\cup E(M)$ is a cocircuit.
\item
$n\geq 2$ and for some $\emptyset\neq P'\subseteq P$,
$P'\cup E(M)$ is a circuit.
\end{enumerate}



As in  \cite{Ellis-Monaghan-Traldi}, we draw the conclusion that
if $e\in D$ and $f\not\in D$ then $e, f$ are either series or parallel. 
It was further proven that a series pair and a parallel pair cannot
have exactly one element in common.  Therefore, the pairs $e,f$ satisfying 
the conditions are either all series pairs or all parallel pairs.  By 
minimality of $n$, $E(M)$ is either an $n$-element parallel class or an 
$n$-element series class.  The last two cases are distinguished from
the first three according to whether or not $E(M)$ is disconnected or not
from elements of $P$ in matroid $M$.  We now use \eqref{TA} and \eqref{TSSM}
to show that, in each case, the calculations that start with $e$ and 
those that start with $f$
have the same result, which contradicts $e\in D$ and $f\not\in D$.

We give the details for case 4.  By hypothesis, 
each of $T(M/e)$, $T(M\setminus e)$, $T(M/f)$, $T(M\setminus f)$,
$T(M/f/e)=T(M/e/f)$, $T(M/e\setminus f)$ and 
$T(M/f\setminus e)$ is well-defined.  (Remark:
By Proposition \ref{SameMinorProp},
$M\setminus f/e$ $=$ 
$M\setminus e/f$.  We don't need this and 
$N\setminus f/e$ $=$ 
$N\setminus e/f$ might be false when proving Theorem \ref{ZBRWellBehaved}.)

Starting with $e$ and with $f$, \eqref{TA} gives the two expressions:
\[
V= x_e T(M/e) + y_e x_f T(M\setminus e /f) + y_e y_f T(M\setminus e\setminus f)
\]
\[
V\neq 
   x_f T(M/f) + y_f x_e T(M\setminus f /e) + y_f y_e T(M\setminus f\setminus e)
\]
Let $M'$ be the $P$-minor obtained by deleting each element
in $E(M)$ except for $e$ and $f$ ($M'=M$ if $n=2$.)
Since 
$E(M')=\{e,f\}$ and $e,f$ are in parallel connected to $P$ within
$M'$, (d) of the hypotheses  tells us that
\[
I(Q) (x_e Y_f - y_f x_e) =
I(Q') (x_f Y_e - y_e x_f ),
\]
where $Q=M'/e \setminus f$, $Q'=M'/f \setminus e$.  
Since $e,f$ are in parallel within $M'$,
matroids
or oriented matroids $Q=Q'$ by Proposition \ref{SameMinorProp}.
(When adapting this proof to Theorem \ref{ZBRWellBehaved}, 
$Q=N'/e\setminus f$ and $Q'=N'/f\setminus e$ might be
different objects.)
Since $A=E(M)\setminus\{e,f\}$ is a set of loops ($\emptyset$ if $n=2$)
in
$M/e\setminus f$ ($N/e\setminus f$) and 
in $M/f\setminus e$ ($N/f\setminus e$), we 
write $Y_A=\prod_{a\in A}Y_a$ ($1$ if $A=\emptyset$)
and use \eqref{TSSM} (and well-behavedness) to write
\[
T(M/e\setminus f) = Y_A I(Q ),\;\;
T(M/f\setminus e) = Y_A I(Q' ),
\]
\[
T(M/e) = Y_f Y_A I(Q ),\;\;\text{ and }\;\;
T(M/f) = Y_e Y_A I(Q' ).
\]
So, since $M\setminus f\setminus e=M\setminus e\setminus f$ 
($N\setminus f\setminus e=N\setminus e\setminus f$ by well-behavedness),
\[
x_e T(M/e) + y_e x_f T(M\setminus e /f)
=
x_f T(M/f) + y_f x_e T(M\setminus f /e)
\]
contradicts 
$V\neq 
x_f T(M/f) + y_f x_e T(M\setminus f /e) + y_f y_e T(M\setminus f\setminus e)$
The remaining cases can be completed analogously.  It might be
noted that our proof differs slightly from \cite{Ellis-Monaghan-Traldi}
in that the cases of $n=3$ and $n\ge 4$ are not distinguished.

\subsection{Universal Tutte Polynomial}
\label{UniversalSec}
It is easy to follow 
\cite{BollobasRiordanTuttePolyColored,Ellis-Monaghan-Traldi} to define
a universal, i.e., most general $P$-ported parametrized
Tutte function $T^{\mathcal{C}}$ for the $P$-minor closed class 
$\mathcal{C}$ given without parameters or initial values.  To do this,
we take indeterminates $x_e, y_e, X_e, Y_e$ for each $e\in E(\mathcal{C})$
and an indeterminate $[Q]$ for each $P$-quotient $Q\in\mathcal{C}$.
Let $\mathbb{Z}[x,y,X,Y]$ denote the integer polynomial ring generated by
the $x_e,y_e,X_e,Y_e$ indeterminates, and define $\widetilde{\mathbb{Z}}$
to be the $\mathbb{Z}[x,y,X,Y]$-module generated by the $[Q]$.  
Let $I^{\mathcal{C}}$ denote the ideal of $\widetilde{\mathbb{Z}}$ 
generated by the identities of Theorem \ref{BigTheorem}, comprising 
for example $[Q](x_eY_f+y_eX_f-x_fY_e-y_fX_e)$ for each subcase of
case (a), etc.  The universal Tutte function has values in the
quotient module $\widetilde{\mathbb{Z}}/I^{\mathcal{C}}$.  Finally,
observe that the range of Tutte function $T$  can be considered to be the
$R$-module generated by the values $I(Q)$ where ring $R$ contains the
$x,y,X,Y$ parameters. If the $I(Q)\in R$, consider
the ring $R$ to be the $R$-module generated by $R$.
We follow \cite{Ellis-Monaghan-Traldi} to write
the corresponding consequence of Theorem \ref{BigTheorem}:

\begin{cor}
\label{UniversalCor}
Let $\mathcal{C}$ be a $P$-minor closed class of matroids or 
oriented matroids.  Then there is a 
$\widetilde{\mathbb{Z}}/I^{\mathcal{C}}$-valued function 
$T^{\mathcal{C}}$ on $\mathcal{C}$ with $T^{\mathcal{C}}(Q)=[Q]$ for each $P$-quotient
$Q\in\mathcal{C}$ that is a $P$-ported parametrized Tutte function
on $\mathcal{C}$ where the parameters are the 
$x, y, X, Y \in \widetilde{\mathbb{Z}}$.  Moreover, if $T$ is any $R$-parametrized
Tutte function with parameters $x'_e, y'_e, X'_e, Y'_e$, then $T$ is the
composition of $T^{\mathcal{C}}$ with the homomorphism determined by
$[Q]\rightarrow I(Q)=T(Q)$ for $P$-quotients $Q$ and 
$x_e\rightarrow x'_e$, etc., for each $e\in E(\mathcal{C})$.
\end{cor}

\section{Tutte Computation Trees and Activities}
\label{Activity}

Several authors \cite{Ellis-Monaghan-Merino-1,Ellis-Monaghan-Traldi}
surveyed the two ways that the 
two-variable Tutte polynomial can be defined:  It 
may be defined either as a universal solution to the
recursive strong Tutte equations, or as a generating function.
Further, two kinds of generating function definitions have been given.
The first is what Tutte originally used for graphs 
\cite{TutteDich,TutteGraphBook} and is
called the \emph{basis} or \emph{activities}
expansion. It enumerates each basis $B\subseteq E$ 
by a term $X^{i(B)}Y^{e(B)}$ (compare to \eqref{PAE} below
in Proposition \ref{TuttePolyExpression}), where the 
\emph{numbers of internally and externally active elements} 
$i(B)$ and $e(B)$ are determined from a given linear order
on the elements of $E$ (see Definition \ref{Activities-Ordered-Def}).
It was shown that even though $(i(B),e(B))$ for particular $B$
might vary with the order, the resulting polynomial is
independent of this order, and that it satisfies the Tutte equations.
The second, called the \emph{corank-nullity} generating function, 
is well-defined automatically
because it enumerates each subset $A\subseteq E$ with
the term $(X-1)^{(r(E)-r(A))}(Y-1)^{(|A|-r(A))}$.  
This generating function is then shown to satisfy the Tutte equations.
Zaslavsky noted that the activities expansion remains universal when parameters
are included whereas the corank-nullity generating function expresses only the
proper subset of Tutte functions which he called normal\cite{MR93a:05047}.
They are characterized by \eqref{CNF}.
See sec. \ref{NormalSubSec}.


Ellis-Monaghan and 
Traldi \cite{Ellis-Monaghan-Traldi} remarked that the Tutte equation
approach appears to give a shorter proof of the ZBR theorem
than the activities expansion approach. 
The proofs by induction on $|E|$
demonstrate that every calculation of $T(M)$ from Tutte equations
produces the same result when the conditions on the parameters
and initial values are satisfied.  We then know immediately that
the polynomial expression
resulting from
a particular calculation equals the Tutte function value.
We show that every recursive calculation (see below)
gives rise to an 
activities expansion, when the activities are defined in the
more general way given by McMahon and Gordon \cite{GordonMcMachonGreedoid}.
We suggest a heuristic reason why
the Tutte equation approach is more succinct:
The induction assures that
\emph{every} computation
with smaller $|E|$ gives the same result, not just those 
computations that are determined by linear orders on $E$.

To be precise, \emph{recursive}
means that the computation uses Tutte equations
to find $T(M)$ in terms of the initial values
and/or  some $T(M')$ with $|E(M')|<|E(M)|$
using recursive computations.  (Note the inductive
definition.)
All the recursive computations of $T(M)$ 
are expressible by ``computation trees,''
formally defined by McMahon and Gordon 
\cite{GordonMcMachonGreedoid}.
Their motivation was to generalize activities expansions and
the corresponding interval partitions of the subset lattice from
matroids to greedoids.  Unlike matroids, some greedoids
do not have an activities expansion for their Tutte polynomial
that derives from an element ordering.

Proofs of activities expansions for matroids, and their generalizations
for $P$-ported matroids, seem more informative and certainly
no harder when the expansions are derived from a general Tutte
computation tree, than when the expansions are only those
that result from an element order.   
From the retrospective that the Tutte equations
specify a non-deterministic recursive computation
\cite{Garey-Johnson}, it seems artificial to start with
element-ordered computations and then prove
first that all linear orders give the same result and 
second that it 
satisfies the Tutte equations, in order prove that 
all recursions give the same result.
We therefore
take advantage of the Tutte computation tree formalism
and the more general expansions it enables.


\subsection{Computation Tree Expansion}

We begin with the ported generalization
of the matroid bases in the classical activities expansion.
Let $\mathcal{B}(M)$ denotes the set of bases in $M$.

\begin{definition}
Given $P$-ported matroid or oriented matroid $M$,
a \textbf{$P$-subbasis} $F$
is an independent set  with $F\subseteq E(M)$
(so $F\cap P=\emptyset$) for which $F\dunion P$ is a spanning set
for $M$
(in other words, $F$ spans $M/P$).
$\mathcal{B}_P(M)$ denotes the set of $P$-subbases.
\end{definition}

An equivalent definition was given in \cite{SetPointedLV}.
The following proposition shows our definition is equivalent to
that given in \cite{RelTuttePoly}.  $C$ and $D$ below are called 
``contracting and deleting sets'' in that paper.

\begin{prop}
$C$ is a $P$-subbasis if and only if
$C \subseteq E(\mathcal(M))$ has no circuits and 
$D=E(M)\setminus C$ has no cocircuits.
\end{prop}

\begin{proof}
$C$ has no circuits means $C$ is an independent set in 
$M$.
$D$ has no cocircuits means $D$ is independent in
the dual of $M$, i.e., $D$ is coindependent.  
$D$ is coindependent if and only if
$S(M)\setminus D=P\dunion C$ is spans
$M$.
\end{proof}


\begin{prop}
For every $P$-subbasis $F$ there exists an independent set $Q\subseteq P$
that extends $F$ to a basis $F\dunion Q\in \mathcal{B}(M)$.
Conversely, if $B\in\mathcal{B}(M)$ then $F=B\cap E=B\setminus P$
is a $P$-subbasis.
\end{prop}

\begin{proof} Immediate. \end{proof}


The next definition 
is also equivalent to one in \cite{RelTuttePoly}. It generalizes
Tutte's definitions based on element orderings \cite{TutteGraphBook,TutteDich} 
extended to
matroids \cite{CrapoAct}.  We will see that expansions based on 
computation trees generalize these further.

\begin{definition}[Activities with respect to a $P$-subbasis and an element
ordering $O$]
\label{Activities-Ordered-Def}
Let ordering $O$ have every $p\in P$ before every
$e\in E$.  Let $F$ be a $P$-subbasis.  Let $B$ be any basis for 
$M$ with $F\subseteq B$.
\begin{itemize}
\item Element $e\in F$
is internally active if $e$ is the least element
within its principal cocircuit with respect to $B$.  Thus, this principal
cocircuit contains no ports.  The reader can verify this definition is 
independent of the $B$ chosen to extend $F$.  Elements $e\in F$ that are
not internally active are called internally inactive.
\item Dually, element $e\in E$ with $e\not\in F$ is externally 
active if $e$ is the least element within its principal circuit with
respect to $B$.  Thus, each externally active element is spanned by 
$F$.  Elements $e\in E\setminus F$ that are not externally active
are called externally inactive.
\end{itemize}
\end{definition}

\begin{definition}[Computation Tree, following \cite{GordonMcMachonGreedoid}]
\label{CompTreeDef}
A $P$-ported (Tutte) computation tree for $M$ is a
binary tree whose root is labeled by $M$ and which satisfies:
\begin{enumerate}
\item If $M$ has non-separating elements not in $P$, then 
the root has two subtrees and there exists one such element $e$ for which 
one subtree is a computation tree
for $M/e$ and the other subtree is a computation tree for 
$M\setminus e$.

The branch to $M/e$ is labeled with ``$e$ contracted'' and 
the other branch is labeled ``$e$ deleted''.
\item Otherwise (i.e., every element in $E(M)$
is separating) the root is a leaf.
\end{enumerate}
\end{definition}

Figure \ref{c4p2TwoFigure} illustrates two computation trees; it 
is conventional to slant contraction edges to the left and
deletion edges to the right.
An immediate consequence is
\begin{prop}
Each leaf of a $P$-ported Tutte computation tree for $M$
is labeled by the direct sum of some $P$-quotient
(oriented if $M$ is oriented) 
summed with loop and/or coloop matroids with 
ground sets $\{e\}$ for various distinct $e\in E$ (possibly none).
\end{prop}

It sometimes helps to revise Definition \ref{CompTreeDef} to require
that every leaf be labelled with an indecomposible; and then allow
a single branch from $M$ to $M/e$ or to $M\setminus e$
labelled ``$e$ contracted as a coloop'' or ``$e$ deleted as a coloop'' 
depending on what kind of separator is $e\in E(M)$.  We leave
the corresponding revisions of further definitions to the reader.


\begin{definition}[Activities with respect to a leaf]
\label{ActivityTreeDef}
For a $P$-ported Tutte computation tree for $M$,
a given leaf, and the path from the root to this leaf:
\begin{itemize}
\item Each $e\in E(M)$ labeled ``contracted'' along this path
is called \textbf{internally passive}.
\item Each coloop $e\in E(M)$ in the leaf's matroid is
called \textbf{internally active}.
\item Each $e\in E(M)$ labeled ``deleted'' along this path
is called \textbf{externally passive}.
\item Each loop $e\in E(M)$ in the leaf's matroid is
called \textbf{externally active}.
\end{itemize}
\end{definition}

\begin{prop}
Given a leaf of a $P$-ported Tutte computation tree for $M$,
the set of internally active or internally passive elements 
constitutes a 
$P$-subbasis of $M$ which we say 
\textbf{belongs to the leaf}.  
Furthermore, every $P$-subbasis $F$ of $N$ belongs to a unique leaf.
\end{prop}

\begin{proof}
For the purpose of this proof, let us extend Definition \ref{ActivityTreeDef}
so that, given a computation tree with a given node $i$ 
labeled by matroid $M_i$,
$e\in E$ is called internally passive when $e$ is labeled 
``contracted'' along the path from root $M$ to
node $i$.  Let $IP_i$ denote the set of such internally passive 
elements.

It is easy to prove by induction on the length of the root to node $i$ path
that
(1) $IP_i\cup S(M_i)$ spans $M$ and 
(2) $IP_i$ is an independent set in $M$.  The proof
of (1) uses the fact that elements labeled deleted are non-separators.  The
proof of (2) uses the fact that for each non-separator 
$f\in M/IP_i$, $f\cup IP_i$ is independent in $M$.

These properties applied to a leaf demonstrate the first conclusion,
since each $e\in E$ in the leaf's matroid must be a separator by Definition 
\ref{CompTreeDef}.

Given a $P$-subbasis $F$, we can find the unique leaf with the
algorithm below.  Note that it also operates on arbitrary subsets of $E$.

\textbf{Tree Search Algorithm:} Beginning
at the root, descend the tree according to the rule: At each branch node,
descend along the edge labeled ``$e$-contracted'' if $e\in F$ and along
the edge labeled ``$e$-deleted'' otherwise (when $e\not\in F$).
\end{proof}

The above definitions and properties lead us to 
reproduce element order based activities:
\begin{prop}
Given element ordering $O$ in which every $p\in P$ is ordered
before each $e\not\in P$, suppose we construct the unique $P$-ported
computation tree $\mathcal{T}$ in which the greatest non-separator $e\in E$ is
deleted and contracted in the matroid at each tree node.

The activity of each $e\in E$ relative to ordering $O$ and
$P$-subbasis $F\subseteq E$ is the same as the activity
of $e$ defined with respect to the leaf 
belonging to $F$ in $\mathcal{T}$.
\end{prop}

\begin{definition}
\label{ActivitySymbolsDef}
Given a computation tree for 
$P$-ported (oriented) matroid $M$,
each $P$-subbasis $F\subseteq E$
is associated with the following subsets of non-port elements
defined according to Definition \ref{ActivityTreeDef}
from the unique leaf determined by the algorithm given above.
\begin{itemize}
\item $IA(F)\subseteq F$ denotes the set of internally active elements,
\item $IP(F)\subseteq F$ denotes the set of internally passive elements,
\item $EA(F)\subseteq E\setminus F$ 
denotes the set of externally active elements,
and 
\item $EP(F)\subseteq E\setminus F$ denotes the set of externally
passive elements.
\item $A(F)=IA(F)\cup EA(F)$ denotes the set of active elements.
\end{itemize}
\end{definition}

\begin{prop}
\label{PartitionProposition}
Given a $P$-ported Tutte computation tree for
$M$, 
the boolean lattice of subsets of $E=E(M)$
is partitioned by the collection of
intervals $[IP(F),F\cup EA(F)]$ (note $F\cup EA(F)=IP(F)\cup A(F)$)
determined from the collection
of $P$-subbases $F$, which correspond to the leaves.

The boolean lattice of subsets of $E=E(M)$
is also partitioned by the collection of
intervals $[EP(F),E\setminus F\cup IA(F)]$ 
(note $E\setminus F\cup IA(F)$ $=$ $EP(F)\cup A(F)$).

For a given $F\in\mathcal{B}_P(M)$, $A\subseteq E$ satisfies
$A\in [IP(F),F\cup EA(F)]$ if and only if 
$(E\setminus A)\in [EP(F),E\setminus F\cup IA(F)]$. 
\end{prop}

\begin{proof}
Every subset $A\subseteq E=E(M)\setminus P$ belongs to the
unique interval corresponding to the unique leaf found by the tree search 
algorithm given at the end of the previous proof.  

The dual of that tree search algorithm, which descends along
the edge labelled ``$e$-deleted'' if $e\in A'$, etc., will find the
unique leaf whose interval $[EP(F),E\setminus F\cup IA(F)]$ contains
$A'$.

When $A\in[IP(F),F\cup EA(F)]$, the dual algorithm applied to
$A'=E\setminus A$ will find the same leaf.
\end{proof}

%%%%%%%%%%%%%%%%%%%%%%%%%%%%%%%%%%%%%%%%%%%%%%%%%%%%%%%%

The following generalizes the activities expansion expression given
in \cite{MR93a:05047} to ported (oriented) matroids, as well as 
Theorem 8.1 of \cite{SetPointedLV}. 

\begin{prop}
\label{TuttePolyExpression}
Given parameters $x_e$, $y_e$, $X_e$, $Y_e$, and 
$P$-ported matroid or oriented matroid $M$
the Tutte polynomial expression
determined by the sets in Definition 
\ref{ActivitySymbolsDef} 
from a computation tree is 
given by
\begin{equation}
\tag{PAE}
\label{PAE}
\sum_{F\in \mathcal{B}_P}[M/F|P]
\;X_{IA(F)}\;x_{IP(F)}\;Y_{EA(F)}\;y_{EP(F)}.
\end{equation}
\end{prop}

\begin{proof}
\eqref{PAE} is an expression constructed by applying some of the
Tutte equations.  One monomial results from each leaf.
It that leaf's matroid,
each active element is a separator, and the active elements
contribute $X_{IA(F)}Y_{EA(F)}$ to the monomial.
The passive elements which contribute
$x_{IP(F)}y_{EP(F)}$
are the tree
edge labels in the path from the root to the leaf.
Each $M/F|P$ denotes a $P$-quotient of $M$, so the
expression is a polynomial in the parameters and in the
initial values.  
Therefore, \eqref{PAE} expressions the result of
the calculation when one substitutes
$[M/F|P]=I(M/F|P)$.
\end{proof}

From Corollary \ref{UniversalCor} we conclude:
\begin{thm}
\label{ActivitiesTheorem}
For every $P$-ported parametrized Tutte function $T$ 
on $\mathcal{C}$ into 
ring $R$ or an $R$-module,
for every computation tree for $M\in\mathcal{C}$
(and so for every ordering of $E(M)$), 
the polynomial expression \eqref{PAE} equals 
$T^{\mathcal{C}}(M)$ of Corollary \ref{UniversalCor}.
\end{thm}

\subsection{Expansions of Normal Tutte Functions}
\label{NormalSubSec}.

After a notational translation, 
Zaslavsky's \cite{MR93a:05047} definition of \emph{normal} Tutte 
functions becomes
those for which $T(\emptyset)=1$, and for which there exist
$u$, $v\in R$ so that for each $e\in E(M)$,

\begin{equation}
\tag{CNF}
\label{CNF}
X_e = x_e + uy_e \text{ and } Y_e = y_e + vx_e.
\end{equation}

Let us drop the $T(\emptyset)=1$ constraint and then
note that \eqref{CNF} applies immediately to $P$-ported Tutte functions.
(Unfortunately, we use 
$(x_e, y_e, X_eT(\emptyset), Y_eT(\emptyset))$ for Zaslavsky's
notations $(b_e, a_e, x_e, y_e)$.)  The normal Tutte functions
include the classical two variable Tutte polynomial.
Please observe that the equations of Theorem \ref{BigTheorem}
are satisfied by \eqref{CNF} independently of the
initial values.  Hence all the expressions for normal Tutte
functions will be in a ring freely generated by
$u$, $v$, the $x_e, y_e$ and the $[Q]$.  We will therefore
call them expansions for a Tutte polynomial.  This
Tutte polynomial is universal for $P$-ported separator-strong
normal parametrized Tutte functions of matroids or oriented matroids.
We can now generalize some known expansions.

\subsubsection{Boolean Interval Expansion}

\begin{cor}
\label{NormalActProp}
The following activities and boolean interval expansion formula
is universal for normal Tutte functions and
is obtained by substituting \eqref{CNF} 
into $T^{\mathcal{C}}(M)$.
\[
T^{\mathcal{C}}(M)=
\sum_{F\in \mathcal{B}_P}[M/F|P]
%\left(
\Big(
\sum_{\substack{
       IP(F)\subseteq K \subseteq F\\
       EP(F)\subseteq L \subseteq E\setminus F
      }}
 x_{K\cup (E\setminus F\setminus L)}\;
 v^{\Card{E\setminus F\setminus L}}\;
 y_{L\cup (F\setminus K)}\;
 u^{\Card{F\setminus K}}\;\;
%\right)
\Big)
\]
\end{cor}

\begin{proof} After substituting \eqref{CNF} we get
\[
T^{\mathcal{C}}(M)=
\sum_{F\in \mathcal{B}_P}[M/F|P]
%\left(
\Big(
x_{IP(F)}\;\;
\prod_{e\in IA(F)}\left(x_e+y_eu\right)\;\;
y_{EP(F)}\;\;
\prod_{e\in EA(F)}\left(y_e+x_ev\right)
%\right)
\Big)
\]
and then, by Definition \ref{ActivitySymbolsDef}, 
$IP(F)\dunion IA(F) =F$ and 
$EP(F)\dunion EA(F)=E\setminus F$.
 \end{proof}


\begin{lem}
\label{KLAlemma}
Given $F\in\mathcal{B}_P$,
$IP(F)$ spans $EA(F)$.

The pairs $(K,L)$ for which 
       $IP(F)\subseteq K \subseteq F$ and 
       $EP(F)\subseteq L \subseteq E\setminus F$
are in a one-to-one correspondance
with the $A$ satisfying $IP(F)\subseteq A\subseteq F\dunion EA(F)$
given by $A=K\dunion (E\setminus F)\setminus L$.


For every such $A$, 
\[
\Card{F\setminus K}=r(M)-r(M/F|P)-r(A)
\]
and
\[
\Card{E\setminus F\setminus L} = \Card{A}-r(A).
\]
\end{lem}

\begin{proof}
(See Figure \ref{Venn} in Appendix.) By our definition of activities,
after all the elements of $IP(F)$ are contracted, all elements 
in $EA(F)$ are loops.
(Note none of these elements are ports.)

Let 
$A=K\dunion (E\setminus F)\setminus L$.
By our definition of activities,
$IP(F)\dunion IA(F)=F$, so $IP(F)\subseteq A$.
Similarly,  
$EP(F)\dunion EA(F)=E\setminus F$, so 
$A\cap(E\setminus F)\subseteq EA(F)$.
Hence 
$IP(F)\subseteq A\subseteq F\dunion EA(F)$.
Conversely, given such an $A$, 
take $K=A\cap F$ and $L=(E\setminus F)\setminus A$.

Since $IP(F)$ spans $EA(F)$ and $K\supseteq IP(F)$, $K$ spans $EA(F)$.
Since $A\subseteq K\dunion EA(F)$, $K$ spans $A$.
$K\subseteq F$, $F$ is a $P$-subbasis, so $K$ and $F$ are independent,
hence $\Card{K}=r(K)=r(A)$ and $\Card{F}=r(F)$.  
Therefore, $\Card{F\setminus K}=r(F)-r(A)$.

Since $F$ is a $P$-subbasis, $r(F\cup P)=r(M)$.
By definition of contraction, $r(M/F|P)=r(F\cup P) - r(F)$,
so $r(M/F|P)=r(M)-r(F)$.  We conclude 
$\Card{F\setminus K}=r(M)-r(M/F|P)-r(A)$.

$E\setminus F\setminus L = A\setminus K$, so
$\Card{E\setminus F\setminus L} = \Card{A}-\Card{K}$.
As above, $\Card{K}=r(A)$, so the last equation follows.
\end{proof}


\begin{cor}
\begin{equation}
\label{FIntervalExpansion}
T^{\mathcal{C}}(M)=
\sum_{F\in \mathcal{B}_P}[M/F|P]
%\left(
\Big(
\sum_{IP(F)\subseteq A \subseteq F\dunion EA(F)}
 x_{A}
 y_{E\setminus A}
 u^{r(M)-r(M/F|P)-r(A)}
 v^{\Card{A}-r(A)}
%\right)
\Big)
\end{equation}
\end{cor}

\begin{proof}
Apply Lemma \ref{KLAlemma} to the inner sum in
Proposition \ref{NormalActProp}.
\end{proof}

\subsubsection{Corank-nullity Expansion}


\begin{lem}
\label{FAMinorlemma}
Given $F\in\mathcal{B}_P(M)$, $(K,L)$ and $A=K\dunion E\setminus F\setminus L$
as in Lemma \ref{KLAlemma},
\[ M/F|P = M/A|P\]
(as matroids or oriented matroids).
\end{lem}
\begin{proof}
Writing the contractions and deletions explicitly, $M/F|P$ 
$=$ $M/F\setminus (E\setminus F)$.  By our definition of
activities, all the $e\not\in P$ in $M/IP(F)\setminus EP(F)$ are 
loops or coloops.  Hence, when all these elements are removed
from $M/IP(F)\setminus EP(F)$ whether by contraction or deletion, the
result is the same matroid or oriented matroid.  Since
$IP(F)\subseteq A\subseteq F$ and 
$EP(F)\subseteq (E\setminus A)\subseteq (E\setminus F)$, we 
can construct $M/F|P$ or $M/A|P$ by forming $M/IP(F)\setminus EP(F)$
first, contracting the remaining elements of $F$ or $A$,
and last deleting all the remaining $e\not\in P$.  
Therefore  $M/A|P$ $=$ $M/F|P$.
\end{proof}

\begin{thm}
\begin{equation}
\tag{PGF}
\label{PGF}
T^{\mathcal{C}}(M) = \sum_{A\subseteq E(M)}[M/A \mid P]x_A y_{E\setminus A}
u^{r(M)-r(M/A\mid P)-r(A)}
v^{|A|-r(A)}.
\end{equation}
\end{thm}

Remark: This extends with parameters, oriented matroids and 
\eqref{TSSM} replacing \eqref{TSM} an expression from
\cite{MR0419272} reproduced in \cite{sdcPorted}.  It can
also be proved by those methods.

\begin{proof}
By Proposition \ref{PartitionProposition}, given any Tutte computation tree,
the lattice of subsets of $E(M)$ is partitioned
into intervals corresponding to $P$-subbases $\mathcal{B}_P$.  
However,  given $F\in\mathcal{B}_P$, for every $A$ satisfying
Lemma \ref{FAMinorlemma}, the $P$-quotient $M/F|P$ is equal to 
$M/A|P$ (as a matroid or oriented matroid).  Hence we can 
interchange the summations in \eqref{FIntervalExpansion} and write
\eqref{PGF}.
\end{proof}

\subsubsection{Geometric Lattice Flat Expansion}


\begin{prop}
\label{GFlatProp}
Let $M$ be an oriented or unoriented.
In the formula below,
$F$ and $G$ range over the geometric lattice of flats 
$\mathcal{L}=\mathcal{L}(M|E)$ contained
in $M$ restricted to $E = E(M)$.  
($E$ is the top of $\mathcal{L}(M|E)$ and $\le$
is that lattice's partial order.)
\[
T^{\mathcal{C}}(M) = \sum_{Q} [Q]
      \sum_{\substack{F\leq E\\
                     [M/F|P]=[Q]
           }}
                   u^{r(M)-r(Q)-r(F)}
                   v^{-r(F)}
                   \sum_{G\le F}
                   \mu(G,F)
                   \prod_{e\in G}
                    (y_e+x_ev)
\]
\end{prop}

\begin{proof}
This 
generalizes and 
follows the steps for theorem 8 in \cite{sdcPorted}.  (See Appendix.)
\end{proof}

\part{Objects, Graphs and Sums}


When $P=\emptyset$, the facts about separator-strong Tutte functions
of matroid direct sums easily follow from the formula
$T(M^1\oplus M^2)T(\emptyset)=T(M^1)T(M^2)$.  For example,
$T$ is strong if and only if $T(\emptyset)=T(\emptyset)^2$.
The theory of separator-strong Tutte functions of graphs covered
in \cite{Ellis-Monaghan-Traldi}  follows from the fact that any
minor closed family of graphs (see below) $\mathcal{G}$ is partitioned into
subfamilies $\mathcal{G}_k$, each with just one indecomposible,
$E_k$, the edgeless graph with $k$ unlabelled vertices, 
if $\mathcal{G}_k\neq\emptyset$.
Tutte function formulas for disjoint and one-point graph unions, 
and the conditions for strongness (defined $T(G^1)T(G^2)=T(G)$ if
matroids $M(G^1)\oplus M(G^2)=M(G)$) are then 
derived \cite{Ellis-Monaghan-Traldi}
in terms of the values $\alpha_k=I(E_k)$.  Life is simple because
matroid $M(E_k)=\emptyset$ for all $k$.

The corresponding facts become more complex when the definitions
are naturally extended to $P$-ported matroids and graphs
or to vertex labelled graphs.  As with
matroids, a $P$-ported graph $G$ has some of the edges in $P$ and the rest,
$E(G)$, satisfy $E(G)\cap P=\emptyset$.  Deletion, contraction,
$P$-minors, $P$-families and $P$-quotients (i.e., indecomposibles)
are also defined as they are for $P$-ported matroids or oriented matroids.
As in \cite{Ellis-Monaghan-Traldi},
deletion of an isthmus (i.e., coloop in the matroid) and
contraction of a loop is forbidden within Tutte equations.

The main difficulty is illustrated by the following example.
Let
$G$ be the circle graph of the five edges
ordered $(e,p,f,q,r)$ and take $P=\{p,q,r\}$.  
So, $e$ and $f$ are a series pair connected to $P$, but
the $P$-quotient graphs $Q_1=G/e\setminus f$ and 
$Q_2=G/f\setminus e$ are different graphs, even though they have 
the same matroid.  $Q_1$ is the path $qrp$ and $Q_2$ is the path $pqr$.
Function $T$ might satisfy \eqref{TA}
and \eqref{TSSM} even if $T(Q_1)=I(Q_1)\neq I(Q_2)=T(Q_2)$.
So, if this $G\in\mathcal{G}$, a necessary condition 
for $T$ to be a $P$-ported
(separator-strong, as always) Tutte function would be
\[
I(Q_1)(x_ey_f - y_fX_e) = I(Q_2)(x_fy_e - y_eX_f).
\]
This equation does not have the form of 
those in the ZBR theorem for graphs \cite{Ellis-Monaghan-Traldi}
because the latter's equations, like the equations in
Theorem \ref{BigTheorem}, each has a single factor 
$I(Q)$ depending on one indecomposible.


The example relies on the elements of $P$ being labelled.
This leads us to formulate a extension of Tutte function theory
for vertex labelled graphs.  
When the outcome, Theorem \ref{ZBRWellBehaved} is applied to graphs as in 
\cite{Ellis-Monaghan-Traldi}, we get Corollary \ref{PZBRGraphCor2}
which demonstrates that the above example illustrates the
\emph{only} situation where the $P$-ported ZBR equations
of Theorem \ref{BigTheorem} must be modified.

Figure \ref{c4p2TwoFigure} illustrates 
the same phenonemon as the first, in a smaller graph, when
the objects in $\mathcal{G}$ are graphs whose vertices are 
labelled by disjoint sets.  Again, two different graphs have the
same oriented matroid.

\begin{figure}
\input{c4p2Two.pdf_t}
\caption{\label{c4p2TwoFigure}
Two Tutte computation trees for the same graph with set labelled
vertices.  The indecomposibles at the bottom of the first tree
are named $Q_1$, $Q_2$ and $Q_3$ from left to right.  For non-separators,
contraction 
edges are slanted left and  deletion edges are slanted right.  An edge
representing a separator is drawn vertically.}
\end{figure}

The expressions from the two Tutte computation trees in Figure
\ref{c4p2TwoFigure}
are 
\[
I(Q_1)x_ex_f+I(Q_2)x_ey_f+I(Q_3)y_eX_f
\]
and
\[
I(Q_1)x_ex_f+I(Q_3)x_fY_e+I(Q_2)y_fX_e=
I(Q_1)x_ex_f+I(Q_2)y_fX_e+I(Q_3)x_fY_e,
\]
which are equal if and only if
\begin{equation}
\label{BadZBRExampleEq}
I(Q_2)(y_fX_e - x_ey_f)=
I(Q_3)(y_eX_f - x_fY_e).
\end{equation}
$Q_2$ and $Q_3$ are isomorphic as edge-labelled graphs
but are different when the vertex
labels are present.

There are two  complications introduced
into $P$-families of matroids when $P\neq\emptyset$.
First, one matroid 
might have more than one $P$-quotient, i.e., indecomposible.
The most simple 
example is a dyad matroid composed of one port and one
non-port element; and its $P$-minors.
Therefore, minimal $P$-minor closed families might have 
more than one indecomposible.  Some will share
matroids or oriented matroids and others will not.
The second, which also occurs with the 
minor closed families of graphs \cite{Ellis-Monaghan-Traldi}
in the original $P=\emptyset$ form, is that
the family is partitioned into disjoint $P$-minor closed
subfamilies.  Each subclass has its own indecomposibles,
$E_k$ in the case of graphs.  Again, 
indecomposibles in different subclasses 
share matroids as do
the $E_k$ all of whose matroids are $\emptyset$.
When $P\neq 0$, the indecomposibles of different 
subclasses might or might not share matroids
or oriented matroids.

The ZBR theorem for graphs in \cite{Ellis-Monaghan-Traldi}
has conditions analogous to those in Corollary \ref{ZBRmatroids},
except the factor $\alpha$ is replaced by $\alpha_k=I(E_k)$
depending on the subfamily.  

$P$-ported matroids or oriented matroids can be combined 
by matroid direct sum $\oplus$.  Graphs can be combined
by disjoint union $\amalg$ or by a one-point union;
then each such combination $G$ of $G^1$ and $G^2$, if defined,
satisfies $M(G)=M(G^1)\oplus M(G^2)$.  
Zaslavsky's definition\cite{MR93a:05047} for 
matroids is immediately extended:


\begin{definition}
A \textbf{strong} $P$-ported Tutte function $T$ on a $P$-family 
$\mathcal{C}$ of matroids
or oriented matroids satisfies $T(M^1)T(M^2)=T(M^1\oplus M^2)$
when $M^1, M^2$ and $M^1\oplus M^2$ are all in $\mathcal{C}$.
\end{definition}

Note that such a strong Tutte function is a separator-strong
Tutte function with $X_e=T(U^e_1)$ and
$Y_e=T(U^e_0)$ for all $e\in E(\mathcal{N})$.

We will give extensions of definitions of strong Tutte functions
and of multiplicative Tutte functions of graphs below when
we define $P$-families of objects with matroids or oriented 
matroids.



\section{Objects with Matroids or Oriented Matroids}

It is useful to think that a $P$-ported Tutte computation tree may have
objects $N$ for its node labels such as graphs.
Each object $N$ has an associated
a $P$-ported matroid or oriented matroid $M(N)$.
Elements are defined $S(N)=S(M(N))$, 
each $p\in S(N)\cap P$ is called a port, and
$E(N)=S(N)\setminus P$.  Loops, coloops and non-separators of $N$
are characterized by their status in $M(N)$.  So we
say $N$ is an \emph{object with a matroid or an oriented matroid}.
Often, but not always, $N$ will be some matroid or oriented
matroid representation.


Contraction $N/e$ and deletion $N\setminus e$ 
of object $N$ are defined when $e\in E(N)$, and
$e$ is not a coloop in $M(N)$ and 
$e$ is not a loop in $M(N)$, respectively.
Under those conditions, $M(N/e)=M(N)/e$ and 
$M(N\setminus e)=M(N)\setminus e$ (as matroids or
oriented matroids).
Thus $P$-minors are defined,
and an indecomposible or $P$-quotient
is a $P$-minor $Q$ for which $S(Q)=S(M(Q))\subseteq P$.

\begin{definition}
\label{OMOMdef}
An $P$-ported object $N$ with a matroid or oriented matroid 
is described above together with $M(N)$, $E(N)$, $S(N)$,
$P$-minors, etc.
A \textbf{$P$-family of objects $\mathcal{N}$}
is a $P$-minor closed class of 
objects with matroid or oriented matroids.
\end{definition}

Tutte computation trees are defined for such $N$.
The matroid $M(N)$ of course
constrains the structure of these trees.  
It is possible
(as when the edgeless graphs $G_k$ have different vertex sets but all
$M(G_k)=\emptyset$)
for different objects, 
even different indecomposibles, to have the same 
matroid or oriented matroid.  It also natually occurs 
that
$N/e\setminus f$ $\neq$ 
$N/f\setminus e$ (as objects) even though 
$M(N)/e\setminus f$ $=$ 
$M(N)/f\setminus e$.  The latter equation when 
$e,f$ are in parallel or in series (see Proposition
\ref{SameMinorProp}) is critical to the matroid ZBR theorems.
It is also conceivable that $N/e/f\neq N/f/e$ or
$N\setminus e\setminus f\neq N\setminus f\setminus e$.

Since every $P$-minor $N'$ of $N$ has matroid
or oriented matroid $M(N')$ the same as the corresponding
minor of $M(N)$, we observe:

\begin{lem}
\label{ObjectTreeValueLemma}
The Tutte computation trees for $M(N)$ are in a one-to-one
correspondance with the 
Tutte computation trees for $N$ 
where corresponding trees are isomorphic.  In each
isomorphism,
corresponding
branches have the same labels ``$e$-contracted'' or
``$e$-deleted'' with $e\in E(N)=E(M(N))$, and a node
labelled $N_i$ in the tree for $N$ corresponds to 
a node labelled $M(N_i)$ in the tree for $M(N)$.

Each computation tree value is given by
the activities expansion \eqref{PAE} reinterpreted for
objects.
\end{lem}


We can still talk about Tutte decompositions and a
Tutte computation tree for $N$ even  without 
a Tutte function.  If we are given values $I(Q)$
for the indecomposibles, each Tutte computation tree 
for $N$ yields a value in the $R$-module generated by
the $I(Q)$.
The Tutte decompositions, and the universal
Tutte polynomial (if it exists!)
of each $N\in \mathcal{N}$
are determined by $M(N)$ and the indecomposibles, i.e., $P$-quotients
$Q$ in $N$, which of course satisfy $Q \in \mathcal{N}$.
This generalizes Zaslavsky's discussion\cite{MR93a:05047}.

\begin{definition}[Separator-strong $P$-ported Tutte function on objects]

Function
$T$ on $\mathcal{N}$
is a $P$-ported separator-strong Tutte function
on $\mathcal{N}$ into 
the ring $R$ 
containing parameters $x_e, y_e, X_e, Y_e$,
or an $R$-module containing the initial values
$T(Q)=I(Q)$ for indecomposibles,
when for all $N\in\mathcal{N}$,
\eqref{TA} and \eqref{TSSM} are
satisfied for each
$e\in E(N)$.
\end{definition}

Therefore:

\begin{prop}
$T$ is a $P$-ported separator-strong Tutte function
on $\mathcal{N}$ if and only if for each $N\in\mathcal{N}$,
all Tutte computation trees for $T(N)$ yield polynomial expressions
that are equal in the range ring or $R$-module.
\end{prop}

We develop our first ZBR-type theorem for $P$-ported objects
with matroids or oriented matroids.  It is the
generalization of the ZBR theorem for graphs as given
by Ellis-Monaghan and Traldi\cite{Ellis-Monaghan-Traldi}.  
It depends on a lemma similar to one of theirs.


\begin{lem}
\label{DisjSubclassLem}
Suppose $P$-family $\mathcal{N}$ is partitioned into
disjoint $P$-minor closed subfamilies $\{\mathcal{N}_{\pi}\}$.
Then $T$ is a Tutte function on $\mathcal{N}$ if and
only if $T$ restricted to $\mathcal{N}_{\pi}$ is
a Tutte function for each $\mathcal{N}_{\pi}$.
\end{lem}

\begin{thm}
\label{ZBRWildFamily}
Suppose $P$-family $\mathcal{N}$ is partitioned into
disjoint $P$-minor closed subfamilies $\{\mathcal{N}_{\pi}\}$,
and each initial value $I(Q)$ depends only on
the matroid or oriented matroid $M(Q)$ and on the
$\pi$ for which $Q\in\mathcal{N}_{\pi}$,

Then $T$ is a Tutte function with given parameters $(x,y,X,Y)$
and initial values $I(Q)$ if and only if it satisfies
the equations of Theorem \ref{BigTheorem}, interpreted
for families of objects with matroids or oriented matroids.
\end{thm}

\begin{proof}
As in \cite{Ellis-Monaghan-Traldi}, lemma
\ref{DisjSubclassLem} lets us prove the
theorem for each $\pi$ separately.


By Lemma \ref{ObjectTreeValueLemma}, $T$ is a 
$P$-ported Tutte function of family
of objects $\mathcal{N}_{\pi}$ if and only
if function $T'(M(N))=T(N)$ on the $P$-family 
$\mathcal{C}^{\pi}=\{M(N)\mid N\in\mathcal{N}_{\pi}\}$ is a $P$-ported Tutte 
function, since by hypothesis 
$I'(M(Q))=I(Q)=I(M(Q))$ for corresponding
indecomposibles $Q\in\mathcal{N}_{\pi}$ and 
$M(Q)\in\mathcal{C}^{\pi}$.
The conclusion follows
from Theorem \ref{BigTheorem} applied to $\mathcal{C}^{\pi}$.
\end{proof}


Ellis-Monaghan and Traldi's ZBR theorem for graphs 
refers to one initial value $\mathcal{\alpha}_k=I(E_k)$
for each non-empty subclass of graphs, with unlabelled vertices, that
have $k$ graph components.  One natural ported generalization
is to partition the $P$-ported graphs $G$
according to (1) how many
graph components $k$, (2) $P'=P\cap S(G)$ and (3)
$\nu:P'\rightarrow\{1,\ldots,k\}$, where $\nu(p)$ is which
component contains edge $p$.  Theorem
\ref{ZBRWildFamily} tells us:

\begin{cor}
\label{PZBRGraphCor1}

Let a $P$-minor closed collection $\mathcal{G}$ of graphs with unlabelled
vertices be partitioned into $\mathcal{G}_{k,P',\nu}$.  Suppose initial
values $I(G)=I_{k,P',\nu}(M(G))$ are given that depend only on the 
part and the matroid or oriented matroid of $G\in\mathcal{G}_{k,P',\nu}$.
Then there is $T$, 
a $P$-ported separator-strong parametrized Tutte function
of graphs $\mathcal{G}$ satisfying $T(Q)=I(Q)=I_{k,P',\nu}(M(Q))$ whenever
$P$-quotient $Q\in\mathcal{G}_{k,P',\nu}$ if and only if the identities
of Theorem \ref{BigTheorem}, interpreted for graphs, are satisfied with 
the given $I(Q)$.
\end{cor}

The next ZBR-type theorem  addresses the problem 
illustrated by equation \eqref{BadZBRExampleEq}. It requires that
the $P$-family satisfy the following:

\begin{definition}
Object $N\in\mathcal{N}$ is \textbf{well-behaved} when
for every independent set $C\subseteq E(N)$ and
coindependent set $D\subseteq E(N)$ for which
$C\cap D=\emptyset$, each of the
$|C\dunion D|!$ orders 
of contracting $C$ and deleting $D$ produces the
same $P$-minor (which is an object) of $N$.

Specifically, let 
$C=\{c_1,\ldots,c_j\}$,
$D=\{d_{j+1},\ldots,d_k\}$
and $R_i(N')=N'/c_i$ if $1\le i \le j$
and $N'\setminus d_i$ if $j+1\le i \le k$.
The condition is 
$R_1\circ\cdots\circ R_k(N) = R_{\sigma_1}\circ\cdots\circ R_{\sigma_k}(N)$
for every permutation $\sigma$ of $\{1,\ldots,k\}$.

$\mathcal{N}$ is \textbf{well-behaved} when 
each $N\in\mathcal{N}$ is well-behaved.

\end{definition}

By definition \ref{OMOMdef} 
all the minors are defined and
$M(R_1\circ\cdots\circ R_k(N)) = M(R_{\sigma_1}\circ\cdots\circ R_{\sigma_k}(N))$
independently of whether $N$ is well-behaved or not.
The point is that the objects themselves are the same.
We give two examples.

\begin{definition}[Graphs with set-labelled vertices]
\label{GSLVDefinition}
The elements of such a graph $S(G)=E(G)\dunion(P\cap S(G))$ are edges.
The vertices are labelled with non-empty
finite sets so the two sets labelling distinct vertices in one
graph are disjoint.  Only non-loop edges $e\not\in P$
can be contracted; when an edge is contracted, its two endpoints
are replaced by one vertex whose label is the union of the 
labels of the two endpoints.  Only non-isthmus edges $e\not\in P$
can be deleted; deletion doesn't change labels.  The graph
has its graphic matroid if it is undirected and its
oriented graphic matroid if it is directed.
\end{definition}

A graph with set-labelled vertices is well-behaved because
the minor obtained  by contracting forest $C$
and deleting $D$ is determined by merging all the vertex labels
of each graph component of $C$ and removing edges $C\cup D$.  
The deletions do not affect the vertex labels.  Hence
the vertex labels are not affected by the order
of the operations.


\begin{definition}[Graphs with set-labelled components]
\label{GSLCDefinition}
The elements of such a graph $S(G)=E(G)\dunion(P\cap S(G))$ are edges.
The path-connected components are labelled by
non-empty finite sets so two components in the same
graph always have disjoint labels.  In other words,
the set labels of the components are a partition $\pi_V$.
Only non-loop edges $e\not\in P$
can be contracted and only non-isthmus edges 
$e\not\in P$
can be deleted.
The component labels are unchanged by these minor operations.
Definition \ref{GSLVDefinition} specifies the
matroids or oriented matroids.
\end{definition}


A non-well-behaved $P$-family $\mathcal{C}!$
can be constructed from
any $P$-family of matroids $\mathcal{C}$ with
some $M\in\mathcal{C}$ with $|E(M)|\geq 2$.  
Each member of $\mathcal{C}!$ is formed from
some $M\in\mathcal{C}$ together with some
history of deletions and contractions that
can be applied to $M$.  Let $c_e$ and $d_e$ be symbols
for contracting and deleting $e\in E(\mathcal{C})$
respectively; a history $h$ is a string of
such symbols.  
Let $M|h$ be the $P$-minor obtained by 
performing history $h$ on $M$, assuming each step
is defined.  The objects of $\mathcal{C}!$
are all pairs $(M,h)$ for which 
$P$-minor $M|h\in\mathcal{C}$ is defined.
The matroid of $(M,h)$ is $M|h$, which
determines the element set, loops and coloops.
If $e\in E(M|h)$ is not
a loop, then define $(M,h)/e=(M,hc_e)$.
Similarly, if $e\in E(M|h)$ is not
a coloop, $(M,h)\setminus e=(M,hd_e)$.

The point of this example is that even if
the $P$-family is not well-behaved and so the
indecomposibles do carry information about their
history, Theorem \ref{ZBRWildFamily} tells 
us that the Tutte function
is still well defined if the initial values depend only
on the matroid, or the oriented matroid, of the 
indecomposible. (See Questions, \ref{Qsubsection}.)


The examples forced us to recognize that
for $N$ an object with a matroid or oriented matroid $M(N)$
with
${e,f}\in E(N)$ in series or in parallel,  it might
happen that $N/e\setminus f\neq N/f\setminus e$ even though,
by Proposition \ref{SameMinorProp}, 
$M(N)/e\setminus f= M(N)/f\setminus e$.  Note that 
Proposition \ref{SameMinorProp} is about 
$\cdot/e\setminus f$ and 
$\cdot/f\setminus e$ which are not commutations
of the same two operations.

\begin{thm}[ZBR Theorem for well-behaved 
$P$-families of objects with matroids or
oriented matroids]
\label{ZBRWellBehaved}

Supposed $\mathcal{N}$ is well-behaved.
The following two statements are equivalent.
\begin{enumerate}
\item $T$ from $\mathcal{N}$ to $R$ or an $R$-module is a $P$-ported 
separator-strong parametrized
$P$-ported Tutte function with $R$-parameters $(x, y, X, Y)$ whose values 
$T(Q)$ on $P$-quotients $Q\in\mathcal{N}$ are the initial
values $I(Q)$.
\item
For every $N\in\mathcal{N}$:
\begin{enumerate}
\item 
If $M(N)=U^{ef}_1\oplus M(Q)=U^{ef}_1\oplus M(Q')$ with 
$P$-quotients $Q=N/e\setminus f$ and $Q'=N/f\setminus e$,
\[
I(Q)(x_e Y_f - y_f X_e ) = 
I(Q')(x_f Y_e - y_e X_f).
\]
\item
If $M(N)=U^{efg}_2\oplus M(Q)=U^{efg}_2\oplus M(Q')$ with 
$P$-quotients $Q=N/e\setminus f/g$ 
and $Q'=N/f\setminus e/g$,
\[
I(Q)X_g(x_e y_f - y_f X_e ) = 
I(Q')X_g(x_f y_e - y_e X_f ).
\]
\item
If $M(N)=U^{efg}_1\oplus M(Q)=U^{efg}_1\oplus M(Q')$ with 
$P$-quotients $Q=N/e\setminus f\setminus g$  
and $Q'=N/f\setminus e\setminus g$,
\[
I(Q)Y_g(x_e Y_f - y_f x_e) = 
I(Q')Y_g(x_f Y_e - y_e x_f).
\]
\item
If $\{e,f\}=E(M(N))$ is a parallel pair connected to $P$, 
\[
I(Q)(x_e Y_f - y_f x_e) = 
I(Q')(x_f Y_e - y_e x_f)
\]
where $P$-quotients $Q=N/e\setminus f$
and $Q'=N/f\setminus e$.
\item
If $\{e,f\}=E(M(N))$ is a series pair connected to $P$, 
\[
I(Q)(x_e y_f - y_f X_e) = 
I(Q')(x_f y_e - y_e X_f)
\]
where $P$-quotients
$Q=N/e\setminus f$ 
and $Q'=N/f\setminus e$.
\end{enumerate}
\end{enumerate}
\end{thm}

\begin{proof}
All the steps of the proof given for 
Theorem \ref{BigTheorem} can be adapted.
The hypothesis the $\mathcal{N}$ is well-behaved
means that $N'/e/f = N'/f/e$, $N'/e\setminus f = N'\setminus f/e$, etc.,
(equalities of objects) are true respectively whenever a step of the
form 
$M'/e/f = M'/f/e$, $M'/e\setminus f = M'\setminus f/e$, etc.,
respectively occurs for matroids or oriented matroids.

The calculations done in all cases of Theorem \ref{BigTheorem}
of where 
Proposition \ref{SameMinorProp} is applied to write
$Q = M'/e\setminus f = M'/f\setminus e$ when $e,f$ are in
series or parallel are replaced by 
$Q = N'/e\setminus f$ and $Q' = N'/f\setminus e$.  
Matroids or oriented matroids are replaced
by objects with matroids or oriented matroids in 
the calculations.
Notes to guide the reader appear in the proof.
\end{proof}



\begin{cor}
\label{IVEqualSerParCor}
A $P$-family of objects with matroids satisfies a ZBR-type theorem
with the identities given in Theorem \ref{BigTheorem} if, 
in 
addition $\mathcal{N}$ being well-behaved,
the initial values $I$ satisfy
$I(N/e\setminus f)=I(N/f\setminus e)$ when $\{e,f\}=E(N)$ is a series
or parallel pair,
$I(N/e\setminus f/g)=I(N/f\setminus e/g)$ when $\{e,f,g\}=E(N)$ is a 
triangle and 
$I(N/e\setminus f\setminus g)=I(N/f\setminus e\setminus g)$ 
when $\{e,f,g\}=E(N)$ 
is a triad.
\end{cor}


\begin{cor}
A $P$-family of objects with matroids satisfies a ZBR-type theorem
with the identities given in Theorem \ref{BigTheorem} if, in 
addition to $\mathcal{N}$ being well-behaved,
the object $P$-quotients 
$N/e\setminus f=N/f\setminus e$ when $\{e,f\}=E(N)$ is a series
or parallel pair,
$N/e\setminus f/g=N/f\setminus e/g$ when $\{e,f,g\}=E(N)$ is a 
triangle and 
$N/e\setminus f\setminus g=N/f\setminus e\setminus g$ when $\{e,f,g\}=E(N)$ 
is a triad.
\end{cor}

\begin{proof}
Clearly, if $N/e\setminus f=N/f\setminus e$ then 
$I(N/e\setminus f)=I(N/f\setminus e)$, etc.
\end{proof}

We conclude with second ported generalization of 
Ellis-Monaghan and Traldi's ZBR theorem for graphs,
besides Corollary \ref{PZBRGraphCor1}.

\begin{cor}
\label{PZBRGraphCor2}
Let $\mathcal{G}$ be a ported $P$-family of graphs with
unlabelled vertices, as in Corollary \ref{PZBRGraphCor1}.
Then there is $T$, $P$-ported separator-strong parametrized Tutte function
of graphs $\mathcal{G}$ satisfying $T(Q)=I(Q)$ 
for all $P$-quotients $Q\in\mathcal{G}$ if and only if 
for every $G\in\mathcal{G}$, each case applies:
\begin{enumerate}
\item[a-d]
Revise the corresponding case of Theorem \ref{ZBRWellBehaved}
by writing $Q'=Q$, i.e., replace $Q'$ by $Q$.
\item[e]
If $E(G)=\{e,f\}$ is a series pair connected to $P$, 
\[
I(Q)(x_e y_f - y_f X_e) = 
I(Q')(x_f y_e - y_e X_f)
\]
where $P$-quotients
$Q=G/e\setminus f$ 
and $Q'=G/f\setminus e$.  (This is case (e) of Theorem \ref{ZBRWellBehaved}
\emph{verbatim}.)
\end{enumerate}
\end{cor}

\begin{proof}
$\mathcal{G}$ is a well-behaved $P$-family of ported objects
with matroids or oriented matroids, so Theorem \ref{ZBRWellBehaved}
applies.
In all cases of Theorem \ref{ZBRWellBehaved} but the
last, the two object minors are the same graph 
because the contracted edges are path-connected,
so the equations of Theorem \ref{ZBRWellBehaved} are simplified.
\end{proof}

Both Corollaries \ref{PZBRGraphCor1} and \ref{PZBRGraphCor2} reduce
to the ZBR theorem for graphs when $P=\emptyset$.  The first
uses the property that $\mathcal{G}$ is partitioned into
minor-closed subclasses with indecomposibles $E_k$ 
for which the
initial values depend only on the matroid or oriented matroid to generalize
the original ZBR equations.  As with matroids, we find again that
the Tutte functions can distinguish different orientations of the 
the same undirected graphs.
The second relies on the 
commutivity of the graph minor operations and generalizes
the fact that different initial values may be assigned to
different indecomposibles, but then the 
conditions sufficient for the initial values to
extend to a Tutte function must be stronger.





\section{Direct and Other Sums}

\label{DirectSec}

It is a common situation that $\{N^1,N^2,N\}\subseteq\mathcal{N}$
and their matroids or oriented matroids 
satisfy $M(N^1)\oplus M(N^2)=M(N)$.
Tutte computation trees help.  The proposition below applies
even to non-well-behaved $N$ when the symbols
$/B^j_i|P_j$ refer to sequences of deletions and contractions.

\begin{definition}
If $\mathcal{T}_1$ and $\mathcal{T}_2$ are Tutte computation trees then
$\mathcal{T}_1\cdot \mathcal{T}_2$ is the tree obtained by appending
a separate copy of $\mathcal{T}_2$ at each leaf of $\mathcal{T}_1$.
The root is the root of the expanded $\mathcal{T}_1$.
\end{definition}

\begin{prop}
\label{SumProp2}

Suppose $N$, $N^1$ and $N^2$ are all in $\mathcal{N}$
and $M(N^1)\oplus M(N^2)=M(N)$.  Then if
$\mathcal{T}_1$ and $\mathcal{T}_2$ are Tutte computation
trees for $N^1$ and $N^2$ respectively with values given
by \eqref{DS1} and \eqref{DS2}, then there is a Tutte
computation tree for $N$ that yields the value given
by \eqref{DS}.

\begin{equation}
\label{DS1}
\tag{DS1}
\sum_{Q^1_i}I(Q^1_{i})c_1(Q^1_{i})\text{ where }Q^1_i=N/B^1_i|P_1.
\end{equation}
\begin{equation}
\label{DS2}
\tag{DS2}
\sum_{Q^2_j}I(Q^2_{j})c_2(Q^2_{j})\text{ where }Q^2_j=N/B^2_j|P_2.
\end{equation}
\begin{equation}
\tag{DS}
\label{DS}
\sum_{Q^1_i,Q^2_j}I(Q_{i,j})c_1(Q^1_{i})c_2(Q^2_{j})
\text{ where }Q_{i,j}=N/B^1_i|(P_1\cup S(N^2))/B^2_j|P_2.
\end{equation}

Furthermore, if $T$ is a Tutte function on $\mathcal{N}$
and $T(N^1)$ and $T(N^2)$ equal the Tutte polynomials
given by \eqref{DS1} and \eqref{DS2} then 
$T(N)$ equals the polynomial given by \eqref{DS}.
\end{prop}

\begin{proof}

We show how to relabel $\mathcal{T}_1\cdot\mathcal{T}_2$ to obtain 
a Tutte computation tree for $N$.  $M(N^1)\oplus M(N^2)=M(N)$ is defined
means $S(M(N^1))\cap S(M(N^2))=\emptyset$ and 
$S(M(N))=S(M(N^1))\cup S(M(N^2))$.   Each node of 
$\mathcal{T}_1\cdot\mathcal{T}_2$ is determined by 
by deleting and/or contracting some elements of
$E(M(N^1))\cup E(M(N^2))$.  Relabel that node with the 
$P$-minor of $N$ obtaining deleting and/or contracting the
same elements respectively in the same order, those in $N^1$
preceding those in $N^2$.  The result is a computation tree for $N$
because $M(N^1)\oplus M(N^2)=M(N)$.  Assume $P\subseteq S(M(N))$
(otherwise, take a smaller $P$) and let $P^1=S(M(N^1))\cap P$
and $P^2=S(M(N^2))\cap P$.
At a leaf of the relabelled
tree, there will be the $P$-quotient $N/B_1/B_2|P$ where
$B_1$ is a $P^1$-subbasis of $M(N^1)$ and $B_2$ is a $P^2$-subbasis
of $M(N^2)$.  

\end{proof}

\subsection{Strong Tutte Functions}

Let us extend the definition of strong parametrized Tutte function
to $P$-families $\mathcal{N}$ of objects with matroids and oriented matroids,
in the way that abstracts the known notion of strong Tutte functions
on minor closed families of graphs\cite{Ellis-Monaghan-Traldi}.
We can then specialize 
$\mathcal{N}$ to $\mathcal{C}$, a $P$-family  of 
matroids or oriented matroids.
There might still be indecomposibles besides or instead of $\emptyset$.

\begin{definition}
A $P$-ported separator-strong Tutte function $T$ on a $P$-family of 
objects $\mathcal{N}$
with matroids is called \textbf{strong} if
whenever $\{N^1, N^2, N\}\subseteq\mathcal{N}$
and 
$M(N^1)\oplus M(N^2)=M(N)$, 
then $T(N^1)T(N^2)=T(N)$.
\end{definition}

We now generalize to $P\neq\emptyset$
the $T(\emptyset)T(\emptyset)=T(\emptyset)$ 
characterization of strong Tutte functions.

\begin{thm}
\label{StrongTheorem}
A $P$-ported separator-strong Tutte function $T$ on a $P$-family of 
objects 
with matroids or oriented matroids $\mathcal{N}$ 
is strong if and only if
$T$ restricted to the indecomposibles of 
$\mathcal{N}$ is strong; i.e.,
whenever $Q^1$, $Q^2$ and 
$Q$ are indecomposibles and $M(Q^1)\oplus M(Q^2)=M(Q)$ then
$T(Q^1)T(Q^2)=T(Q)$.
\end{thm}

\begin{proof}
Every $P$-quotient is in $\mathcal{N}$, so clearly $T$ restricted
to the $P$-quotients is strong.

Conversely, suppose $N^1$, $N^2$ and $N$ are in $\mathcal{N}$ and
$M(N^1)\oplus M(N^2)=M(N)$, so Proposition \ref{SumProp2} applies.

Since $M(N^1)\oplus M(N^2)=M(N)$, 
$M(N/(B_1\cup B_2)|P)$ $=$ $(M(N^1)/B_1|P)\oplus (M(N^2)/B_2|P)$
$=$ $M(N^1/B_1|P)\oplus M(N^2/B_2|P)$.  
We now use the fact
that $Q_{ij}=N/B_1/B_2|P$, $Q^1_i=N^1/B_1|P$ and $Q^2_j=N^2/B_2|P$ are 
$P$-quotients and the hypothesis to
write $I(Q_{ij})=I(Q^1_i)I(Q^2_j)$.  


We therefore conclude $T(N)=T(N^1)T(N^2)$
from \eqref{DS1}, \eqref{DS2} and  \eqref{DS}.

\end{proof}

\subsection{Multiplicative Tutte Functions}

Graphs can be combined in several ways all so
the matroid of the combination is the direct
sum of the matroids of the parts. This motivates:

\begin{definition}
A partially defined binary operation ``$*$'' on
a $P$-family of objects with matroids or oriented matroids
$\mathcal{N}$ is \textbf{a matroidal direct sum} if 
whenever $N^1*N^2\in\mathcal{N}$ is defined for
$\{N^1,N^2\}\subseteq\mathcal{N}$, the 
matroids or oriented matroids satisfy
$M(N^1)\oplus M(N^2) = M(N^1*N^2)$.
\end{definition}

Proposition \ref{SumProp2} applies
when $N^1*N^2=N$ is defined.
It gives a general recipe for $T(N^1*N^2)$ which generalizes
the identity \cite{Ellis-Monaghan-Traldi}
$T(M^1\oplus M^2)T(\emptyset)=T(M^1)T(M^2)$ 
for separator-strong Tutte functions of
matroids.
The $P$-ported generalization is more complicated and generally cannot
be expressed by a product in the domain ring of $T$.  

\begin{prop}
\label{SumProp}
Suppose $*$ is a matroidal direct sum and
$N^1$, $N^2$ and $N^1*N^2$ are each members of a $P$-family
for which $T$ is a Tutte function.

If for $P$-quotients $Q^j_i$ and $R$-coefficients $c_j(Q^j_i)$, $j=1$ and $2$,
\begin{equation*}
%\label{MD1}
%\tag{MD1}
T(N^1) = \sum_{Q^1_i}T(Q^1_{i})c_1(Q^1_{i})
\end{equation*}
and
\begin{equation*}
%\label{MD2}
%\tag{MD2}
T(N^2) = \sum_{Q^2_j}T(Q^2_{i})c_2(Q^2_{j})
\end{equation*}
then
\begin{equation*}
%\label{MD}
%\tag{MD}
T(N^1 * N^2) = \sum_{Q^1_i,Q^2_j}T(Q^1_{i}*Q^2_{j})c_1(Q^1_{i})c_2(Q^2_{j}).
\end{equation*}
\end{prop}

\begin{proof}
Substitute $Q_{i,j}=Q^1_i*Q^2_j$ in \eqref{DS} of Proposition \ref{SumProp2}.
\end{proof}


When $\mathcal{N}$ is a $P$-family of matroids or oriented matroids,
direct matroid or oriented matroid sum is obviously a matroidal direct
sum operation, and so Proposition \ref{SumProp} is applicable.

\begin{cor}\cite{Ellis-Monaghan-Traldi}
Let $P=\emptyset$.
$T(M^1\oplus M^2)T(\emptyset) = T(M^1)T(M^2)$ for Tutte
function $T$ of matroids.
\end{cor}

\begin{proof}
Our proof demonstrates how Proposition \ref{SumProp}
generalizes this formula to $P$-families.
The expansions \ref{DS1} and \ref{DS2} take the one-term form
$T(M^j) = T(\emptyset) c_j(\emptyset)$, $j$ $=$ $1,2$,
so $T(M^1)T(M^2)=T(\emptyset)^2 c_1(\emptyset) c_2(\emptyset)$.
Expansion \ref{DS} is then
$T(M^1\oplus T^2)=T(\emptyset) c_1(\emptyset) c_2(\emptyset)$.
\end{proof}

Following the definitions for graphs in \cite{Ellis-Monaghan-Traldi}, we write:

\begin{definition}
Given a matroidal direct sum $*$ on $\mathcal{N}$, 
a Tutte function $T$ on $\mathcal{N}$ is
\textbf{multiplicative} (with respect to ``$*$'')
if whenever $N^1*N^2$ is defined
for $\{N^1,N^2\}\subseteq \mathcal{N}$, the Tutte
function values satisfy 
$T(N^1)T(N^2)=T(N^1*T^2)$.
\end{definition}

A strong Tutte function is certainly multiplicative
for any ``$*$'', but not conversely.

\begin{cor}
A $P$-ported Tutte function $T$ on $P$-family $\mathcal{N}$ 
is multiplicative with respect to matroidal direct product
``$*$'' if and only if 
for every pair of indecomposibles $\{Q_i, Q_j\}\in\mathcal{N}$ for which 
$Q_i*Q_j\in\mathcal{N}$ is defined, $T(Q_i)T(Q_j)=T(Q_i*Q_j)$.
\end{cor}

\begin{proof} 
When $N^1*N^2$ is defined, Proposition \ref{SumProp2}
applies because $M(N^1)\oplus  M(N^2)=M(N^1*N^2)$.
$T(N^1*N^2)=T(N^1)T(N^2)$ is then a consequence
of $T(Q_i*Q_j)=T(Q_i)T(Q_j)$.
\end{proof}

\begin{cor}
Suppose $T$ is parametrized Tutte function on a minor-closed class
of graphs $\mathcal{G}$ with unlabelled vertices
and $P=\emptyset$.
$T$ is strong if and only if 
there is an idempotent $\alpha=\alpha^2\in R$ and $T(E_k)=\alpha$
whenever $\mathcal{G}_k\neq\emptyset$.
\end{cor}


\begin{proof}
$M(E_k)=\emptyset$ for all $k\geq 1$ and 
$\emptyset\oplus\emptyset=\emptyset$, so 
$T(E_i)T(E_j)=T(E_k)$ whenever $\mathcal{G}_i$,
$\mathcal{G}_j$ and 
$\mathcal{G}_k$ are all non-empty.  Hence, if
$\mathcal{G}_k\neq\emptyset$ then $T(E_k)T(E_k)=T(E_k)=\alpha$.
Further, if $\mathcal{G}_j\neq\emptyset$ with $j\neq k$,
$\alpha=T(E_k)=T(E_k)T(E_k)=T(E_j)$.
\end{proof}

This
strengthens part of Corollary 3.13 in \cite{Ellis-Monaghan-Traldi}.
The other part does require its 
additional conditions to establish strongness of 
a Tutte function on graphs given as multiplicative on
both disjoint union $\amalg$ and one-point unions.
Consider $\mathcal{N}=\{E_3, E_4, E_5,  \ldots \}$,
$T(E_k)=1$ for $k\geq 3$, $k\neq 5$ and 
$T(E_5)=0$.  $T(E_3)T(E_4)=1\neq T(E_5)$, so 
$T$ is not strong, but $T$ is multiplicative
on disjoint and one-point unions because $E_5$ cannot be expressed
as either kind of union of graphs in $\mathcal{G}$.  The other conditions
are that $\mathcal{G}_k\neq\emptyset$ for all $k$ and that 
$\mathcal{G}$ is closed under one-point unions and removal of isolated
vertices.

\part{Remarks, Background, Problems, and References}

\section{Non-commutativity}
We presented in  \cite{TutteEx}
a new kind of strong Tutte-like function on $P$-ported oriented graphic
matroids (more generally, unimodular, i.e. regular oriented matroids)
whose values vary with the orientation.   
Each function value $F(G)$ is
in the exterior algebra over $R^{2p}$, where $R$ is
the reals extended by the $x_e, y_e$ 
and $|P|=p$.  The function
obeys an \emph{anti-commutative} variant of \eqref{TSM}
with exterior multiplication $\wedge$.  
(When $P=\emptyset$, $F$ is
the reduced Laplacian determinant in the
famous Matrix Tree Theorem \cite{HararyBook,sdcMTT}.)  
It is the first example we know
of
``the possibility of making use of a noncommutative generalization of the Tutte
polynomial at some point in the future.'' mentioned by Bollob\'{a}s and
Riordan in \cite{BollobasRiordanTuttePolyColored}.  We won't say
more beyond that (1) each of the $\binom{2p}{p}$
Pl\"{u}cker coodinates of $F(G)$ is a $P$-ported Tutte function
of the kind we covered; and (2) that quadratic inequalities among
some of them express negative correlation between edges in spanning trees,
results also known as 
Rayleigh's inequality\cite{ChoeRayleigh,SempleWelshNegCorr08,HalfPlaneStuff}.

A second non-commutative possibility might be found in 
section \ref{DirectSec}.
The abstraction applies to situations where
an object represents an initial matroid, graph, etc. plus
a history of deletions and contractions.  
The key feature is that the Tutte decompositions
of an object are identical to the Tutte decompositions
of that object's matroid.
The initial value
on indecomposible object $Q$
might then depend on the order of the deletion and contractions
reductions to 
to obtain it from $N$.  This helps us understand the
theory, but whether objects with minors that depend on reduction 
order have useful applications remains to be seen.

\section{Background and Other Related Work}
\label{BackgroundSec}

Besides Brylawski's 
work,  another early appearance of Tutte decomposition
of a matroid or graph with a basepoint is \cite{SmithPatroids}.
Ellis-Monaghan and Traldi \cite{Ellis-Monaghan-Traldi}
explain that by leaving the reduction
by $e_0$ to last so $e_0$ is always contracted as a coloop or deleted
as a loop, the Tutte function value can be expressed by
$T(M) = (rX_{e_o} + sY_{e_0})T(\emptyset)$ were $r, s$ are 
not-necessarilly-unique elements in $R$.  As one application,
they 
give a formula for the parametrized Tutte polynomial
for the parallel connection across $e_0$
which generalized Brylawski's.
These $r,s$ appear in
the $P$-ported Tutte function expression 
$rT(U^{e_0}_1)$ $+$ $sT(U^{e_0}_0)$ when $P=\{e_0\}$.  They
are parametrized generalizations of the coefficients of
$z'$ and $x'$ in Brylawski's four variable Tutte polynomial.

Las Vergnas defined and gave basic properties of ``set-pointed'' Tutte
polynomials (with no parameters) and used them to study matroid perspectives.
The polynomial given in \cite{MR0419272,SetPointedLV} has a variable
$\xi_l$ for each subset in a collection of $k$ subsets
$P_l\subseteq P$, $l=1,\ldots,k$.  Each term in \eqref{PAE} had
$\prod\xi_l^{r_i(P_l)}$ for $[Q_i]$ where $r_i$ is the rank function 
of matroid $Q_i$.  Therefore \eqref{TSM} was satisfied and the association of the
term to (non-oriented) $Q_i$ could be assured by taking all
$2^{|P|}$ subsets for the $P_l$.  The matroid perspective is the 
strong map $M\setminus E(M)\rightarrow M/E(M)$ given by the identity on
$P$.  

In \cite{sdcPorted}, we
reproduced Las Vergnes' theory with explicit $P$-quotient (matroid)
variables (see the $[Q_i]$ symbols in Corollary \ref{UniversalCor}
in sec. \ref{UniversalSec}) in place of 
$\prod\xi_l^{r_i(P_l)}$.  We then
gave formulas for the $P$-ported Tutte polynomial for the
union and its dual of matroids whose common elements are in $P$.
These formulas  work in a way similar to what appears in
sec. \ref{UniversalSec}.
We extend to algebras
the $\mathbb{Z}[u,w]$-module
generated by the $[Q_i]$ by defining multiplications 
$\tilde{*}$ with the rules
$[Q_i]\tilde{*}[Q_j]=r_{ij}[Q_{i,j}]$, with $r_{i,j}\in\mathbb{Z}[u,w]$ and
$Q_{i,j}=Q_i*Q_j$ depending on $(Q_i,Q_j)$ and whether $*$ represents 
union or its dual.  It is not often recognized that series and parallel
connection of matroids across basepoint $p$ 
is equivalent to matroid union and its dual on matroids with only
element $p$ in common.  We plan to investigate whether the formulas
for parametrized Tutte polynomials of parallel connections
in \cite{Ellis-Monaghan-Traldi} can be generalized to the dual
of union when $|P|>1$, and to detail the relationship when $|P|=1$.

Recent work on a different generalization, weak Tutte functions
(see sec. \ref{Complications}), has been done by 
Ellis-Monaghan and Zaslavsky \cite{ZaslavskyOct18}.
The distinction between weak Tutte functions (satisfying
an additive identity only)
and strong Tutte functions
(which satisfy \eqref{TSM} and \eqref{TA}) seems first to
have been made by Zaslavsky\cite{MR93a:05047}, for matroids.
That paper also defined weak and strong Tutte functions of
graphs.  However, we use using Ellis-Monaghan and Traldi's
definition of strong Tutte functions of 
graphs\cite{Ellis-Monaghan-Traldi}.
The latter restricts \eqref{TA} to non-separators (not just 
non-loops);  and it 
requires $T(G^1)T(G^2)=T(G)$ whenever 
matroids $M(G^1)\oplus M(G_2)=M(G)$ 
(which we sometimes interpret as oriented), not just when
$G$ is the disjoint union of $G^1$ and $G^2$.
The term separator-strong 
(I learned\cite{JoAndTom} 
after \cite{Ellis-Monaghan-Traldi} appeared.)
is used for Tutte functions of matroids and graphs 
as defined in \cite{Ellis-Monaghan-Traldi}; recall that
they satisfy \eqref{TA} and \eqref{TSSM}.
Normal is used in the same way as in\cite{MR93a:05047}.


We introduced $P$-ported parametrized Tutte polynomials
for normal strong Tutte functions
in \cite{TutteEx}, i.e., 
those with
corank-nullity polynomial expressions.
Most of the results in the current paper, when so restricted
appeared in \cite{TutteEx}
or can be derived by adding parameters and oriented matriod considerations
to material in \cite{sdcPorted}.  
These include computation tree\cite{GordonMcMachonGreedoid} based activities 
expansions 
with terms corresponding to $P$-subbases (which are called
``contracting sets'' in \cite{RelTuttePoly}).
%The one non-elementary fact about
%oriented matroids needed is that two subsets $B,B'\in E(M)$ that span
%the same flat define the same $\emph{oriented}$ minor
%$M/B|P=M/B'|P$.  See \cite{OMBOOK}.  
In the normal case, the initial values can be assigned arbitrarilly.
We used this to show the our extensor-valued Tutte-like function 
\cite{sdcPorted} is expressible
by assigning extensors as the initial values.  In this
electrical network application, the indecomposibles are oriented 
graphic matroids
and different values \emph{are} assigned to different orientations of
the same matroid.  

\subsection{Questions}
\label{Qsubsection}
Diao and Hetyei's work \cite{RelTuttePoly} is part of a
larger literature on knot invariants which
leads us to ask if the $P$-ported objects with matroids abstraction, 
and its related Tutte computation tree expansions for parametrized
Tutte functions (sec. \ref{DirectSec}), can be usefully
applied to objects besides graphs or directed graphs, such
as various kinds of knot diagrams.

Another open project is to classify the solutions to
the conditions of Theorem \ref{BigTheorem} (about separator-strong
$P$-ported Tutte functions) and 
Theorem  \ref{StrongTheorem} (about the strong ones) for rings and
for fields along the lines of 
\cite{MR93a:05047} and \cite{BollobasRiordanTuttePolyColored}.

Finally, the theory of objects with matroids or oriented
matroids compared with $P$-ported Tutte functions of 
oriented matroids
raises the question of what role orientation plays in 
$P$-ported Tutte
function theory.  An oriented matroid can be considered to
be an object with a non-oriented matroid.  In other
words, the orientation is a property of the object
that is constrained by, but not uniquely determined by 
the object's non-oriented 
matroid.  However, our results show that when the initial
values
depend only on the orientations in addition
to depending on the matroids, the Tutte function is 
characterized by 
the equations of Theorem \ref{BigTheorem}, not
Theorem \ref{ZBRWellBehaved}.
The reason is that when interchanging the elements
when deleting and contracting a series or parallel pair, 
both the resulting matroids and 
their orientations are the same.  The question we raise
is whether there are interesting objects with matroids
and certain other properties besides matroid orientation
such that Theorem \ref{BigTheorem}
characterizes the Tutte functions, when the initial
values depend only on the matroids and those other properties.

\section{Acknowledgements}

I thank
the Newton Institute for Mathematical Sciences
of Cambridge University for hospitality and support of my
participation in the Combinatorics and Statistical Mechanics
Programme, January to July 2008, during which some of this
work and many related subjects were reviewed and discussed.

I thank the organizers of 
the Thomas H. Brylawsky Memorial Conference, Mathematics Department of
The University of North Carolina, Chapel Hill, October 2008, 
Henry Crapo, Gary Gordon and James Oxley
for
fostering collaboration beween people who carry
on the memory of Prof. Brylawsky and editing this journal
issue.  

I thank Lorenzo Traldi for bringing to my attention and
discussing Diao and Hetyei's work, as well as
Joanne Ellis-Monaghan, Gary Gordon, Elizabeth McMahon
and Thomas Zaslavsky for helpful conversations and communications
at the Newton Institute and elsewhere, and
my University at Albany colleague Eliot Rich for mutual assistance with
writing.

This work is also supported by a Sabbatical leave granted
by the University at Albany, Sept. 2008 to Sept. 2009.



\bibliographystyle{abbrv}
\bibliography{ParamTutte}

\appendix

\begin{figure}
\input{Venn.pdf_t}
\caption{\label{Venn} Illustration for proof of Lemma \ref{KLAlemma}.}
\end{figure}



\section{Proof of Proposition \ref{GFlatProp}}

For each subset $A\subseteq E$, 
$[Q] = [M/A|P] = [M/F|P]$ is determined by the unique flat $F$
in $L(E)$ spanned by $A\subseteq E$.  So, we write \eqref{PGF}
by
\[
T^{\mathcal{C}}(M) = 
     \sum_{Q} [Q] \sum_{\substack{
                     F\in \mathcal{L}\\
                     [M/F|P]=[Q]
                      }}
      \sum_{\substack{
             A\subseteq F\\
             A\text{ spans }F
          }}
      x_A y_{E\setminus A}
      u^{r(M)-r(M/A\mid P)-r(A)}
      v^{|A|-r(A)}.
\]
Factoring, we get
\[
T^{\mathcal{C}}(M) = 
     \sum_{Q} [Q] \sum_{\substack{
                     F\in \mathcal{L}\\
                     [M/F|P]=[Q]
                      }}
      u^{r(M)-r(Q)-r(F)}
      v^{-r(F)}
      \sum_{\substack{
             A\subseteq F\\
             A\text{ spans }F
          }}
      x_A y_{E\setminus A}
      v^{|A|}.
\]
$A$ is summed over the spanning sets of $F$.  Let $Z(F)$ denote
this last sum.  Since every subset of $F$ spans some flat
in $\mathcal{L}$,
\[
\sum_{0\le G \le F}Z(G) = \sum_{e\in F}(y_e + x_ev).
\]
M\"{o}bius inversion gives
\[
Z(F)= \sum_{0\le G\le F}\mu(G,F)(y_e +  x_ev).
\]


\end{document}


