\documentclass[12pt,leqno]{amsart}




\usepackage{amsmath,amssymb,amsfonts,amsthm}
\usepackage{eucal,graphicx}
\usepackage{color}
\usepackage[pdftex]{hyperref}


\setlength{\textwidth}{6.5in}
\setlength{\oddsidemargin}{0.0in}
\setlength{\evensidemargin}{0.0in}
\setlength{\textheight}{9in}
\setlength{\topmargin}{-.4in}

%\renewcommand{\baselinestretch}{2}     % Activate for double spacing.
%\renewcommand{\baselinestretch}{1.6}   % Activate for 1-1/2 spacing.
%\renewcommand{\baselinestretch}{1.3}   % Activate for 1-1/3 spacing.


\newcommand \comment[1]{}			%  Silent version.
%\renewcommand \comment[1]{\emph{#1}}		%  Comment revealed.
\newcommand \dateadded[1]{\comment{[Date added: #1.]}}
\newcommand \mylabel[1]{\label{#1}\comment{{\rm \{#1\} }}}
\newcommand \myref[1]{\ref{#1}\comment{{\{#1\}}}}

\newtheorem{lem}{Lemma}
\newtheorem{cor}[lem]{Corollary}
\newtheorem{prop}[lem]{Proposition}
\newtheorem{thm}[lem]{Theorem}
\newtheorem{definition}[lem]{Definition}


\theoremstyle{remark}
\newtheorem{exam}{Example}%[section]
\newcommand \myexam[1]{\smallskip\begin{exam}[\emph{#1}]}

\renewcommand{\phi}{\varphi}
\newcommand\eset{\varnothing}
\newcommand\inv{^{-1}}
\newcommand\setm{\setminus}
\newcommand\chiz{\chi^\bbZ}
\newcommand\bbR{\mathbb{R}}
\newcommand\bbZ{\mathbb{Z}}
\newcommand\cH{\mathcal{H}}


\newcommand\Ueloop{\ensuremath{U^e_{0}}}
\newcommand\Uecoloop{\ensuremath{U^e_{1}}}
\newcommand\Uefdyad{\ensuremath{U^{ef}_{1}}}
\newcommand\Uefgtriad{\ensuremath{U^{efg}_{1}}}
\newcommand\Uefgtriangle{\ensuremath{U^{efg}_{2}}}

%   Disjoint Union
%\newcommand{\dunion}{\uplus}
\newcommand{\dunion}
%{\mbox{\hbox{\hskip4pt$\cdot$\hskip-4.62pt$\cup$\hskip2pt}}}
{\mbox{\hbox{\hskip6pt$\cdot$\hskip-5.50pt$\cup$\hskip2pt}}}
%
% Dot inside a cup.
% If there is a better, more Latex like way 
% (more invariant under font size changes) way,
% I'd like to know.


\newcommand{\Bases}[1]{\ensuremath{{\mathcal{B}}(#1)}}
\newcommand{\Reals}{\ensuremath{\mathbb{R}}}
\newcommand{\FieldK}{\ensuremath{K}}
\newcommand{\Perms}{\ensuremath{\mathfrak{S}}}
\newcommand{\rank}{{\rho}}% {{\mbox{rank}}}
\newcommand{\Rank}{{\rho}}% {{\mbox{rank}}}
\newcommand{\Card}[1]{\ensuremath{{\left|#1\right|}}}
\newcommand{\ext}[1]{\ensuremath{\mathbf{#1}}}

% Set Complement
% command to mess with overline, bar or custom 
% alternatives for sequence or set complement
%
%\newcommand{\scomp}[1]{\ensuremath{\;\overline{#1}\;}}
%\newcommand{\scomp}[1]{\ensuremath{\bar{#1}}}
%\newcommand{\scomp}[1]{\ensuremath{\genfrac{}{}{}{}{}{#1}}}
\newcommand{\scomp}[1]{\ensuremath{\overline{#1}}}

\allowdisplaybreaks


%%%%%%%%%%%%%%%%%%%%%%%%%%%%%%%%%%%%%%%%%

\begin{document}

\title[Direct Sums and Other Objects with Matroids]
{Direct Sums and Tutte Functions of Graphs and Other Objects
with Matroids}


\author{Seth Chaiken}
\address{Computer Science Department\\
The University at Albany (SUNY)\\
Albany, NY 12222, U.S.A.}
\email{\tt sdc@cs.albany.edu}



\begin{abstract}

The polynomials expressing the conditions that
a parametrized $P$-ported Tutte function is well-defined
for matroids
each have one factor $I(Q)$, where $Q$ is a $P$-quotient.  
Since the elements are 
labelled, the methodology applies
to objects such as graphs with ports
for which similar theorems can be proven.  
We abstract graphs to objects that have 
ported Tutte functions
because they have matroids, but might
have different Tutte function values
on two objects with the same matroid.
Two new theorems are given
and are used to generalize known conditions
for graphs without port edges to graphs with port edges.
In some cases, the condition becomes
$I(Q)r+I(Q')s=0$, where $r,s$ are 
polynomials in $x,y,X,Y$.
The abstraction is then used to
characterize ported Tutte functions
of an object or a combination of two
objects, whose matroid or oriented 
matroid is a direct matroid or
oriented matroid sum.  
This extends with ports some 
known strong Tutte 
function and multiplicative 
Tutte function results.
\end{abstract}

\subjclass[2000]{Primary 05B35; Secondary 05C99, 05C15, 57M25, 94C05}



\keywords{Tutte function, Tutte polynomial}

\thanks{Version of \today.}

\maketitle
\pagestyle{headings}


%%%%%%%%%%%%%%%%%%%%%%%%

\section{Introduction}

When $P=\emptyset$, the facts about separator-strong Tutte functions
of matroid direct sums easily follow from the formula
$T(M^1\oplus M^2)T(\emptyset)=T(M^1)T(M^2)$.  For example,
$T$ is strong if and only if $T(\emptyset)=T(\emptyset)^2$.
The theory of separator-strong Tutte functions of graphs covered
in \cite{Ellis-Monaghan-Traldi}  follows from the fact that any
minor closed family of graphs (see below) $\mathcal{G}$ is partitioned into
subfamilies $\mathcal{G}_k$, each with just one indecomposible,
$E_k$, the edgeless graph with $k$ unlabelled vertices, 
if $\mathcal{G}_k\neq\emptyset$.
Tutte function formulas for disjoint and one-point graph unions, 
and the conditions for strongness (defined $T(G^1)T(G^2)=T(G)$ if
matroids $M(G^1)\oplus M(G^2)=M(G)$) are then 
derived \cite{Ellis-Monaghan-Traldi}
in terms of the values $\alpha_k=I(E_k)$.  Life is simple because
matroid $M(E_k)=\emptyset$ for all $k$.

The corresponding facts become more complex when the definitions
are naturally extended to $P$-ported matroids and graphs
or to vertex labelled graphs.  As with
matroids, a $P$-ported graph $G$ has some of the edges in $P$ and the rest,
$E(G)$, satisfy $E(G)\cap P=\emptyset$.  Deletion, contraction,
$P$-minors, $P$-families and $P$-quotients (i.e., indecomposibles)
are also defined as they are for $P$-ported matroids or oriented matroids.
As in \cite{Ellis-Monaghan-Traldi},
deletion of an isthmus (i.e., coloop in the matroid) and
contraction of a loop is forbidden within Tutte equations.

The main difficulty is illustrated by the following example.
Let
$G$ be the circle graph of the five edges
ordered $(e,p,f,q,r)$ and take $P=\{p,q,r\}$.  
So, $e$ and $f$ are a series pair connected to $P$, but
the $P$-quotient graphs $Q_1=G/e\setminus f$ and 
$Q_2=G/f\setminus e$ are different graphs, even though they have 
the same matroid.  $Q_1$ is the path $qrp$ and $Q_2$ is the path $pqr$.
Function $T$ might satisfy \eqref{TA}
and \eqref{TSSM} even if $T(Q_1)=I(Q_1)\neq I(Q_2)=T(Q_2)$.
So, if this $G\in\mathcal{G}$, a necessary condition 
for $T$ to be a $P$-ported
(separator-strong, as always) Tutte function would be
\[
I(Q_1)(x_ey_f - y_fX_e) = I(Q_2)(x_fy_e - y_eX_f).
\]
This equation does not have the form of 
those in the ZBR theorem for graphs \cite{Ellis-Monaghan-Traldi}
because the latter's equations, like the equations in
Theorem \ref{BigTheorem}, each has a single factor 
$I(Q)$ depending on one indecomposible.


The example relies on the elements of $P$ being labelled.
This leads us to formulate a extension of Tutte function theory
for vertex labelled graphs.  
When the outcome, Theorem \ref{ZBRWellBehaved} is applied to graphs as in 
\cite{Ellis-Monaghan-Traldi}, we get Corollary \ref{PZBRGraphCor2}
which demonstrates that the above example illustrates the
\emph{only} situation where the $P$-ported ZBR equations
of Theorem \ref{BigTheorem} must be modified.

Figure \ref{c4p2TwoFigure} illustrates 
the same phenonemon as the first, in a smaller graph, when
the objects in $\mathcal{G}$ are graphs whose vertices are 
labelled by disjoint sets.  Again, two different graphs have the
same oriented matroid.

\begin{figure}
\input{c4p2Two.pdf_t}
\caption{\label{c4p2TwoFigure}
Two Tutte computation trees for the same graph with set labelled
vertices.  The indecomposibles at the bottom of the first tree
are named $Q_1$, $Q_2$ and $Q_3$ from left to right.  For non-separators,
contraction 
edges are slanted left and  deletion edges are slanted right.  An edge
representing a separator is drawn vertically.}
\end{figure}

The expressions from the two Tutte computation trees in Figure
\ref{c4p2TwoFigure}
are 
\[
I(Q_1)x_ex_f+I(Q_2)x_ey_f+I(Q_3)y_eX_f
\]
and
\[
I(Q_1)x_ex_f+I(Q_3)x_fY_e+I(Q_2)y_fX_e=
I(Q_1)x_ex_f+I(Q_2)y_fX_e+I(Q_3)x_fY_e,
\]
which are equal if and only if
\begin{equation}
\label{BadZBRExampleEq}
I(Q_2)(y_fX_e - x_ey_f)=
I(Q_3)(y_eX_f - x_fY_e).
\end{equation}
$Q_2$ and $Q_3$ are isomorphic as edge-labelled graphs
but are different when the vertex
labels are present.

There are two  complications introduced
into $P$-families of matroids when $P\neq\emptyset$.
First, one matroid 
might have more than one $P$-quotient, i.e., indecomposible.
The most simple 
example is a dyad matroid composed of one port and one
non-port element; and its $P$-minors.
Therefore, minimal $P$-minor closed families might have 
more than one indecomposible.  Some will share
matroids or oriented matroids and others will not.
The second, which also occurs with the 
minor closed families of graphs \cite{Ellis-Monaghan-Traldi}
in the original $P=\emptyset$ form, is that
the family is partitioned into disjoint $P$-minor closed
subfamilies.  Each subclass has its own indecomposibles,
$E_k$ in the case of graphs.  Again, 
indecomposibles in different subclasses 
share matroids as do
the $E_k$ all of whose matroids are $\emptyset$.
When $P\neq 0$, the indecomposibles of different 
subclasses might or might not share matroids
or oriented matroids.

The ZBR theorem for graphs in \cite{Ellis-Monaghan-Traldi}
has conditions analogous to those in Corollary \ref{ZBRmatroids},
except the factor $\alpha$ is replaced by $\alpha_k=I(E_k)$
depending on the subfamily.  

$P$-ported matroids or oriented matroids can be combined 
by matroid direct sum $\oplus$.  Graphs can be combined
by disjoint union $\amalg$ or by a one-point union;
then each such combination $G$ of $G^1$ and $G^2$, if defined,
satisfies $M(G)=M(G^1)\oplus M(G^2)$.  
Zaslavsky's definition\cite{MR93a:05047} for 
matroids is immediately extended:


\begin{definition}
A \textbf{strong} $P$-ported Tutte function $T$ on a $P$-family 
$\mathcal{C}$ of matroids
or oriented matroids satisfies $T(M^1)T(M^2)=T(M^1\oplus M^2)$
when $M^1, M^2$ and $M^1\oplus M^2$ are all in $\mathcal{C}$.
\end{definition}

Note that such a strong Tutte function is a separator-strong
Tutte function with $X_e=T(U^e_1)$ and
$Y_e=T(U^e_0)$ for all $e\in E(\mathcal{N})$.

We will give extensions of definitions of strong Tutte functions
and of multiplicative Tutte functions of graphs below when
we define $P$-families of objects with matroids or oriented 
matroids.



\section{Objects with Matroids or Oriented Matroids}

It is useful to think that a $P$-ported Tutte computation tree may have
objects $N$ for its node labels such as graphs.
Each object $N$ has an associated
a $P$-ported matroid or oriented matroid $M(N)$.
Elements are defined $S(N)=S(M(N))$, 
each $p\in S(N)\cap P$ is called a port, and
$E(N)=S(N)\setminus P$.  Loops, coloops and non-separators of $N$
are characterized by their status in $M(N)$.  So we
say $N$ is an \emph{object with a matroid or an oriented matroid}.
Often, but not always, $N$ will be some matroid or oriented
matroid representation.


Contraction $N/e$ and deletion $N\setminus e$ 
of object $N$ are defined when $e\in E(N)$, and
$e$ is not a coloop in $M(N)$ and 
$e$ is not a loop in $M(N)$, respectively.
Under those conditions, $M(N/e)=M(N)/e$ and 
$M(N\setminus e)=M(N)\setminus e$ (as matroids or
oriented matroids).
Thus $P$-minors are defined,
and an indecomposible or $P$-quotient
is a $P$-minor $Q$ for which $S(Q)=S(M(Q))\subseteq P$.

\begin{definition}
\label{OMOMdef}
An $P$-ported object $N$ with a matroid or oriented matroid 
is described above together with $M(N)$, $E(N)$, $S(N)$,
$P$-minors, etc.
A \textbf{$P$-family of objects $\mathcal{N}$}
is a $P$-minor closed class of 
objects with matroid or oriented matroids.
\end{definition}

Tutte computation trees are defined for such $N$.
The matroid $M(N)$ of course
constrains the structure of these trees.  
It is possible
(as when the edgeless graphs $G_k$ have different vertex sets but all
$M(G_k)=\emptyset$)
for different objects, 
even different indecomposibles, to have the same 
matroid or oriented matroid.  It also natually occurs 
that
$N/e\setminus f$ $\neq$ 
$N/f\setminus e$ (as objects) even though 
$M(N)/e\setminus f$ $=$ 
$M(N)/f\setminus e$.  The latter equation when 
$e,f$ are in parallel or in series (see Proposition
\ref{SameMinorProp}) is critical to the matroid ZBR theorems.
It is also conceivable that $N/e/f\neq N/f/e$ or
$N\setminus e\setminus f\neq N\setminus f\setminus e$.

Since every $P$-minor $N'$ of $N$ has matroid
or oriented matroid $M(N')$ the same as the corresponding
minor of $M(N)$, we observe:

\begin{lem}
\label{ObjectTreeValueLemma}
The Tutte computation trees for $M(N)$ are in a one-to-one
correspondance with the 
Tutte computation trees for $N$ 
where corresponding trees are isomorphic.  In each
isomorphism,
corresponding
branches have the same labels ``$e$-contracted'' or
``$e$-deleted'' with $e\in E(N)=E(M(N))$, and a node
labelled $N_i$ in the tree for $N$ corresponds to 
a node labelled $M(N_i)$ in the tree for $M(N)$.

Each computation tree value is given by
the activities expansion \eqref{PAE} reinterpreted for
objects.
\end{lem}


We can still talk about Tutte decompositions and a
Tutte computation tree for $N$ even  without 
a Tutte function.  If we are given values $I(Q)$
for the indecomposibles, each Tutte computation tree 
for $N$ yields a value in the $R$-module generated by
the $I(Q)$.
The Tutte decompositions, and the universal
Tutte polynomial (if it exists!)
of each $N\in \mathcal{N}$
are determined by $M(N)$ and the indecomposibles, i.e., $P$-quotients
$Q$ in $N$, which of course satisfy $Q \in \mathcal{N}$.
This generalizes Zaslavsky's discussion\cite{MR93a:05047}.

\begin{definition}[Separator-strong $P$-ported Tutte function on objects]

Function
$T$ on $\mathcal{N}$
is a $P$-ported separator-strong Tutte function
on $\mathcal{N}$ into 
the ring $R$ 
containing parameters $x_e, y_e, X_e, Y_e$,
or an $R$-module containing the initial values
$T(Q)=I(Q)$ for indecomposibles,
when for all $N\in\mathcal{N}$,
\eqref{TA} and \eqref{TSSM} are
satisfied for each
$e\in E(N)$.
\end{definition}

Therefore:

\begin{prop}
$T$ is a $P$-ported separator-strong Tutte function
on $\mathcal{N}$ if and only if for each $N\in\mathcal{N}$,
all Tutte computation trees for $T(N)$ yield polynomial expressions
that are equal in the range ring or $R$-module.
\end{prop}

We develop our first ZBR-type theorem for $P$-ported objects
with matroids or oriented matroids.  It is the
generalization of the ZBR theorem for graphs as given
by Ellis-Monaghan and Traldi\cite{Ellis-Monaghan-Traldi}.  
It depends on a lemma similar to one of theirs.


\begin{lem}
\label{DisjSubclassLem}
Suppose $P$-family $\mathcal{N}$ is partitioned into
disjoint $P$-minor closed subfamilies $\{\mathcal{N}_{\pi}\}$.
Then $T$ is a Tutte function on $\mathcal{N}$ if and
only if $T$ restricted to $\mathcal{N}_{\pi}$ is
a Tutte function for each $\mathcal{N}_{\pi}$.
\end{lem}

\begin{thm}
\label{ZBRWildFamily}
Suppose $P$-family $\mathcal{N}$ is partitioned into
disjoint $P$-minor closed subfamilies $\{\mathcal{N}_{\pi}\}$,
and each initial value $I(Q)$ depends only on
the matroid or oriented matroid $M(Q)$ and on the
$\pi$ for which $Q\in\mathcal{N}_{\pi}$,

Then $T$ is a Tutte function with given parameters $(x,y,X,Y)$
and initial values $I(Q)$ if and only if it satisfies
the equations of Theorem \ref{BigTheorem}, interpreted
for families of objects with matroids or oriented matroids.
\end{thm}

\begin{proof}
As in \cite{Ellis-Monaghan-Traldi}, lemma
\ref{DisjSubclassLem} lets us prove the
theorem for each $\pi$ separately.


By Lemma \ref{ObjectTreeValueLemma}, $T$ is a 
$P$-ported Tutte function of family
of objects $\mathcal{N}_{\pi}$ if and only
if function $T'(M(N))=T(N)$ on the $P$-family 
$\mathcal{C}^{\pi}=\{M(N)\mid N\in\mathcal{N}_{\pi}\}$ is a $P$-ported Tutte 
function, since by hypothesis 
$I'(M(Q))=I(Q)=I(M(Q))$ for corresponding
indecomposibles $Q\in\mathcal{N}_{\pi}$ and 
$M(Q)\in\mathcal{C}^{\pi}$.
The conclusion follows
from Theorem \ref{BigTheorem} applied to $\mathcal{C}^{\pi}$.
\end{proof}


Ellis-Monaghan and Traldi's ZBR theorem for graphs 
refers to one initial value $\mathcal{\alpha}_k=I(E_k)$
for each non-empty subclass of graphs, with unlabelled vertices, that
have $k$ graph components.  One natural ported generalization
is to partition the $P$-ported graphs $G$
according to (1) how many
graph components $k$, (2) $P'=P\cap S(G)$ and (3)
$\nu:P'\rightarrow\{1,\ldots,k\}$, where $\nu(p)$ is which
component contains edge $p$.  Theorem
\ref{ZBRWildFamily} tells us:

\begin{cor}
\label{PZBRGraphCor1}

Let a $P$-minor closed collection $\mathcal{G}$ of graphs with unlabelled
vertices be partitioned into $\mathcal{G}_{k,P',\nu}$.  Suppose initial
values $I(G)=I_{k,P',\nu}(M(G))$ are given that depend only on the 
part and the matroid or oriented matroid of $G\in\mathcal{G}_{k,P',\nu}$.
Then there is $T$, 
a $P$-ported separator-strong parametrized Tutte function
of graphs $\mathcal{G}$ satisfying $T(Q)=I(Q)=I_{k,P',\nu}(M(Q))$ whenever
$P$-quotient $Q\in\mathcal{G}_{k,P',\nu}$ if and only if the identities
of Theorem \ref{BigTheorem}, interpreted for graphs, are satisfied with 
the given $I(Q)$.
\end{cor}

The next ZBR-type theorem  addresses the problem 
illustrated by equation \eqref{BadZBRExampleEq}. It requires that
the $P$-family satisfy the following:

\begin{definition}
Object $N\in\mathcal{N}$ is \textbf{well-behaved} when
for every independent set $C\subseteq E(N)$ and
coindependent set $D\subseteq E(N)$ for which
$C\cap D=\emptyset$, each of the
$|C\dunion D|!$ orders 
of contracting $C$ and deleting $D$ produces the
same $P$-minor (which is an object) of $N$.

Specifically, let 
$C=\{c_1,\ldots,c_j\}$,
$D=\{d_{j+1},\ldots,d_k\}$
and $R_i(N')=N'/c_i$ if $1\le i \le j$
and $N'\setminus d_i$ if $j+1\le i \le k$.
The condition is 
$R_1\circ\cdots\circ R_k(N) = R_{\sigma_1}\circ\cdots\circ R_{\sigma_k}(N)$
for every permutation $\sigma$ of $\{1,\ldots,k\}$.

$\mathcal{N}$ is \textbf{well-behaved} when 
each $N\in\mathcal{N}$ is well-behaved.

\end{definition}

By definition \ref{OMOMdef} 
all the minors are defined and
$M(R_1\circ\cdots\circ R_k(N)) = M(R_{\sigma_1}\circ\cdots\circ R_{\sigma_k}(N))$
independently of whether $N$ is well-behaved or not.
The point is that the objects themselves are the same.
We give two examples.

\begin{definition}[Graphs with set-labelled vertices]
\label{GSLVDefinition}
The elements of such a graph $S(G)=E(G)\dunion(P\cap S(G))$ are edges.
The vertices are labelled with non-empty
finite sets so the two sets labelling distinct vertices in one
graph are disjoint.  Only non-loop edges $e\not\in P$
can be contracted; when an edge is contracted, its two endpoints
are replaced by one vertex whose label is the union of the 
labels of the two endpoints.  Only non-isthmus edges $e\not\in P$
can be deleted; deletion doesn't change labels.  The graph
has its graphic matroid if it is undirected and its
oriented graphic matroid if it is directed.
\end{definition}

A graph with set-labelled vertices is well-behaved because
the minor obtained  by contracting forest $C$
and deleting $D$ is determined by merging all the vertex labels
of each graph component of $C$ and removing edges $C\cup D$.  
The deletions do not affect the vertex labels.  Hence
the vertex labels are not affected by the order
of the operations.


\begin{definition}[Graphs with set-labelled components]
\label{GSLCDefinition}
The elements of such a graph $S(G)=E(G)\dunion(P\cap S(G))$ are edges.
The path-connected components are labelled by
non-empty finite sets so two components in the same
graph always have disjoint labels.  In other words,
the set labels of the components are a partition $\pi_V$.
Only non-loop edges $e\not\in P$
can be contracted and only non-isthmus edges 
$e\not\in P$
can be deleted.
The component labels are unchanged by these minor operations.
Definition \ref{GSLVDefinition} specifies the
matroids or oriented matroids.
\end{definition}


A non-well-behaved $P$-family $\mathcal{C}!$
can be constructed from
any $P$-family of matroids $\mathcal{C}$ with
some $M\in\mathcal{C}$ with $|E(M)|\geq 2$.  
Each member of $\mathcal{C}!$ is formed from
some $M\in\mathcal{C}$ together with some
history of deletions and contractions that
can be applied to $M$.  Let $c_e$ and $d_e$ be symbols
for contracting and deleting $e\in E(\mathcal{C})$
respectively; a history $h$ is a string of
such symbols.  
Let $M|h$ be the $P$-minor obtained by 
performing history $h$ on $M$, assuming each step
is defined.  The objects of $\mathcal{C}!$
are all pairs $(M,h)$ for which 
$P$-minor $M|h\in\mathcal{C}$ is defined.
The matroid of $(M,h)$ is $M|h$, which
determines the element set, loops and coloops.
If $e\in E(M|h)$ is not
a loop, then define $(M,h)/e=(M,hc_e)$.
Similarly, if $e\in E(M|h)$ is not
a coloop, $(M,h)\setminus e=(M,hd_e)$.

The point of this example is that even if
the $P$-family is not well-behaved and so the
indecomposibles do carry information about their
history, Theorem \ref{ZBRWildFamily} tells 
us that the Tutte function
is still well defined if the initial values depend only
on the matroid, or the oriented matroid, of the 
indecomposible. (See Questions, \ref{Qsubsection}.)


The examples forced us to recognize that
for $N$ an object with a matroid or oriented matroid $M(N)$
with
${e,f}\in E(N)$ in series or in parallel,  it might
happen that $N/e\setminus f\neq N/f\setminus e$ even though,
by Proposition \ref{SameMinorProp}, 
$M(N)/e\setminus f= M(N)/f\setminus e$.  Note that 
Proposition \ref{SameMinorProp} is about 
$\cdot/e\setminus f$ and 
$\cdot/f\setminus e$ which are not commutations
of the same two operations.

\begin{thm}[ZBR Theorem for well-behaved 
$P$-families of objects with matroids or
oriented matroids]
\label{ZBRWellBehaved}

Supposed $\mathcal{N}$ is well-behaved.
The following two statements are equivalent.
\begin{enumerate}
\item $T$ from $\mathcal{N}$ to $R$ or an $R$-module is a $P$-ported 
separator-strong parametrized
$P$-ported Tutte function with $R$-parameters $(x, y, X, Y)$ whose values 
$T(Q)$ on $P$-quotients $Q\in\mathcal{N}$ are the initial
values $I(Q)$.
\item
For every $N\in\mathcal{N}$:
\begin{enumerate}
\item 
If $M(N)=U^{ef}_1\oplus M(Q)=U^{ef}_1\oplus M(Q')$ with 
$P$-quotients $Q=N/e\setminus f$ and $Q'=N/f\setminus e$,
\[
I(Q)(x_e Y_f - y_f X_e ) = 
I(Q')(x_f Y_e - y_e X_f).
\]
\item
If $M(N)=U^{efg}_2\oplus M(Q)=U^{efg}_2\oplus M(Q')$ with 
$P$-quotients $Q=N/e\setminus f/g$ 
and $Q'=N/f\setminus e/g$,
\[
I(Q)X_g(x_e y_f - y_f X_e ) = 
I(Q')X_g(x_f y_e - y_e X_f ).
\]
\item
If $M(N)=U^{efg}_1\oplus M(Q)=U^{efg}_1\oplus M(Q')$ with 
$P$-quotients $Q=N/e\setminus f\setminus g$  
and $Q'=N/f\setminus e\setminus g$,
\[
I(Q)Y_g(x_e Y_f - y_f x_e) = 
I(Q')Y_g(x_f Y_e - y_e x_f).
\]
\item
If $\{e,f\}=E(M(N))$ is a parallel pair connected to $P$, 
\[
I(Q)(x_e Y_f - y_f x_e) = 
I(Q')(x_f Y_e - y_e x_f)
\]
where $P$-quotients $Q=N/e\setminus f$
and $Q'=N/f\setminus e$.
\item
If $\{e,f\}=E(M(N))$ is a series pair connected to $P$, 
\[
I(Q)(x_e y_f - y_f X_e) = 
I(Q')(x_f y_e - y_e X_f)
\]
where $P$-quotients
$Q=N/e\setminus f$ 
and $Q'=N/f\setminus e$.
\end{enumerate}
\end{enumerate}
\end{thm}

\begin{proof}
All the steps of the proof given for 
Theorem \ref{BigTheorem} can be adapted.
The hypothesis the $\mathcal{N}$ is well-behaved
means that $N'/e/f = N'/f/e$, $N'/e\setminus f = N'\setminus f/e$, etc.,
(equalities of objects) are true respectively whenever a step of the
form 
$M'/e/f = M'/f/e$, $M'/e\setminus f = M'\setminus f/e$, etc.,
respectively occurs for matroids or oriented matroids.

The calculations done in all cases of Theorem \ref{BigTheorem}
of where 
Proposition \ref{SameMinorProp} is applied to write
$Q = M'/e\setminus f = M'/f\setminus e$ when $e,f$ are in
series or parallel are replaced by 
$Q = N'/e\setminus f$ and $Q' = N'/f\setminus e$.  
Matroids or oriented matroids are replaced
by objects with matroids or oriented matroids in 
the calculations.
Notes to guide the reader appear in the proof.
\end{proof}



\begin{cor}
\label{IVEqualSerParCor}
A $P$-family of objects with matroids satisfies a ZBR-type theorem
with the identities given in Theorem \ref{BigTheorem} if, 
in 
addition $\mathcal{N}$ being well-behaved,
the initial values $I$ satisfy
$I(N/e\setminus f)=I(N/f\setminus e)$ when $\{e,f\}=E(N)$ is a series
or parallel pair,
$I(N/e\setminus f/g)=I(N/f\setminus e/g)$ when $\{e,f,g\}=E(N)$ is a 
triangle and 
$I(N/e\setminus f\setminus g)=I(N/f\setminus e\setminus g)$ 
when $\{e,f,g\}=E(N)$ 
is a triad.
\end{cor}


\begin{cor}
A $P$-family of objects with matroids satisfies a ZBR-type theorem
with the identities given in Theorem \ref{BigTheorem} if, in 
addition to $\mathcal{N}$ being well-behaved,
the object $P$-quotients 
$N/e\setminus f=N/f\setminus e$ when $\{e,f\}=E(N)$ is a series
or parallel pair,
$N/e\setminus f/g=N/f\setminus e/g$ when $\{e,f,g\}=E(N)$ is a 
triangle and 
$N/e\setminus f\setminus g=N/f\setminus e\setminus g$ when $\{e,f,g\}=E(N)$ 
is a triad.
\end{cor}

\begin{proof}
Clearly, if $N/e\setminus f=N/f\setminus e$ then 
$I(N/e\setminus f)=I(N/f\setminus e)$, etc.
\end{proof}

We conclude with second ported generalization of 
Ellis-Monaghan and Traldi's ZBR theorem for graphs,
besides Corollary \ref{PZBRGraphCor1}.

\begin{cor}
\label{PZBRGraphCor2}
Let $\mathcal{G}$ be a ported $P$-family of graphs with
unlabelled vertices, as in Corollary \ref{PZBRGraphCor1}.
Then there is $T$, $P$-ported separator-strong parametrized Tutte function
of graphs $\mathcal{G}$ satisfying $T(Q)=I(Q)$ 
for all $P$-quotients $Q\in\mathcal{G}$ if and only if 
for every $G\in\mathcal{G}$, each case applies:
\begin{enumerate}
\item[a-d]
Revise the corresponding case of Theorem \ref{ZBRWellBehaved}
by writing $Q'=Q$, i.e., replace $Q'$ by $Q$.
\item[e]
If $E(G)=\{e,f\}$ is a series pair connected to $P$, 
\[
I(Q)(x_e y_f - y_f X_e) = 
I(Q')(x_f y_e - y_e X_f)
\]
where $P$-quotients
$Q=G/e\setminus f$ 
and $Q'=G/f\setminus e$.  (This is case (e) of Theorem \ref{ZBRWellBehaved}
\emph{verbatim}.)
\end{enumerate}
\end{cor}

\begin{proof}
$\mathcal{G}$ is a well-behaved $P$-family of ported objects
with matroids or oriented matroids, so Theorem \ref{ZBRWellBehaved}
applies.
In all cases of Theorem \ref{ZBRWellBehaved} but the
last, the two object minors are the same graph 
because the contracted edges are path-connected,
so the equations of Theorem \ref{ZBRWellBehaved} are simplified.
\end{proof}

Both Corollaries \ref{PZBRGraphCor1} and \ref{PZBRGraphCor2} reduce
to the ZBR theorem for graphs when $P=\emptyset$.  The first
uses the property that $\mathcal{G}$ is partitioned into
minor-closed subclasses with indecomposibles $E_k$ 
for which the
initial values depend only on the matroid or oriented matroid to generalize
the original ZBR equations.  As with matroids, we find again that
the Tutte functions can distinguish different orientations of the 
the same undirected graphs.
The second relies on the 
commutivity of the graph minor operations and generalizes
the fact that different initial values may be assigned to
different indecomposibles, but then the 
conditions sufficient for the initial values to
extend to a Tutte function must be stronger.





\section{Direct and Other Sums}

\label{DirectSec}

It is a common situation that $\{N^1,N^2,N\}\subseteq\mathcal{N}$
and their matroids or oriented matroids 
satisfy $M(N^1)\oplus M(N^2)=M(N)$.
Tutte computation trees help.  The proposition below applies
even to non-well-behaved $N$ when the symbols
$/B^j_i|P_j$ refer to sequences of deletions and contractions.

\begin{definition}
If $\mathcal{T}_1$ and $\mathcal{T}_2$ are Tutte computation trees then
$\mathcal{T}_1\cdot \mathcal{T}_2$ is the tree obtained by appending
a separate copy of $\mathcal{T}_2$ at each leaf of $\mathcal{T}_1$.
The root is the root of the expanded $\mathcal{T}_1$.
\end{definition}

\begin{prop}
\label{SumProp2}

Suppose $N$, $N^1$ and $N^2$ are all in $\mathcal{N}$
and $M(N^1)\oplus M(N^2)=M(N)$.  Then if
$\mathcal{T}_1$ and $\mathcal{T}_2$ are Tutte computation
trees for $N^1$ and $N^2$ respectively with values given
by \eqref{DS1} and \eqref{DS2}, then there is a Tutte
computation tree for $N$ that yields the value given
by \eqref{DS}.

\begin{equation}
\label{DS1}
\tag{DS1}
\sum_{Q^1_i}I(Q^1_{i})c_1(Q^1_{i})\text{ where }Q^1_i=N/B^1_i|P_1.
\end{equation}
\begin{equation}
\label{DS2}
\tag{DS2}
\sum_{Q^2_j}I(Q^2_{j})c_2(Q^2_{j})\text{ where }Q^2_j=N/B^2_j|P_2.
\end{equation}
\begin{equation}
\tag{DS}
\label{DS}
\sum_{Q^1_i,Q^2_j}I(Q_{i,j})c_1(Q^1_{i})c_2(Q^2_{j})
\text{ where }Q_{i,j}=N/B^1_i|(P_1\cup S(N^2))/B^2_j|P_2.
\end{equation}

Furthermore, if $T$ is a Tutte function on $\mathcal{N}$
and $T(N^1)$ and $T(N^2)$ equal the Tutte polynomials
given by \eqref{DS1} and \eqref{DS2} then 
$T(N)$ equals the polynomial given by \eqref{DS}.
\end{prop}

\begin{proof}

We show how to relabel $\mathcal{T}_1\cdot\mathcal{T}_2$ to obtain 
a Tutte computation tree for $N$.  $M(N^1)\oplus M(N^2)=M(N)$ is defined
means $S(M(N^1))\cap S(M(N^2))=\emptyset$ and 
$S(M(N))=S(M(N^1))\cup S(M(N^2))$.   Each node of 
$\mathcal{T}_1\cdot\mathcal{T}_2$ is determined by 
by deleting and/or contracting some elements of
$E(M(N^1))\cup E(M(N^2))$.  Relabel that node with the 
$P$-minor of $N$ obtaining deleting and/or contracting the
same elements respectively in the same order, those in $N^1$
preceding those in $N^2$.  The result is a computation tree for $N$
because $M(N^1)\oplus M(N^2)=M(N)$.  Assume $P\subseteq S(M(N))$
(otherwise, take a smaller $P$) and let $P^1=S(M(N^1))\cap P$
and $P^2=S(M(N^2))\cap P$.
At a leaf of the relabelled
tree, there will be the $P$-quotient $N/B_1/B_2|P$ where
$B_1$ is a $P^1$-subbasis of $M(N^1)$ and $B_2$ is a $P^2$-subbasis
of $M(N^2)$.  

\end{proof}

\subsection{Strong Tutte Functions}

Let us extend the definition of strong parametrized Tutte function
to $P$-families $\mathcal{N}$ of objects with matroids and oriented matroids,
in the way that abstracts the known notion of strong Tutte functions
on minor closed families of graphs\cite{Ellis-Monaghan-Traldi}.
We can then specialize 
$\mathcal{N}$ to $\mathcal{C}$, a $P$-family  of 
matroids or oriented matroids.
There might still be indecomposibles besides or instead of $\emptyset$.

\begin{definition}
A $P$-ported separator-strong Tutte function $T$ on a $P$-family of 
objects $\mathcal{N}$
with matroids is called \textbf{strong} if
whenever $\{N^1, N^2, N\}\subseteq\mathcal{N}$
and 
$M(N^1)\oplus M(N^2)=M(N)$, 
then $T(N^1)T(N^2)=T(N)$.
\end{definition}

We now generalize to $P\neq\emptyset$
the $T(\emptyset)T(\emptyset)=T(\emptyset)$ 
characterization of strong Tutte functions.

\begin{thm}
\label{StrongTheorem}
A $P$-ported separator-strong Tutte function $T$ on a $P$-family of 
objects 
with matroids or oriented matroids $\mathcal{N}$ 
is strong if and only if
$T$ restricted to the indecomposibles of 
$\mathcal{N}$ is strong; i.e.,
whenever $Q^1$, $Q^2$ and 
$Q$ are indecomposibles and $M(Q^1)\oplus M(Q^2)=M(Q)$ then
$T(Q^1)T(Q^2)=T(Q)$.
\end{thm}

\begin{proof}
Every $P$-quotient is in $\mathcal{N}$, so clearly $T$ restricted
to the $P$-quotients is strong.

Conversely, suppose $N^1$, $N^2$ and $N$ are in $\mathcal{N}$ and
$M(N^1)\oplus M(N^2)=M(N)$, so Proposition \ref{SumProp2} applies.

Since $M(N^1)\oplus M(N^2)=M(N)$, 
$M(N/(B_1\cup B_2)|P)$ $=$ $(M(N^1)/B_1|P)\oplus (M(N^2)/B_2|P)$
$=$ $M(N^1/B_1|P)\oplus M(N^2/B_2|P)$.  
We now use the fact
that $Q_{ij}=N/B_1/B_2|P$, $Q^1_i=N^1/B_1|P$ and $Q^2_j=N^2/B_2|P$ are 
$P$-quotients and the hypothesis to
write $I(Q_{ij})=I(Q^1_i)I(Q^2_j)$.  


We therefore conclude $T(N)=T(N^1)T(N^2)$
from \eqref{DS1}, \eqref{DS2} and  \eqref{DS}.

\end{proof}

\subsection{Multiplicative Tutte Functions}

Graphs can be combined in several ways all so
the matroid of the combination is the direct
sum of the matroids of the parts. This motivates:

\begin{definition}
A partially defined binary operation ``$*$'' on
a $P$-family of objects with matroids or oriented matroids
$\mathcal{N}$ is \textbf{a matroidal direct sum} if 
whenever $N^1*N^2\in\mathcal{N}$ is defined for
$\{N^1,N^2\}\subseteq\mathcal{N}$, the 
matroids or oriented matroids satisfy
$M(N^1)\oplus M(N^2) = M(N^1*N^2)$.
\end{definition}

Proposition \ref{SumProp2} applies
when $N^1*N^2=N$ is defined.
It gives a general recipe for $T(N^1*N^2)$ which generalizes
the identity \cite{Ellis-Monaghan-Traldi}
$T(M^1\oplus M^2)T(\emptyset)=T(M^1)T(M^2)$ 
for separator-strong Tutte functions of
matroids.
The $P$-ported generalization is more complicated and generally cannot
be expressed by a product in the domain ring of $T$.  

\begin{prop}
\label{SumProp}
Suppose $*$ is a matroidal direct sum and
$N^1$, $N^2$ and $N^1*N^2$ are each members of a $P$-family
for which $T$ is a Tutte function.

If for $P$-quotients $Q^j_i$ and $R$-coefficients $c_j(Q^j_i)$, $j=1$ and $2$,
\begin{equation*}
%\label{MD1}
%\tag{MD1}
T(N^1) = \sum_{Q^1_i}T(Q^1_{i})c_1(Q^1_{i})
\end{equation*}
and
\begin{equation*}
%\label{MD2}
%\tag{MD2}
T(N^2) = \sum_{Q^2_j}T(Q^2_{i})c_2(Q^2_{j})
\end{equation*}
then
\begin{equation*}
%\label{MD}
%\tag{MD}
T(N^1 * N^2) = \sum_{Q^1_i,Q^2_j}T(Q^1_{i}*Q^2_{j})c_1(Q^1_{i})c_2(Q^2_{j}).
\end{equation*}
\end{prop}

\begin{proof}
Substitute $Q_{i,j}=Q^1_i*Q^2_j$ in \eqref{DS} of Proposition \ref{SumProp2}.
\end{proof}


When $\mathcal{N}$ is a $P$-family of matroids or oriented matroids,
direct matroid or oriented matroid sum is obviously a matroidal direct
sum operation, and so Proposition \ref{SumProp} is applicable.

\begin{cor}\cite{Ellis-Monaghan-Traldi}
Let $P=\emptyset$.
$T(M^1\oplus M^2)T(\emptyset) = T(M^1)T(M^2)$ for Tutte
function $T$ of matroids.
\end{cor}

\begin{proof}
Our proof demonstrates how Proposition \ref{SumProp}
generalizes this formula to $P$-families.
The expansions \ref{DS1} and \ref{DS2} take the one-term form
$T(M^j) = T(\emptyset) c_j(\emptyset)$, $j$ $=$ $1,2$,
so $T(M^1)T(M^2)=T(\emptyset)^2 c_1(\emptyset) c_2(\emptyset)$.
Expansion \ref{DS} is then
$T(M^1\oplus T^2)=T(\emptyset) c_1(\emptyset) c_2(\emptyset)$.
\end{proof}

Following the definitions for graphs in \cite{Ellis-Monaghan-Traldi}, we write:

\begin{definition}
Given a matroidal direct sum $*$ on $\mathcal{N}$, 
a Tutte function $T$ on $\mathcal{N}$ is
\textbf{multiplicative} (with respect to ``$*$'')
if whenever $N^1*N^2$ is defined
for $\{N^1,N^2\}\subseteq \mathcal{N}$, the Tutte
function values satisfy 
$T(N^1)T(N^2)=T(N^1*T^2)$.
\end{definition}

A strong Tutte function is certainly multiplicative
for any ``$*$'', but not conversely.

\begin{cor}
A $P$-ported Tutte function $T$ on $P$-family $\mathcal{N}$ 
is multiplicative with respect to matroidal direct product
``$*$'' if and only if 
for every pair of indecomposibles $\{Q_i, Q_j\}\in\mathcal{N}$ for which 
$Q_i*Q_j\in\mathcal{N}$ is defined, $T(Q_i)T(Q_j)=T(Q_i*Q_j)$.
\end{cor}

\begin{proof} 
When $N^1*N^2$ is defined, Proposition \ref{SumProp2}
applies because $M(N^1)\oplus  M(N^2)=M(N^1*N^2)$.
$T(N^1*N^2)=T(N^1)T(N^2)$ is then a consequence
of $T(Q_i*Q_j)=T(Q_i)T(Q_j)$.
\end{proof}

\begin{cor}
Suppose $T$ is parametrized Tutte function on a minor-closed class
of graphs $\mathcal{G}$ with unlabelled vertices
and $P=\emptyset$.
$T$ is strong if and only if 
there is an idempotent $\alpha=\alpha^2\in R$ and $T(E_k)=\alpha$
whenever $\mathcal{G}_k\neq\emptyset$.
\end{cor}


\begin{proof}
$M(E_k)=\emptyset$ for all $k\geq 1$ and 
$\emptyset\oplus\emptyset=\emptyset$, so 
$T(E_i)T(E_j)=T(E_k)$ whenever $\mathcal{G}_i$,
$\mathcal{G}_j$ and 
$\mathcal{G}_k$ are all non-empty.  Hence, if
$\mathcal{G}_k\neq\emptyset$ then $T(E_k)T(E_k)=T(E_k)=\alpha$.
Further, if $\mathcal{G}_j\neq\emptyset$ with $j\neq k$,
$\alpha=T(E_k)=T(E_k)T(E_k)=T(E_j)$.
\end{proof}

This
strengthens part of Corollary 3.13 in \cite{Ellis-Monaghan-Traldi}.
The other part does require its 
additional conditions to establish strongness of 
a Tutte function on graphs given as multiplicative on
both disjoint union $\amalg$ and one-point unions.
Consider $\mathcal{N}=\{E_3, E_4, E_5,  \ldots \}$,
$T(E_k)=1$ for $k\geq 3$, $k\neq 5$ and 
$T(E_5)=0$.  $T(E_3)T(E_4)=1\neq T(E_5)$, so 
$T$ is not strong, but $T$ is multiplicative
on disjoint and one-point unions because $E_5$ cannot be expressed
as either kind of union of graphs in $\mathcal{G}$.  The other conditions
are that $\mathcal{G}_k\neq\emptyset$ for all $k$ and that 
$\mathcal{G}$ is closed under one-point unions and removal of isolated
vertices.

\part{Remarks, Background, Problems, and References}

\section{Non-commutativity}
We presented in  \cite{TutteEx}
a new kind of strong Tutte-like function on $P$-ported oriented graphic
matroids (more generally, unimodular, i.e. regular oriented matroids)
whose values vary with the orientation.   
Each function value $F(G)$ is
in the exterior algebra over $R^{2p}$, where $R$ is
the reals extended by the $x_e, y_e$ 
and $|P|=p$.  The function
obeys an \emph{anti-commutative} variant of \eqref{TSM}
with exterior multiplication $\wedge$.  
(When $P=\emptyset$, $F$ is
the reduced Laplacian determinant in the
famous Matrix Tree Theorem \cite{HararyBook,sdcMTT}.)  
It is the first example we know
of
``the possibility of making use of a noncommutative generalization of the Tutte
polynomial at some point in the future.'' mentioned by Bollob\'{a}s and
Riordan in \cite{BollobasRiordanTuttePolyColored}.  We won't say
more beyond that (1) each of the $\binom{2p}{p}$
Pl\"{u}cker coodinates of $F(G)$ is a $P$-ported Tutte function
of the kind we covered; and (2) that quadratic inequalities among
some of them express negative correlation between edges in spanning trees,
results also known as 
Rayleigh's inequality\cite{ChoeRayleigh,SempleWelshNegCorr08,HalfPlaneStuff}.

A second non-commutative possibility might be found in 
section \ref{DirectSec}.
The abstraction applies to situations where
an object represents an initial matroid, graph, etc. plus
a history of deletions and contractions.  
The key feature is that the Tutte decompositions
of an object are identical to the Tutte decompositions
of that object's matroid.
The initial value
on indecomposible object $Q$
might then depend on the order of the deletion and contractions
reductions to 
to obtain it from $N$.  This helps us understand the
theory, but whether objects with minors that depend on reduction 
order have useful applications remains to be seen.

\section{Background and Other Related Work}
\label{BackgroundSec}

Besides Brylawski's 
work,  another early appearance of Tutte decomposition
of a matroid or graph with a basepoint is \cite{SmithPatroids}.
Ellis-Monaghan and Traldi \cite{Ellis-Monaghan-Traldi}
explain that by leaving the reduction
by $e_0$ to last so $e_0$ is always contracted as a coloop or deleted
as a loop, the Tutte function value can be expressed by
$T(M) = (rX_{e_o} + sY_{e_0})T(\emptyset)$ were $r, s$ are 
not-necessarilly-unique elements in $R$.  As one application,
they 
give a formula for the parametrized Tutte polynomial
for the parallel connection across $e_0$
which generalized Brylawski's.
These $r,s$ appear in
the $P$-ported Tutte function expression 
$rT(U^{e_0}_1)$ $+$ $sT(U^{e_0}_0)$ when $P=\{e_0\}$.  They
are parametrized generalizations of the coefficients of
$z'$ and $x'$ in Brylawski's four variable Tutte polynomial.

Las Vergnas defined and gave basic properties of ``set-pointed'' Tutte
polynomials (with no parameters) and used them to study matroid perspectives.
The polynomial given in \cite{MR0419272,SetPointedLV} has a variable
$\xi_l$ for each subset in a collection of $k$ subsets
$P_l\subseteq P$, $l=1,\ldots,k$.  Each term in \eqref{PAE} had
$\prod\xi_l^{r_i(P_l)}$ for $[Q_i]$ where $r_i$ is the rank function 
of matroid $Q_i$.  Therefore \eqref{TSM} was satisfied and the association of the
term to (non-oriented) $Q_i$ could be assured by taking all
$2^{|P|}$ subsets for the $P_l$.  The matroid perspective is the 
strong map $M\setminus E(M)\rightarrow M/E(M)$ given by the identity on
$P$.  

In \cite{sdcPorted}, we
reproduced Las Vergnes' theory with explicit $P$-quotient (matroid)
variables (see the $[Q_i]$ symbols in Corollary \ref{UniversalCor}
in sec. \ref{UniversalSec}) in place of 
$\prod\xi_l^{r_i(P_l)}$.  We then
gave formulas for the $P$-ported Tutte polynomial for the
union and its dual of matroids whose common elements are in $P$.
These formulas  work in a way similar to what appears in
sec. \ref{UniversalSec}.
We extend to algebras
the $\mathbb{Z}[u,w]$-module
generated by the $[Q_i]$ by defining multiplications 
$\tilde{*}$ with the rules
$[Q_i]\tilde{*}[Q_j]=r_{ij}[Q_{i,j}]$, with $r_{i,j}\in\mathbb{Z}[u,w]$ and
$Q_{i,j}=Q_i*Q_j$ depending on $(Q_i,Q_j)$ and whether $*$ represents 
union or its dual.  It is not often recognized that series and parallel
connection of matroids across basepoint $p$ 
is equivalent to matroid union and its dual on matroids with only
element $p$ in common.  We plan to investigate whether the formulas
for parametrized Tutte polynomials of parallel connections
in \cite{Ellis-Monaghan-Traldi} can be generalized to the dual
of union when $|P|>1$, and to detail the relationship when $|P|=1$.

Recent work on a different generalization, weak Tutte functions
(see sec. \ref{Complications}), has been done by 
Ellis-Monaghan and Zaslavsky \cite{ZaslavskyOct18}.
The distinction between weak Tutte functions (satisfying
an additive identity only)
and strong Tutte functions
(which satisfy \eqref{TSM} and \eqref{TA}) seems first to
have been made by Zaslavsky\cite{MR93a:05047}, for matroids.
That paper also defined weak and strong Tutte functions of
graphs.  However, we use using Ellis-Monaghan and Traldi's
definition of strong Tutte functions of 
graphs\cite{Ellis-Monaghan-Traldi}.
The latter restricts \eqref{TA} to non-separators (not just 
non-loops);  and it 
requires $T(G^1)T(G^2)=T(G)$ whenever 
matroids $M(G^1)\oplus M(G_2)=M(G)$ 
(which we sometimes interpret as oriented), not just when
$G$ is the disjoint union of $G^1$ and $G^2$.
The term separator-strong 
(I learned\cite{JoAndTom} 
after \cite{Ellis-Monaghan-Traldi} appeared.)
is used for Tutte functions of matroids and graphs 
as defined in \cite{Ellis-Monaghan-Traldi}; recall that
they satisfy \eqref{TA} and \eqref{TSSM}.
Normal is used in the same way as in\cite{MR93a:05047}.


We introduced $P$-ported parametrized Tutte polynomials
for normal strong Tutte functions
in \cite{TutteEx}, i.e., 
those with
corank-nullity polynomial expressions.
Most of the results in the current paper, when so restricted
appeared in \cite{TutteEx}
or can be derived by adding parameters and oriented matriod considerations
to material in \cite{sdcPorted}.  
These include computation tree\cite{GordonMcMachonGreedoid} based activities 
expansions 
with terms corresponding to $P$-subbases (which are called
``contracting sets'' in \cite{RelTuttePoly}).
%The one non-elementary fact about
%oriented matroids needed is that two subsets $B,B'\in E(M)$ that span
%the same flat define the same $\emph{oriented}$ minor
%$M/B|P=M/B'|P$.  See \cite{OMBOOK}.  
In the normal case, the initial values can be assigned arbitrarilly.
We used this to show the our extensor-valued Tutte-like function 
\cite{sdcPorted} is expressible
by assigning extensors as the initial values.  In this
electrical network application, the indecomposibles are oriented 
graphic matroids
and different values \emph{are} assigned to different orientations of
the same matroid.  

\subsection{Questions}
\label{Qsubsection}
Diao and Hetyei's work \cite{RelTuttePoly} is part of a
larger literature on knot invariants which
leads us to ask if the $P$-ported objects with matroids abstraction, 
and its related Tutte computation tree expansions for parametrized
Tutte functions (sec. \ref{DirectSec}), can be usefully
applied to objects besides graphs or directed graphs, such
as various kinds of knot diagrams.

Another open project is to classify the solutions to
the conditions of Theorem \ref{BigTheorem} (about separator-strong
$P$-ported Tutte functions) and 
Theorem  \ref{StrongTheorem} (about the strong ones) for rings and
for fields along the lines of 
\cite{MR93a:05047} and \cite{BollobasRiordanTuttePolyColored}.

Finally, the theory of objects with matroids or oriented
matroids compared with $P$-ported Tutte functions of 
oriented matroids
raises the question of what role orientation plays in 
$P$-ported Tutte
function theory.  An oriented matroid can be considered to
be an object with a non-oriented matroid.  In other
words, the orientation is a property of the object
that is constrained by, but not uniquely determined by 
the object's non-oriented 
matroid.  However, our results show that when the initial
values
depend only on the orientations in addition
to depending on the matroids, the Tutte function is 
characterized by 
the equations of Theorem \ref{BigTheorem}, not
Theorem \ref{ZBRWellBehaved}.
The reason is that when interchanging the elements
when deleting and contracting a series or parallel pair, 
both the resulting matroids and 
their orientations are the same.  The question we raise
is whether there are interesting objects with matroids
and certain other properties besides matroid orientation
such that Theorem \ref{BigTheorem}
characterizes the Tutte functions, when the initial
values depend only on the matroids and those other properties.

\section{Acknowledgements}

I thank
the Newton Institute for Mathematical Sciences
of Cambridge University for hospitality and support of my
participation in the Combinatorics and Statistical Mechanics
Programme, January to July 2008, during which some of this
work and many related subjects were reviewed and discussed.

I thank the organizers of 
the Thomas H. Brylawsky Memorial Conference, Mathematics Department of
The University of North Carolina, Chapel Hill, October 2008, 
Henry Crapo, Gary Gordon and James Oxley
for
fostering collaboration beween people who carry
on the memory of Prof. Brylawsky and editing this journal
issue.  

I thank Lorenzo Traldi for bringing to my attention and
discussing Diao and Hetyei's work, as well as
Joanne Ellis-Monaghan, Gary Gordon, Elizabeth McMahon
and Thomas Zaslavsky for helpful conversations and communications
at the Newton Institute and elsewhere, and
my University at Albany colleague Eliot Rich for mutual assistance with
writing.

This work is also supported by a Sabbatical leave granted
by the University at Albany, Sept. 2008 to Sept. 2009.



\bibliographystyle{abbrv}
\bibliography{DSumLike}

\end{document}


