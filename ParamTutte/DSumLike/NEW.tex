\documentclass[12pt,leqno]{amsart}
\usepackage{amsmath,amssymb,amsfonts,amsthm}
\usepackage{eucal,graphicx}

\setlength{\textwidth}{6.5in}
\setlength{\oddsidemargin}{0.0in}
\setlength{\evensidemargin}{0.0in}
\setlength{\textheight}{9in}
\setlength{\topmargin}{-.4in}

%\renewcommand{\baselinestretch}{2}     % Activate for double spacing.
%\renewcommand{\baselinestretch}{1.6}   % Activate for 1-1/2 spacing.


\newcommand \comment[1]{}			%  Silent version.
%\renewcommand \comment[1]{\emph{#1}}		%  Comment revealed.
\newcommand \dateadded[1]{\comment{[Date added: #1.]}}
\newcommand \mylabel[1]{\label{#1}\comment{{\rm \{#1\} }}}
\newcommand \myref[1]{\ref{#1}\comment{{\{#1\}}}}

\newtheorem{lem}{Lemma}
\newtheorem{cor}[lem]{Corollary}
\newtheorem{prop}[lem]{Proposition}
\newtheorem{thm}[lem]{Theorem}


\theoremstyle{remark}
\newtheorem{exam}{Example}%[section]
\newcommand \myexam[1]{\smallskip\begin{exam}[\emph{#1}]}

\renewcommand{\phi}{\varphi}
\newcommand\eset{\varnothing}
\newcommand\inv{^{-1}}
\newcommand\setm{\setminus}
\newcommand\chiz{\chi^\bbZ}
\newcommand\bbR{\mathbb{R}}
\newcommand\bbZ{\mathbb{Z}}
\newcommand\cH{\mathcal{H}}

\allowdisplaybreaks

%%%%%%%%%%%%%%%%%%%%%%%%%%%%%%%%%%%%%%%%%

\begin{document}

\title{Ported, Restricted, Set Pointed Parametrized Tutte Functions}

\author{Seth Chaiken}
\address{Computer Science Department\\
The University at Albany (SUNY)\\
Albany, NY 12222, U.S.A.}
\email{\tt sdc@cs.albany.edu}



\begin{abstract}
 \end{abstract}

\subjclass[2000]{...}



\keywords{Tutte function, Tutte polynomial}

\thanks{Version of \today.}

\maketitle
\pagestyle{headings}


%%%%%%%%%%%%%%%%%%%%%%%%
\section*{Introduction}

The Tutte polynomial is a well-known invariant of graphs and matroids.  Several
related polynomials have applications that involve graphs whose edges each bear
one or more weights and/or some of whose edges (or vertices) are distinguished.
In this context, the additive parametrized Tutte identity applied to graph $M$
with edge set $S(M)$
\begin{equation}
\label{A-P-T-identity}
T(M) = x_e T(M/e) + y_e T(M\setminus e) \text{ if $e\in S(M)$ is neither a loop nor a coloop in $M$}
\end{equation}
is assumed only when $e$ is \emph{not} a distinguished edge.  (A distinguished vertex
remains an identifyable element within a set of collapsed original vertices
during each contraction of edges.)  The applications, questions
and results have natural generalizations to matroids.  So, for now, we will
use matroid language and refer to graphs via their graphic matroids.

The purpose of this paper is to extend, to matroids and graphs
a set of distinguished elements $P$,  known results and methods pertaining
to solutions to \ref{A-P-T-identity} together with the multiplicative
identities
\begin{equation}
\label{M-P-T-identities}
T(M)=X_e T(M/e) \text{ if $e\in S(M)$ is a coloop}\\
T(M)=Y_e T(M\setminus e) \text{ if $e\in S(M)$ is a loop}\\
T(M_1\oplus M_2)=T(M_1)T(M_2)
\end{equation}
when these identities are restricted to apply only when
$e\not\in P$.  

When all the $x_e, y_e$ parameters are $0$ and,
independently of $e$, each coloop multiplier $X_e=z$
and each loop multiplier $Y_e=t$, and $T(\emptyset)=1$, 
the solutions have been characterized by Las Vergnas .... .
The theory is ...

Interesting complications arise with other the parameter and multiplier
values.  We extend known results about this to the case when $P\neq\emptyset$.


When the Tutte identities are parametrized, it is especially important to carefully
distinguish between solutions (functions of matroids that satisfy all the relevent
identities) and formal polynomials that result from using a subset of the 
identities to try to calculate $T(M)$ for one $M$.  Unlike in the non-parametrized
case, even without distinguished elements, different formal polynomials
result from different calculations.  These formal polynomials
are in the parameters, values for $T$ of loops, coloops and the initial value $T(\emptyset)$.
Additional polynomial identities in these
values must be satisfied in order for the values together with the to have 
a solution.  A solution of course means that $T(M')$ for all the (matroid) minors of $M$
satisfy all the Tutte identities; equivalently, all calculation sequences give
the same result.  When the parameters are all $1$, the value for each loop is
$t$, the value for each coloop is $z$ and the initial value $T(\emptyset)=1$,
it is well-known that the identities have a unique solution which is a 
polynomial in $z$ and $t$.  That is the famous Tutte polynomial.  It is well-known
as a universal Tutte invariant $U(M)(t,z)$--all Tutte invariants $F(M)$ are obtained
as $F(M)=U(M)(F(U_{0,1}),F(U_{1,1})$, where $U_{0,1}$ is the loop matroid on 
any ground set $\{e\}$ and $U_{1,1}$ is the coloop matroid.
Since additional
conditions involving parameters, loop/coloop, and empty matroid values are
needed for a solution, we will avoid the term ``Tutte polynomial'' for now
and use the term Tutte function to denote any solution to the parametrized Tutte equations.

Several recent papers have demonstrated the essential equivalences between most
different, seemingly incompatible forms for solutions of parametrized Tutte equations
(which have been called ``Tutte polynomials'').  Their conclusions are that, like
for Tutte invariants, universal solutions can be given.





\end{document}
