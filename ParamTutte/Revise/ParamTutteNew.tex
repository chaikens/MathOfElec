\documentclass[12pt,leqno]{amsart}

\usepackage{rcs}


\usepackage{amsmath,amssymb,amsfonts,amsthm}
\usepackage{eucal,graphicx}
\usepackage{color}
\usepackage[pdftex]{hyperref}


\setlength{\textwidth}{6.5in}
\setlength{\oddsidemargin}{0.0in}
\setlength{\evensidemargin}{0.0in}
\setlength{\textheight}{9in}
\setlength{\topmargin}{-.4in}

%\renewcommand{\baselinestretch}{2}     % Activate for double spacing.
%\renewcommand{\baselinestretch}{1.6}   % Activate for 1-1/2 spacing.
%\renewcommand{\baselinestretch}{1.3}   % Activate for 1-1/3 spacing.


\newcommand \comment[1]{}			%  Silent version.
%\renewcommand \comment[1]{\emph{#1}}		%  Comment revealed.
\newcommand \dateadded[1]{\comment{[Date added: #1.]}}
\newcommand \mylabel[1]{\label{#1}\comment{{\rm \{#1\} }}}
\newcommand \myref[1]{\ref{#1}\comment{{\{#1\}}}}

\newtheorem{lem}{Lemma}
\newtheorem{cor}[lem]{Corollary}
\newtheorem{prop}[lem]{Proposition}
\newtheorem{thm}[lem]{Theorem}
\newtheorem{definition}[lem]{Definition}


\theoremstyle{remark}
\newtheorem{exam}{Example}%[section]
\newcommand \myexam[1]{\smallskip\begin{exam}[\emph{#1}]}

%\renewcommand{\phi}{\varphi}
\newcommand\eset{\varnothing}
\newcommand\inv{^{-1}}
\newcommand\setm{\setminus}
\newcommand\chiz{\chi^\bbZ}
\newcommand\bbR{\mathbb{R}}
\newcommand\bbZ{\mathbb{Z}}
\newcommand\cH{\mathcal{H}}


\newcommand\Ueloop{\ensuremath{U^e_{0}}}
\newcommand\Uecoloop{\ensuremath{U^e_{1}}}
\newcommand\Uefdyad{\ensuremath{U^{ef}_{1}}}
\newcommand\Uefgtriad{\ensuremath{U^{efg}_{1}}}
\newcommand\Uefgtriangle{\ensuremath{U^{efg}_{2}}}

%   Disjoint Union
%\newcommand{\dunion}{\uplus}
\newcommand{\dunion}
%{\mbox{\hbox{\hskip4pt$\cdot$\hskip-4.62pt$\cup$\hskip2pt}}}
{\mbox{\hbox{\hskip6pt$\cdot$\hskip-5.50pt$\cup$\hskip2pt}}}
%
% Dot inside a cup.
% If there is a better, more Latex like way 
% (more invariant under font size changes) way,
% I'd like to know.
\newcommand{\dunionsubscript}
{\mbox{\scriptsize\hbox{\hskip4pt$\cdot$\hskip-4.35pt$\cup$\hskip2pt}}}


\newcommand{\Bases}[1]{\ensuremath{{\mathcal{B}}(#1)}}
\newcommand{\Reals}{\ensuremath{\mathbb{R}}}
\newcommand{\FieldK}{\ensuremath{K}}
\newcommand{\Perms}{\ensuremath{\mathfrak{S}}}
\newcommand{\rank}{{\rho}}% {{\mbox{rank}}}
\newcommand{\Rank}{{\rho}}% {{\mbox{rank}}}
\newcommand{\Card}[1]{\ensuremath{{\left|#1\right|}}}
\newcommand{\ext}[1]{\ensuremath{\mathbf{#1}}}

% Set Complement
% command to mess with overline, bar or custom 
% alternatives for sequence or set complement
%
%\newcommand{\scomp}[1]{\ensuremath{\;\overline{#1}\;}}
%\newcommand{\scomp}[1]{\ensuremath{\bar{#1}}}
%\newcommand{\scomp}[1]{\ensuremath{\genfrac{}{}{}{}{}{#1}}}
\newcommand{\scomp}[1]{\ensuremath{\overline{#1}}}

\allowdisplaybreaks


%%%%%%%%%%%%%%%%%%%%%%%%%%%%%%%%%%%%%%%%%

\begin{document}

\title[Ported Separator-Strong Parametrized Tutte Functions]
{Ported Separator-Strong 
Parametrized Tutte Functions}

\author{Seth Chaiken}
\address{Computer Science Department\\
The University at Albany (SUNY)\\
Albany, NY 12222, U.S.A.}
\email{\tt sdc@cs.albany.edu}



\begin{abstract}
We discuss Tutte functions of parametrized matroids in which
certain ``port'' elements have been distinguished.
The port elements are held back from deletion and contraction
during Tutte decompositions.  We prove conditions, specified on the parameters
and other relevant values,
that are necessary and sufficient for a ported Tutte function to be well-defined
on ported minor-closed classes.
They generalize together known  
conditions for definedness on minor-closed sets but with no ports, and
for definedness everywhere but with the port restriction.  

The ported Tutte functions might have different values
on different orientations of the same matroid.
The 
application to graphs where these
differences do occur is motivated by
expressions of solutions to electrical resistive network
and other discrete Laplacian-based problems by new kinds of Tutte functions.

An activities based expression and the corresponding
interval partition of the boolean lattice of port-free subsets
are 
given for any recursive computation of a Tutte function value, 
not just one based on an element order.  These are then specialized to
a corank-nullity and a geometric lattice expansion for the ported
generalization of normal Tutte functions.
\end{abstract}

\subjclass[2000]{Primary 05B35; Secondary 05C99, 05C15, 57M25, 94C05}



\keywords{Tutte function, Tutte polynomial, Kirchhoff, Oriented Matroid,
pointed, restricted, relative, ported}

\thanks{New Version \today.}

\maketitle
\pagestyle{headings}

\RCSID $Id: ParamTutteNew.tex,v 1.28 2010/12/31 06:05:05 seth Exp seth $

%%%%%%%%%%%%%%%%%%%%%%%%

\section{Introduction}

Two natural and useful generalizations
of the Tutte equations and the resulting
Tutte polynomials $T$ and decompositions are known wherein:
\begin{enumerate}
\item A set $P$ of distinguished elements, which
we call ``ports'', is specified.  The
operations of deleting and contracting 
non-separator $e$,
and of removing a separator $e$
so that $T(G\oplus e)=XT(G)$ or $YT(G)$ depending
on whether $e$ is a coloop or loop, are then
restricted only to $e\not\in P$.  See \cite{SetPointedLV,MR0419272}
where the Tutte polynomials are called ``set-pointed'',
\cite{sdcPorted}, and Brylawsky's early work 
\cite{BrylawskiPointed} on the $|P|=1$ case.
\item
Four parameters $x_e, y_e, X_e, Y_e$ are 
given for each element $e$.  $x_e$ and $y_e$
are coefficients in the additive Tutte 
equation for contracting and deleting
$e$ when $e$ is a non-separator.  $X_e$
and $Y_e$ generalize the common $X$ and $Y$
in (1) to different parameters for
different $e$.
The resulting additive 
Tutte equation is $T(G)=x_eT(G/e)+y_eT(G\setminus e)$.
When a set $P$ of port elements (1) is specified,
parameters are given only for $e\not\in P$.
\end{enumerate}

We use the separator multiplicative Tutte equation shown
in (1) because the general multiplicative equation
$T(G_1\oplus G_2) = T(G_1)T(G_2)$ can be derived from it given
additional assumptions, in particular 
$T(\emptyset)=T(\emptyset)T(\emptyset)$ when $P=\emptyset$
\cite[Corollary 2.3]{Ellis-Monaghan-Traldi}.
For this note, we only address consequences of generalizations
(1) and (2).  We adopt Zaslavsky's terminology\cite{MR93a:05047} to 
call the resulting functions \textbf{separator
strong Tutte functions}.
Ported generalizations of 
the 
formula
$T(\emptyset) T(G_1\oplus G_2) 
= T(G_1) T(G_2))$ \cite{Ellis-Monaghan-Traldi}
and related results about strong Tutte functions\cite{MR93a:05047},
direct sums 
and graphs (based on 
\cite{MR93a:05047,BollobasRiordanTuttePolyColored,Ellis-Monaghan-Traldi}) are 
left for future publication.

The underlying problem with the parametrized generalizion (2)
is that different computations of $T(G)$ from the separator
multiplicative and additive Tutte equations and the value of $T$ on $\emptyset$
give different results.  Only for special
parameter values (such as $X_e = X$, $Y_e = Y$ and $x_e = y_e = 1$
for all $e$,
which define classical Tutte polynomials) do the Tutte equations have a
solution.  
This problem has been resolved by Zaslavsky 
who published the so-called ZBR conditions\cite{Ellis-Monaghan-Traldi} 
on the parameter values for collections of Tutte equations to
have a solution 
\cite{MR93a:05047,Ellis-Monaghan-Traldi}.
Those conditions were applied to classify the Tutte functions into
fields\cite{MR93a:05047} and rings\cite{BollobasRiordanTuttePolyColored}.
Two equivalent algebraic formulations 
of the ZBR conditions have been given.  One can specify a set of 
polynomial equations on the 
parameters necessary and sufficient for $T(G)$ to be defined from
the Tutte equations for all $G$ in a given domain.  Alternatively, given the
domain, the range of
$T$ can be specified by the quotient module or ring 
$R[\{x_e, y_e, X_e, Y_e\}]/I$ 
%$R[\Lambda]/I$ 
where
$I$ is the ideal generated by the polynomials $f$, for each equation
$f = 0$ in the set.  

In either formulation, a solution to the Tutte equations on some domain
is called a Tutte function.  A polynomial expression for a Tutte function
that is universal for all those Tutte functions is called a Tutte polynomial.
When the parameters (for us, the $x_e, y_e, X_e, Y_e$) and
values ($T(Q)$ for indecomposibles $Q$) are considered to be the variables,
the Tutte polynomial is non-unique\cite{TomJul09}.  We can then
say that the ZBR conditions (on the values of the parameters
and the $T(Q)$) are necessary and sufficient for all the
Tutte polynomials for the same graph or matroid to express the same value.
  
The subject of this note is extending the ZBR conditions
to the combination of generalizations of (1) and (2) slightly
beyond the work of Diao and Hetyei, together with
ported generalizations of various expansions of Tutte functions,
and the emergance of paramatrized Tutte functions depending
on matroid orientation when the port constraint (1) is imposed.
The development suggests that it is more natural to consider
a Tutte function to be determined by the separator multiplicative
equations with parameters ($X_e$ and $Y_e$) 
plus the \textbf{initial value} $T(\emptyset)$
rather than by considering the loop and coloop graphs or matroids
to be the indecomposible objects.  
With the port constraint, our indecomposibles
are (or in the case of graphs, have) matroids, including $\emptyset$,
on zero or more port elements only.


\subsection{ZBR conditions and indecomposibles}

The first step towards characterizing Tutte functions
over various choices of field or ring values and function domain
was taken by Zaslavsky\cite{MR93a:05047}. He used induction on the
number of matroid elements to show that the conditions are exactly
those for the existance of Tutte functions on only three very simple
kinds of matroids and all their minors: 
The ``dyad'' or uniform rank 1 matroid on two elements, the
``triad'' or uniform rank 1 matroid on three elements and the 
``triangle'' or uniform rank 2 matroid
on three elements.  See the top of
figure \ref{small}.  
Diao and Hetyei found that the three ZBR conditions on the 
parameters are necessary
and sufficient for ported (which they called ``relative'')
Tutte functions to be defined 
for all graphs and matroids.  Our generalization consists of
ZBR-type conditions that apply to five simple kinds of 
configurations within an appropriately restricted minor closed 
class.  These conditions are necessary and sufficient for
ported Tutte functions to be defined on the given class.
They are illustrated in the bottom of figure \ref{small}.
Each condition applies to only certain 
matroids in the class with only two or three non-port elements.
The minor $Q$ mentioned in the condition is an 
\textbf{indecomposible}, which means that
none of the Tutte equations apply to $T(Q)$.

When $T(G)$ is well-defined, it can then
be expressed by a not necessarilly unique polynomial in the parameters
and $T(Q)$ on indecomposibles $Q$.  Generalizing $T(\emptyset)$, we call
such $T(Q)$ \textbf{initial values}.
Generally speaking, a function $T$,
not necessarilly a classical
polynomial, is called a \textbf{separator-strong} \cite{JoAndTom} 
\textbf{ported} \cite{sdcPorted} \textbf{Tutte function} when it
satisfies the parametrized additive and separator multiplicative
Tutte equations.
Such functions are sometimes only defined
on a particular domain of matroids, graphs, and
certain matroid representations\cite{TutteEx}.

%\begin{enumerate}
%\item

Theorem \ref{BigTheorem} demonstrates how
the 
characterizations\cite{MR93a:05047,BollobasRiordanTuttePolyColored,Ellis-Monaghan-Traldi} of Tutte functions
defined on minor-closed subsets $\mathcal{G}$ generalize the
conditions given by Diao and Hetyei for relative
Tutte functions\footnote{It is unfortunate that the term ``Tutte polynomial''
is often used in the sense of our usage of ``Tutte function'', even in
situations where the difference matters.}  
to be defined everywhere.  Thus the
programs of Zaslavsky \cite{MR93a:05047}, 
and of Bollob\'{a}s and Riordan 
\cite{BollobasRiordanTuttePolyColored} of classifying
all Tutte functions may be carried out in the 
ported or relative case.  

Diao and Hetyei's conditions consist
of Zaslavsky's 
conditions on all dyads, triads and 
triangles\footnote{When the conditions apply to \textit{all} dyads 
and triads over a ground set, it is immediate that they apply to all
the triangles also.}
of non-port elements
together with a ``block invariant'' property of
the Tutte function restricted to graphs or matroids with
port elements only.

Our's consist of 
one equation $T(Q)f=0$ for each instance of 
of one of five types of $G\in\mathcal{G}$, where
all but two or three of $G$'s elements
are ports
and $Q$ is a minor of $G$ obtained by removing 
the non-port elements.  In the first three,
$f=0$ expresses Zaslavsky's conditions on dyads,
triangles and triads on non-ports.  The last two resemble the
triangle and triad conditions in that one of the 
three elements is replaced with a quotient $Q$ of 
$Q_0$, where $Q_0$ is connected to the other two elements
in either a series or parallel pair.  See figure \ref{small}.

\begin{figure}
\input{smallmatroids.pdf_t}
\caption{\label{small} Dyad, triangle and triad 
\cite{MR93a:05047} represented by graphs.
Below them are the five kinds of matroids for which our conditions
are derived by using that two computations from Tutte equations
must give the same value.  Each condition has the
form $T(Q)f=0$ where $f$ is a polynomial in the parameters
for $e$, $f$, and, in two of the five cases, $g$.
In the last case, $f=(x_eY_f+y_ex_f)-(x_fY_e+y_fx_e)$.}
\end{figure}

We defer our proof to the end because the 
result is such a mild generalization of 
Diao and Hetyei's work.  Our proof avoids 
introducing an activities-based expression
based on a linear ordering of the elements.

It should be noted that if Tutte functions
over proper subclasses of matroids or graphs are considered,
then each condition derives from a particular subclass
member.  The three independent kinds of conditions found by
Zaslavsky must be replaced by five independent kinds when we
generalize with a non-empty set of ports. 
We were surprised that although each condition
is $F=0$ for some a polynomial $F$ that includes 
parameters for 2 or 3 elements of $E$,
each such $F=T(Q)f$ includes 
\emph{one and only one $T(Q)$} which occurs as 
a factor.

%\item
\subsection{Tutte functions depending on orientations}

The resulting
ported Tutte functions might depend on the orientation
of an oriented matroid $M$, not just on the
underlying matroid.  Here is the simplest case.
Let $Q^+$ be the oriented matroid
on port elements $\{p,q\}$ consisting of a 2-circuit
with signature $\pm(+,+)$ and $Q^-$ be the one with
signature $\pm(+,-)$.  An example is any ported Tutte
function $T$ with $T(Q^+)=1$ and $T(Q^-)=-1$.  Note that
the distinct oriented matroids $Q^{\pm}$ have identical 
underlying matroids.

An electrical resistor network is one of many analogs
modeled by the 
discrete Laplacian matrix\cite{DoyleSnellBook}.
It has been known from Kirchhoff\cite{Kirchhoff} that 
the transfer resistance
(called transpedance in \cite{BSST,TutteGraphBook}), 
is the ratio of two polynomials whose monomials 
are $\pm$ products of conductances $g_e$ over sets of edges,
where each set is a kind of forest.  
Transfer resistance is usually defined given two 
pairs of nodes.  It is the potential difference
across one pair due to unit current flowing in and
out at the other pair.
Suppose two new edges $\{p,q\}$ are introduced
solely to demark the two given pairs of nodes.
We take $p$ and $q$ to be ports and we do not 
supply parameters or conductances $g$ for them.
The denominator is the ported Tutte function that enumerates
the spanning trees containing neither $p$ nor $q$.  
Although the number of terms in the numerator
is also ported Tutte function of the underlying graphic, non-oriented 
matroid\cite{sdcPorted}, 
the numerator itself cannot be because the signs of the terms
vary.  

It is an immediate result of our explicit extension of
Diao and Hetyei's theory to oriented matroids 
that the above numerator is \textit{also} a ported Tutte function.
It is the extension of $T$ above with $T(Q')=0$ for
all decomposibles except for $U^r_{pq}$ and
all $x_e=X_e=g_e$, $y_e=Y_e=1$.   The positive terms
(multiples of $T(U^+_{ef})=1$) correspond to spanning forests
$F$ with exactly two trees such that $F\cup \{p,q\}$ contains
a unique circuit and, when oriented, that signed circuit contains 
$p$ and $q$ in the same direction.  The negative terms 
include $T(U^-_{ef})=-1$.
%are analogous.
  
We had developed this subject in \cite{TutteEx}
where the Tutte equations had to be verified directly.
We had demonstrated and applied a variation of Tutte functions
of linear matroids whose values lie in an exterior algebra,
so this is not subsumed by the present work.

%\item
\subsection{Tree-based activity expressions}

When the conditions are satisfied, all recursive
computations based on the ported Tutte equations give the
same result.  The Tutte computation tree based formulation
of activities
\cite{GordonMcMachonGreedoid}
enables us to 
define an activities-type expansion for every such
computation, not just those based on a linear
ordering of $E$.  The known theory generalizes to
the ported case, including for example, interval
partitions of the Boolean lattice $2^{E}$.  
There is 
one term and boolean interval for
each independent subset $F\subseteq E$ for
which $F\cup P$ is spanning; such $F$ specialize
to bases when $P=\emptyset$.
The blocks
of such interval partitions are then partitioned according
to the common $Q=M/A|P$ for $A$ in each block.

Our inductive, non-activities proof 
facilitates
the generalization Diao and Hetyei's activities based
expression of $T(G)$ (needed in their proof)
so that the activities are not based on a fixed
ordering of elements.  Instead, the activities 
are based on an arbitrary recursive calculation of
$T(G)$ expressed as a tree as in \cite{NegamiPoly}.  This result applies the
idea of the ``computation tree'' based activities expression
for Tutte polynomials of greedoids given
by Gordon and MacMahon to ported or restricted Tutte 
polynomials of matroids or graphs\cite{GordonMcMachonGreedoid}.

%\item 
\subsection{Ported normal Tutte functions}
We conclude the note by addressing the 
ported generalization
of \textbf{normal} Tutte functions, named by 
Zaslavsky as those that have corank-nullity expressions,
in contrast 
to the activity-based expressions that exist for every
Tutte function.

%\end{enumerate}

\section{Matroid Tutte Function Characterization}
\label{ParamTutteSec}


\begin{definition}
\label{SSTMDefinition}
Let $P$ be a set of \textbf{ports} and $\mathcal{C}$ be a class of matroids or
oriented matroids closed under taking minors or oriented minors
by deleting or contracting only elements $e\not\in P$.
Such a $\mathcal{C}$ is called a \textbf{$P$-family}.  
Suppose $\mathcal{C}$ is
given with four parameters $(x_e,y_e,X_e,Y_e)$,
each in a commutative ring $R$, for each $e\not\in P$ that is an element
in some $M\in\mathcal{C}$.

The ground set of $M\in\mathcal{C}$ is denoted by $S(M)$ and
$E(M) = S(M)\setminus P$, the ground set elements of $M$ that are 
not ports.  


A \textbf{separator-strong parametrized $P$-ported Tutte function}
$T$ maps $\mathcal{C}$ to $R$ or to an $R$-module and satisfies 
conditions 
\eqref{TA} and \eqref{TSSM} below for all $M\in\mathcal{C}$ and
all $e$ in $E(M)$.

\begin{equation}
\label{TA}
\tag{TA}
\begin{gathered}
T(M) = x_e T(M/e) + y_e T(M\setminus e) \\
\text{ if $e\not\in P$ and $e$ is a non-separator, 
       i.e., neither a loop nor a coloop.}
\end{gathered}
\end{equation}


\begin{equation}
\label{TSSM}
\tag{TSSM}
\begin{gathered}
\text{If } e\not\in P\text{ is a coloop in }M\text{ then }
T(M)=X_e T(M/e).\\
\text{If } e\not\in P\text{ is a loop in }M\text{ then }
T(M)=Y_e T(M\setminus e).
\end{gathered}
\end{equation}

\end{definition}
The $Q\in \mathcal{C}$ for which neither \eqref{TA} nor
\eqref{TSSM} apply, equivalently, $E(Q)=\emptyset$,
are called the \textbf{indecomposibles} or
\textbf{$P$-quotients} in $\mathcal{C}$.
$U^{ef\ldots}_r$ denotes the uniform rank $r$ matroid 
on $\{e,f,\ldots\}\subseteq E$.

\subsection{Paramaters and Initial Values}

Classifications of parametrized Tutte functions $T$ for
given rings (or fields as rings) and given function domain
are partly based on the value $T(\emptyset)$.  For example,
Tutte functions traditionally satisfy 
$T(G_1\oplus G_2) = T(G_1)T(G_2)$ for all direct summands,
while in \cite{MR93a:05047} separator-strong Tutte functions are
the generalization where $G_1$ or $G_2$ is required to be a
separator.   Hence $T(\emptyset)\neq T(\emptyset)T(\emptyset)$ is 
possible for separator-strong Tutte functions.  

Here we take a different approach which seems to be more natural for
separator-strong ported Tutte functions with $P\neq \emptyset$.  
Instead of specifying a Tutte function by parameters denoted above by
$x_e, y_e$ together with ``initial values'' $T(U^r_e)$ for each rank $r=0$ or
$r=1$ matroid on a singleton ground set $\{e\}$, we provide
four parameters $x_e,y_e, X_e, Y_e$ for each $e\in E$, and
$T(U^r_e)$ is determined by a Tutte equation $T(U^0_e)=Y_eT(\emptyset)$
or $T(U^0_e)=Y_eT(\emptyset)$.  We consider $\emptyset$ (assuming
$\emptyset\in\mathcal{G}$) to be an indecomposible.  The
separator multiplicative Tutte equations 
$T(U^0_e\oplus G)=Y_eT(G)$
and 
$T(U^1_e\oplus G)=X_eT(G)$ therefore subsume the special case
of an ``initial value'' needed for a loop or coloop $e$ when $e\not\in P$.

We use the term \textbf{initial value} for $T(Q)$ whenever $Q$ is 
an indecomposible.  (In \cite{MR93a:05047}, the $T(U^r_e)$ are called initial
values, whereas for us, for example, $T(U^1_e)=X_eT(\emptyset)$ where
$X_e$ is called a parameter and $T(\emptyset)$ is called an initial value.) 

\subsection{Ported Matroid ZBR Theorem}


Proposition \ref{SameMinorProp} helps simplify the statement
of Theorem \ref{BigTheorem}


\begin{prop}
\label{SameMinorProp}
Suppose $e,f$ are in series, or are in parallel, in matroid or
oriented matroid $M$.
%\begin{enumerate}
%\item 
(1) The minors $M/e\setminus f=M/f\setminus e$
are equal as matroids.
%\item 
(2) If $M$ is oriented, the oriented minors 
$M/e\setminus f=M/f\setminus e$ are equal as oriented matroids.
%\end{enumerate}
\end{prop}

\begin{proof}
If $e,f$ are in series, note that $M/e\setminus f$ $=$
$M\setminus f/e$.  $e$ is a coloop in $M\setminus f$,
so $M\setminus f/e$ $=$ $M\setminus\{e,f\}$, which is
clearly the same matroid or oriented matroid if $e,f$
are interchanged.  The relevant theory of minors of oriented matroids
can be found in \cite{OMBOOK}.

If $e,f$ are in parallel, $e,f$ are in series in 
the matroid or oriented matroid dual $M^*$ of $M$.
By the first case, $M^*\setminus e/ f$ $=$
$M^*\setminus f/e$ as matroids or as oriented matroids.
Thus $M/e\setminus f$ $=$ $(M^*\setminus e/ f)^*$ $=$
$(M^*\setminus f/ e)^*$ $=$ $M/f\setminus e$ as matroids or
as oriented matroids.
\end{proof}

\begin{thm}
\label{BigTheorem}
The following two statements are equivalent.
\begin{enumerate}
\item 
Given an initial value $T(Q)$ in an $R$-module for each
$P$-quotient $Q\in\mathcal{C}$, the Tutte equations
\eqref{TA} and \eqref{TSSM}
extend $T$ to a unique $P$-ported separator-strong
parametrized Tutte function with
with $R$-parameters $(x, y, X, Y)$.
\item
\begin{enumerate}
\item For every $M=U^{ef}_1\oplus Q\in\mathcal{C}$ with 
$P$-quotient $Q$ ($U^{ef}_1$ is a dyad), 
\[
T(Q)(x_e Y_f + y_e X_f) = 
T(Q)(x_f Y_e + y_f X_e).
\]
\item
For every $M=U^{efg}_2\oplus Q\in\mathcal{C}$ with 
$P$-quotient $Q$ ($U^{efg}_2$ is a triangle), 
\[
T(Q)X_g(x_e y_f + y_e X_f) = 
T(Q)X_g(x_f y_e + y_f X_e).
\]
\item
For every $M=U^{efg}_1\oplus Q\in\mathcal{C}$ with 
$P$-quotient $Q$  ($U^{efg}_1$ is a triad), 
\[
T(Q)Y_g(x_e Y_f + y_e x_f) = 
T(Q)Y_g(x_f Y_e + y_f x_e).
\]
\item
If $\{e,f\}=E(M)$ is a parallel pair connected to $P$, 
\[
T(Q)(x_e Y_f + y_e x_f) = 
T(Q)(x_f Y_e + y_f x_e)
\]
where $P$-quotient $Q=M/e\setminus f=M/f\setminus e$.
\item
If $\{e,f\}=E(M)$ is a series pair connected to $P$, 
\[
T(Q)(x_e y_f + y_e X_f) = 
T(Q)(x_f y_e + y_f X_e)
\]
where $P$-quotient $Q=M/e\setminus f=M/f\setminus e$.
\end{enumerate}
\end{enumerate}
\end{thm}

When $P=\emptyset$, this reduces to the
Zaslavsky-Bollob\'{a}s-Riordan theorem 
for matroids\cite{Ellis-Monaghan-Traldi}:
In that case, only $Q=\emptyset$ occurs and the last two conditions become
vacuous.

\subsection{Universal Tutte Polynomial}
\label{UniversalSec}
It is easy to follow 
\cite{BollobasRiordanTuttePolyColored,Ellis-Monaghan-Traldi} to define
a universal, i.e., most general $P$-ported parametrized
Tutte function $T^{\mathcal{C}}$ for $\mathcal{C}$.
We take indeterminates $x_e, y_e, X_e, Y_e$ for each $e\in E(M), 
M\in\mathcal{C}$
and an indeterminate $[Q]$ for each $P$-quotient $Q\in\mathcal{C}$.
Let $\mathbb{Z}[x,y,X,Y]$ denote the integer polynomial ring generated by
the $x_e,y_e,X_e,Y_e$ indeterminates, and define $\widetilde{\mathbb{Z}}$
to be the $\mathbb{Z}[x,y,X,Y]$-module generated by the $[Q]$.  
Let $I^{\mathcal{C}}$ denote the ideal of $\widetilde{\mathbb{Z}}$ 
generated by the identities of Theorem \ref{BigTheorem}, comprising 
for example $[Q](x_eY_f+y_eX_f-x_fY_e-y_fX_e)$ for each instance of
case (a), etc.  The universal Tutte function has values in the
quotient module $\widetilde{\mathbb{Z}}/I^{\mathcal{C}}$.  Finally,
observe that the range of Tutte function $T$  can be considered to be the
$R$-module generated by the values $T(Q)$ where ring $R$ contains the
$x,y,X,Y$ parameters. If the $T(Q)\in R$, consider
the ring $R$ to be the $R$-module generated by $R$.
We follow \cite{Ellis-Monaghan-Traldi} to write
the corresponding consequence of Theorem \ref{BigTheorem}:

\begin{cor}
\label{UniversalCor}
Let $\mathcal{C}$ be a $P$-minor closed class of matroids or 
oriented matroids.  Then there is a 
$\widetilde{\mathbb{Z}}/I^{\mathcal{C}}$-valued function 
$T^{\mathcal{C}}$ on $\mathcal{C}$ with $T^{\mathcal{C}}(Q)=[Q]$ for each $P$-quotient
$Q\in\mathcal{C}$ that is a $P$-ported parametrized Tutte function
on $\mathcal{C}$ where the parameters are the 
$x, y, X, Y \in \widetilde{\mathbb{Z}}$.  Moreover, if $T$ is any $R$-parametrized
Tutte function with parameters $x'_e, y'_e, X'_e, Y'_e$, then $T$ is the
composition of $T^{\mathcal{C}}$ with the homomorphism determined by
$[Q]\rightarrow T(Q)$ for $P$-quotients $Q$ and 
$x_e\rightarrow x'_e$, etc., for each $e\in E(\mathcal{C})$.
\end{cor}

%\section{A Tutte function depending on matroid orientation}

%Our synthesis began with
%\cite{TutteEx}, where the term ``port'' is
%motivated by an application;

%{\LARGE Whoppie!  Write me soon!  OR DELETE ME!! UGH...}

\section{Tutte Computation Trees and Activities}
\label{Activity}

Zaslavsky\cite{MR93a:05047} noted that the 
\emph{basis} or \emph{activities}
expansion, introduced by Tutte
\cite{TutteDich,TutteGraphBook} for graphs, applies
to all well-defined parametrized 
Tutte functions whereas the 
parametrized corank-nullity generating function
expresses only a proper subset of them, which he called
normal.  We show
how to give activities-type expansions comprised of
one expansion for every recursive computation of
a ported Tutte function value, not just those
determined by a linear element ordering.
We apply to 
ported matroids the expansions based on
activities defined with Tutte computation trees, which 
Gordon and McMahon \cite{GordonMcMachonGreedoid}
used to demonstrate
that a definition of Tutte polynomials
of greedoids is consistant with Tutte equations.
Unlike matroids, some greedoids
do not have activities expansions for their Tutte polynomials
that derive from element orderings\cite{GordonMcMachonGreedoid}.  
When we have $P\neq\emptyset$,
each tree leaf is generally one of many indecomposibles. 
The use of trees whose leaves are indecomposibles 
for analyzing a variation of Tutte decompositions for graph
appears in \cite{NegamiPoly}.

Ellis-Monaghan and 
Traldi \cite{Ellis-Monaghan-Traldi} remarked that the Tutte equation
approach appears to give a shorter proof of the ZBR theorem
than the activities expansion approach. 
The proofs by induction on $|E|$
demonstrate that every calculation of $T(M)$ from Tutte equations
produces the same result when the conditions on the parameters
and initial values are satisfied.  We then know immediately that
the polynomial expression
resulting from
a particular calculation equals the Tutte function value.
We suggest a heuristic reason why
the Tutte equation approach is more succinct:
The induction assures that
\emph{every} computation
with smaller $|E|$ gives the same result, not just those 
computations that are determined by linear orders on $E$.
%
Proofs of activities expansions for matroids, and their generalizations
for $P$-ported matroids, seem more informative and certainly
no harder when the expansions are derived from a general Tutte
computation tree, than when the expansions are only those
that result from an element order.   
From the retrospective that the Tutte equations
specify a non-deterministic recursive computation
\cite{Garey-Johnson}, it seems artificial to start with
element-ordered computations and then prove
first that all linear orders give the same result and 
second that it 
satisfies the Tutte equations, in order prove that 
all recursions give the same result.
We therefore
take advantage of the Tutte computation tree formalism
and the more general expansions it enables.

\subsection{Computation Tree Expansion}
\begin{definition}
Given $P$-ported matroid or oriented matroid $M$,
a \textbf{$P$-subbasis} $F$
is an independent set  with $F\subseteq E(M)$
(so $F\cap P=\emptyset$) for which $F\dunion P$ is a spanning set
for $M$
(in other words, $F$ spans $M/P$).
$\mathcal{B}_P(M)$ denotes the set of $P$-subbases.
\end{definition}

Equivalent definitions were given in \cite{SetPointedLV}
and in \cite{RelTuttePoly}.   In the latter, $F$ is called a
``contracting set.'' The following is immediate and
useful:

\begin{prop}
For every $P$-subbasis $F$ there exists an independent set $Q\subseteq P$
that extends $F$ to a basis $F\dunion Q\in \mathcal{B}(M)$ of $M$.
Conversely, if $B\in\mathcal{B}(M)$ then $F=B\cap E=B\setminus P$
is a $P$-subbasis.
\end{prop}

\begin{definition}[Computation Tree, following \cite{GordonMcMachonGreedoid}]
\label{CompTreeDef}
A $P$-ported (Tutte) computation tree for $M$ is a
binary tree whose root is labeled by $M$ and which satisfies:
\begin{enumerate}
\item If $M$ has non-separating elements not in $P$, then 
the root has two subtrees and there exists one such element $e$ for which 
one subtree is a computation tree
for $M/e$ and the other subtree is a computation tree for 
$M\setminus e$.

The branch to $M/e$ is labeled with ``$e$ contracted'' and 
the other branch is labeled ``$e$ deleted''.
\item Otherwise (i.e., every element in $E(M)$
is separating) the root is a leaf.  

Each leaf is labelled with a $P$-quotient $Q$ plus a direct sum
of loop or coloop matroids on elements respectively $e$ or $f$ in $E$, 
expressed
as $[Q]\prod X_e \prod Y_f$.
\end{enumerate}
\end{definition}

%Figure \ref{c4p2TwoFigure} illustrates two computation trees; it 
%is conventional to slant contraction edges to the left and
%deletion edges to the right.

\begin{definition}[Activities with respect to a leaf]
\label{ActivityTreeDef}
For a $P$-ported Tutte computation tree for $M$,
a given leaf, and the path from the root to this leaf:
\begin{itemize}
\item Each $e\in E(M)$ labeled ``contracted'' along this path
is called \textbf{internally passive}.
\item Each coloop $e\in E(M)$ in the leaf's matroid is
called \textbf{internally active}.
\item Each $e\in E(M)$ labeled ``deleted'' along this path
is called \textbf{externally passive}.
\item Each loop $e\in E(M)$ in the leaf's matroid is
called \textbf{externally active}.
\end{itemize}
\end{definition}

\begin{prop}
Given a leaf of a $P$-ported Tutte computation tree for $M$,
the set of internally active or internally passive elements 
constitutes a 
$P$-subbasis of $M$ which we say 
\textbf{belongs to the leaf}.  
Furthermore, every $P$-subbasis $F$ of $N$ belongs to a unique leaf.
\end{prop}

\begin{proof}
For the purpose of this proof, let us extend Definition \ref{ActivityTreeDef}
so that, given a computation tree with a given node $i$ 
labeled by matroid $M_i$,
$e\in E$ is called internally passive when $e$ is labeled 
``contracted'' along the path from root $M$ to
node $i$.  Let $IP_i$ denote the set of such internally passive 
elements.

It is easy to prove by induction on the length of the root to node $i$ path
that
(1) $IP_i\cup S(M_i)$ spans $M$ and 
(2) $IP_i$ is an independent set in $M$.  The proof
of (1) uses the fact that elements labeled deleted are non-separators.  The
proof of (2) uses the fact that for each non-separator 
$f\in M/IP_i$, $f\cup IP_i$ is independent in $M$.

These properties applied to a leaf demonstrate the first conclusion,
since each $e\in E$ in the leaf's matroid must be a separator by Definition 
\ref{CompTreeDef}.

Given a $P$-subbasis $F$, we can find the unique leaf with the
algorithm below.  Note that it also operates on arbitrary subsets of $E$.

\textbf{Tree Search Algorithm:} Beginning
at the root, descend the tree according to the rule: At each branch node,
descend along the edge labeled ``$e$-contracted'' if $e\in F$ and along
the edge labeled ``$e$-deleted'' otherwise (when $e\not\in F$).
\end{proof}

We leave the reader to check that 
the classsical element-order based activities
expansion, as extended with ports explicitly
in\cite{RelTuttePoly}, is reproduced with 
the unique $P$-ported
computation tree in which the greatest non-separator $e\in E$ is
deleted and contracted in the matroid at each tree node,
when the elements are ordered
so $p\in P$ is 
before each $e\not\in P$.

\begin{definition}
\label{ActivitySymbolsDef}
Given a computation tree for 
$P$-ported (oriented) matroid $M$,
each $P$-subbasis $F\subseteq E$
is associated with the following subsets of non-port elements
defined according to Definition \ref{ActivityTreeDef}
from the unique leaf determined by the algorithm given above.
\begin{itemize}
\item $IA(F)\subseteq F$ denotes the set of internally active elements,
\item $IP(F)\subseteq F$ denotes the set of internally passive elements,
\item $EA(F)\subseteq E\setminus F$ 
denotes the set of externally active elements,
and 
\item $EP(F)\subseteq E\setminus F$ denotes the set of externally
passive elements.
\item $A(F)=IA(F)\dunion EA(F)$ denotes the set of active elements.
\end{itemize}
\end{definition}

\begin{prop}
\label{PartitionProposition}
Given a $P$-ported Tutte computation tree for
$M$, 
the boolean lattice of subsets of $E=E(M)$
is partitioned by the collection of
intervals $[IP(F),F\dunion EA(F)]$ 
determined from the collection
of $P$-subbases $F$, which correspond to the leaves.
(Note $F\dunion EA(F)=IP(F)\dunion A(F)$.)

Each $A\in [IP(F),F\dunion EA(F)]$ in one interval determines the
same matroid or oriented matroid  $P$-quotient
by $M/A|P=M/IP(F)|P$.

The boolean lattice of subsets of $E=E(M)$
is also partitioned by the collection of
intervals $[EP(F),(E\setminus F)\dunion IA(F)]$.
(Note $(E\setminus F)\dunion IA(F)$ $=$ $EP(F)\dunion A(F)$.)

Each $A'\in [EP(F), (E\setminus F)\dunion IA(F)]$ of one of these
intervals determines the same matroid or oriented matroid
$P$-quotient by $M\setminus B'/(E\setminus B')$
$=$ $M/F|P$.

For a given $F\in\mathcal{B}_P(M)$, $A\subseteq E$ satisfies
$A\in [IP(F),F\dunion EA(F)]$ if and only if 
$(E\setminus A)\in [EP(F),(E\setminus F)\dunion IA(F)]$. 
\end{prop}

\begin{proof}
Every subset $A\subseteq E=E(M)\setminus P$ belongs to the
unique interval corresponding to the unique leaf found by the tree search 
algorithm given at the end of the previous proof.  

$P$-quotient $M/A|P$ is independent of the order of the
deletions and contractions.  So let $IP(F)\subseteq A$ be contracted
and $(E\setminus EA(F))\subseteq (E\setminus A)$ be deleted first.
The remaining elements of $E$ are loops or coloops, so the $P$-quotient
is independent of whether they are deleted or contracted.

The dual of that tree search algorithm, which descends along
the edge labelled ``$e$-deleted'' if $e\in A'$, etc., will find the
unique leaf whose interval $[EP(F),E\setminus F\cup IA(F)]$ contains
$A'$.

When $A\in[IP(F),F\cup EA(F)]$, the dual algorithm applied to
$A'=E\setminus A$ will find the same leaf.  The $P$-quotient is
determined by $IP(F)$.
\end{proof}

%%%%%%%%%%%%%%%%%%%%%%%%%%%%%%%%%%%%%%%%%%%%%%%%%%%%%%%%

The following generalizes the activities expansion expression given
in \cite{MR93a:05047} to ported (oriented) matroids, as well as 
Theorem 8.1 of \cite{SetPointedLV}. 

\begin{prop}
\label{TuttePolyExpression}
Given parameters $x_e$, $y_e$, $X_e$, $Y_e$, and 
$P$-ported matroid or oriented matroid $M$
the Tutte polynomial expression
determined by the sets in Definition 
\ref{ActivitySymbolsDef} 
from a computation tree is 
given by
\begin{equation}
\tag{PAE}
\label{PAE}
\sum_{F\in \mathcal{B}_P}[M/F|P]
\;X_{IA(F)}\;x_{IP(F)}\;Y_{EA(F)}\;y_{EP(F)}.
\end{equation}
\end{prop}

\begin{proof}
\eqref{PAE} is an expression constructed by applying some of the
Tutte equations.  One monomial results from each leaf.
In that leaf's matroid,
each active element is a separator, and the active elements
contribute $X_{IA(F)}Y_{EA(F)}$ to the monomial.
The passive elements which contribute
$x_{IP(F)}y_{EP(F)}$
are the tree
edge labels in the path from the root to the leaf.
Each $M/F|P$ denotes a $P$-quotient of $M$, so the
expression is a polynomial in the parameters and in the
initial values.  
Therefore, \eqref{PAE} expresses the result of
the calculation when one substitutes
$[M/F|P]=T(M/F|P)$.
\end{proof}

From Corollary \ref{UniversalCor} we conclude:
\begin{thm}
\label{ActivitiesTheorem}
For every $P$-ported parametrized Tutte function $T$ 
on $\mathcal{C}$ into 
ring $R$ or an $R$-module,
for every computation tree for $M\in\mathcal{C}$,
(and so for every ordering of $E(M)$), 
the polynomial expression \eqref{PAE} equals 
$T^{\mathcal{C}}(M)$ of Corollary \ref{UniversalCor}.
\end{thm}


\subsection{Expansions of Normal Tutte Functions}
\label{NormalSubSec}
After a notational translation, 
Zaslavsky's \cite{MR93a:05047} definition of {\bf normal} Tutte 
functions discriminates
those for which $T(\emptyset)=1$, and those for which there exist
$u$, $v\in R$ so that for each $e\in E(M)$,

\begin{equation}
\tag{CNF}
\label{CNF}
X_e = x_e + uy_e \text{ and } Y_e = y_e + vx_e.
\end{equation}

The equations of
Theorem \ref{BigTheorem} are readily verified and
so we naturally extend this definition, with the
$T(\emptyset)=1$ condition dropped, to 
ported separator-strong parametrized Tutte functions.
All the expressions for normal Tutte
functions are therefore in the ring freely generated by
$u$, $v$, the $x_e, y_e$ and the $[Q]$.
We can now generalize some known 
expansions for the Tutte polynomial $T^{\mathcal{C}}$
after the \eqref{CNF} substitution.
The rank function for $M$ is denoted by $r$.


\begin{cor}[Boolean Interval Expansion]
\label{NormalActProp}
The following activities and boolean interval expansion formula
is universal for normal Tutte functions:
\[
T^{\mathcal{C}}(M)=
\sum_{F\in \mathcal{B}_P}[M/F|P]
%\left(
\Big(
\sum_{\substack{
       IP(F)\subseteq K \subseteq F\\
       EP(F)\subseteq L \subseteq E\setminus F
      }}
 x_{K\cup (E\setminus F\setminus L)}\;
 v^{\Card{E\setminus F\setminus L}}\;
 y_{L\cup (F\setminus K)}\;
 u^{\Card{F\setminus K}}\;\;
%\right)
\Big)
\]
\end{cor}

\begin{proof} 
Substitute 
\eqref{CNF} 
into $T^{\mathcal{C}}(M)$
and use
$IP(F)\dunion IA(F) =F$ and 
$EP(F)\dunion EA(F)=E\setminus F$
from
Definition \ref{ActivitySymbolsDef}.
\end{proof}

\begin{lem}
\label{KLAlemma}
Given $F\in\mathcal{B}_P$,
$IP(F)$ spans $EA(F)$.

The pairs $(K,L)$ for which 
       $IP(F)\subseteq K \subseteq F$ and 
       $EP(F)\subseteq L \subseteq E\setminus F$
are in a one-to-one correspondance
with the $A$ satisfying $IP(F)\subseteq A\subseteq F\dunion EA(F)$
given by $A=K\dunion (E\setminus F)\setminus L$.


For every such $A$, 
\[
\Card{F\setminus K}=r(M)-r(M/F|P)-r(A)
\]
and
\[
\Card{E\setminus F\setminus L} = \Card{A}-r(A).
\]
\end{lem}

\begin{proof}
(See Figure \ref{Venn} in Appendix.) By our definition of activities,
after all the elements of $IP(F)$ are contracted, all elements 
in $EA(F)$ are loops.
(Note none of these elements are ports.)

Let 
$A=K\dunion (E\setminus F\setminus L)$.
By our definition of activities,
$IP(F)\dunion IA(F)=F$, so $IP(F)\subseteq A$.
Similarly,  
$EP(F)\dunion EA(F)=E\setminus F$, so 
$A\cap(E\setminus F)\subseteq EA(F)$.
Hence 
$IP(F)\subseteq A\subseteq F\dunion EA(F)$.
Conversely, given such an $A$, 
take $K=A\cap F$ and $L=(E\setminus F)\setminus A$.

Since $IP(F)$ spans $EA(F)$ and $K\supseteq IP(F)$, $K$ spans $EA(F)$.
Since $A\subseteq K\dunion EA(F)$, $K$ spans $A$.
$K\subseteq F$, $F$ is a $P$-subbasis, so $K$ and $F$ are independent,
hence $\Card{K}=r(K)=r(A)$ and $\Card{F}=r(F)$.  
Therefore, $\Card{F\setminus K}=r(F)-r(A)$.

Since $F$ is a $P$-subbasis, $r(F\cup P)=r(M)$.
By definition of contraction, $r(M/F|P)=r(F\cup P) - r(F)$,
so $r(M/F|P)=r(M)-r(F)$.  We conclude 
$\Card{F\setminus K}=r(M)-r(M/F|P)-r(A)$.

$E\setminus F\setminus L = A\setminus K$, so
$\Card{E\setminus F\setminus L} = \Card{A}-\Card{K}$.
As above, $\Card{K}=r(A)$, so the last equation follows.
\end{proof}


\begin{cor}
\begin{equation}
\label{FIntervalExpansion}
T^{\mathcal{C}}(M)=
\sum_{F\in \mathcal{B}_P}[M/F|P]
%\left(
\Big(
\sum_{IP(F)\subseteq A \subseteq (F\!\dunionsubscript\! EA(F))}
 x_{A}
 y_{E\setminus A}
 u^{r(M)-r(M/F|P)-r(A)}
 v^{\Card{A}-r(A)}
%\right)
\Big)
\end{equation}
\end{cor}

\begin{proof}
Apply Lemma \ref{KLAlemma} to the inner sum in
Proposition \ref{NormalActProp}.
\end{proof}

\begin{thm}[Corank-nullity expansion]
\begin{equation}
\tag{PGF}
\label{PGF}
T^{\mathcal{C}}(M) = \sum_{A\subseteq E(M)}[M/A \mid P]x_A y_{E\setminus A}
u^{r(M)-r(M/A\mid P)-r(A)}
v^{|A|-r(A)}.
\end{equation}
\end{thm}

Remark: This extends to separator-strong ported Tutte functions
with parameters on possibly oriented matroids
an expression from \cite{MR0419272} reproduced in \cite{sdcPorted}.  It can
also be proved by corank-nullity generating function methods.

\begin{proof}
By Proposition \ref{PartitionProposition}, given any Tutte computation tree,
the lattice of subsets of $E(M)$ is partitioned
into intervals corresponding to $P$-subbases $\mathcal{B}_P$.  
In each interval, 
$P$-quotient $M/F|P$ is equal to 
$M/A|P$ (as a matroid or oriented matroid)
for 
$A\in [IP(F),F\cup EA(F)]$.
Hence we can 
interchange the summations in \eqref{FIntervalExpansion} and write
\eqref{PGF}.
\end{proof}


\begin{prop}[Geometric Lattice Flat Expansion]
\label{GFlatProp}
Let $M$ be an oriented or unoriented.
In the formula below,
$F$ and $G$ range over the geometric lattice of flats 
$\mathcal{L}=\mathcal{L}(M|E)$ contained
in $M$ restricted to $E = E(M)$.  
($E$ is the top of $\mathcal{L}$ and $\le$
is its partial order.)
\[
T^{\mathcal{C}}(M) = \sum_{Q} [Q]
      \sum_{\substack{F\leq E\\
                     [M/F|P]=[Q]
           }}
                   u^{r(M)-r(Q)-r(F)}
                   v^{-r(F)}
                   \sum_{G\le F}
                   \mu(G,F)
                   \prod_{e\in G}
                    (y_e+x_ev)
\]
\end{prop}

\begin{proof}
This 
generalizes and 
follows the steps for theorem 8 in \cite{sdcPorted}.  (See Appendix.)
\end{proof}


\section{Proof of the Ported ZBR Theorem}

The proof is independent of Diao and Hetyei's.  It is different in
that it is not based on element activities determined by a linear element
order.  
It results from adding considerations of ports to 
Zaslavsky's proofs in \cite{MR93a:05047}.  We base our sketch 
below on the ``straightforward adaption'' presented by
Ellis-Monaghan and Traldi \cite{Ellis-Monaghan-Traldi}, giving
a few details and pointing out differences.

\begin{proof}
As in \cite{Ellis-Monaghan-Traldi}, 
the necessary relations are easy to deduce by 
applying \eqref{TA} and \eqref{TSSM} in two different
orders to a general configuration in each of the five families.

Now on to the converse.
As in \cite{Ellis-Monaghan-Traldi}, the strategy
is to first verify
that the conditions imply $T(M)$ is well-defined for 
$n = \Card{E(M)} = 0, 1$ and $2$.  Note that
for us, the count $n$ does not include $|P\cap S(M)|$.
Second, we rely on the hypothesis the $\mathcal{C}$ is closed under
$P$-minors in order to verify
that in a larger minimum $n$ counterexample, the elements of
$E(M)$ are either all in series or all in parallel, 
and then the conditions
imply that all calculation orders give the same result.  All the cases involve
two different combinations of deleting and contracting of several elements
in $E(M)$ where both combinations produce the same $P$-quotients.

Let $M\in\mathcal{C}$ be a counterexample with minimum $n=|E(M)|$.
Therefore, whenever $M'$ is a proper $P$-minor of $M$,
$T(M')$ is well-defined.  \eqref{TA} and \eqref{TSSM}
have the property that given $M$ and $e\in E(M)$, exactly one equation
applies.  Therefore, 
calculations that yield different values for $T(M)$ must start with
reducing by different elements of $E(M)$.  Since $T(M)$ is an already given 
initial value when $n=0$, 
we can assume $n\geq 2$.

$M$ cannot contain a separator $e\in E(M)$, because
this $e$ is a separator in every $P$-minor of $M$ containing $e$.
Therefore, as observed in \cite{Ellis-Monaghan-Traldi}, every
computation has the same result $X_e T(M/e)$ or $Y_e T(M\setminus e)$ 
depending on whether $e$ is a coloop or a loop.

Let $e$ be one element in $E(M)$.  Since no element in $E(M)$ is a 
separator, $V=x_{e} T(M/{e}) + y_{e} T(M\setminus {e})$ is well-defined, 
and so is $x_{e'} T(M/{e'}) + y_{e'} T(M\setminus {e'})$ 
for each other $e'\in E$.
We follow \cite{Ellis-Monaghan-Traldi} and define
$D=\{e'\in E(M) \mid V=x_{e'} T(M/{e'}) + y_{e'} T(M\setminus {e'})\}$.  The 
induction
hypothesis then tells us 
that there is at least 
one  $f\in E(M)\setminus D$.  
(Recall $e, e', f\not\in P$.)

Suppose that $e$ is a separator in both $M\setminus f$ and 
$M/f$ and $f$ is a separator in both 
$M\setminus e$ and $M/e$.  Then, 
$T$ would
be well-defined for all four of these $P$-minors and so
we can write 
\[
T(M)=x_e x_f T(M/\{e,f\}) + x_e y_f T(M/e\setminus f)
+ y_e x_f T(M\setminus e/f)
+ y_e y_f T(M\setminus\{e,f\}).
\]
Both computations give the same value because in this situation
the reductions by $e$ and $f$ commute. 
So, for $M$ to be a counterexample, there must be $e\in D$
and $f\not\in D$ ($e, f\not\in P$)
to which one case of the following lemma applies:

\begin{lem}
\cite{MR93a:05047}
Let $e , f$ be nonseparators in a matroid $M$. Within each column the
statements are equivalent:

\begin{minipage}{0.5\textwidth}
\begin{enumerate}
\item
$e$ is a separator in $M\setminus f$.
\item
$e$ is a coloop in $M\setminus f$.
\item
$e$ and $f$ are in series in $M$.
\item
$f$ is a separator in $M\setminus e$.
\end{enumerate}
\end{minipage}
\begin{minipage}{0.5\textwidth}
\begin{enumerate}
\item
$e$ is a separator in $M/ f$.
\item
$e$ is a loop in $M/ f$.
\item
$e$ and $f$ are in parallel in $M$.
\item
$f$  is a separator in $M/e$.
\end{enumerate}
\end{minipage}
\end{lem}

We claim that one of the following five cases must be satisfied:
\begin{enumerate}
\item 
$n=2$ and $E(M)=\{e,f\}$ is a dyad.
\item
$n\geq 3$ and $E(M)$ is a circuit not connected to $P$.
\item
$n\geq 3$ and $E(M)$ is a cocircut not connected to $P$.
\item
$n\geq 2$ and for some $\emptyset\neq P'\subseteq P$,
$P'\cup E(M)$ is a cocircuit.
\item
$n\geq 2$ and for some $\emptyset\neq P'\subseteq P$,
$P'\cup E(M)$ is a circuit.
\end{enumerate}

As in  \cite{Ellis-Monaghan-Traldi}, we draw the conclusion that
if $e\in D$ and $f\not\in D$ then $e, f$ are either series or parallel. 
It was further proven that a series pair and a parallel pair cannot
have exactly one element in common.  Therefore, the pairs $e,f$ satisfying 
the conditions are either all series pairs or all parallel pairs.  By 
minimality of $n$, $E(M)$ is either an $n$-element parallel class or an 
$n$-element series class.  The last two cases are distinguished from
the first three according to whether or not $E(M)$ is disconnected or not
from elements of $P$ in matroid $M$.  We now use \eqref{TA} and \eqref{TSSM}
to show that, in each case, the calculations that start with $e$ and 
those that start with $f$
have the same result, which contradicts $e\in D$ and $f\not\in D$.

We give the details for case (d).  By hypothesis, 
each of $T(M/e)$, $T(M\setminus e)$, $T(M/f)$, $T(M\setminus f)$,
$T(M/f/e)=T(M/e/f)$, $T(M/e\setminus f)$ and 
$T(M/f\setminus e)$ is well-defined.  (Remark:
By Proposition \ref{SameMinorProp},
$M\setminus f/e$ $=$ 
$M\setminus e/f$.  

Starting with $e$ and with $f$, \eqref{TA} gives the two expressions:
\[
V= x_e T(M/e) + y_e x_f T(M\setminus e /f) + y_e y_f T(M\setminus e\setminus f)
\]
\[
V\neq 
   x_f T(M/f) + y_f x_e T(M\setminus f /e) + y_f y_e T(M\setminus f\setminus e)
\]
Let $M'$ be the $P$-minor obtained by deleting each element
in $E(M)$ except for $e$ and $f$ ($M'=M$ if $n=2$.)
Since 
$E(M')=\{e,f\}$ and $e,f$ are in parallel connected to $P$ within
$M'$, (d) of the hypotheses  tells us that
\[
T(Q) (x_e Y_f - y_f x_e) =
T(Q') (x_f Y_e - y_e x_f ),
\]
where $Q=M'/e \setminus f$, $Q'=M'/f \setminus e$.  
Since $e,f$ are in parallel within $M'$,
matroids
or oriented matroids $Q=Q'$ by Proposition \ref{SameMinorProp}.
Since $A=E(M)\setminus\{e,f\}$ is a set of loops ($\emptyset$ if $n=2$)
in
$M/e\setminus f$ ($N/e\setminus f$) and 
in $M/f\setminus e$ ($N/f\setminus e$), we 
write $Y_A=\prod_{a\in A}Y_a$ ($1$ if $A=\emptyset$)
and use \eqref{TSSM} to write
\[
T(M/e\setminus f) = Y_A T(Q ),\;\;
T(M/f\setminus e) = Y_A T(Q' ),
\]
\[
T(M/e) = Y_f Y_A T(Q ),\;\;\text{ and }\;\;
T(M/f) = Y_e Y_A T(Q' ).
\]
So, since $M\setminus f\setminus e=M\setminus e\setminus f$,
\[
x_e T(M/e) + y_e x_f T(M\setminus e /f)
=
x_f T(M/f) + y_f x_e T(M\setminus f /e)
\]
contradicts 
$V\neq 
x_f T(M/f) + y_f x_e T(M\setminus f /e) + y_f y_e T(M\setminus f\setminus e)$
The remaining cases can be completed analogously.  It might be
noted that our proof differs slightly from \cite{Ellis-Monaghan-Traldi}
in that the cases of $n=3$ and $n\ge 4$ are not distinguished.
\end{proof}

\section{Acknowledgements}

I thank
the Newton Institute for Mathematical Sciences
of Cambridge University for hospitality and support of my
participation in the Combinatorics and Statistical Mechanics
Programme, January to July 2008, during which some of this
work and many related subjects were reviewed and discussed.

I thank Lorenzo Traldi for bringing to my attention and
discussing Diao and Hetyei's work, as well as
Joanne Ellis-Monaghan, Gary Gordon, Elizabeth McMahon
and Thomas Zaslavsky for helpful conversations and communications
at the Newton Institute and elsewhere.

This work is also supported by a Sabbatical leave granted
by the University at Albany, Sept. 2008 to Sept. 2009.

\bibliographystyle{abbrv}
\bibliography{ParamTutteNew}

\appendix

\begin{figure}
\input{Venn.pdf_t}
\caption{\label{Venn} Illustration for proof of Lemma \ref{KLAlemma}.}
\end{figure}



\section{Proof of Proposition \ref{GFlatProp}}

For each subset $A\subseteq E$, 
$[Q] = [M/A|P] = [M/F|P]$ is determined by the unique flat $F$
in $L(E)$ spanned by $A\subseteq E$.  So, we write \eqref{PGF}
by
\[
T^{\mathcal{C}}(M) = 
     \sum_{Q} [Q] \sum_{\substack{
                     F\in \mathcal{L}\\
                     [M/F|P]=[Q]
                      }}
      \sum_{\substack{
             A\subseteq F\\
             A\text{ spans }F
          }}
      x_A y_{E\setminus A}
      u^{r(M)-r(M/A\mid P)-r(A)}
      v^{|A|-r(A)}.
\]
Factoring, we get
\[
T^{\mathcal{C}}(M) = 
     \sum_{Q} [Q] \sum_{\substack{
                     F\in \mathcal{L}\\
                     [M/F|P]=[Q]
                      }}
      u^{r(M)-r(Q)-r(F)}
      v^{-r(F)}
      \sum_{\substack{
             A\subseteq F\\
             A\text{ spans }F
          }}
      x_A y_{E\setminus A}
      v^{|A|}.
\]
$A$ is summed over the spanning sets of $F$.  Let $Z(F)$ denote
this last sum.  Since every subset of $F$ spans some flat
in $\mathcal{L}$,
\[
\sum_{0\le G \le F}Z(G) = \sum_{e\in F}(y_e + x_ev).
\]
M\"{o}bius inversion gives
\[
Z(F)= \sum_{0\le G\le F}\mu(G,F)(y_e +  x_ev).
\]

\end{document}

``Let $G$ be a connected graph and $H\subseteq E(G)$. 
Assume we are given a
mapping $c$ from $E(G) \setminus H$ to a color set $\Lambda$. 
Assume further that $\psi$ is a
block invariant associating an element of a fixed integral domain $R$a to
each connected graph. For any contracting set $C$ of $G$ with respect to
$\mathcal{H}$, let $\mathcal{H}_C$ 
be the graph obtained by deleting all edges in $D$ and
contracting all edges in $C$ (so that the only edges left in $\mathcal{H}_C$ 
are the zero edges).''

