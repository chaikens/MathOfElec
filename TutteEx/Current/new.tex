\documentclass{letter}
\usepackage{geometry}
\geometry{letterpaper}

%%%%%%%%%% Start TeXmacs macros
\newenvironment{tmparmod}[3]{\begin{list}{}{\setlength{\topsep}{0pt}\setlength{\leftmargin}{#1}\setlength{\rightmargin}{#2}\setlength{\parindent}{#3}\setlength{\listparindent}{\parindent}\setlength{\itemindent}{\parindent}\setlength{\parsep}{\parskip}} \item[]}{\end{list}}
%%%%%%%%%% End TeXmacs macros

\begin{document}

\title{New TutteEx Paper--for Math }\author{}\maketitle

Theme:Electrical Networks and the Matrix Tree Theorem are the main historical
and current context of this paper.

\begin{tmparmod}{1in}{0pt}{0pt}
  \begin{tmparmod}{1cm}{0pt}{0pt}
    \begin{tmparmod}{0pt}{1cm}{0pt}
      Electrical networks have a long history of (giving, providing, donating,
      ???) motivations for graph and combinatorial theory. \ A graphs
      Laplacian matrix L appears in the equation $L  = I$ that can be solved
      for the electrical node potential  in terms of the electrical currents
      $I$ flowing into the nodes. \ [Matrix Tree Theorem and things like
      Maxwell's rule]
      
      Theme: My discovery is interesting because it domonstrates a new
      relationship between two interesting topics.
      
      The spanning tree enumerator is of course the special case for graphic
      matroids of the basis enumerator, which is one member of the important
      family of Tutte matroid invariants.
      
      The [SUPER] consequence of our result is that it is not a coincidence
      that electrical network solutions are ratios of minors of Laplacians and
      that each minor satisfies the Tutte equations. We have found that when
      the solution set to a suitably formulated system of electrical network
      equations is expressed as an extensor, the function that maps the
      network's graphic matroid to this extensor satisfies the Tutte
      equations. Of course, this requires that those equations [Tutte?] be
      extended [rewritten? formulated?] from commutative rings to exterior
      algebra. \ This is because the [wedge] exterior product, or wedge, is
      anti-commutative. \ This paper presents the technicalities that seem to
      be necessary [to make this program work?] for this generaliztion to
      work. \ 
    \end{tmparmod}
    
    Theme: Understanding my discovery requires that each topic be looked at in
    a somewhat unfamiliar way to most readers of this journal.
    
    Theme: What is the definition of $M_E (N)$ with the minimum of theory?
    
    Theme: What is the main point of the TutteEx paper, informed by the refs
    report?
    
    We can formuate a [Laplace/Dirichlet problem, CHECK] so the [solution,
    kernel, Green's function CHECK] satisfies the [anticommutive, exterior
    algebra?? these are new to this paper] Tutte equations.
    
    When we express the [??] of the [??] problem in exterior algebra, the [??]
    satisfies the Tutte equations. \ Our formulation requires that some
    elements are distinguished as ports: The ports determine the coordinate
    system for the [solution??] There is one instance of the Tutte eq. for
    each of the remaining elements: For each $e$ not a port, $M (N) = r_e M (N
    e) + g_e M (N / e) .$We believe that this [??] underscores the importance
    of distinguishing port elements [the ports] [to obtain results of this
    kind?]
    
    Our other line of work investingates the [consequences of ] distinguishing
    of port elements within matroid theory. \ There is less novelty. \ This
    idea has been studied before, but not in the electrical network context
    [by mathematicians, but by EEs] We present the fairly straightforward
    generaliztions of Tutte function theory. \ We can then see how our
    algebraic result fits into the discrete theory.
  \end{tmparmod}
\end{tmparmod}

\end{document}
