Parametrized, ring valued
Tutte equations and functions (with empty port set), and 
the universal ported Tutte invariant (called the big Tutte 
polynomial by Las Vergnas) have been studied.  So has
the (reduced) Laplacian matrix's determinant for a graph--This 
determinant, according to the famous matrix tree theorem, equals
the number of bases in the corresponding graphic matroid.  It is
well-known that the number of bases in a Tutte invariant defined
for all matroids.  More strongly, the following homogeneous generating
function for the bases
\[
F(\mathcal{N}(\emptyset,E))=
\sum_{B\in\mathcal{B}(\mathcal{N})}
\left(\prod_{e\in B} g_e
\right)
\left(\prod_{e\in E\setminus B} r_e
\right)
=
\sum_{B\in\mathcal{B}(\mathcal{N})}
g_B\;\;r_{\scomp{B}}
\]
is a Tutte function (with $P=\emptyset$).
These background topics will be surveyed and extended as needed.


This paper's main topic is a parametrized function from certain
(ported) extensors to extensors.  An extensor is a 
(fully or completely\footnote{We will follow authors such as
\cite{JacobsonI} in omitting this qualifier for such decomposible elements})
decomposible element in an exterior algebra (i.e., algebra of antisymmetric
tensors).  Our main result (Theorem \textbf{FILLIN}) is that 
if extensor deletion and contraction are defined in a way that is
consistant with our function definition, then our function obeys 
identities in exterior algebra that are analogs of the Tutte 
equations.  


One corollary applies to the class of ported unimodular oriented
matroids:  Our function, restricted to the unimodular extensors
which represent this class, yields an extensor-valued Tutte function 
that sometimes distinguishes different orientations of the same 
matroid.

When the corollary is applied with the
parameters all set to $1$, we get new Tutte-like invariants
that can sometimes distinguish different orientations of the same matroid.

\subsection{Ports and Parameters}

The idea to restrict deletion/contraction decomposition so it does
not apply to some distinguished elements 
is a natural one.  It was applied to invariants of
matroid morphisms (or strong maps) by Las 
Vergnas\cite{MR0419272,SetPointedLV,EtienneLasVergnasMorphismVectorial}.
The equations for 
ported Tutte invariants (i.e., $F$ with the $r_e=g_e=1$ and 
invariant under port-preserving matroid isomorphisms) determine
the 
\textbf{big Tutte polynomial} defined by Las Vergnas in \cite{SetPointedLV}.
Whereas the traditional Tutte polynomial has only two variables,
which correspond to the isomorphism classes of the indecomposible 
loop and coloop matroids, the generalization adds additional
variables corresponding to the additional indecomposible matroids
on sets of port elements. 

Our motivation there and here
comes from distinguishing as ports some edges
in a directed graph model of an electrical network.  The ports
model two-terminal plugs or sockets where the network interacts
with an enviroment; the other edges model resistors.

In this paper we add the observation that if ported Tutte 
decomposition is done on an \emph{oriented} matroid, the indecomposible
minors include connected \emph{oriented} matroids on sets of port elements.
Therefore, if $\mathcal{N}_1$ and $\mathcal{N}_2$ are different orientations
of the same orientable matroid, then 
$F(\mathcal{N}_1)\neq F(\mathcal{N}_2)$ is possible for a ported
Tutte function $F$.  
