% Template for ICIP-2000 paper; to be used with:
%          spconf.sty  - ICASSP/ICIP LaTeX style file, and
%          IEEEbib.bst - IEEE bibliography style file.
% --------------------------------------------------------------------------
\documentclass{article}
\usepackage{spconf,amsmath,epsfig}

% Example definitions.
% --------------------
\def\x{{\mathbf x}}
\def\L{{\cal L}}
\def\Reals{\ensuremath{\mathbf R}}
\newtheorem{theorem}{Theorem}
\newtheorem{definition}{Definition}

\newcommand{\supp}[1]{{{\mbox{supp\ }#1}}}

%    Absolute value notation
\newcommand{\abs}[1]{\lvert#1\rvert}


%    Rank
\DeclareMathOperator{\rank}{rank}

%    Matroid Union 
\newcommand{\munion}{\lor}

%   Set minus
\newcommand{\sminus}{\backslash}


%   Two matrices horizontally concatenated, space between can be
%   adjusted here
%
\newcommand{\hmat}[2]{[#1\;#2]}

%   Oriented Matroid Pair
\newcommand{\OMP}[2]{#1,\,#2}

\newcommand{\extra}[1]{{\small{#1}}}


% Title.
% ------
\title{\extra{REPORT:}
(\today)
AN ORIENTED MATROID PAIR MODEL FOR ELECTRICAL AND MECHANICAL NETWORKS
}
%
% Single address.
% ---------------
\name{Seth Chaiken\thanks{Part of this reseach was done during a Sabbatical
from the University at Albany in 2001.}}
\address{University at Albany\\
	Department of Computer Science\\
	Albany, NY 12222\\
	\texttt{sdc@cs.albany.edu}}
%
\begin{document}
%\ninept
%
\maketitle
%

\begin{abstract}
Both resistive electrical networks and elastic mechanical systems such
as trusses have a topological or geometric structure together with constitutive
laws for the elements prior to their interconnection.
Oriented matroids provide a common discrete mathematical model 
for such 
structure
%structures
in which relationships on the signs of element 
quantities can be expressed.
Pairing of oriented matroids
enables non-linear monotone constitutive laws to be fit
into the abstraction in a way that accomodates port and nullor
insertions as well.

The resulting mathematical model clarifies some mechanical analogies for 
these 
circuit theory concepts, 
relates apparently dissimilar published theories
for existance and uniqueness and shows how to 
handle elastic mechanical systems with small displacements.
It also enables constraints on the signs of system quantities
to be predicted from the structure when this is possible.
\extra{Finally, it 
derives topological solution formulas for linearized mechanical 
systems in which the analog of a tree-sum is a sum over minimally rigid
trusses.}
\end{abstract}
%

\section{Introduction}
\label{sec:intro}

%Our work \cite{sdcOMP} shows that the model and results of 
%\cite{HaslerDApplMath,HaslerNeirynck} generalize from a graph model to a
%linear subspace pair model
%for which the discrete structure 
%%in which the topological conditions are expressed 
%that expresses 
%%the 
%topological conditions
%for existance or uniqueness of solutions
%is pair of oriented
%matroids, rather than a graph with designated resistor, source, nullator
%and norator edges.  
Our work \cite{sdcOMP} shows that the model and results of 
\cite{HaslerDApplMath,HaslerNeirynck} generalize from a graph model to a
linear subspace pair model;
the pair of linear subspaces defines a pair of oriented matroids;
this oriented matroid pair is a discrete structure that
generalizes a graph with designated resistor, source, nullator
and norator edges and enables 
topological conditions
for the existance or uniqueness of solutions
to be expressed. Two real matrices, which can be easily generated
from the system design, represent these oriented matroids so that 
it is practical to work with them.  (Some oriented matroids, indeed 
most, are not representable by real matrices and would require much more
space to store, but they do not occur in our application.)

\extra{

In the electrical circuit theory literature, the circuit ``topology''
means the network graph (to which Kirchhoff's laws apply)
together with particular kinds of ``device elements''
such as resistors, capacitors, voltage sources (batteries), 
current sources, etc., associated with single graph edges, and possibly
``multiport'' elements associated with multiple graph edges.
Problems with multiport elements are reduced to 
those with only single edge elements, in order to apply
the theory of  \cite{HaslerDApplMath},  through the use of 
``nullator'' and ``norator'' elements.  Detailed exposition of the problems,
reductions, theory and applications is given in \cite{HaslerNeirynck}.

}

In many cases, interesting conclusions can be reached by 
``calculations'' done in a very simple ``algebra of signs'', a 
form of qualitative reasoning, 
starting with the sign patterns of the matrices.  In cases when
the qualitative calculations show the outcome depends on numeric
values, the numeric information can then be used; or inequalities
among system parameters for each outcome can be derived.  
Our work
applies to models whose non-linearities are monotone.  However, in order
to understand results like those of 
\cite{HaslerDApplMath,HaslerNeirynck} matroid theoretically, 
we believe
a pair of oriented matroids (not just one) 
on a common ground set is necessary.
Then, the non-linear monotonicity is modeled by the sign
of an element in one oriented matroid covector being equal to the sign of
the same element in the other.
We further motivate pairs
by drawing 
the analogy between the electrical and mechanical elastic (whose subspace
pairs are not graphic cycle and cocycle spaces)
applications by way of the common model.  


The present report will clarify how our 
generalizations of Foss\'{e}prez, Hasler and Neirynck's conditions
for unique solvability
are equivalent to the characterization of $\mathcal{W}_0$ matrix pairs
shown by Sandberg and Willson \cite{SWExistancePf,W0APPLpaper}:
They are equivalent combinatorial 
conditions for the unique solvability
of a \textit{common}
%(not just similar)
problem with monotone non-linearities.
A third known variant due to Nishi and Chua 
\cite{NishiChuaCactus,NishiChuaCCCS} is based on 
deletion/contraction operations applicable
to various kinds of primative
1 and 2-port elements to produce ``cactus graph'' networks,
which are then analyzed with the signs of determinants 
of their matrices.

Theory valid for all oriented matroid pairs, not just those represented by
a pair of linear subspaces, was presented in \cite{sdcOMP}.  For lack of
space, we omit determinant sign conditions shown equivalent
in \cite{sdcOMP} to the cases of common covector conditions that we cover.
Although oriented matroids can also be axiomatized with ``chirotopes''
which abstract determinant signs, the oriented matroid 
common covector approach has the intuitive advantage of relating the 
covector directly to qualitative properties of a system state or differences.

The introduction continues with
single oriented matroids and then
succinct example of our approach
that reproduces a case of 
a known result.  


\subsection{Single Oriented Matroids}

We think of the oriented matroid $\mathcal{M}(M)$ 
\textit{represented} by matrix $M$ 
as its finite set of \textit{covectors} $\mathcal{L}(M)$, where
each covector 
is the tuple of the \textit{signs} $\{+,-,0\}$
or \textit{signature} $X=\sigma(l)$ of the real coordinates 
of a member $l$ of the linear subspace $L(M)$ in $\Reals^U$
spanned by the rows of $M$. (We use italics to denote a 
\textit{term being defined}.)  Hence $\mathcal{L}(M)$ has at most 
$3^{|U|}$ covectors.  
For example,
when $M$ is the signed incidence matrix of a network graph, each covector
represents a combination of branch voltage drop signs feasible under Kirchhoff's
voltage law; the finite \textit{ground set} $U$ labels the branches.
One can call $X$ a \textit{signed set}, in
which elements of subset $X^+$ occur with $+$ sign and those in $X^-$ have
$-$ sign. 
The \textit{support} $\supp{X}$ is the subset
of $e\in U$ for which $X_e \neq 0$, i.e., $\supp{X}=X^+\cup X^-$.
Bachem and Kern's book \cite{BachemKern}
motivates oriented matroids from linear subspaces this way.
For the sake of
brevity, we 
take for the defining axioms of oriented matroids
some properties of the ``sign algebra'' operations
that we will use.

Given two sign tuples $X^1$, $X^2$, their \textit{composition}
$Z=X^1 \circ X^2$ has $\supp{Z}$ $=$ 
$\supp{X^1}\cup \supp{X^2}$
and 
for $e\in\supp{Z}$,
$X_e=X^i_e$ where $i$ is the smallest index for which $X^i_e\neq 0$.
Note that if $X^i=\sigma(l^i)$ for $l^i\in\Reals^U$, then $X^1\circ X^2$
$=$ $\sigma(l^1 + \epsilon l^2)$ for some sufficiently small $\epsilon >0$.
Hence $\mathcal{L}(M)$ is closed under the $\circ$ operation.

\begin{definition}
The collection $\mathcal{L}$ of sign tuples over $U$ is the set of 
\textit{covectors} of an oriented matroid if it satisfies:\\
(L0)
$0\in\mathcal{L}$.\\
(L1) If $X\in\mathcal{L}$ then $-X\in\mathcal{L}$.\\
(L2) For all $X$, $Y \in \mathcal{L}$, $X\circ Y$ $\in$ $\mathcal{L}$.\\
(L3) For all $X$, $Y \in \mathcal{L}$, $e\in X^+\cap Y^-$
there is
$Z\in\mathcal{L}$ such that 
$Z^+\subset(X^+\cup Y^+)-e$,
$Z^-\subset(X^-\cup Y^-)-e$, 
and 
$(\supp{X}-\supp{Y})\cup(\supp{Y}-\supp{X})\cup
(X^+\cup Y^+)\cup(X^-\cup Y^-)\subset\supp{Z}$.
\end{definition}
Note that property (L3) says $Z_e=0$ and it predicts $Z_g$ for all
$g\neq e$ except when $X_gY_g = -$; i.e., $g$ has opposite signs in
$X$ and $Y$.  The logical
equivalence of this definition to various apparently weaker
axiomatizations is due to work of Edmonds, Fukada and 
Mandel cited and surveyed in \cite{OMBOOK}.

Other oriented matroid 
notions such as orthogonality and 
independence can be expressed by properties of covector sets
that are motivated by linear algebra.
The covectors $\mathcal{L}(L^\perp)$ of the orthogonal complement
of linear subspace $L\subset\Reals^U$ form another oriented matroid.
We say $X\perp Y$ for signed sets $X,Y$ when 
either $\supp{X}\cap\supp{Y}=\emptyset$ or
there are $e,f\in U$
with $X_fY_f = -X_eY_e \neq 0$.
This abstracts a necessary condition for 
two real vectors to be orthogonal.
In fact, for every oriented matroid $\mathcal{M}$ with covectors
$\mathcal{L}(\mathcal{M})$
the collection $\mathcal{V}=\mathcal{L}^\perp$ defined by
$\{Y | Y\perp X \mbox{\ for all\ }X\in\mathcal{L}\}$ satisfies the
covector axioms (\cite{OMBOOK}, Prop. 3.7.12); it is called the 
set of \textit{vectors} of $\mathcal{M}$ and is the set of covectors
of the \textit{dual} or \textit{orthogonal} oriented matroid 
$\mathcal{M}^\perp$. The vectors code all combinations of coefficient 
sign that occur among all linear dependencies of the columns of $M$,
when $\mathcal{M}=\mathcal{M}(M)$.
\extra{More directly, an independent set $I\subset U$ is characterized by: 
for all $3^{|I|}$ ``input'' assignments $i$ of $e\in I$ to $\{+,-,0\}$, 
there exists a covector $l\in \mathcal{L}(\mathcal{M})$ for which
$l_e = i_e$ for all $e\in I$.}  Abstractly, an \textit{independent set}
$I\subset U$ satisfies $\supp{V}\not\subset I$ for all non-zero 
vectors $\mathit{V}(\mathit{M})$.

KVL, KCL and analogous mechanical structural or geometric laws 
are each formulated by a constraint of the form 
$v\in L$ $=$ $\mbox{row space}(M)$
where $L\subset\Reals^U$.  The problem of
reformulating such a law by a system of linear equations is
solved as follows:  A maximal subset $B\subset U$ corresponding to
a linearly independent set of columns of $M$ is found.  Such a $B$ is
a maximal independent set,
called a \textit{basis in the matroid} $\mathcal{M}(M)$.
Row operations and possibly deletion of zero rows can transform
$M$ to $( I\ ;\ M^{\overline{B}} )$ (after column permutation)
where $I$ is the $r\times r$ identity matrix, where $r$ $=$ 
$\mbox{rank}(M)$ $=$ $\mbox{dim}(L)$ $=$ $\mbox{rank}(L)$
$=$ $\mbox{rank}(\mathcal{M}(L))$.    
It is now clear that $v\in L$ is characterized by 
$v_{\overline{B}}$ $=$ $v_{B}M^{\overline{B}}$.
For each independently chosen
$v_{B}\in\Reals^B,$ $v=(v_B;v_{\overline{B}})$ is unique tuple
for which the $B$ coordinates equal $v_B$.

The \textit{cocircuits} (resp. \textit{circuits})
of an oriented matroid are the non-zero covectors (resp. vectors)
whose support is minimal.  Minty's painting property,
most popularly known as a theorem about 
directed graphs\cite{VandewalleChua}, is generally true about
the cocircuit $\mathcal{C}^*$ and circuit 
$\mathcal{C}$ collections of an oriented matroid $\mathcal{M}$.
(Note $\mathcal{C}$ is the
cocircuits of the orthogonal oriented matroid $\mathcal{M}$.)
In fact, when the 
simple non-triviality, symmetry, and minimal support properties are assumed, 
the painting property characterizes when 
$\mathcal{C}^*$ and $\mathcal{C}$ are the cocircuit/circuit collections of 
an oriented matroid.

\begin{theorem}
(\cite{OMBOOK}, Th. 3.4.4(4P); \cite{BachemKern}, Prop. 5.12) 
For every partition $U=R\cup G \cup B \cup W$ and for every $e\in R\cup G$,
\textbf{either} %\\
%\begin{itemize}
%\item[(a)] 
(a) There exists $X\in\mathcal{C}^*$ so $e\in\mbox{Supp}{X}$,
$X_R\geq 0$, $X_G\leq 0$, $X_B$ unrestriced and $X_W=0$ %\\
%\item[]
\textbf{or}\\
%\item[(b)]
(b) There exists $Y\in\mathcal{C}$ so $e\in\mbox{Supp}{Y}$,
$Y_R\geq 0$, $Y_G\leq 0$, $Y_B = 0$ and $Y_W$ unrestricted %\\
%\item[]
\textbf{but not both.}
%\end{itemize}
\end{theorem}

This led us to generalize in \cite{sdcOMP}
Hasler and Neirynck's notion of a ``pair of 
conjugate trees'' to a ``complementary pair of bases''; and of 
a ``non-trivial uniform partial orientation of the resistors''
to a ``common (non-zero) covector''.   

\subsection{Example}

We illustrate the oriented matroid approach by reproducing the result of
\cite{TrajWillNDR} that a particular configuration of a ``feedback structure''
with two Ebers-Moll transistors and one port
cannot exhibit negative differential resistance by itself, and it can 
exhibit NDR if one resistor is added and $\alpha_1+\alpha_2-1>0$.  
Under KVL,
the voltages across the port, resistor
and two linearized 
Ebers-Moll diodes are given by the row space member of $M_V$ when the
three rows are multiplied by the 3 independent voltages
$V_1$, $V_2$ and the port voltage $V$.  The space of current values
feasible in the same 4 elements under KCL and the two Ebers-Moll current 
controlled current source laws is only one dimensional; it is spanned by the one
row of $M_I$.  Beginning with the signatures of the rows of the matrices,
we can apply the covector axioms to explore what common covectors are possible
under several variations.




\section{Basic}
\label{sec:Basic}

A \textit{subspace pair} $(L_V, L_I)$ is a pair of linear subspaces of
$\Reals^U$, where the elements of finite set $U$ index the coordinates.
The scalar product $v\cdot w = \sum_{e\in U}v_e w_e$ is used to define that
$v, w \in \Reals^U$ are \textit{orthogonal} when $v\cdot w = 0$.  An 
\textit{orthogonal subspace pair}  satisfies $v\cdot w = 0$ for
all $v\in L_V$ and $w\in L_I$.  A subspace pair has \textit{full rank} when
$\mathrm{rank}(L_V)+\mathrm{rank}(L_I)= |U|$.  Hence an orthogonal full rank 
subspace pair is a linear subspace paired with its orthogonal complement.

The structure of an electrical network is defined 
beginning with the \textit{network graph}
$\mathcal{N}$ with \textit{nodes} $N$  and \textit{arcs} $U$.  
(The generality obtainable by port, nullator/norator or nullor, and 
device characteristic insertions will be treated later.)
Each arc has 
a fixed but arbitrary direction to define the sign of its voltage drop and
current flow.  The \textit{incidence matrix} $M_V$, with rows indexed by $N$
and columns indexed by $U$ is defined so $M_V(n,e)=+1$ when the tail of $e$ is
$n$, $-1$ if the head of $e$ is $n$, and $0$ if $n$ and $e$ are not incident.

When $L_V$ is the row space of $M_V$ and $L_I$ is the orthogonal complement
of $L_V$, the members of $L_V$ are voltage drop tuples in $\mathcal{N}$ 
feasible under Kirchhoff's voltage law and the members of $L_I$ are the 
current flow tuples feasible under Kirchhoff's current law.  These facts 
restate Kirchhoff's laws and Tellegen's theorem.  Note that we can determine
$(L_V, L_I)$ from one of these subspaces given and Tellegen's theorem: The
role of nodes here is not strictly necessary.

\extra{ Kirchhoff's voltage law can be expressed by the statement:  
The feasible voltage drops are the image of the map 
$\Reals^N\rightarrow\Reals^U$ given by $\phi\rightarrow\phi M_V$.  
Kirchhoff's current law says the feasible current flows are the kernel
of the map $\Reals^U\rightarrow\Reals^N$ given by $u\rightarrow u M_I^t$.
Tellegen's theorem is the observation that $M_V$ and $M_I^t$ are adjoints.
See \cite{WyattTele}.}

The definition of mechanical network structure begins with the 
(undirected) \textit{framework graph} $\mathcal{F}$ with \textit{vertices}
$N$ and \textit{edges} $U$.  A \textit{framework} $\mathcal{F}(\mathbf{p})$ 
in d $dimentions$ is a framework graph $\mathcal{F}$ and an \textit{embedding} 
$\mathbf{p}:N\rightarrow \Reals^d$.  
The embedding assigns each vertex to a point in
$d$-dimensional space.  The \textit{rigidity matrix} $M_V$ 
has $d|N|$ rows, each
indexed by one coordinate in $\Reals^d$ of the point that embeds one vertex.
For edge $e=(i,j)\in U$, column $M_V(e)$ of the rigidity matrix is defined 
(when vertices are numbered 0 through $|N|-1$):\\
\begin{minipage}{\linewidth}
\begin{align*}
(0, \ldots,0, &\mathbf{p}(n_i)-\mathbf{p}(n_j),\\
              &\mathrm{positions\ } di\ldots di+d-1
\end{align*}
\end{minipage}
\hfill\begin{minipage}{.85\linewidth}
\begin{align*}
\hfill 0,\ldots,0,&\mathbf{p}(n_j)-\mathbf{p}(n_i), 0, \ldots,0)^T\\
                  &\mathrm{positions\ } dj\ldots dj+d-1
\end{align*}
      \end{minipage}\\
This definition is echoed from the literature \cite{RigidityBook} on rigidity theory, except
we interchange rows and columns.  Just as we deemphasized nodes of electrical
networks, we will use merely the row space of $M_V$ for most of what follows.

\extra{
The rigidity matrix as a function of the embedding $\mathbf{p}$ is denoted
$M_V(\mathbf{p})$.  The row vector $\mathbf{p}$ left multiplied with
$M_V(\mathbf{p})$
is the row tuple denoted $\mathbf{L}=\mathbf{p}M_V(\mathbf{p})$  Then, 
$\mathbf{L}_e$ $=$ 
$(\mathbf{p}M_V(\mathbf{p}))_e$ $=$ $|\mathbf{p}_i-\mathbf{p}_j|^2$ for
each edge $e=(i,j)$.  Now if each $\mathbf{p}_i$ is a differentiable function
of $t$, $d\mathbf{L}/dt$ $=$ $2\mathbf{p}'M_V(\mathbf{p})$.  Framework
$\mathcal{F}(\mathbf{p})$ is \textit{first-order rigid} when 
$d\mathbf{L}/dt$ $=$ $0$ for all $\mathbf{p}'$ implies 
$|\mathbf{p}_i-\mathbf{p}_j|^2$ is constant for all pairs $i$, $j$, not just
endpoints of edges. 
}%extra

Let $\mathbf{p}(i)-\mathbf{p}(j)$ for edge $e=\{i,j\}$ be called the
\textit{vector from } $j$ \textit{to} $i$.  
It is known that the row space $L_V$ 
of $M_V$ consists of tuples in $\Reals^{U}$
such that component $v(e)\in\Reals$ for $e\in U$ is the projection of the
relative velocity of vertex $i$ with respect to vertex $j$ projected onto
the vector from $j$ to $i$, for some combination of vertex velocities 
$\mathbf{v}:N\rightarrow\Reals^d$:
$v(e)=(\mathbf{v}(i)-\mathbf{v}(j))\cdot(\mathbf{p}(i)-\mathbf{p}(j))$.

It is also known that the $L_I$, the orthogonal complement of $L_V$, 
is comprised of the tuples $\sigma:U\rightarrow\Reals$ 
of scalars for which the framework is in static
equilibrium when each edge $e$ exerts force 
$\sigma(e)(\mathbf{p}(j)-\mathbf{p}(i))$ on vertex $i$.  By this convention, 
$\sigma(e)>0$ means $e$ is under tension and $\sigma(e)<0$ means $e$ is under 
compression.  Each tuple $\sigma\in L_I$ is called a \textit{self-stress}.

Under this analogy, 
(1) KVL corresponds to geometric consistancy of first order
edge length changes under changes in the embedding, (2) KCL corresponds to
Newton's laws of static equilibrium, and (3) Tellegen's theorem corresponds
to a virtual work principle, that static equilibrium is
characterized by 
the internal forces of every virtual embedding change 
doing zero virtual work.

\extra{\subsection{Elastic Analog of the Nodal Admittance Matrix}

Reduced nodal admittance matrix.  Nodal resistance matrix.
Interaction with a physical framework with it's environment.
A framework is first order rigid iff it ``resolves all applications of
static equilibrium forces''.  However, every physical bar has some
elasticity:  An ideal rigid bar is analogous to an ideal voltage 
source.  Hence, given an elastic framework, for every application
of static equilibrium forces on the vertices, the vertex positions
will change as the bars stretch or shrink under the forces they now
carry to resolve the applied force.  These first order vertex position 
changes are given by $Z\mathbf{f}$.

The environment might interact by ``forcing'' some vertices to change position
relative to one another.
Intuitively, the framework will ``push back''.  The other vertices are free
to move as adjacent vertices move and incident bars change length in
response to the forces developed in them to resolve the forces required
to hold the framework in its new position.  The position changes of the
free vertices $V$ can be calculated by solving for the unknown position changes
in the system of equations $(Y\mathbf{v}_V)(V)=0$.

For our purposes, we insert port elements in order to make interactions 
with the environment explicit.  This enables a coordinate of an 
environmental interaction quantity to correspond to an oriented matroid
element, so that its sign can be read off from the corresponding entry
in a covector.

It's yet to be done to handle simutaneous application of force to more 
than 2 vertices....
}

\section{THE SUBSPACE PAIR MODEL AND PORTS}


Questions of existance and 
uniquenss of solution for various combinations of 
kinds of sources can be formulated after 
modeling devices.  
Each port element provides separate output and input
variables
for an 
electrical current or 
voltage \textit{kind of} source, or its mechanical analog.
Unlike device variables, the two variables of each port
are not directly related by a constitutive law which is 
part of the system model.
%Ports are introduced so the response of an electrical network to current 
%and/or voltage sources, and the mechanical analogs, can be formulated.

\extra{ Ports also facilitate formal operations to compose larger systems
from smaller ones.  We believe ports are important for investigations of
rigidity because they model how a framework interacts with its environment,
for example, what a mechanical model ``feels like'' when you squeeze it.
We have also found that electrical port characteristics of unit resistance
ported electrical networks are ratios of coefficients in certain 
partial evaluations 
of a generalization of the Tutte polynomial\cite{sdcPorted}.}


Familiar topological conditions on dependencies among source values
pertain to the \textit{matroids} of the subspaces $L_V$ and $L_I$.
Questions about existance and uniqueness of solution
will be answered in terms of supplementary subspace pair models
which are obtained by the familiar operations of opening and shorting ports.
Finally, operations on subspace pairs that model nullor insertion are
defined, so that such ideal elements can be modeled combinatorially or
geometrically.  

The supplemental subspace pair derived after nullor 
insertion will typically not be orthogonal.  One might also choose to
model linearized CCCSs or VCVSs within one of the subspaces.  
Each port insertion generally increases $\rank(L_V)+\rank(L_I)-|U|$;
system behavior for linearized constitutive laws will be shown to be
represented by the intersection of two linear spaces.
Hence we do not assume any rank or orthogonality conditions on subspace 
pairs in the definitions below.

Given a subspace pair $(L_V, L_I)$ and element $p\in U$ not already a port, 
we define the 
operation of \textit{inserting a port at  $p$} 
as follows: A new subspace pair $(L'_V, L'_I)$
is defined with $U'=U-\{p\}\cup\{p_V,p_I\}$, $L'_V=L_V\oplus\Reals$ (direct 
sum) with $p$ replaced by $p_V$;
and the coordinate of the added $\mathbf{R}$ 
indexed by $p_I$, together with
$L'_I=L_I\oplus\Reals$ with $p$ now replaced by $p_I$
and the added subspace indexed by $p_V$.
Note that (going to $(L'_V, L'_I)$) the ranks of
$L_V$ and $L_I$ each increase by 1, and $|U'|$ $=$ $|U|+1$.  After $p$ port
insertions, we denote the final $U$ by $E\cup P_V \cup P_I$ 
with pairwise disjoint
$E$, $P_V$ and $P_I$, $|P_V|$ $=$ $|P_I|$ $=$ $p$, $P_V\cup P_I$ being the 
replacement elements.  (((??? Let $P$ denote $P_V \cup P_I$.)))

The \textit{subspace pair model} $\mathbf{M}$ $=$ 
$(E, \Gamma, P, (L_V, L_I))$ consists of finite set $E$ of 
\textit{device elements}, \textit{constitutive law relations}
$\Gamma = \{\Gamma_e\subset\Reals\times\Reals | e \in E\}$, a finite set
$P=P_V \cup P_I$ that result from inserting ports as defined above, 
and a subspace pair $(L_V, L_I)$ over $\Reals^U$ with $U=E\cup P$.

The \textit{variables} of $\mathbf{M}$ are 
$\{u_{Ve}, u_{Ie} | e \in E\}$  $\cup$ \\
$\{ u_{Vp}, u_{Ip} | p_I, p_V \in P \}$.  
(For brevity, subscript ``$Vp$'' means port element
$p_V\in P_V$, etc.)
A \textit{subspace pair model with sources} $S$ 
is a subspace pair model $(E, \Gamma, P, (L_V, L_I), S)$
together with a subset $S$ of exactly $|P|$ of the $2|P|$ elements
in $P$.  A \textit{$V$-driven port} is a port $p\in P$ for which 
$p_V\in S$ and $p_I\not\in S$, then $u_{Vp}$ is called 
an \textit{input variable}.   Reverse $V$ and $I$ to define an 
\textit{$I$-driven port} and its input variable.

A \textit{solution} of $\mathbf{M}$ with sources
is a real valued extension to \textit{all} variables of $\mathbf{M}$ 
of a given \textit{input} assignment to the input variables
that satisfies\\
$(u_{VP}, u_{IP}, u_V) \in L_V$,
$(u_{VP}, u_{IP}, u_I) \in L_I$
and
$(u_{Ve}, u_{Ie}) \in \Gamma_e$ for all
$e\in E$.  Note that in this model, the constraint 
$(u_{VP}, u_{IP}, u_V) \in L_V$ does not (by itself) imply any constraint
on a ``I'' type port variable $u_{Ip}$, similarly, $u_{Vp}$ is not constrained
by $(u_{VP}, u_{IP}, u_I) \in L_V$.  Port variables are not constrained
by the constitutive laws $\Gamma$ (by themselves) either.

In the language of matroid theory, we can call the element $p_I$ an 
\textit{isthmus} of the matroid $\mathcal{M}(L_V)$; similarly, $p_V$ is
an isthmus of $\mathcal{M}(L_I)$.  In general, the matroid represented by
a matrix is characterized by the collection $\mathcal{I}$ 
of \textit{independent sets} 
of matrix columns, where a set of columns is called independent when it is
linearly independent.  (Matroid theory studies what can be deduced by 
the following three axioms satisfied by $\mathcal{I}$: (1) 
$\mathcal{I}\neq\emptyset$. 
(2) If $A\subset B\in\mathcal{I}$ then $A\in\mathcal{I}$.  (3) 
If $A$, $B$ $\in\mathcal{I}$ and $|A|<|B|$, then there exists $e\in B-A$ 
for which $A\cup\{e\}\in\mathcal{I}$.  For example, an isthmus $e$ is 
characterized by $A\cup\{e\}\in\mathcal{I}$ for all $A\in\mathcal{I}$.
The \textit{rank} of a subset $C\in U$ is the size of the largest independent
subset of $C$.  
%A set $D$ is called \textit{co-independent} if $\rank(U-D)$
%$=$ $\rank(U)$; in other words, removing $D$ does not diminish the original
%matroid's rank.)

We say a subspace pair problem with sources $S$ 
is \textit{well-posed} when for all input assignments there is a unique 
solution.

The condition that there is no cycle of voltage source branches 
in the ``voltage'' graph nor a
cutset of current source branches in the ``current graph''
is well-known to be necessary for 
an electrical network to have a unique 
solution for all choices of source values.
%Assume as usual no branch is taken to be both a current and voltage source.
This generalizes to:



\begin{theorem}
\label{Feasiblity1}
(1) If all ``V'' source port values are feasible under
the $L_V$ constraint then 
$\{p_V|p \mbox{\ is V-driven}\}$
is an independent set in the matroid $\mathcal{M}(L_V)$.
(2) If every solution is unique then 
$\{p_I|p \mbox{\ is V-driven}\}$
must be co-independent in $\mathcal{M}(L_I)$.
(3) If all ``I'' source port values are feasible under
the $L_I$ constraint then 
$\{p_I|p \mbox{\ is I-driven}\}$
is an independent set in the matroid $\mathcal{M}(L_I)$.
(4) If every solution is unique 
then the set of elements
$\{p_V|p \mbox{\ is I-driven}\}$
must be co-independent in $\mathcal{M}(L_V)$.
\end{theorem}

Proof of (1) and (3):
If set $S$ of input variables is dependent, then there is
some combination of input values that is not feasible.
Proof of (2) and (4):
A non-co-independent set $S$ must contain a cocircuit, so
there is a non-zero covector supported by $S$.  Hence there is a 
feasible variable assignment that is non-zero on the some port output
variables only.


When the constitutive laws are linear, the solutions of $\textbf{M}$ are
found from the \textit{intersection} of two linear subspaces:  Let $G$ 
be the diagonal matrix with ``conductances'' $g_e$ in its positions
indexed by $e \in E$ (so $\Gamma_e$ $=$ $\{ (v, g_e v) | v \in \Reals \}$
and 1 in its other diagonal positions.  The solution set projected onto
the $w_I$ variables is $L_VG\cap L_I$.

\section{DELETION AND CONTRACTION}

Given a subspace $L\subset\Reals^U$ and $e\in U$, the subspace $L-e$ 
``$L$ \textit{with} $e$ \textit{deleted}'' is the
subspace $L-e\subset\Reals^{U-\{e\}}$ defined by $L-e$ $=$
$\{ l(U-e) | l(U)\in L\}$, where $l(U-e)$ denotes the tuple $l(U)\in\Reals^U$
with component labeled by $e$ dropped.  
Thus, $L-e$ is the \textit{projection} of $L$ into $\Reals^{U-\{e\}}$.
If $L$ is the row space of matrix $M$,
then $L-e$ is the row space of $M(U-e)$, which is $M$ with column $e$ deleted.

The subspace $L/e$ ``$L$ \textit{with} $e$ \textit{contracted}'' is the 
subspace
of $\Reals^{\{U-e\}}$ defined by 
$L/e$ $=$
$\{ l(U-e) | l(U)\in L \mathrm{\ and\ } l(e)=0\}$.  
In other words, $L/e$ is the 
intersection of $L$ with
the (hyperplane) subspace of $\Reals^U$ with $l(e)=0$ 
projected into 
$\Reals^{\{U-e\}}$.

We now define deletion and contraction on subspace pairs:  
$(L_V, L_I)-e$ $=$ $(L_V-e, L_I/e)$ and 
$(L_V, L_I)/e$ $=$ $(L_V/e, L_I-e)$.  One can recognize that 
deleting element $e\in S$ from a subspace pair modeling an electrical network
corresponds to \textit{opening} the corresponding branch.  Dually, 
contraction corresponds to \textit{shorting} the branch.  Mechanically,
deletion of an edge corresponds to ``breaking'' the corresponding bar:
ignore any distance change between its ends and transmit no force.
Contraction corresponds to 
declaring the bar to be rigid, which rules out 
all (first order) distance changes between the endpoints and allows 
the bar 
to transmit arbitrary force of tension or compression.

\subsection{Nullators and Norators}
A \textit{nullator} element $e\in E$ 
expresses the ideal constitutive law $u_{Ve}=0$
and $u_{Ie}=0$ which approximates conditions at the input
to an amplifier when a system is stabilized by feedback.
Hence a nullator is declared by 
\textit{contracting $e$ in both
$L_V$ and $L_I$.}  Ordinarilly, this reduces both their ranks by 1.

A \textit{norator} element $e\in E$ indicates that the constitutive
law put no direct constraint on $u_{Ve}$ and $u_{Ie}$; the amplifiers
approximately adjust the output state so the feedback produces zero input.
Hence a norator is declared by 
\textit{deleting $e$ in both
$L_V$ and $L_I$.}  Ordinarilly, their ranks don't change.

Thus, for each ordinary nullator/norator pair, the rank balance conditions
are preserved.  

The conditions in Theorem \ref{Feasiblity1} apply to the subspace pair
obtained from all declarations of nullators, norators, opens and shorts.

\extra{ There is a subtle difference between declaring a V-source with 0
input value and contracting the same element.  If a set $S$ 
of $k$ such elements
is not independent in $\mathcal{M}(L_V)$, then the rank of $\mathcal{M}(L_V/S)$
will be more than
$\mbox{rank}(\mathcal{M}(L_V))-k$ but the given combination of input values
will still be feasible.  If they are not co-independent in
$\mathcal{M}(L_I)$ (which will certainly be true when there are no nullors).
then the rank of $\mathcal{M}(L_I-S)$ will be less than 
$\mbox{rank}(\mathcal{M}(L_I))$ but there will be a non-zero combination of
output only varibles.  (Physically, that corresponds to non-zero current 
circulating in a loop of ideal wires; or a non-zero self-stress in an
overbraced subframework of rigid bars.  The dual dual physical situation
is that is possible for the disconnected parts of an electrical network
to differ in electical potential when a cut-set of branches are removed;
mechanically, more flexes of the framework can exist when some edges are
removed. For this reason, we are careful to distinguish deletion/contraction
from port insertion.)}

\extra{
\subsection{Topological Formulas}

They can come out of the subspace pair formulation three ways:
\begin{itemize}
\item  Besides the ``$g$'' or ``$r$'' variables, do not insert ports but
do use an extra variable ``$x$'' to relate the voltage to current of one port.
Then the equation $\det(M_VG;M_I)=0$ has the form $Ax+B=0$, so $x=-B/A$.
\item Use the Pl\"{u}cker coordinate formulation of the intersection subspace to 
identify a minor of a hybrid or other description matrix of as the determinant
of the solution submatrix for a system of equations; then use the Cramer's
rule generalization to find the ratios of minors of the equation 
matrix to analyzed.  
This was done for my ISCAS 98 paper.
\item Use Rota's Grassmann-Cayley algrabra meet formula to extract expansion
directly from subspace pair matrices, together with the previous way to
identify Pl\"{u}cker coordinate ratios with description matrix minors.
\end{itemize}

}

\section{SUPPLEMENTAL SUBSPACE PAIR}

The \textit{supplemental subspace pair} of a subspace pair model
with sources is constructed by \textit{zeroing} all the sources.
Specifically, (1) for each ``V'' source element $p$, $p_V$ is 
contracted in both $L_V$ and $L_I$,
(2) for each ``I'' source element $p$, 
$p_I$ is contracted in both $L_V$ and $L_I$.


\extra{
\begin{figure}[htb]

\begin{minipage}[c]{.48\linewidth}
  \centering
 \centerline{\input{2res.pstex_t}}
%  \vspace{2.0cm}
\end{minipage}
%
\hfill
\begin{minipage}[c]{.48\linewidth}
\[
\begin{array}{cccc}
\multicolumn{4}{c}{\mbox{($M_V$ matrix)}} \\
0   &  1  &  0  &  0  \\
1   &  0  &  1  &  1  \\ \hline \hline
p_V & p_I & e_1 & e_2 \\ \hline \hline
1   &  0  &  0  &  0  \\
0   &  1  & -1  &  0  \\
0   &  0  &  1  &  -1 \\
\multicolumn{4}{c}{\mbox{($M_I$ matrix)}} \\
\end{array}
\]
\end{minipage}
%
\begin{minipage}[b]{.48\linewidth}
\[
\begin{array}{ccc}
  1  &  0  &  0  \\ \hline
 p_I & e_1 & e_2 \\ \hline
  0  &  0  &  0  \\
  1  & -1  &  0  \\
  0  &  1  &  -1
\end{array}
\]
ZIR analysis for voltage source input:
$p_V$ contracted.  Zero response (unique solution) even if
negative resistances $\neq 0$ are allowed.
\end{minipage}
\hfill
\begin{minipage}[b]{.48\linewidth}
\[
\begin{array}{ccc}
0   &  0  &  0  \\
1   &  1  &  1  \\ \hline
p_V & e_1 & e_2 \\ \hline
1   &  0  &  0  \\
0   &  1  &  -1
\end{array}
\]
ZIR analysis for current source input:
$p_I$ deleted.
Zero response (unique solution) \textit{unless}
$g_1 = -g_2$.  
\end{minipage}
\caption{Simple example that illustrates the solution is unique when the 
port is V-driven provided each $g_e\neq 0$, but the I-driven system has a
unique solution provided each $g_e>0$ because the resulting supplemental
oriented matroid pair has complementary bases and no common covector.}
\label{Simple}
%
\end{figure}
}

\section{NO-COMMON-COVECTOR PROPERTY AND $\mathcal{W}_0$ PAIRS}

The following theorem
demonstrates, by means of Sandberg and Willson's theory of 
$\mathcal{W}_0$ pairs, that any 
subspace pair model 
\extra{(with  its separation of
geometric/topological and constitutive constraints)}
can be analyzed
for unique solvability from the oriented matroid pair it generates.
Conversely, a matrix pair 
$(A,B)\in\mathcal{W}_0$ is characterized by a rank condition and a 
no-common-covector property.

\begin{theorem}
The subspace pair model has a unique solution for all source 
values when the supplementary oriented matroid pair has a complementary 
base pair and no common covector.
\end{theorem}

Proof:  The following matrices must be square and order $|U|$
for solutions of (..) to be unique:
$A$ $=$ $\left(\begin{array}{c}M_V\\ 0\end{array}\right)$ and
$B$ $=$ $\left(\begin{array}{c}0\\ -M_I\end{array}\right)$.
The theorem follows with these matrices used in the equivalance of 
2. and 5. in the theorem below.

\begin{theorem}
\label{sandwillompairtheorem}
For a pair 
%
% EDITdone
% delete ``order'' --- we understand that when we see 
% ``$n\times n$'', and this avoids confusing with order as in
% a linear ordering
%
of $n\times n$ 
matrices $(A,B)$, the following conditions are 
equivalent.
%
% EDITdone
% period, not colon
%
\begin{enumerate}
\item 
$(A,B)\in {\mathcal W}_0$ in the sense of 
Sandberg and 
Willson~\cite{SWExistancePf,W0APPLpaper};
e.g., $|AD+B|\neq 0$ for all positive diagonal $D$, etc.
\item 
$\rank{\mathcal M}\hmat{A}{B}=n$ and 
${\mathcal L}\hmat{A}{B} \cap {\mathcal L}\hmat{I}{-I}=\{0\}.$
\item 
$\rank{\mathcal M}\hmat{A}{B}=n$ and 
${\mathcal V}\hmat{A}{B} \cap {\mathcal V}\hmat{I}{-I}=\{0\}.$
\item 
\textrm{(}Fundamental theorem of Sandberg and 
Willson~\cite{SWExistancePf,W0APPLpaper}\textrm{)}\ 
%
% EDITdone
% use ~ before \cite so a space is left between the word and the reference
%
For all functions $F:{\bf R}^n\rightarrow {\bf R}^n$ of the form
$F(x)_k=f_k(x_k)$ where each $f_k$ is a strictly 
monotone increasing 
function from 
${\bf R}$ {\em onto} $\bf R$ and for all $c\in {\bf R}^n$, the 
equation
\[
AF(x) + Bx = c
\]
has a unique solution $x$.% \cite{W0APPLpaper,W0paper}.
\item 
For all functions $G:{\bf R}^n\rightarrow {\bf R}^n$ of the form
$G(w)_k=g_k(w_k)$ where each $g_k$ is a strictly monotone increasing 
function from ${\bf R}$ {\em onto} $\bf R$ and for all 
$d^{\prime}, d^{\prime\prime}\in {\bf R}^n$, the 
equations
\begin{equation}
\label{W0dualproblem}
u^t=z^tA+d^{\prime},\;\; w^t=z^tB+d^{\prime\prime},\;\; u=-G(w)
\end{equation}
have a unique solution $(u,w,z)$.
\end{enumerate}
\end{theorem}

Note:  A direct inductive proof is obtainable by generalizing 
the proofs given in \cite{HaslerNeirynck}.  This approach has the advantage
of revealing circuit theoretic concepts that occur.  See also
\cite{Fosseprez}.  One of the steps is to prove that if the no-common-covector
property is true for $(M_V,M_I)$, then it is true for the matrix pair 
from the system obtained by replacing one of the non-linear elements by a
source.

\begin{theorem}
With constitutive laws given by monotone increasing functions from $\Reals$ onto
$\Reals$, every subspace pair problem can be posed as a case of theorem
\ref{sandwillompairtheorem}, and every case of theorem
\ref{sandwillompairtheorem} can be posed as a subspace pair problem.
\end{theorem}


% References should be produced using the bibtex program from suitable
% BiBTeX files (here: strings, refs, manuals). The IEEEbib.bst bibliography
% style file from IEEE produces unsorted bibliography list.
% -------------------------------------------------------------------------
\bibliographystyle{IEEEbib}
%\bibliography{strings,refs,manuals}
\bibliography{OMPEMech}

\end{document}
