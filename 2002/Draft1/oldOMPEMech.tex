% Template for ICIP-2000 paper; to be used with:
%          spconf.sty  - ICASSP/ICIP LaTeX style file, and
%          IEEEbib.bst - IEEE bibliography style file.
% --------------------------------------------------------------------------
\documentclass{article}
\usepackage{spconf,amsmath,epsfig}

% Example definitions.
% --------------------
\def\x{{\mathbf x}}
\def\L{{\cal L}}
\def\Reals{\ensuremath{\mathbf R}}
\newtheorem{theorem}{Theorem}[section]

%    Absolute value notation
\newcommand{\abs}[1]{\lvert#1\rvert}


%    Rank
\DeclareMathOperator{\rank}{rank}

%    Matroid Union 
\newcommand{\munion}{\lor}

%   Set minus
\newcommand{\sminus}{\backslash}


%   Two matrices horizontally concatenated, space between can be
%   adjusted here
%
\newcommand{\hmat}[2]{[#1\;#2]}

%   Oriented Matroid Pair
\newcommand{\OMP}[2]{#1,\,#2}

\newcommand{\extra}[1]{{\small{#1}}}


% Title.
% ------
\title{\extra{REPORT:}
(\today)
AN ORIENTED MATROID PAIR MODEL FOR ELECTRICAL AND MECHANICAL NETWORKS
}
%
% Single address.
% ---------------
\name{Seth Chaiken\thanks{Part of this reseach was done during a Sabbatical
from the University at Albany in 2001.}}
\address{University at Albany\\
	Department of Computer Science\\
	Albany, NY 12222\\
	\texttt{sdc@cs.albany.edu}}
%
\begin{document}
%\ninept
%
\maketitle
%

\begin{abstract}
Both resistive electrical networks and elastic mechanical systems such
as trusses have a topological or geometric structure together with constitutive
laws for the elements prior to their interconnection.
Oriented matroids provide a common discrete mathematical model 
for such structures in which relationships on the signs of element 
quantities can be expressed.
Pairing of oriented matroids
enables non-linear monotone constitutive laws to be fit
into the abstraction in a way that accomodates port and nullor
insertions as well.

The resulting mathematical model clarifies some mechanical analogies for 
these 
circuit theory concepts, enables constraints on the signs of system response 
quantities to be predicted from the structure when this is possible, 
and derives topological solution formulas for linearized mechanical 
systems in which the analog of a tree-sum is a sum over minimally rigid
trusses.  It also enables the theories for existance and uniqueness to
be applied to mechanical systems with small displacements.
\end{abstract}
%

\section{Introduction}
\label{sec:intro}

Our work \cite{sdcOMP} shows that the model and results of 
\cite{HaslerDApplMath,HaslerNeirynck} generalize from a graph model to a
linear subspace pair model
for which 
a pair of oriented
matroids (with a common ground set that generalizes the
set of resistor and other edges)
rather than a graph with designated resistor, source, nullator
and norator edges,
is the discrete structure 
%in which the topological conditions are expressed 
that expresses 
%the 
topological conditions
for existance or uniqueness of solutions.  These results, like
those of \cite{HaslerDApplMath,HaslerNeirynck} apply to
models whose non-linearities are monotone.
Each oriented matroid codes which combinations of \textit{signs} $\{+,-,0\}$
are feasible as \textit{signatures} of the real coordinate tuple  members
of each subspace.
Bachem and Kern's book \cite{BachemKern}
motivates oriented matroids from this direction.  For example, a dual 
pair of oriented matroids abstracts an orthogonally complementary pair
of (finite dim. real) linear subspaces.  Minty's painting property,
most popularly known as a theorem about directed graphs, then emerges
as an excellent \textit{defining axiom} for oriented matroids.
This led us to generalize in \cite{sdcOMP}
Hasler and Neirynck's notion of a ``pair of 
conjugate trees'' to a ``complementary pair of bases''; and of 
a ``non-trivial uniform partial orientation of the resistors''
to a ``common (non-zero) covector''.   
Theory valid for all oriented matroid pairs, not just those represented by
a pair of linear subspaces, was presented in \cite{sdcOMP}.
We will clarify below how our 
generalizations of Foss\'{e}prez, Hasler and Neirynck's conditions
for unique solvability
are equivalent to the characterization of $\mathcal{W}_0$ matrix pairs
shown by Sandberg and Willson \cite{SWExistancePf,W0APPLpaper}
to be the condition for unique solvability
of a \textit{common} (not just similar) 
problem with monotone non-linearities.
\extra{

In the electrical circuit theory literature, the circuit ``topology''
means the network graph (to which Kirchhoff's laws apply)
together with particular kinds of ``device elements''
such as resistors, capacitors, voltage sources (batteries), 
current sources, etc., associated with single graph edges, and possibly
``multiport'' elements associated with multiple graph edges.
Kirchhoff's current law is that current, a scalar in each directed edge,
satisfies conservation of flow.  Kirchhoff's voltage law amounts to the 
condition that 
the voltage drop, a scalar in each directed edge, is equal to the 
difference of the scalar potential values at the endpoints.
Problems with multiport elements are reduced to 
those with only single edge elements, in order to apply
the theory of  \cite{HaslerDApplMath},  through the use of 
``nullator'' and ``norator'' elements, discussed below.  
Detailed exposition of the problems,
reductions, theory and applications is given in \cite{HaslerNeirynck}.

}We will introduce our model by using it to demonstrate
the analogy between the electrical and mechanical elastic 
applications.   
The  subspace 
pairs of the latter 
are not graphic cycle and cocycle spaces.  
Unlike in the electrical case, the structural model is ``inexact'' because
it depends of the precise sizes and positions of
the bars in addition to their interconnection; 
the constitutive laws describe elastic characteristics of the bars.
\extra{

Interesting purely ``discrete'' results (and open problems)
are known in rigidity theory for \textit{generic rigidity}, where
the positions are assumed to have no algebraic dependencies.  Then,
the oriented matroid of the rigidity matrix (defined below)
doesn't change with small position perturbations, and the property
of first order rigidity coincides with local rigidity:  The framework
embedding is determined up to Euclidean motions by the lengths of the
bars, when restricted to a neighborhood of the original embedding.

}We believe that our formulation
has the advantage over conventional matrix formulations
of making more explicit the discrete structure model for circuit 
topology and some conditions when this model determines qualitative
relationships among circuit quantity signs independently of 
particular monotone constitutive laws.

\section{Basic}
\label{sec:Basic}

A \textit{subspace pair} $(L_V, L_I)$ is a pair of linear subspaces of
$\Reals^U$, where the elements of finite set $U$ index the coordinates.
The scalar product $v\cdot w = \sum_{e\in U}v_e w_e$ is used to define that
$v, w \in \Reals^U$ are \textit{orthogonal} when $v\cdot w = 0$.  An 
\textit{orthogonal subspace pair}  satisfies $v\cdot w = 0$ for
all $v\in L_V$ and $w\in L_I$.  A subspace pair has \textit{full rank} when
$\mathrm{rank}(L_V)+\mathrm{rank}(L_I)= |U|$.  Hence an orthogonal full rank 
subspace pair is a linear subspace paired with its orthogonal complement.

The structure of an electrical network is defined 
beginning with the \textit{network graph}
$\mathcal{N}$ with \textit{nodes} $N$  and \textit{arcs} $U$.  
(The generality obtainable by port, nullator/norator or nullor, and 
device characteristic insertions will be treated later.)
Each arc has 
a fixed but arbitrary direction to define the sign of its voltage drop and
current flow.  The \textit{incidence matrix} $M_V$, with rows indexed by $N$
and columns indexed by $U$ is defined so $M_V(n,e)=+1$ when the tail of $e$ is
$n$, $-1$ if the head of $e$ is $n$, and $0$ if $n$ and $e$ are not incident.

When $L_V$ is the row space of $M_V$ and $L_I$ is the orthogonal complement
of $L_V$, the members of $L_V$ are voltage drop tuples in $\mathcal{N}$ 
feasible under Kirchhoff's voltage law and the members of $L_I$ are the 
current flow tuples feasible under Kirchhoff's current law.  These facts 
restate Kirchhoff's laws and Tellegen's theorem.  Note that we can determine
$(L_V, L_I)$ from one of these subspaces given and Tellegen's theorem: The
role of nodes here is not strictly necessary.

The definition of mechanical network structure begins with the 
(undirected) \textit{framework graph} $\mathcal{F}$ with \textit{vertices}
$N$ and \textit{edges} $U$.  A \textit{framework} $\mathcal{F}(\mathbf{p})$ 
in d $dimentions$ is a framework graph $\mathcal{F}$ and an \textit{embedding} 
$\mathbf{p}:N\rightarrow \Reals^d$.  
The embedding assigns each vertex to a point in
$d$-dimensional space.  The \textit{rigidity matrix} $M_V$ 
has $d|N|$ rows, each
indexed by one coordinate in $\Reals^d$ of the point that embeds one vertex.
For edge $e=(i,j)\in U$, column $M_V(e)$ of the rigidity matrix is defined 
(when vertices are numbered 0 through $|N|-1$):\\
\begin{minipage}{\linewidth}
\begin{align*}
(0, \ldots,0, &\mathbf{p}(n_i)-\mathbf{p}(n_j),\\
              &\mathrm{positions\ } di\ldots di+d-1
\end{align*}
\end{minipage}
\hfill\begin{minipage}{.85\linewidth}
\begin{align*}
\hfill 0,\ldots,0,&\mathbf{p}(n_j)-\mathbf{p}(n_i), 0, \ldots,0)^T\\
                  &\mathrm{positions\ } dj\ldots dj+d-1
\end{align*}
      \end{minipage}\\
This definition is echoed from the literature \cite{RigidityBook} on rigidity theory, except
we interchange rows and columns.  Just as we deemphasized nodes of electrical
networks, we will use merely the row space of $M_V$ for most of what follows.

\extra{
The rigidity matrix as a function of the embedding $\mathbf{p}$ is denoted
$M_V(\mathbf{p})$.  The row vector $\mathbf{p}$ left multiplied with
$M_V(\mathbf{p})$
is the row tuple denoted $\mathbf{L}=\mathbf{p}M_V(\mathbf{p})$  Then, 
$\mathbf{L}_e$ $=$ 
$(\mathbf{p}M_V(\mathbf{p}))_e$ $=$ $|\mathbf{p}_i-\mathbf{p}_j|^2$ for
each edge $e=(i,j)$.  Now if each $\mathbf{p}_i$ is a differentiable function
of $t$, $d\mathbf{L}/dt$ $=$ $2\mathbf{p}'M_V(\mathbf{p})$.  Framework
$\mathcal{F}(\mathbf{p})$ is \textit{first-order rigid} when 
$d\mathbf{L}/dt$ $=$ $0$ for all $\mathbf{p}'$ implies 
$|\mathbf{p}_i-\mathbf{p}_j|^2$ is constant for all pairs $i$, $j$, not just
endpoints of edges. 
}%extra

Let $\mathbf{p}(i)-\mathbf{p}(j)$ for edge $e=\{i,j\}$ be called the
\textit{vector from } $j$ \textit{to} $i$.  
It is known that the row space $L_V$ 
of $M_V$ consists of tuples in $\Reals^{U}$
such that component $v(e)\in\Reals$ for $e\in U$ is the projection of the
relative velocity of vertex $i$ with respect to vertex $j$ projected onto
the vector from $j$ to $i$, for some combination of vertex velocities 
$\mathbf{v}:N\rightarrow\Reals^d$:
$v(e)=(\mathbf{v}(i)-\mathbf{v}(j))\cdot(\mathbf{p}(i)-\mathbf{p}(j))$.

It is also known that the $L_I$, the orthogonal complement of $L_V$, 
is comprised of the tuples $\sigma:U\rightarrow\Reals$ 
of scalars for which the framework is in static
equilibrium when each edge $e$ exerts force 
$\sigma(e)(\mathbf{p}(j)-\mathbf{p}(i))$ on vertex $i$.  By this convention, 
$\sigma(e)>0$ means $e$ is under tension and $\sigma(e)<0$ means $e$ is under 
compression.  Each tuple $\sigma\in L_I$ is called a \textit{self-stress}.

Under this analogy, 
(1) KVL corresponds to geometric consistancy of first order
edge length changes under changes in the embedding, (2) KCL corresponds to
Newton's laws of static equilibrium, and (3) Tellegen's theorem corresponds
to a virtual work principle, that static equilibrium is
characterized by 
the internal forces of every virtual embedding change 
doing zero virtual work.

\subsection{Independent Variables and Bases}
KVL, KCL and analogous mechanical structural or geometric laws 
are each formulated above by a constraint of the form 
$v\in L$ $=$ $\mbox{row space}(M)$
where $v$ is a tuple of variables indexed by $U$
and $L\subset\Reals^U$ is a linear subspace.  The problem of
reformulating such a law by a system of linear equations is
solved as follows:  A maximal subset $B\subset U$ corresponding to
a linearly independent set of columns of $M$ is found.  Such a $B$ is
called a \textit{basis in the matroid} $\mathcal{M}(L)$ 
\textit{represented by} the columns of matrix $M$ or the linear subspace
$L$.  Row operations and possibly deletion of zero rows can transform
$M$ to $( I\ ;\ M^{\overline{B}} )$ (after column permutation)
where $I$ is the $r\times r$ identity matrix, where $r$ $=$ 
$\mbox{rank}(M)$ $=$ $\mbox{dim}(L)$ $=$ $\mbox{rank}(L)$
$=$ $\mbox{rank}(\mathcal{M}(L))$.    
It is now clear that $v\in L$ is characterized by 
$v_{\overline{B}}$ $=$ $v_{B}M^{\overline{B}}$.
For each independently chosen
$v_{B}\in\Reals^B,$ $v=(v_B;v_{\overline{B}})$ is unique tuple
for which the $B$ coordinates equal $v_B$.


\extra{\section{Elastic Analog of the Nodal Admittance Matrix}

Reduced nodal admittance matrix.  Nodal resistance matrix.
Interaction with a physical framework with it's environment.
A framework is first order rigid iff it ``resolves all applications of
static equilibrium forces''.  However, every physical bar has some
elasticity:  An ideal rigid bar is analogous to an ideal voltage 
source.  Hence, given an elastic framework, for every application
of static equilibrium forces on the vertices, the vertex positions
will change as the bars stretch or shrink under the forces they now
carry to resolve the applied force.  These first order vertex position 
changes are given by $Z\mathbf{f}$.

The environment might interact by ``forcing'' some vertices to change position
relative to one another.
Intuitively, the framework will ``push back''.  The other vertices are free
to move as adjacent vertices move and incident bars change length in
response to the forces developed in them to resolve the forces required
to hold the framework in its new position.  The position changes of the
free vertices $V$ can be calculated by solving for the unknown position changes
in the system of equations $(Y\mathbf{v}_V)(V)=0$.

For our purposes, we insert port elements in order to make interactions 
with the environment explicit.  This enables a coordinate of an 
environmental interaction quantity to correspond to an oriented matroid
element, so that its sign can be read off from the corresponding entry
in a covector.

It's yet to be done to handle simutaneous application of force to more 
than 2 vertices....



to be written..}


\section{THE SUBSPACE PAIR MODEL}

Ports are introduced so the response of an electrical network to current 
and/or voltage sources, and the mechanical analogs, can be formulated.
After modeling device characteristics, questions of existance and 
uniquenss of solution for various kinds of sources can be formulated.
Familiar topological conditions on dependencies among source values
pertain to the \textit{matroids} of the subspaces $L_V$ and $L_I$.
Questions about existance and uniqueness of solution
will be answered in terms of supplementary subspace pair models
which are obtained by the familiar operations of opening and shorting ports.
Finally, operations on subspace pairs that model nullor insertion are
defined, so that such ideal elements can be modeled combinatorially or
geometrically.  

The supplemental subspace pair derived after nullor 
insertion will typically not be orthogonal.  One might also choose to
model linearized CCCSs or VCVSs within one of the subspaces.  
Each port insertion generally increases $\rank(L_V)+\rank(L_I)-|U|$;
system behavior for linearized constitutive laws will be shown to be
represented by the intersection of two linear spaces.
Hence we do not assume any rank or orthogonality conditions on subspace 
pairs in the definitions below.

Given a subspace pair $(L_V, L_I)$ and element $p\in U$ not already a port, 
we define the 
operation of \textit{inserting a port at  $p$} 
as follows: A new subspace pair $(L'_V, L'_I)$
is defined with $U'=U-\{p\}\cup\{p_V,p_I\}$, $L'_V=L_V\oplus\Reals$ (direct 
sum) with $p_V$ replacing $p$ and $p_I$ indexing the coordinate of the added
subspace, and $L'_I=L_I\oplus\Reals$ with $p_I$ replacing $p$ and $p_V$ 
indexing the coordinate of the added subspace.  Note (going to 
$(L'_V, L'_I)$) that the ranks of
$L_V$ and $L_I$ each increase by 1, and $|U'|$ $=$ $|U|+1$.  After $p$ port
insertions, we denote the final $U=E\cup P_V \cup P_I$ with pairwise disjoint
$E$, $P_V$ and $P_I$, $|P_V|$ $=$ $|P_I|$ $=$ $p$, $P_V\cup P_I$ being the 
replacement elements.  Let $P$ denote $P_V \cup P_I$.  

The \textit{subspace pair model} $\mathbf{M}$ $=$ 
$(E, \Gamma, P, (L_V, L_I))$ consists of finite set $E$ of 
\textit{device elements}, \textit{constitutive law relations}
$\Gamma = \{\Gamma_e\subset\Reals\times\Reals | e \in E\}$, a finite set
$P=P_V \cup P_I$ that result from inserting ports as defined above, 
and a subspace pair $(L_V, L_I)$ over $\Reals^U$ with $U=E\cup P$.

The \textit{variables} of $\mathbf{M}$ are 
$\{u_{Ve}, u_{Ie} | e \in E\}$  $\cup$ \\
$\{ u_{Vp}, u_{Ip} | p_I, p_V \in P \}$.  
(For brevity, subscript ``$Vp$'' means port element
$p_V\in P_V$, etc.)
A \textit{subspace pair model with sources} $S$ 
is a subspace pair model $(E, \Gamma, P, (L_V, L_I), S)$
together with a subset $S$ of exactly $|P|$ of the $2|P|$ elements
in $P$.  A \textit{$V$-driven port} is a port $p\in P$ for which 
$p_V\in S$ and $p_I\not\in S$, then $u_{Vp}$ is called 
an \textit{input variable}.   Reverse $V$ and $I$ to define an 
\textit{$I$-driven port} and its input variable.

A \textit{solution} of $\mathbf{M}$ with sources
is a real valued extension to \textit{all} variables of $\mathbf{M}$ 
of a given \textit{input} assignment to the input variables
that satisfies\\
$(u_{VP}, u_{IP}, u_V) \in L_V$,
$(u_{VP}, u_{IP}, u_I) \in L_I$
and
$(u_{Ve}, u_{Ie}) \in \Gamma_e$ for all
$e\in E$.  Note that in this model, the constraint 
$(u_{VP}, u_{IP}, u_V) \in L_V$ does not (by itself) imply any constraint
on a ``I'' type port variable $u_{Ip}$, similarly, $u_{Vp}$ is not constrained
by $(u_{VP}, u_{IP}, u_I) \in L_V$.  Port variables are not constrained
by the constitutive laws $\Gamma$ (by themselves) either.

In the language of matroid theory, we can call the element $p_I$ an 
\textit{isthmus} of the matroid $\mathcal{M}(L_V)$; similarly, $p_V$ is
an isthmus of $\mathcal{M}(L_I)$.  In general, the matroid represented by
a matrix is characterized by the collection $\mathcal{I}$ 
of \textit{independent sets} 
of matrix columns, where a set of columns is called independent when it is
linearly independent.  (Matroid theory studies what can be deduced by 
the following three axioms satisfied by $\mathcal{I}$: (1) 
$\mathcal{I}\neq\emptyset$. 
(2) If $A\subset B\in\mathcal{I}$ then $A\in\mathcal{I}$.  (3) 
If $A$, $B$ $\in\mathcal{I}$ and $|A|<|B|$, then there exists $e\in B-A$ 
for which $A\cup\{e\}\in\mathcal{I}$.  For example, an isthmus $e$ is 
characterized by $A\cup\{e\}\in\mathcal{I}$ for all $A\in\mathcal{I}$.
The \textit{rank} of a subset $C\in U$ is the size of the largest independent
subset of $C$.  A set $D$ is called \textit{co-independent} if $\rank(U-D)$
$=$ $\rank(U)$; in other words, removing $D$ does not diminish the original
matroid's rank.)

We say a subspace pair problem with sources $S$ 
is \textit{well-posed} when for all input assignments there is a unique 
solution.


The condition that there is no cycle of voltage source branches nor a
cutset of current source branches is well-known to be necessary for 
an electrical network to have a solution for all choices of source values.
Assume as usual no branch is taken to be both a current and voltage source.
Our generalization of the following:
The set of elements $p_V$ 
for ports driven by ``V'' sources
is an independent set in the matroid $\mathcal{M}(L_V)$ and 
co-independent set
in $\mathcal{M}(L_I)$,
and the elements $p_I$ for ports driven by ``I'' sources comprise an
independent set in $\mathcal{M}(L_I)$ and a co-independent set in
$\mathcal{M}(L_V)$.  

When the constitutive laws are linear, the solutions of $\textbf{M}$ are
found from the \textit{intersection} of two linear subspaces:  Let $G$ 
be the diagonal matrix with ``conductances'' $g_e$ in its positions
indexed by $e \in E$ (so $\Gamma_e$ $=$ $\{ (v, g_e v) | v \in \Reals \}$
and 1 in its other diagonal positions.  The solution set projected onto
the $w_I$ variables is $L_VG\cap L_I$.

\section{DELETION AND CONTRACTION}

Given a subspace $L\subset\Reals^U$ and $e\in U$, the subspace $L-e$ 
``$L$ \textit{with} $e$ \textit{deleted}'' is the
subspace $L-e\subset\Reals^{U-\{e\}}$ defined by $L-e$ $=$
$\{ l(U-e) | l(U)\in L\}$, where $l(U-e)$ denotes the tuple $l(U)\in\Reals^U$
with component labeled by $e$ dropped.  If $L$ is the row space of matrix $M$,
then $L-e$ is the row space of $M(U-e)$, which is $M$ with column $e$ deleted.

The subspace $L/e$ ``$L$ \textit{with} $e$ \textit{contracted}'' is the 
subspace
of $\Reals^{\{U-e\}}$ defined by 
$L/e$ $=$
$\{ l(U-e) | l(U)\in L \mathrm{\ and\ } l(e)=0\}$.  
In other words, $L/e$ is the 
intersection of $L$ with
the (hyperplane) subspace of $\Reals^U$ with $l(e)=0$ followed by the $e$ 
coordinate removed.  

We now define deletion and contraction on subspace pairs:  
$(L_V, L_I)-e$ $=$ $(L_V-e, L_I/e)$ and 
$(L_V, L_I)/e$ $=$ $(L_V/e, L_I-e)$.  One can recognize that 
deleting element $e\in S$ from a subspace pair modeling an electrical network
corresponds to \textit{opening} the corresponding branch.  Dually, 
contraction corresponds to \textit{shorting} the branch.  Mechanically,
deletion of an edge corresponds to ``breaking'' the corresponding bar:
ignore any distance change between its ends and transmit no force.
Contraction corresponds to 
declaring the bar to be rigid, which rules out 
all (first order) distance changes between the endpoints and allows 
the bar 
to transmit arbitrary force.

\section{SIMPLE EXAMPLES}


\begin{figure}[htb]

\begin{minipage}[c]{.48\linewidth}
  \centering
 \centerline{\input{2res.pstex_t}}
%  \vspace{2.0cm}
\end{minipage}
%
\hfill
\begin{minipage}[c]{.48\linewidth}
\[
\begin{array}{cccc}
\multicolumn{4}{c}{\mbox{($M_V$ matrix)}} \\
0   &  1  &  0  &  0  \\
1   &  0  &  1  &  1  \\ \hline \hline
p_V & p_I & e_1 & e_2 \\ \hline \hline
1   &  0  &  0  &  0  \\
0   &  1  & -1  &  0  \\
0   &  0  &  1  &  -1 \\
\multicolumn{4}{c}{\mbox{($M_I$ matrix)}} \\
\end{array}
\]
\end{minipage}
%
\begin{minipage}[b]{.48\linewidth}
\[
\begin{array}{ccc}
  1  &  0  &  0  \\ \hline
 p_I & e_1 & e_2 \\ \hline
  0  &  0  &  0  \\
  1  & -1  &  0  \\
  0  &  1  &  -1
\end{array}
\]
ZIR analysis for voltage source input:
$p_V$ contracted.  Zero response (unique solution) even if
negative resistances $\neq 0$ are allowed.
\end{minipage}
\hfill
\begin{minipage}[b]{.48\linewidth}
\[
\begin{array}{ccc}
0   &  0  &  0  \\
1   &  1  &  1  \\ \hline
p_V & e_1 & e_2 \\ \hline
1   &  0  &  0  \\
0   &  1  &  -1
\end{array}
\]
ZIR analysis for current source input:
$p_I$ deleted.
Zero response (unique solution) \textit{unless}
$g_1 = -g_2$.  
\end{minipage}
\caption{Simple Example.}
\label{Simple}
%
\end{figure}


\section{NO-COMMON-COVECTOR PROPERTY AND $\mathcal{W}_0$ PAIRS}



The following theorem
demonstrates, by means of Sandberg and Willson's theory of 
$\mathcal{W}_0$ pairs, that any 
subspace pair model (with  its separation of
geometric/topological and constitutive constraints) can be analyzed
for unique solvability from the oriented matroid pair it generates.
Conversely, a matrix pair 
$(A,B)\in\mathcal{W}_0$ is characterized by a rank condition and a 
no-common-covector property.

\begin{theorem}
The subspace pair model has a unique solution for all source 
values when the corresponding oriented matroid pair has a complementary 
base pair and no common covector.
\end{theorem}

Proof:  The following matrices must be square and order $|U|$
for solutions of (..) to be unique:
$A$ $=$ $\left(\begin{array}{c}M_V\\ 0\end{array}\right)$ and
$B$ $=$ $\left(\begin{array}{c}0\\ -M_I\end{array}\right)$.
The theorem follows with these matrices used in the equivalance of 
2. and 5. in the theorem below.

\begin{theorem}
\label{sandwillompairtheorem}
For a pair 
%
% EDITdone
% delete ``order'' --- we understand that when we see 
% ``$n\times n$'', and this avoids confusing with order as in
% a linear ordering
%
of $n\times n$ 
matrices $(A,B)$, the following conditions are 
equivalent.
%
% EDITdone
% period, not colon
%
\begin{enumerate}
\item 
$(A,B)\in {\mathcal W}_0$ in the sense of 
Sandberg and 
Willson~\cite{SWExistancePf,W0APPLpaper};
e.g., $|AD+B|\neq 0$ for all positive diagonal $D$, etc.
\item 
$\rank{\mathcal M}\hmat{A}{B}=n$ and 
${\mathcal L}\hmat{A}{B} \cap {\mathcal L}\hmat{I}{-I}=\{0\}.$
\item 
$\rank{\mathcal M}\hmat{A}{B}=n$ and 
${\mathcal V}\hmat{A}{B} \cap {\mathcal V}\hmat{I}{-I}=\{0\}.$
\item 
\textrm{(}Fundamental theorem of Sandberg and 
Willson~\cite{SWExistancePf,W0APPLpaper}\textrm{)}\ 
%
% EDITdone
% use ~ before \cite so a space is left between the word and the reference
%
For all functions $F:{\bf R}^n\rightarrow {\bf R}^n$ of the form
$F(x)_k=f_k(x_k)$ where each $f_k$ is a strictly 
monotone increasing 
function from 
${\bf R}$ {\em onto} $\bf R$ and for all $c\in {\bf R}^n$, the 
equation
\[
AF(x) + Bx = c
\]
has a unique solution $x$.% \cite{W0APPLpaper,W0paper}.
\item 
For all functions $G:{\bf R}^n\rightarrow {\bf R}^n$ of the form
$G(w)_k=g_k(w_k)$ where each $g_k$ is a strictly monotone increasing 
function from ${\bf R}$ {\em onto} $\bf R$ and for all 
$d^{\prime}, d^{\prime\prime}\in {\bf R}^n$, the 
equations
\begin{equation}
\label{W0dualproblem}
u^t=z^tA+d^{\prime},\;\; w^t=z^tB+d^{\prime\prime},\;\; u=-G(w)
\end{equation}
have a unique solution $(u,w,z)$.
\end{enumerate}
\end{theorem}

Note:  A direct inductive proof is obtainable by generalizing 
the proofs given in \cite{HaslerNeirynck}.  This approach has the advantage
of revealing circuit theoretic concepts that occur.  See also
\cite{Fosseprez}.  One of the steps is to prove that if the no-common-covector
property is true for $(M_V,M_I)$, then it is true for the matrix pair 
from the system obtained by replacing one of the non-linear elements by a
source.



% References should be produced using the bibtex program from suitable
% BiBTeX files (here: strings, refs, manuals). The IEEEbib.bst bibliography
% style file from IEEE produces unsorted bibliography list.
% -------------------------------------------------------------------------
\bibliographystyle{IEEEbib}
%\bibliography{strings,refs,manuals}
\bibliography{OMPEMech}

\end{document}
