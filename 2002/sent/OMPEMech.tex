% Template for ICIP-2000 paper; to be used with:
%          spconf.sty  - ICASSP/ICIP LaTeX style file, and
%          IEEEbib.bst - IEEE bibliography style file.
% --------------------------------------------------------------------------
\documentclass{article}
\usepackage{spconf,amsmath,epsfig,verbatim}
% verbatim needed for comment environment

% Example definitions.
% --------------------
\def\x{{\mathbf x}}
\def\L{{\cal L}}
\def\Reals{\ensuremath{\mathbf R}}
\newtheorem{theorem}{Theorem}
\newtheorem{definition}{Definition}

\newcommand{\supp}[1]{{{\mbox{supp\ }#1}}}

%    Absolute value notation
\newcommand{\abs}[1]{\lvert#1\rvert}


%    Rank
\DeclareMathOperator{\rank}{rank}

%    Matroid Union 
\newcommand{\munion}{\lor}

%   Set minus
\newcommand{\sminus}{\backslash}


%   Two matrices horizontally concatenated, space between can be
%   adjusted here
%
\newcommand{\hmat}[2]{[#1\;#2]}

%   Oriented Matroid Pair
\newcommand{\OMP}[2]{#1,\,#2}

%\newcommand{\extra}[1]{{\small{#1}}}
\newcommand{\extra}[1]{}

% Title.
% ------
\title{\extra{REPORT:}
%(\today
(Proposal for ISCAS 2002)
%\the\time)
THE ORIENTED MATROID PAIR MODEL FOR MONOTONE
DC ELECTRICAL AND ELASTIC NETWORK UNIQUE SOLVABILITY
}
%
% Single address.
% ---------------
\name{Seth Chaiken\thanks{Part of this reseach was done during a Sabbatical
from the University at Albany in 2001.}}
\address{University at Albany,
	Department of Computer Science\\
	Albany, NY 12222 (\texttt{sdc@cs.albany.edu})}
%
\begin{document}
%\renewcommand{\baselinestretch}{0.90}
\ninept
%
\maketitle
%

\begin{abstract}
Resistive electrical networks and elastic mechanical systems such
as trusses have a topological or geometric structure together with constitutive
laws for the elements prior to their interconnection.
Oriented matroids provide a common discrete mathematical model 
for such 
structure
%structures
in which relationships on the signs of element 
quantities can be expressed.
Pairing of oriented matroids
enables non-linear monotone constitutive laws to be fit
into the abstraction in a way that allows port and nullor
insertions and provides discrete unique solvability conditions.

The resulting mathematical model clarifies some mechanical analogies for 
these 
circuit theory concepts, 
relates apparently dissimilar published theories
for existence and uniqueness and shows how to 
handle elastic mechanical systems with small displacements.
It also enables constraints on the signs of system quantities
to be predicted from the structure when this is possible.
\extra{Finally, it 
derives topological solution formulas for linearized mechanical 
systems in which the analog of a tree-sum is a sum over minimally rigid
trusses.}
\end{abstract}
%

\section{Introduction}
\label{sec:intro}

%Our work \cite{sdcOMP} shows that the model and results of 
%\cite{HaslerDApplMath,HaslerNeirynck} generalize from a graph model to a
%linear subspace pair model
%for which the discrete structure 
%%in which the topological conditions are expressed 
%that expresses 
%%the 
%topological conditions
%for existance or uniqueness of solutions
%is pair of oriented
%matroids, rather than a graph with designated resistor, source, nullator
%and norator edges.  

Our topic in non-linear systems is DC equations, 
say for operating points/resistive circuits, whose only
non-linearities are 
monotone increasing bijections $\Reals\rightarrow\Reals$.  
Special case conditions
for existence and uniqueness of solutions for all 
%choices of 
such non-linear
functions and additive 
%(source value)
constants were given by Duffin, Minty,
and Rockafellar.  
The more general determinant based theory
of $\mathcal{W}_0$ matrix pairs due to Sandberg and Willson 
\cite{SWExistancePf} extended Fiedler and Pt\'{a}k's work \cite{FiedlerPtak}.
Nielsen and Willson \cite{NielWillpaper}
%applied it \cite{NielWillpaper} to demonstrate
%proved with it  
used it to prove 
that disallowing the 2 transistor feedback structure in a transistor circuit
is sufficient for uniqueness.
Graph based theories that identified structures forbidden for general
solvability and uniqueness were given by 
Hasler, Neirynck, and others \cite{HaslerDApplMath,HaslerNeirynck,Fosseprez}
for nullor/resistor networks; and by Nishi and Chua 
\cite{NishiChuaCactus,NishiChuaCCCS} for networks with all kinds of 2-port
controlled sources, applied in \cite{NishiChuaTransFB} to 
reproduce Nielsen and Willson's result.  
Hasler \textit{et.\ als.\ } conclusion is that 
a ``pair of conjugate spanning trees'' and the absence of a
``non-trivial uniform partial orientation of the resistor [edges]''
are necessary and sufficient for existence and uniqueness of solutions
for all suitable functions and source values.  Nishi and Chua's structures
are ``cactus graph'' networks with negative determinants obtained by 
deletion/contraction operations particular to each kind of controlled source.

%The present 
Our paper shows 
%that 
oriented matroid (OM) theory \cite{BachemKern,OMBOOK}
covers
%generalizes 
Hasler \textit{et.\ als.}
``conjugate spanning tree'' and ``orientation'' concepts to 
provide a theory equivalent to Sandberg and Willson's theory of 
$\mathcal{W}_0$ 
%matrix 
pairs 
%that solves 
to solve 
the same problem.  
We also review the key oriented matroid concepts and 
demonstrate them on one feedback structure case of
\cite{TrajWillNDR}.
%and show how electric and elastic structures are modeled
%analogously.  
%(The matroids of the elastic structure model geometry 
% in addition to topology.)
We believe distinguished port elements,
pairings of oriented matroids with a common ground set,
and common covectors (explained below) are crucial.
%to this enterprise.  
Although oriented matroids can 
also be axiomatized with ``chirotopes''
which abstract determinant signs, 
(so suitable abstractions of graph orientation properties are
mathematically equivalent to principles behind determinant signs)
the oriented matroid pair
common covector approach has an intuitive
advantage for qualitative reasoning because
the common covector displays precisely 
the signs of all state quantities or their differences.

\extra{Preliminary connections between our approach and the 
issue of DC operating point stability \cite{Green} were published
\cite{sdcISCAS98}, but relating these determinant based results
to common covectors is still under investigation.}

The recognition of Minty's painting property\cite{VandewalleChua} 
and other facts
from OM
%oriented matroid theory being 
theory used by Hasler \textit{et. al.} 
\cite{HaslerNeirynck,Fosseprez,HaslerDApplMath} led us to 
generalize in \cite{sdcOMP}
their graph model notion of a ``pair of 
conjugate trees''
to a ``complementary pair of bases''
in a pair of matroids (which abstract the ``voltage and current graphs''
of \cite{Recski} and others); and of 
a ``non-trivial uniform partial orientation of the resistors''
to a ``common (non-zero) covector in an oriented matroid pair''.   
We showed that the graph model 
generalizes to a linear subspace pair model;
the pair of linear subspaces defines a pair of oriented matroids.
This OM pair is the discrete structure that
generalizes a graph with designated resistor, source, nullator
and norator edges.
Topological conditions
for the existence or uniqueness of solutions
are expressed in it. 
Two real matrices, which can be easily generated
from the system design, represent the OMs so that 
it is practical to work with them.  (Some OMs, indeed 
most, are not representable by real matrices and would require much more
space to store, but they do not occur in our application.)
Theory valid for all OM pairs, not just those represented by
a pair of linear subspaces, was presented in \cite{sdcOMP}.  For lack of
space, we omit determinant sign conditions shown equivalent
in \cite{sdcOMP} to the cases of common covector conditions that we cover.


\extra{

In the electrical circuit theory literature, the circuit ``topology''
means the network graph (to which Kirchhoff's laws apply)
together with particular kinds of ``device elements''
such as resistors, capacitors, voltage sources (batteries), 
current sources, etc., associated with single graph edges, and possibly
``multiport'' elements associated with multiple graph edges.
Problems with multiport elements are reduced to 
those with only single edge elements, in order to apply
the theory of  \cite{HaslerDApplMath},  through the use of 
``nullator'' and ``norator'' elements.  Detailed exposition of the problems,
reductions, theory and applications is given in \cite{HaslerNeirynck}.

}

In many cases, interesting conclusions can be reached
from the sign patterns of feasible subspace members by 
``calculations'' in an ``algebra of signs'', a 
form of qualitative reasoning that uses
easy, fundamental oriented matroid theoretic 
operations 
upon matrix sign patterns.  In cases when
the qualitative calculations show the outcome depends on numeric
values, the numeric information can then be used, say to 
calculate a new matrix
whose signs reveal better information;
or else, inequalities
on system parameters for each case of outcome can be derived.  
The pairing seems to be needed because
the non-linear monotonicity
constrains two quantities to only have a common
sign. The list of those signs for one state or state difference
is the common covector.  (Our structural/constitutive law
separation by OM pairing handles issues different from
``imprecise constitutive law constants'' of \cite{Murota} and others.)

\subsection{Single Oriented Matroids}

See \cite{OMBOOK} for full details about OMs;
%oriented matroids;
\cite{BachemKern} is a good introduction to our point of view.
Other ways to apply matroids to 
electrical and other systems are given in \cite{Recski,Murota,RigidityBook}
and EE literature on symbolic simulation.  A full survey is omitted.

We think of the oriented matroid $\mathcal{M}(M)$ 
\textit{represented} by matrix $M$ 
as its finite set of \textit{covectors} $\mathcal{L}(M)$, where
each covector 
is the tuple of the \textit{signs} $\{+,-,0\}$
or \textit{signature} $X=\sigma(l)$ of the real coordinates 
of a member $l$ of the \textit{linear subspace} $L(M)$ in $\Reals^U$
spanned by the rows of $M$. 
%(We use italics to denote a 
(Italics denote a 
\textit{term or symbol being defined}.)  Hence 
$\mathcal{L}(M)=\mathcal{L}(L(M))$ has at most 
$3^{|U|}$ covectors.  
For example,
when $M$ is the signed incidence matrix of a network graph, each covector
represents a combination of branch voltage drop 
signs feasible under Kirchhoff's
voltage law; the finite \textit{ground set} $U$ labels the branches.
One can call $X$ a \textit{signed set}, in
which elements of subset $X^+$ occur with $+$ sign and those in $X^-$ have
$-$ sign. 
The \textit{support} $\supp{X}$ is the subset
of $e\in U$ for which $X_e \neq 0$, i.e., $\supp{X}=X^+\cup X^-$.
Bachem and Kern's book \cite{BachemKern}
motivates oriented matroids from linear subspaces this way.
%For the sake of brevity,
For brevity's sake,
we define oriented matroids 
using some ``sign algebra'' operation properties 
we will then use.

Given two sign tuples $X^1$, $X^2$, their \textit{composition}
$Z=X^1 \circ X^2$ has $\supp{Z}$ $=$ 
$\supp{X^1}\cup \supp{X^2}$
and 
for $e\in\supp{Z}$,
$X_e=X^i_e$ where $i$ is the smallest index for which $X^i_e\neq 0$.
Note that if $X^i=\sigma(l^i)$ for $l^i\in\Reals^U$, then $X^1\circ X^2$
$=$ $\sigma(l^1 + \epsilon l^2)$ for some sufficiently small $\epsilon >0$.
Hence $\mathcal{L}(M)$ is closed under 
%the 
$\circ$.
% operation.

\begin{definition}
\label{OMDEF}
The collection $\mathcal{L}(\mathcal{M})$ 
of signed sets with ground set $U$ is the set of 
\textit{covectors} of an oriented matroid 
$\mathcal{M}$ if it satisfies:\\
\textbf{(L0)} $0\in\mathcal{L}$. \textbf{(L1-2)} If $X,Y\in
      \mathcal{L}$ then $-X$ and $X\circ Y\in\mathcal{L}$.\\
\textbf{(L3)} For all $X$, $Y \in \mathcal{L}$ and $e\in X^+\cap Y^-$
there is
$Z\in\mathcal{L}$ such that 
$Z^+\subset(X^+\cup Y^+)\sminus\{e\}$,
$Z^-\subset(X^-\cup Y^-)\sminus\{e\}$,\\
and 
$(\supp{X}\sminus \supp{Y})\cup(\supp{Y}\sminus \supp{X})\cup$\\
\hspace*{0.5in}$(X^+\cup Y^+)\cup(X^-\cup Y^-)\subset\supp{Z}$.
\end{definition}
%Note that property 
Property (L3) says $Z_e=0$ and it predicts $Z_g$ for all
$g\neq e$ except those with $X_gY_g = -$; i.e., $g$
having opposite signs in
$X$ and $Y$.  The logical
equivalence of this definition to various apparently weaker
axiomatizations is due to work of Edmonds, Fukada and 
Mandel cited and surveyed in \cite{OMBOOK}.

Other oriented matroid 
notions such as orthogonality and 
independence can be expressed by properties of covector sets
that are motivated by linear algebra.
The covectors $\mathcal{L}(L^\perp)$ of the 
\textit{orthogonal complement}
$L^\perp$
of linear subspace $L\subset\Reals^U$ form another oriented matroid.
We say $X\perp Y$ for signed sets $X,Y$ when 
either $\supp{X}\cap\supp{Y}=\emptyset$ or
there are $e,f\in U$
with $X_fY_f = -X_eY_e \neq 0$.
This abstracts a necessary condition for 
two real vectors to be orthogonal
under the usual dot product.
In fact, for every 
%oriented matroid $\mathcal{M}$ with covectors
covector set
$\mathcal{L}(\mathcal{M})$,
the set $\mathcal{V}=\mathcal{L}^\perp$ defined by
$\{Y | Y\perp X \mbox{\ for all\ }X\in\mathcal{L}\}$ satisfies the
covector axioms (\cite{OMBOOK}, Prop. 3.7.12); 
$\mathcal{V}(\mathcal{M})$ is called the 
set of \textit{vectors} of $\mathcal{M}$ and is the set of covectors
of the \textit{dual} or \textit{orthogonal} oriented matroid 
$\mathcal{M}^\perp$. The OM vectors display 
%code 
all combinations of coefficient 
sign that occur among all linear dependencies of the columns of $M$,
when $\mathcal{M}=\mathcal{M}(M)$.
\extra{More directly, an independent set $I\subset U$ is characterized by: 
for all $3^{|I|}$ ``input'' assignments $i$ of $e\in I$ to $\{+,-,0\}$, 
there exists a covector $X\in \mathcal{L}(\mathcal{M})$ for which
$X_e = i_e$ for all $e\in I$.}  Abstractly, an \textit{independent set}
$I\subset U$ satisfies $\supp{V}\not\subset I$ for all non-zero 
vectors $V\in\mathcal{V}(\mathit{M})$.

KVL, KCL and analogous mechanical structural or geometric laws 
are each formulated by a constraint of the form 
$v\in L$ $=$ $\mbox{row space}(M)$
where $L\subset\Reals^U$.  To
reformulate this law by a system of linear equations,
a maximal subset $B\subset U$ corresponding to
a linearly independent set of columns of $M$ is found.  Such a $B$ is
a maximal independent set,
called a \textit{basis in the matroid} $\mathcal{M}(M)$.
The collection of all bases in $\mathcal{M}$ is denoted by 
$\mathcal{B}(\mathcal{M})$.
Row operations and possibly deletion of zero rows can transform
$M$ to $\hmat{I}{M^{\overline{B}}}$ (after column permutation)
where $I$ is the $r\times r$ identity matrix, where $r$ $=$ 
$\mbox{rank}(M)$ $=$ $\mbox{dim}(L)$ $=$ $\mbox{rank}(L)$
$=$ $\mbox{rank}(\mathcal{M}(L))$.    
It is now clear that $v\in L$ is characterized by 
$v_{\overline{B}}$ $=$ $v_{B}M^{\overline{B}}$.
For each independently chosen
$v_{B}\in\Reals^B,$ $v=(v_B;v_{\overline{B}})\in L$ is unique
with its $B$ coordinates equal to $v_B$.

The \textit{cocircuits} (resp. \textit{circuits} $\mathcal{C}$)
of an oriented matroid are the non-zero covectors (resp. vectors)
whose support is minimal.  
\extra{Minty's painting property,
most popularly known as a theorem about 
directed graphs\cite{VandewalleChua}, is generally true about
the cocircuit $\mathcal{C}^*(\mathcal{M})$ and circuit 
$\mathcal{C}(\mathcal{M})$ collections.}
Note $\mathcal{C}$ is the
cocircuits of the orthogonal oriented matroid $\mathcal{M}^\perp$.
\extra{In fact, when the 
simple non-triviality, symmetry, and minimal support properties are assumed, 
the painting property characterizes when 
$\mathcal{C}^*$ and $\mathcal{C}$ are the cocircuit/circuit collections of 
an oriented matroid.}

\extra{
\begin{theorem}
(\cite{OMBOOK}, Th. 3.4.4(4P); \cite{BachemKern}, Prop. 5.12) 
For every partition $U=R\cup G \cup B \cup W$ and for every $e\in R\cup G$,
\textbf{either} %\\
%\begin{itemize}
%\item[(a)] 
(a) There exists $X\in\mathcal{C}^*$ so $e\in\mbox{Supp}{X}$,
$X_R\geq 0$, $X_G\leq 0$, $X_B$ unrestricted and $X_W=0$ %\\
%\item[]
\textbf{or}\\
%\item[(b)]
(b) There exists $Y\in\mathcal{C}$ so $e\in\mbox{Supp}{Y}$,
$Y_R\geq 0$, $Y_G\leq 0$, $Y_B = 0$ and $Y_W$ unrestricted %\\
%\item[]
\textbf{but not both.}
%\end{itemize}
\end{theorem}
}

\section{THE SUBSPACE PAIR MODEL AND APPLICATIONS}

\begin{comment}
A \textit{subspace pair} $(L_V, L_I)$ is a pair of linear subspaces of
$\Reals^U$, where the elements of finite set $U$ index the coordinates.
The scalar product $v\cdot w = \sum_{e\in U}v_e w_e$ is used to define that
$v, w \in \Reals^U$ are \textit{orthogonal} when $v\cdot w = 0$.  An 
\textit{orthogonal subspace pair}  satisfies $v\cdot w = 0$ for
all $v\in L_V$ and $w\in L_I$.  A subspace pair has \textit{full rank} when
$\mathrm{rank}(L_V)+\mathrm{rank}(L_I)= |U|$.  Hence an orthogonal full rank 
subspace pair is a linear subspace paired with its orthogonal complement.
\end{comment}

Some applications begin with basic structural laws
modeled by an orthogonal complementary pair of 
linear subspaces $(L_V, L_I)$, i.e., $L_I=L_V^\perp$.  This pair 
is then modified by port insertion, deletion, contraction, and 
nullator/norator insertion operations which can 
destroy the original orthogonality and/or 
$\mbox{dim\ }(L_V)+\mbox{dim\ }(L_I)=|U|$ properties.
Other applications can begin with one or both of $L_V,L_I$ 
expressing some constitutive laws too (see our example).

Electrical network structure is defined 
%beginning 
with the \textit{network graph}
$\mathcal{N}$ with \textit{nodes} $N$  and \textit{arcs} $U$.  
%(The generality obtainable by port, nullator/norator or nullor, and 
%device characteristic insertions will be treated later.)
%Each arc has 
%a fixed but arbitrary direction to define the sign of its voltage drop and
%current flow.  
The \textit{incidence matrix} $M_V$ has rows indexed by $N$,
columns indexed by $U$, and $M_V(n,e)=+1$ when the tail of $e$ is
$n$, $-1$ if the head of $e$ is $n$, and $0$ if $n$ and $e$ are not incident.
When $L_V=L(M_V)$ and $L_I=L_V^\perp$,
each $u\in L_V$ is a combination of voltage drops in $\mathcal{N}$ 
feasible under KVL and each $w\in L_I$ is 
an arc current flow feasible under KCL.  These facts 
restate Kirchhoff's laws and Tellegen's theorem.  

\extra{ 
We can determine
$(L_V, L_I)$ from one of these subspaces given and Tellegen's theorem: The
role of nodes here is not strictly necessary.
Kirchhoff's voltage law can be expressed by the statement:  
The feasible voltage drops are the image of the map 
$\Reals^N\rightarrow\Reals^U$ given by $\phi\rightarrow\phi M_V$.  
Kirchhoff's current law says the feasible current flows are the kernel
of the map $\Reals^U\rightarrow\Reals^N$ given by $u\rightarrow u M_I^t$.
Tellegen's theorem is the observation that $M_V$ and $M_I^t$ are adjoints.
See \cite{WyattTele}.}

\extra{
A practical and familiar way to generate $M_V$ and $M_I$ matrices whose rows 
provide some covectors to use for qualitative analysis is to 
chose a spanning set of (directed) cuts and cycles respectively, say
the fundamental cutsets and cycles of a spanning tree.  A single spanning
tree however
for both $M_V$ and $M_I$ is not necessary.  There is no advantage, at 
least for hand calculations, for $M_V$ and $M_I$ to have the minimum
number of rows.  Particular cutsets and cycles, or linear combinations of
them, can be devised and used to provide covectors that represent
approximations of particular ``modes'' of system operation or change 
from one operating point to another.}


The definition of mechanical elastic structure begins with the 
(undirected) \textit{framework graph} $\mathcal{F}$ with \textit{vertices}
$N$ and \textit{edges} $U$.  \textit{Framework} $\mathcal{F}(\mathbf{p})$ 
%in d $dimentions$ 
is 
%a framework graph 
$\mathcal{F}$ and an \textit{embedding} 
$\mathbf{p}:N\rightarrow \Reals^d$.  
%The embedding assigns each vertex to a point in
%$d$-dimensional space.  
The \textit{rigidity matrix} $M_V$ 
has $d|N|$ rows in groups of $d$ corresponding to the vertices.
%each indexed by one coordinate in $\Reals^d$ 
%of the point that embeds one vertex.
For $e=(i,j)\in U$, column $M_V(e)$ 
%of the rigidity matrix 
is defined \cite{RigidityBook}\\
%(when vertices are numbered 0 through $|N|-1$):\\
$(0, \ldots,0, \mathbf{p}(n_i)-\mathbf{p}(n_j),0,\ldots,
\mathbf{p}(n_j)-\mathbf{p}(n_i), 0, \ldots,0)^T$\\
where $\mathbf{p}(n_i)-\mathbf{p}(n_j)$ occupies 
%positions $di\ldots di+d-1$ and 
$n_i$'s group of positions
and $\mathbf{p}(n_j)-\mathbf{p}(n_i)$
occupies $n_j$'s.
%positions $dj\ldots dj+d-1$.
\extra{This definition is echoed from the 
literature \cite{RigidityBook} on rigidity theory, except
we interchange rows and columns.  Just as we deemphasized nodes of electrical
networks, we will use merely the row space of $M_V$ for most of what follows.
}
\extra{The rigidity matrix as a 
function of the embedding $\mathbf{p}$ is denoted
$M_V(\mathbf{p})$.  The row vector $\mathbf{p}$ left multiplied with
$M_V(\mathbf{p})$
is the row tuple denoted $\mathbf{L}=\mathbf{p}M_V(\mathbf{p})$  Then, 
$\mathbf{L}_e$ $=$ 
$(\mathbf{p}M_V(\mathbf{p}))_e$ $=$ $|\mathbf{p}_i-\mathbf{p}_j|^2$ for
each edge $e=(i,j)$.  Now if each $\mathbf{p}_i$ is a differentiable function
of $t$, $d\mathbf{L}/dt$ $=$ $2\mathbf{p}'M_V(\mathbf{p})$.  Framework
$\mathcal{F}(\mathbf{p})$ is \textit{first-order rigid} when 
$d\mathbf{L}/dt$ $=$ $0$ for all $\mathbf{p}'$ implies 
$|\mathbf{p}_i-\mathbf{p}_j|^2$ is constant for all pairs $i$, $j$, not just
endpoints of edges. 
}%extra
%Let $\mathbf{p}(i)-\mathbf{p}(j)$ for edge $e=\{i,j\}$ be called the
%\textit{vector from } $j$ \textit{to} $i$.  
From \cite{RigidityBook},
$u\in L_V=L(M_V)\subset\Reals^U$ 
iff 
for some combination of \textit{vertex velocities}
$\mathbf{v}:N\rightarrow\Reals^d$,
$u_e=(\mathbf{v}(i)-\mathbf{v}(j))\cdot(\mathbf{p}(i)-\mathbf{p}(j))$
for each $e=(i,j)\in U$.
Also,
the \textit{self-stress subspace} $L_I=L_V^\perp$%
%, the orthogonal complement of $L_V$, 
is shown to be all $w\in\Reals^U$
for which the framework is in static
equilibrium when each edge $e=(i,j)$ exerts force 
$w_e(\mathbf{p}(j)-\mathbf{p}(i))$ on vertex $i$.  By this convention, 
$w_e>0$ means $e$ is under tension and $w_e<0$ means $e$ is under 
compression.  
%(\cite{RigidityBook} actually defines $M_V^T$, not $M_V$.)
(The rigidity matrix in \cite{RigidityBook} is $M_V^T$, not $M_V$.)

Under this analogy, 
(1) KVL corresponds to geometric consistency of first order
bar length changes under changes in the embedding, (2) KCL corresponds to
Newton's laws of static equilibrium, and (3) Tellegen's theorem corresponds
to a virtual work principle, that static equilibrium is
characterized by 
the internal forces against every virtual embedding change 
doing zero virtual work.

\extra{\subsection{Elastic Analog of the Indefinite Nodal Admittance Matrix}

Reduced nodal admittance matrix.  Nodal resistance matrix.
Interaction with a physical framework with it's environment.
A framework is first order rigid iff it ``resolves all applications of
static equilibrium forces''.  However, every physical bar has some
elasticity:  An ideal rigid bar is analogous to an ideal voltage 
source.  Hence, given an elastic framework, for every application
of static equilibrium forces on the vertices, the vertex positions
will change as the bars stretch or shrink under the forces they now
carry to resolve the applied force.  These first order vertex position 
changes are given by $Z\mathbf{f}$.

The environment might interact by ``forcing'' some vertices to change position
relative to one another.
Intuitively, the framework will ``push back''.  The other vertices are free
to move as adjacent vertices move and incident bars change length in
response to the forces developed in them to resolve the forces required
to hold the framework in its new position.  The position changes of the
free vertices $V$ can be calculated by solving for the unknown position changes
in the system of equations $(Y\mathbf{v}_V)(V)=0$.

For our purposes, we insert port elements in order to make interactions 
with the environment explicit.  This enables a coordinate of an 
environmental interaction quantity to correspond to an oriented matroid
element, so that its sign can be read off from the corresponding entry
in a covector.

It's yet to be done to handle simutaneous application of force to more 
than 2 vertices....
}

%\subsection{Ports, Devices and Solutions}

Given $(L_V, L_I)$ and $p\in U$ not already a port, 
the 
operation of \textit{inserting a port at  $p$} 
defines a new subspace pair $(L'_V, L'_I)$
with $U'=U\sminus \{p\}\cup\{p_V,p_I\}$, $L'_V=L_V\oplus\Reals$ (direct 
sum) with $p$ replaced by $p_V$;
and the coordinate of the added $\mathbf{R}$ 
indexed by $p_I$, together with
$L'_I=L_I\oplus\Reals$ with $p$ now replaced by $p_I$
and the added subspace indexed by $p_V$.
\extra{Note that (going to $(L'_V, L'_I)$) the ranks of
$L_V$ and $L_I$ each increase by 1, and $|U'|$ $=$ $|U|+1$.}
\extra{After $p$ port
insertions, we denote the final $U$ by $E\cup P_V \cup P_I$ 
with pairwise disjoint
$E$, $P_V$ and $P_I$, $|P_V|$ $=$ $|P_I|$ $=$ $p$, $P_V\cup P_I$ being the 
replacement elements.}

\extra{Questions of existence and 
uniqueness of solution for various combinations of 
kinds of sources are formulated after 
modeling devices.  
Each port element provides separate output and input
variables
for an 
electrical current or 
voltage \textit{kind of} source, or its mechanical analog.}
\extra{Unlike device variables, the two variables of each port
are not directly related by a constitutive law which is 
part of the system model.}
%Ports are introduced so the response of an electrical network to current 
%and/or voltage sources, and the mechanical analogs, can be formulated.

\extra{ Ports also facilitate formal operations to compose larger systems
from smaller ones.  We believe ports are important for investigations of
rigidity because they model how a framework interacts with its environment,
for example, what a mechanical model ``feels like'' when you squeeze it.
We have also found that electrical port characteristics of unit resistance
ported electrical networks are ratios of coefficients in certain 
partial evaluations 
of a generalization of the Tutte polynomial\cite{sdcPorted}.}

%Hence we do not assume any rank or orthogonality conditions on subspace 
%pairs in the definitions below.

\pagebreak
The \textit{subspace pair model} $\mathbf{M}$ $=$ 
$(E, \Gamma, P, (L_V, L_I))$ consists of finite set $E$ of 
\textit{device elements}, \textit{constitutive law relations}
$\Gamma = \{\Gamma_e\subset\Reals\times\Reals | e \in E\}$, the finite set
$P=P_V \cup P_I$ that results from inserting ports as defined above, 
and a subspace pair $(L_V, L_I)$ over $\Reals^U$ with $U=E\cup P$.

The \textit{variables} of $\mathbf{M}$ are 
$\{u_{\mathit{Ve}}, u_{\mathit{Ie}} | e \in E\}$  $\cup$ \\
$\{ u_{\mathit{pV}}, u_{\mathit{pI}} | p_I, p_V \in P \}$.  
(For brevity, subscript ``$\mathit{pV}$'' means port element
$p_V\in P_V$, etc.)
A \textit{subspace pair model with sources} $S$ 
is a subspace pair model $(E, \Gamma, P, (L_V, L_I), S)$
together with a subset $S$ of exactly $|P|$ of the $2|P|$ elements
in $P$.  A \textit{$V$-driven port} is 
%a port $p\in P$ 
one for which 
$p_V\in S$ and $p_I\not\in S$, then $u_{\mathit{pV}}$ is called 
an \textit{input variable}.   Reverse $V$ and $I$ to define an 
\textit{$I$-driven port} and its input variable.

A \textit{solution} of $\mathbf{M}$ with sources
is a real valued extension to \textit{all} variables of $\mathbf{M}$ 
of a given \textit{input} assignment to the input variables
that satisfies
$(u_{\mathit{PV}}, u_{\mathit{PI}}, u_V) \in L_V$,
$(u_{\mathit{PV}}, u_{\mathit{PI}}, u_I) \in L_I$
and
$(u_{\mathit{Ve}}, u_{\mathit{Ie}}) \in \Gamma_e$ for all
$e\in E$.
%  Note that in this model, 
The constraint 
$(u_{\mathit{PV}}, u_{\mathit{PI}}, u_V) \in L_V$ 
does not (by itself) imply any constraint
on a I-driven port variable $u_{\mathit{pI}}$, similarly, 
$u_{\mathit{pV}}$ is not constrained
by 
$(u_{\mathit{PV}}, u_{\mathit{PI}}, u_V) \in L_I$.
\extra{Port variables are not constrained
by the constitutive laws $\Gamma$ (by themselves) either.}
In matroid theory an element like $p_I$ of
$\mathcal{M}(L_V)$ that is independent of all others
is called an \textit{isthmus}.

\extra{
In the language of matroid theory, we can call the element $p_I$ an 
\textit{isthmus} of the matroid $\mathcal{M}(L_V)$; similarly, $p_V$ is
an isthmus of $\mathcal{M}(L_I)$.  In general, the matroid represented by
a matrix is characterized by the collection $\mathcal{I}$ 
of \textit{independent sets} 
of matrix columns, where a set of columns is called independent when it is
linearly independent.  (Matroid theory studies what can be deduced by 
the following three axioms satisfied by $\mathcal{I}$: (1) 
$\mathcal{I}\neq\emptyset$. 
(2) If $A\subset B\in\mathcal{I}$ then $A\in\mathcal{I}$.  (3) 
If $A$, $B$ $\in\mathcal{I}$ and $|A|<|B|$, then there exists $e\in B\sminus A$ 
for which $A\cup\{e\}\in\mathcal{I}$.  For example, an isthmus $e$ is 
characterized by $A\cup\{e\}\in\mathcal{I}$ for all $A\in\mathcal{I}$.
The \textit{rank} of a subset $C\in U$ is the size of the largest independent
subset of $C$.  
%A set $D$ is called \textit{coindependent} if $\rank(U\sminus D)$
%$=$ $\rank(U)$; in other words, removing $D$ does not diminish the original
%matroid's rank.)
}

\extra{We say a subspace pair problem with sources $S$ 
is \textit{well-posed} when for all input assignments there is a unique 
solution.}

Assume as usual 
%no branch is taken to be both a current and voltage source.
no port is both I-driven and V-driven.
The 
\extra{well-known necessary condition
for an electrical network to have a unique 
solution for all choices of source values is}
condition of no cycle of voltage source branches
in the ``voltage'' graph nor a
cutset of current source branches in the ``current graph'' (see, e.g.,
\cite{ChensBook})
generalizes to:


\begin{theorem}
\label{Feasiblity1}
\textbf{(1)} If all V input assignments are feasible under
the $L_V$ constraint then 
$\{p_V|p \mbox{\ is V-driven}\}$
is an independent set in the matroid $\mathcal{M}(L_V)$.
\textbf{(2)} If every solution is unique then\\
$\{p_I|p \mbox{\ is V-driven}\}$
must be coindependent in $\mathcal{M}(L_I)$.
\textbf{(3)} 
If all V input assignments are feasible under
the $L_I$ constraint then 
$\{p_I|p \mbox{\ is I-driven}\}$
is an independent set in the matroid $\mathcal{M}(L_I)$.
\textbf{(4)} If every solution is unique 
then
$\{p_V|p \mbox{\ is I-driven}\}$
must be coindependent in $\mathcal{M}(L_V)$.
\end{theorem}

\noindent Pf.\ (1,3):
If set $S$ of input variables is dependent, then there is
some combination of input values that is not feasible.
Pf.\ (2,4):
$S$ is not a \textit{coindependent} set iff $S$ contains a cocircuit, so
%A non-coindependent set $S$ must contain a cocircuit, so
there is a non-zero covector supported by $S$.  Hence there is a 
feasible variable assignment that is non-zero on the some port output
variables only.


Each port insertion increases $\rank(L_V)+\rank(L_I)-|U|$ by 1.
When the constitutive laws are linear, the solutions of $\textbf{M}$ are
found from the \textit{intersection} of two linear subspaces:  Let $G$ 
be the diagonal matrix with ``conductances'' $g_e$ in its positions
indexed by $e \in E$ (so $\Gamma_e$ $=$ $\{ (v, g_e v) | v \in \Reals \}$
and 1 in its other diagonal positions.  The solution set projected onto
the $u_I$ variables is $L_VG\cap L_I$.  
(Here, $L_V G$ means $L(M_V G)$.)


\begin{comment}
Familiar topological conditions on dependencies among source values
pertain to the \textit{matroids} of the subspaces $L_V$ and $L_I$.
Questions about existence and uniqueness of solution
will be answered in terms of supplementary subspace pair models
which are obtained by the familiar operations of opening and shorting ports.
Finally, operations on subspace pairs that model nullor insertion are
defined, so that such ideal elements can be modeled combinatorially or
geometrically.  

The supplemental subspace pair derived after nullor 
insertion will typically not be orthogonal.  One might also choose to
model linearized CCCSs or VCVSs within one of the subspaces.  
\end{comment}

\subsection{Example: a feedback structure case of 
Trajkovi\`{c} and Willson \cite{TrajWillNDR}}

%$M_V$ is written above; $M_I$ below.  
We used the rules of 
Definition \ref{OMDEF} to figure out common covectors in 
$\mathcal{L}(M_V)$, $\mathcal{L}(M_I)$ with 
their first two 
signs,
of port elements pV, pI given,
for the 
tangents and differences drawn boldly on the 
$I_{\mbox{p}}$/$V_{\mbox{p}}$ curve.  
(The algorithm ideas used appear in \cite{sdcOMP}.)
In three cases, (unique)
extensions of the given signs exist when $1-\Sigma\alpha\le 0$ but there is
only one such case otherwise, verifying a condition for NDR from 
\cite{TrajWillNDR}.
Multiple operating points with an I-driven port are impossible.  
The magnitude inequalities come from ``$\epsilon<<1$'' in explaining
``$\circ$'' above Definition \ref{OMDEF}.

\noindent 
Fig. 1: Example.  
$M_V$/$M_I$ are the upper/lower right arrays.\hspace*{-20pt}
\[
\begin{array}{c|ccccc}
\mbox{(row mults.)}u_{\mbox{pI}}
                           & 0 & 1 & 0 & 0 & 0 \\
u_{\mbox{pV}}              & 1 & 0 & 0 & 0 & +1 \\
u_{\mbox{VR}}              & 0 & 0 & 1 & 0 & +1 \\
u_{\mbox{VD1}}             & 0 & 0 & 0 & 1 &-1 \\ \hline
\Gamma\mbox{\ coeffs.}\rightarrow
                           & 1 & 1 & g_{\mbox{R}}
                                       & g_{\mbox{D1}}
                                           & g_{\mbox{D1}} \\ \hline
\mbox{elements}\rightarrow & \mbox{pV}
                               & \mbox{pI}
                                   & \mbox{R} 
                                       & \mbox{D1}
                                           & \mbox{D2} \\ \hline
u_{\mbox{pI}}=-I_{\mbox{p}}& 0 & 1 
                                   & 1-\alpha_1-\alpha_2
                                       & \alpha_1-1
                                           &\alpha_2-1  \\
u_{\mbox{pV}}              & 1 & 0 & 0 & 0 & 0
\end{array}
\]

\noindent
\input{NDR.pstex_t}\vspace*{0.5em}

\noindent
\input{NDRgraph.pstex_t}


\begin{comment}
We illustrate the oriented matroid approach by reproducing the result of
\cite{TrajWillNDR} that a particular configuration of a ``feedback structure''
with two Ebers-Moll transistors and one port
cannot exhibit negative differential resistance by itself, and it can 
exhibit NDR if one resistor is added and $\alpha_1+\alpha_2-1>0$.  
Under KVL,
the voltages across the port, resistor
and two 
Ebers-Moll diodes are given by the row space member of $M_V$ when the
three rows are multiplied by the 3 independent voltages
$V_1$, $V_2$ and the port voltage $V$.  The space of current values
feasible in the same 4 elements under KCL and the two 
Ebers-Moll linear CCCS's
is only one dimensional; it is spanned by the one
row of $M_I$.  Beginning with the signatures of the rows of the matrices,
we can apply the covector axioms to explore what common covectors are possible
under several variations.
\end{comment}


%\subsection{Deletion and contraction}

Given a subspace $L\subset\Reals^U$ and $e\in U$, the subspace 
``$L$ \textit{with} $e$ \textit{deleted}''
is
$L\sminus e$ $=$ 
$\{ l(U\sminus e) | l\in L\}$
$\subset\Reals^{U\sminus \{e\}}$, 
where $l(U\sminus e)$ denotes the $l\in\Reals^U$
with the component labeled by $e$ dropped.  
Thus, $L\sminus e$ is the \textit{projection} of $L$ into 
$\Reals^{U\sminus \{e\}}$.
If $L=L(M)$ 
then $L\sminus e$ $=$ $L(M(U\sminus \{e\}))$ is the row space of 
$M(U\sminus \{e\})$, which is $M$ with column $e$ deleted.
The subspace  ``$L$ \textit{with} $e$ \textit{contracted}'' 
$L/e$ $=$ 
$\{ l(U\sminus e) | l(U)\in L \mathrm{\ and\ } l(e)=0\}$.  
So, $L/e$ is the 
intersection of $L$ with
the (hyperplane) subspace of $\Reals^U$ with $l(e)=0$,
projected into 
$\Reals^{U\sminus \{e\}}$.

We now define \textit{deletion} and \textit{contraction} on subspace pairs:  
$(L_V, L_I)\sminus e$ $=$ $(L_V\sminus e, L_I/e)$ and 
$(L_V, L_I)/e$ $=$ $(L_V/e, L_I\sminus e)$.  
Deleting $e$ in the electrical application
corresponds to \textit{opening} the corresponding branch.
Dually, 
contraction corresponds to \textit{shorting}.  
Mechanically,
deletion of an edge corresponds to ``breaking'' the corresponding bar:
ignore any distance change between its ends and transmit no force.
Contraction corresponds to 
declaring the bar to be inelastic, which rules out 
all (first order) distance changes between the endpoints and 
rules out any constitutive law referring to 
tension or compression in that bar.

%\subsection{Nullators and Norators}
A \textit{nullator} element $e\in E$ 
expresses the ideal constitutive law $u_{Ve}=0$
and $u_{Ie}=0$ which approximates conditions at the input
to a high-gain amplifier when a system is stabilized by feedback.
Hence a nullator is declared by 
\textit{contracting $e$ in both
$L_V$ and $L_I$.}  Ordinarily, this reduces both their ranks by 1.
%
A \textit{norator} element $e\in E$ indicates that the constitutive
law puts no direct constraint on $u_{Ve}$ or $u_{Ie}$; the amplifiers
approximately adjust the output state so the feedback results in zero input.
Hence a norator is declared by 
\textit{deleting $e$ in both
$L_V$ and $L_I$.}  Ordinarily, their ranks don't change.
%
Theorem \ref{Feasiblity1} applies to the subspace pair
obtained from all declarations of nullators, norators, opens and shorts.

\extra{ There is a subtle difference between declaring a V-source with 0
input value and contracting the same element.  If a set $S$ 
of $k$ such elements
is not independent in $\mathcal{M}(L_V)$, then the rank of $\mathcal{M}(L_V/S)$
will be more than
$\mbox{rank}(\mathcal{M}(L_V))-k$ but the given combination of input values
will still be feasible.  If they are not coindependent in
$\mathcal{M}(L_I)$ (which will certainly be true when there are no nullors).
then the rank of $\mathcal{M}(L_I\sminus S)$ will be less than 
$\mbox{rank}(\mathcal{M}(L_I))$ but there will be a non-zero combination of
output only variables.  (Physically, that corresponds to non-zero current 
circulating in a loop of ideal wires; or a non-zero self-stress in an
overbraced subframework of rigid bars.  The dual dual physical situation
is that is possible for the disconnected parts of an electrical network
to differ in electrical potential when a cut-set of branches are removed;
mechanically, more flexes of the framework can exist when some edges are
removed. For this reason, we are careful to distinguish deletion/contraction
from port insertion.)


We can also declare ``display'' quantities, say for the difference of two
independent quantities, by appending the appropriate column to $M_V$ or
$M_I$.  Any dependency of the display quantity on element quantities will
appear as a dependency in the oriented matroid.
}

\extra{
\subsection{Topological Formulas}

They can come out of the subspace pair formulation three ways:
\begin{itemize}
\item  Besides the ``$g$'' or ``$r$'' variables, do not insert ports but
do use an extra variable ``$x$'' to relate the voltage to current of one port.
Then the equation $\det(M_VG;M_I)=0$ has the form $Ax+B=0$, so $x=-B/A$.
\item Use the Pl\"{u}cker coordinate formulation of the intersection subspace to 
identify a minor of a hybrid or other description matrix of as the determinant
of the solution submatrix for a system of equations; then use the Cramer's
rule generalization to find the ratios of minors of the equation 
matrix to analyzed.  
This was done for my ISCAS 98 paper.
\item Use Rota's Grassmann-Cayley algebra meet formula to extract expansion
directly from subspace pair matrices, together with the previous way to
identify Pl\"{u}cker coordinate ratios with description matrix minors.
\end{itemize}

}


The \textit{supplemental subspace pair}
% of a subspace pair model
%with sources 
is constructed by \textit{zeroing} all the sources.
For each V-driven (resp. I-driven) port $p$, $p_V$ 
(resp. $p_I$) is 
contracted in both $L_V$ and $L_I$, and $p_I$ 
(resp. $p_V$) is deleted in
both $L_V$ and $L_I$.
The resulting subspaces, etc.  are denoted $L_V^0$, $L_I^0$, etc.

\extra{
\begin{figure}[htb]

\begin{minipage}[c]{.48\linewidth}
  \centering
 \centerline{\input{2res.pstex_t}}
%  \vspace{2.0cm}
\end{minipage}
%
\hfill
\begin{minipage}[c]{.48\linewidth}
\[
\begin{array}{cccc}
\multicolumn{4}{c}{\mbox{($M_V$ matrix)}} \\
0   &  1  &  0  &  0  \\
1   &  0  &  1  &  1  \\ \hline \hline
p_V & p_I & e_1 & e_2 \\ \hline \hline
1   &  0  &  0  &  0  \\
0   &  1  & -1  &  0  \\
0   &  0  &  1  &  -1 \\
\multicolumn{4}{c}{\mbox{($M_I$ matrix)}} \\
\end{array}
\]
\end{minipage}
%
\begin{minipage}[b]{.48\linewidth}
\[
\begin{array}{ccc}
  1  &  0  &  0  \\ \hline
 p_I & e_1 & e_2 \\ \hline
  0  &  0  &  0  \\
  1  & -1  &  0  \\
  0  &  1  &  -1
\end{array}
\]
ZIR analysis for voltage source input:
$p_V$ contracted.  Zero response (unique solution) even if
negative resistances $\neq 0$ are allowed.
\end{minipage}
\hfill
\begin{minipage}[b]{.48\linewidth}
\[
\begin{array}{ccc}
0   &  0  &  0  \\
1   &  1  &  1  \\ \hline
p_V & e_1 & e_2 \\ \hline
1   &  0  &  0  \\
0   &  1  &  -1
\end{array}
\]
ZIR analysis for current source input:
$p_I$ deleted.
Zero response (unique solution) \textit{unless}
$g_1 = -g_2$.  
\end{minipage}
\caption{Simple example that illustrates the solution is unique when the 
port is V-driven provided each $g_e\neq 0$, but the I-driven system has a
unique solution provided each $g_e>0$ because the resulting supplemental
oriented matroid pair has complementary bases and no common covector.}
\label{Simple}
%
\end{figure}
}

\section{NO-COMMON-COVECTORS, $\mathcal{W}_0$ PAIRS}

%The following Theorem demonstrates, by means of 
Theorem \ref{SubspacePairTh} uses
Sandberg and Willson's 
%theory of 
$\mathcal{W}_0$ pairs
to show every 
subspace pair model 
\extra{(with  its separation of
geometric/topological and constitutive constraints)}
can be analyzed
for unique solvability from the oriented matroid pair it generates.
%By Thm.\ \ref{sandwillompairtheorem} 
Theorem \ref{sandwillompairtheorem} shows how
$(A,B)\in\mathcal{W}_0$ is characterized by a rank condition and a 
no-common-covector property.  
Theorem \ref{PairConclusionTheorem} is our unifying conclusion.

\begin{theorem}
\label{SubspacePairTh}
The subspace pair model has a unique solution for all 
%source 
input assignment 
values and positive monotone constitutive laws iff 
\textbf{(1)}, \textbf{(2)} and \textbf{(3)} are satisfied:\\
\textbf{(1).} There are bases $B_V\in\mathcal{B}(L_V)$,
$B_I\in\mathcal{B}(L_I)$ for which all V-driven 
ports $p$ satisfy $p_V\in B_V$ (note $p_I$ must be 
in $B_V$ since it's an isthmus in $\mathcal{M}_V$) 
and $p_I\not\in B_I$,
and all I-driven ports $p$ satisfy $p_I\in B_I$ and 
$p_V \not\in B_V$.  \\
\textbf{(2).}
$B_V\cup B_I = U$, for the bases in \textbf{(1)}.\\
\textbf{(3).} The oriented matroid pair of supplemental pair $(L_V^0,L_I^0)$ 
have no common (nonzero) covector.
(i.e., $\mathcal{L}(L_V^0)\cap\mathcal{L}(L_I^0)=\{0\}$.)
\end{theorem}

Proof sketch, if:  When $(L_V^0,L_I^0)$ is constructed given
\textbf{(1-2)}, we find $\mbox{rank\ }(L_V^0)+\mbox{rank\ }(L_I^0)\geq|E|$. By
Theorem 1.3 of \cite{sdcOMP}
and $|E|\geq$ rank of the matroid union 
(\cite{Welsh}, 8.3) of $\mathcal{M}_V^0\vee\mathcal{M}_I^0$,
we must have equality else \textbf{(3)} would be 
contradicted.  
%We then verify from (2) the square matrices
%$A$ $=$ $\left(\begin{array}{c}M_V^0\\ 0\end{array}\right)$ and
%$B$ $=$ $\left(\begin{array}{c}0\\ -M_I^0\end{array}\right)$
%satisfy the rank part of condition (2) of Theorem
Form square matrices
$A$ $=$ $\left(\begin{array}{c}M_V^0\\ 0\end{array}\right)$,
$B$ $=$ $\left(\begin{array}{c}0\\ -M_I^0\end{array}\right)$ and 
verify from \textbf{(2)} they satisfy the rank part of cond. \textbf{(2)} of
Th. \ref{sandwillompairtheorem}. 
${\mathcal L}\hmat{A}{B} \cap {\mathcal L}\hmat{I}{-I}=\{0\}.$
follows directly from hypothesis \textbf{(3)}.  We can then
formulate the monotonic 
subspace pair problem as a case of 
Theorem \ref{sandwillompairtheorem} part \textbf{(5)} to complete the proof.

Only if:  Use Theorem \ref{Feasiblity1} to verify \textbf{(1)} and 
\textbf{(2)}.  
If \textbf{(3)} were violated,
positive linear constitutive laws could be constructed
(as in \cite{HaslerNeirynck,SWExistancePf}) for which 
a non-zero solution for zero input exists.

\begin{theorem}
\label{sandwillompairtheorem}
(From \cite{sdcOMP}) For a pair 
%
% EDITdone
% delete ``order'' --- we understand that when we see 
% ``$n\times n$'', and this avoids confusing with order as in
% a linear ordering
%
of $n\times n$ 
matrices $(A,B)$, the following conditions are 
equivalent.\\
%
% EDITdone
% period, not colon
%
\textbf{(1)}
$(A,B)\in {\mathcal W}_0$ in the sense of 
Sandberg and 
Willson~\cite{SWExistancePf};
e.g., $|AD+B|\neq 0$ for all positive diagonal $D$, etc.\\
\textbf{(2)}
$\rank{\mathcal M}\hmat{A}{B}=n$ and 
${\mathcal L}\hmat{A}{B} \cap {\mathcal L}\hmat{I}{-I}=\{0\}.$\\
\textbf{(3)}
$\rank{\mathcal M}\hmat{A}{B}=n$ and 
${\mathcal V}\hmat{A}{B} \cap {\mathcal V}\hmat{I}{-I}=\{0\}.$\\
\textbf{(4)}
\textrm{(}Fundamental theorem of Sandberg and 
Willson~\cite{SWExistancePf,W0APPLpaper}\textrm{)}\ 
%
% EDITdone
% use ~ before \cite so a space is left between the word and the reference
%
For all functions $F:{\bf R}^n\rightarrow {\bf R}^n$ of the form
$F(x)_k=f_k(x_k)$ where each $f_k$ is a strictly 
monotone increasing 
function from 
${\bf R}$ {\em onto} $\bf R$ and for all $c\in {\bf R}^n$, the 
equation $AF(x) + Bx = c$\\
has a unique solution $x$. \cite{W0APPLpaper}.\\
\textbf{(5)}
For all functions $G:{\bf R}^n\rightarrow {\bf R}^n$ of the form
$G(w)_k=g_k(w_k)$ where each $g_k$ is a strictly monotone increasing 
function from ${\bf R}$ {\em onto} $\bf R$ and for all 
$d^{\prime}, d^{\prime\prime}\in {\bf R}^n$, the 
eqns.\ $u^t=z^tA+d^{\prime},\;\; w^t=z^tB+d^{\prime\prime},\;\; u=-G(w)$
have a unique solution.% $(u,w,z)$.
\end{theorem}

%Note:  
A direct inductive proof generalizing 
\cite{HaslerNeirynck,Fosseprez}'s has the advantage
of revealing circuit theoretic concepts that occur.
One step proves that if the no-common-covector
property is true for $(M_V,M_I)$, then it is true for the matrix pair 
from the system obtained by replacing one of the non-linear elements by a
source.

\begin{theorem}
\label{PairConclusionTheorem}
With constitutive laws given by monotone increasing functions from 
$\Reals$ onto
$\Reals$, every subspace pair problem can be posed as a case of Theorem
\ref{sandwillompairtheorem} (see Theorem \ref{SubspacePairTh}), 
and every case of Theorem
\ref{sandwillompairtheorem} can be posed as a subspace pair problem
(with $\hmat{A}{B}=M_V^0$, $\hmat{I}{-I}=M_I^0$.)
\end{theorem}


% References should be produced using the bibtex program from suitable
% BiBTeX files (here: strings, refs, manuals). The IEEEbib.bst bibliography
% style file from IEEE produces unsorted bibliography list.
% -------------------------------------------------------------------------
%\bibliographystyle{IEEEbib}
%%\bibliography{strings,refs,manuals}
%\bibliography{OMPEMech}

%\begin{comment} %comment out our hand hacked bibliography


\begin{thebibliography}{10}
\setlength{\itemsep}{-2pt}

\bibitem{SWExistancePf}
I.~W. Sandberg and A.~N. {Willson, Jr.},
\newblock ``Some theorems on properties of dc equations of non-linear
  networks,''
\newblock {\em Bell Syst. Tech. J.}, vol. 48, pp. 1--34, 1969.

\bibitem{FiedlerPtak}
M.~Fiedler and V.~Ptak,
\newblock ``Some generalizations of positive definiteness and monotonicity,''
\newblock {\em Numer. Math.}, vol. 9, pp. 163--172, 1966.

\bibitem{NielWillpaper}
R.~O. Nielsen and A.~N. {Willson, Jr.},
\newblock ``A fundamental result concerning the topology of transistor circuits
  with multiple operating equilibria,''
\newblock {\em Proc. IEEE}, vol. 68, pp. 196--208, 1980.

\bibitem{HaslerDApplMath}
M.~Hasler, C.~Marthy, A.~Oberlin, and D.~de~Werra,
\newblock ``A discrete model for studying existence and uniqueness of solutions
  in nonlinear resistive circuits,''
\newblock {\em Disc. Appl. Math.}, vol. 50, pp. 169--184, 1994.

\bibitem{HaslerNeirynck}
M.~Hasler and J.~Neirynck,
\newblock {\em Nonlinear Circuits},
\newblock Artech House, Norwood, Mass., 1986.

\bibitem{Fosseprez}
M.~Foss\'{e}prez,
\newblock {\em Non-linear Circuits, Qualitative Analysis of Non-linear,
  Non-reciprocal Circuits},
\newblock John Wiley, 1992.

\bibitem{NishiChuaCactus}
T.~Nishi and L.~O. Chua,
\newblock ``Topological criteria for nonlinear resistive circuits with
  controlled sources to have a unique solution,''
\newblock {\em IEEE Trans. Circ. Syst.}, vol. CAS--31, pp. 722--741, 1984.

\bibitem{NishiChuaCCCS}
T.~Nishi and L.~O. Chua,
\newblock ``Uniqueness of solution for nonlinear resistive circuits containing
  {CCCS}'s or {VCVS}'s whose controlling coefficients are finite,''
\newblock {\em IEEE Trans. Circ. Syst.}, vol. CAS--33, pp. 381--397, 1986.

\bibitem{NishiChuaTransFB}
T.~Nishi and L.~O. Chua,
\newblock ``Topological proof of the {N}ielsen-{W}illson theorem,''
\newblock {\em IEEE Trans. Circ. Syst.}, vol. CAS--33, pp. 398--405, 1986.

\bibitem{BachemKern}
A.~Bachem and W.~Kern,
\newblock {\em Linear Programming Duality, An Introduction to Oriented
  Matroids},
\newblock Springer-Verlag, 1992.

\bibitem{OMBOOK}
A.~Bj\"{o}rner, M.~Las Vergnas, B.~Sturmfels, N.~White, and G.~Ziegler,
\newblock {\em Oriented Matroids}, vol.~46 of {\em Encyc. Math. and its Appl.},
\newblock Cambridge Univ. Press, 2nd edition, 1999.

\bibitem{TrajWillNDR}
L.~Trajkovi\`{c} and A.~N. {Willson, Jr.},
\newblock ``Complementary two-transistor circuits and negative differential
  resistance,''
\newblock {\em IEEE Trans. Circ. Syst.}, vol. CAS--37, pp. 1258--1266, 1990.

\bibitem{VandewalleChua}
J.~Vandewalle and L.~O. Chua,
\newblock ``The colored branch theorem and its applications in circuit
  theory,''
\newblock {\em IEEE Trans. Circuits Syst.}, vol. CAS--27, no. 9, 1980.

\bibitem{sdcOMP}
S.~Chaiken,
\newblock ``Oriented matroid pairs, theory and an electric application,''
\newblock in {\em Matroid Theory, AMS-IMS-SIAM Joint Summer Research
  Conference}, J.~E. Bonin, J.~G. Oxley, and B.~Servatius, Eds. American
  Mathematical Society, 1996, vol. 197 of {\em Contemporary Mathematics}, pp.
  313--331.

\bibitem{Recski}
A.~Recski,
\newblock {\em Matroid Theory and its Applications in Electric Network Theory
  and in Statics},
\newblock Springer-Verlag, 1989.

\bibitem{Murota}
K.~Murota,
\newblock {\em Matrices and matroids for systems analysis},
\newblock Springer, 2000.

\bibitem{RigidityBook}
J.~Graver, B.~Servatius, and H.~Servatius,
\newblock {\em Combinatorial Rigidity}, vol.~2 of {\em Graduate Studies in
  Mathematics},
\newblock American Mathematics Society, 1993.

\bibitem{ChensBook}
W.~K. Chen,
\newblock {\em Applied Graph Theory, Graphs and Electrical Networks},
\newblock North-Holland, 2 edition, 1976.

\bibitem{Welsh}
D.~J.~A. Welsh,
\newblock {\em Matroid Theory},
\newblock Addison-Wesley, 1976.

\bibitem{W0APPLpaper}
A.~N. {Willson, Jr.},
\newblock ``New theorems on the equations of nonlinear dc transistor
  networks,''
\newblock {\em Bell Syst. Tech. J.}, vol. 49, pp. 1713--1738, 1970.

\end{thebibliography}





%\end{comment}


\end{document}
