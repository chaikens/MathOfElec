\documentclass{amsproc-sunycstr}
\usepackage{epsfig,verbatim}
% verbatim needed for comment environment




% My custom definitions.
% --------------------

\def\color[rgb]#1{}
%To enable Tex without the \color macro to handle pstex_t output
%from newer xfigs.

\def\x{{\mathbf x}}
\def\L{{\cal L}}
\def\Reals{\ensuremath{\mathbf R}}

\theoremstyle{plain}
\newtheorem{theorem}{Theorem}
\newtheorem{proposition}{Proposition}
\newtheorem{lemma}{Lemma}[section]

\theoremstyle{definition}
\newtheorem{definition}{Definition}

\theoremstyle{remark}
\newtheorem{remark}{Remark}
\newtheorem{fuzz}{Heuristic Comment (``Fuzz'')}


\newcommand{\qedclaim}{{$\vartriangleleft$}}

\newcommand{\Pnew}{{P_{\mbox{new}}}}
\newcommand{\Mnew}{{\mathbf{M}_{\mbox{new}}}}

\newcommand{\supp}[1]{{{\mbox{supp\ }#1}}}

%    Absolute value notation
\newcommand{\abs}[1]{\lvert#1\rvert}


%    Rank
\DeclareMathOperator{\rank}{rank}

%    Matroid Union 
\newcommand{\munion}{\lor}

%   Set minus
%\newcommand{\sminus}{\backslash}


%   Disjoint Union
%\newcommand{\dunion}{\uplus}
\newcommand{\dunion}
{\mbox{\hbox{\hskip4pt$\cdot$\hskip-4.62pt$\cup$\hskip2pt}}}
%
% Dot inside a cup.
% If there is a better, more Latex like way 
% (more invariant under font size changes) way,
% I'd like to know.

%   Two matrices horizontally concatenated, space between can be
%   adjusted here
%
\newcommand{\hmat}[2]{[#1\;#2]}

%   Oriented Matroid Pair
\newcommand{\OMP}[2]{#1,\,#2}

%\newcommand{\extra}[1]{{\small{#1}}}
\newcommand{\extra}[1]{{{#1}}}
%\newcommand{\extra}[1]{}

% Title.
% ------
\title[Oriented Matroid Pair Model]
{\extra{REPORT}
\extra{(\today
:
\the\time)}
THE ORIENTED MATROID PAIR MODEL FOR MONOTONE
DC ELECTRICAL AND ELASTIC NETWORK UNIQUE SOLVABILITY
}
%
% Single address.
% ---------------
%\name
\author{Seth Chaiken}
\thanks{Part of this reseach was done during a Sabbatical
from the University at Albany in 2001.
\extra{This report represents on-going research.
%, details 
%and background information stemming from a paper submitted to ISCAS-2002
}%\extra
}%\thanks
\address{University at Albany,
	Department of Computer Science\\
	Albany, NY 12222 %(\texttt{sdc@cs.albany.edu})
}
\email{sdc@cs.albany.edu}
%
\begin{document}
%\renewcommand{\baselinestretch}{0.90}
%\ninept
%
%\maketitle
%

\begin{abstract}
Resistive electrical networks and elastic mechanical systems such
as trusses have a topological or geometric structure together with constitutive
laws for the elements prior to their interconnection.
Oriented matroids provide a common discrete mathematical model 
for such 
structure
%structures
in which relationships on the signs of element 
quantities can be expressed.
Pairing of oriented matroids
enables non-linear monotone constitutive laws to be fit
into the abstraction in a way that allows port and nullor
insertions and provides discrete unique solvability conditions.

The resulting mathematical model clarifies some mechanical analogies for 
these 
circuit theory concepts, 
relates apparently dissimilar published theories
for existence and uniqueness and shows how to 
handle elastic mechanical systems with small displacements.
It also enables constraints on the signs of system quantities
to be predicted from the structure when this is possible.
\extra{Finally, it 
derives topological solution formulas for linearized mechanical 
systems in which the analog of a tree-sum is a sum over minimally rigid
trusses.}
\end{abstract}
%
\maketitle





% Abstract preceeds maketitle in ams documentclasses

\begin{center}\verb|$Id: OMPEMech.tex,v 3.7 2002-03-02 03:20:08-05 sdc Exp $|\end{center}

\section{Introduction}
\label{sec:intro}


%Our work \cite{sdcOMP} shows that the model and results of 
%\cite{HaslerDApplMath,HaslerNeirynck} generalize from a graph model to a
%linear subspace pair model
%for which the discrete structure 
%%in which the topological conditions are expressed 
%that expresses 
%%the 
%topological conditions
%for existance or uniqueness of solutions
%is pair of oriented
%matroids, rather than a graph with designated resistor, source, nullator
%and norator edges.  

Our topic in non-linear systems is DC equations, 
say for operating points/resistive circuits, whose only
non-linearities are 
monotone increasing bijections $\Reals\rightarrow\Reals$.  
Special case conditions
for existence and uniqueness of solutions for all 
%choices of 
such non-linear
functions and additive 
%(source value)
constants were given by Duffin, Minty,
and Rockafellar.  
The more general determinant based theory
of $\mathcal{W}_0$ matrix pairs due to Sandberg and Willson 
\cite{SWExistancePf} extended Fiedler and Pt\'{a}k's work \cite{FiedlerPtak}.
Nielsen and Willson \cite{NielWillpaper}
%applied it \cite{NielWillpaper} to demonstrate
%proved with it  
used it to prove 
that disallowing the 2 transistor feedback structure in a transistor circuit
is sufficient for uniqueness.
Graph based theories that identified structures forbidden for general
solvability and uniqueness were given by 
Hasler, Neirynck, and others \cite{HaslerDApplMath,HaslerNeirynck,Fosseprez}
for nullor/resistor networks; and by Nishi and Chua 
\cite{NishiChuaCactus,NishiChuaCCCS} for networks with all kinds of 2-port
controlled sources, applied in \cite{NishiChuaTransFB} to 
reproduce Nielsen and Willson's result.  
Hasler \textit{et.\ als.\ } conclusion is that 
a ``pair of conjugate spanning trees'' and the absence of a
``non-trivial uniform partial orientation of the resistor [edges]''
are necessary and sufficient for existence and uniqueness of solutions
for all suitable functions and source values.  Nishi and Chua's structures
are ``cactus graph'' networks with negative determinants obtained by 
deletion/contraction operations particular to each kind of controlled source.

%The present 
Our paper shows 
%that 
oriented matroid (OM) theory \cite{BachemKern,OMBOOK}
covers
%generalizes 
Hasler \textit{et.\ als.}
``conjugate spanning tree'' and ``orientation'' concepts to 
provide a theory equivalent to Sandberg and Willson's theory of 
$\mathcal{W}_0$ 
%matrix 
pairs 
%that solves 
to solve 
the same problem.  
We also review the key oriented matroid concepts and 
demonstrate them on one feedback structure case of
\cite{TrajWillNDR}.
%and show how electric and elastic structures are modeled
%analogously.  
%(The matroids of the elastic structure model geometry 
% in addition to topology.)
We believe distinguished port elements,
pairings of oriented matroids with a common ground set,
and common covectors (explained below) are crucial.
%to this enterprise.  
Although oriented matroids can 
also be axiomatized with ``chirotopes''
which abstract determinant signs, 
(so suitable abstractions of graph orientation properties are
mathematically equivalent to principles behind determinant signs)
the oriented matroid pair
common covector approach has an intuitive
advantage for qualitative reasoning because
the common covector displays precisely 
the signs of all state quantities or their differences.

\extra{Preliminary connections between our approach and the 
issue of DC operating point stability \cite{GreenWillNDR} were published
\cite{sdcISCAS98}, but relating these determinant based results
to common covectors is still under investigation.}

The recognition of Minty's painting property\cite{VandewalleChua} 
and other facts
from OM
%oriented matroid theory being 
theory used by Hasler \textit{et. al.} 
\cite{HaslerNeirynck,Fosseprez,HaslerDApplMath} led us to 
generalize in \cite{sdcOMP}
their graph model notion of a ``pair of 
conjugate trees''
to a ``complementary pair of bases''
in a pair of matroids (which abstract the ``voltage and current graphs''
of \cite{ChensBook} and others); and of 
a ``non-trivial uniform partial orientation of the resistors''
to a ``common (non-zero) covector in an oriented matroid pair''.   
We showed that the graph model 
generalizes to a linear subspace pair model;
the pair of linear subspaces defines a pair of oriented matroids.
This OM pair is the discrete structure that
generalizes a graph with designated resistor, source, nullator
and norator edges.
Topological conditions
for the existence or uniqueness of solutions
are expressed in it. 
Two real matrices, which can be easily generated
from the system design, represent the OMs so that 
it is practical to work with them.  (Some OMs, indeed 
most, are not representable by real matrices and would require much more
space to store, but they do not occur in our application.)
Theory valid for all OM pairs, not just those represented by
a pair of linear subspaces, was presented in \cite{sdcOMP}.  For lack of
space, we omit determinant sign conditions shown equivalent
in \cite{sdcOMP} to the cases of common covector conditions that we cover.


\extra{

In the electrical circuit theory literature, the circuit ``topology''
means the network graph (to which Kirchhoff's laws apply)
together with particular kinds of ``device elements''
such as resistors, capacitors, voltage sources (batteries), 
current sources, etc., associated with single graph edges, and possibly
``multiport'' elements associated with multiple graph edges.
Each graph edge is associated with one voltage and one current variable.
Kirchhoff's laws and the graph determine homogeneous
linear equations on these variables.

The electrical network model consists of the ``topology'' plus particular
numeric or symbolic constitutive laws and or parameters the define
the characteristics of each device.  Idealizations of certain devices
such as operational amplifiers enable the modeller to use more a general
``topology'' in lieu of these devices.  
In particular, nullators and norators are two-terminal elements that 
obey Kirchhoff's laws but have constitutive relations 
$(v,i)=(0,0)$ and $(v,i)\in\Reals^2$ (unrestricted) respectively.
Such idealization requires 
that the actual system is stabilized by feedback as a dynamical system.
Problems with multiport elements are reduced to 
those with only single edge elements, in order to apply
the theory of  \cite{HaslerDApplMath},  through the use of 
these nullator and norator elements.  Detailed exposition of the problems,
reductions, theory and applications is given in \cite{HaslerNeirynck}.

}

\subsection{Topology and Geometry}
The term ``topology'' 
of an electrical network
applies to the network graph as a 
1-dimensional finite simplicial complex (together with edge labels to
distinguish kinds of edges.)  
This graph determines the particular Kirchhoff 
law constraints on the voltage and current variables
of the network model.
The relevant information is coded by the graph's
oriented matroid.
Matroid theory lets us study
certain combinatorial properties of graphs in geometric 
terms\footnote{Recall that combinatorial geometry has been
used as alternative term for simple matroid.}.  Figure \ref{graphgeo}
illustrates shows how the circuits in a graph,
which are the minimal dependencies among voltage drops under KVL,
are the affine dependencies among points in the 2-dimensional plane.
These dependencies comprise the same matroid.
A theme in our work
is to extend analytical theories of graph based (electrical)
networks to applications where a geometric object plays the role of 
the graph.  In rigidity theory, this geometric object is the embedded
framework (graph with nodes embedded in $\Reals^d$).  The relevent
oriented matroids do depend on the embedding as well as the graph.
\begin{figure}
\label{graphgeo}
\input{graphgeo.pstex_t}
\caption{$\{e_1,e_2,e_3\},\{e_3,e_4,e_5\},\{e_1,e_2,e_4,e_5\}$ are supports of
cycles and are minimal dependencies in the graphic matroid and in its
equivalent affine dependency matroid of the 2-dimensional 
point arrangement.}
\end{figure}

Graphic oriented matroids belong to the special class of \textit{regular}
(oriented) matroids, also called totally unimodular matroids.  
These are the matroids that
be represented with matrices all of whose minors are $+1$, $-1$ or 
$0$.   The rigidity oriented matroids (from frameworks) are generally
not regular.  

\begin{fuzz}
The table below indicates some of the prior results:\\
{\newcommand{\mypage}[1]{\begin{minipage}{3in}#1\end{minipage}}
\begin{tabular}{cc}
Minty\cite{} & 
\mypage{Dual pair of regular matroids, abstracting a graph.} \\ \hline
Rockafellar\cite{RockafellarConvProg} &
\mypage{Orthogonal complementary subspaces} \\ \hline
Hasler, Neirynck\cite{HaslerNeirynck,Fosseprez} & 
\mypage{Pair of graphic and cographic matroids, from 
different graphs.} \\ \hline
First-order elastic frameworks & 
\mypage{A case of a dual pair of realizable oriented matroids.}
\\ \hline
Sandberg, Willson\cite{SWExistancePf,SWExistanceSIAM} & 
\mypage{We show their theory is for a 
pair of realizable oriented matroids with the 
complementary base/no-common-covector property.}
\end{tabular}
}
\end{fuzz}

\subsection{Qualitative Reasoning}
In many cases, interesting conclusions can be reached
from the sign patterns of feasible subspace members by 
``calculations'' in an ``algebra of signs'', a 
form of qualitative reasoning that uses
easy, fundamental oriented matroid theoretic 
operations 
upon matrix sign patterns.  In cases when
the qualitative calculations show the outcome depends on numeric
values, the numeric information can then be used, say to 
calculate a new matrix
whose signs reveal better information;
or else, inequalities
on system parameters for each case of outcome can be derived.  
The pairing seems to be needed because
the non-linear monotonicity
constrains two quantities to only have a common
sign. The list of those signs for one state or state difference
is the common covector.  (Our structural/constitutive law
separation by OM pairing handles issues different from
``imprecise constitutive law constants'' of \cite{Murota} and others.)

\subsubsection{Introductory examples}

Here are two examples of inferences from oriented matroid theory.  First, consider
the \textit{Wheatstone bridge} network below.  
\begin{center}
\input{bridge.pstex_t}
\end{center}
Among the covectors in
$\mathcal{L}_I$ are $X_{e_1}X_{e_2}X_{e_3}X_{e_4}X_{e_5}X_p$ $=$
$+0+0+-$ and $0+-+0-$.  These are indeed cocircuits; they signify the
signature of the current flows in the elementary cycles $\{e_1,e_3,e_5,p\}$ 
$\{e_2,e_3,e_4,p\}$.  It's an immediate consequence of Definition \ref{OMDEF}
(\textbf{L3}) (see \ref{OMDEFsec}) that 
$++0++- \in \mathcal{L}_I$.

When the resistance values are positive or, more generally, the resistors
have nonlinear characteristics satisfying $v_ei_e \geq 0$ and 
$v_ei_e = 0$ if and only if $v_e=0$ and $i_e=0$, covectors 
$X\in\mathcal{L}_V$ and $Y\in\mathcal{L}_I$ indicate signatures of 
feasible electrical solutions only if $X_{e_i}=Y_{e_i}$, $i=1,\ldots,5$.
The duality of $\mathcal{M}_V$ and $\mathcal{M}_I$ then implies
$X_pY_p=-$, so neither can be $0$.  



Our second example is the 2-dimensional framework below.
\begin{center}
\input{bridge2d.pstex_t}
\end{center}
The only non-zero covectors in $\mathcal{L}_I$ now is $\pm(+--+-+)$.
This describes the signature of the only non-zero self-stress.  
The signatures of first-order bar length changes comprise the
covectors of the oriented matroid dual $\mathcal{L}_V$.  Oriented
matroid orthogonality now rules out length change combinations 
of single bars only (the framework is rigid), certain changes in
adjacent pairs of bars (such as $e_1p=\pm(+-)$) or triples (such
as $e_1e_2p=+-+$).


\extra{
\subsubsection{Monotone source dependency}
Hasler, Wang and Chauffoureaux 
\cite{ChauffHaslerMonoDep,HaslerWangMonoDep} gave a condition that distinguished
when a current or voltage in one resistor of a nonlinear resistor/nullator
network depends monotonically on one source value and predicts the 
direction of the dependence.  Their condition generalizes to 
the relative sign of the corresponding 
two elements in all common covectors.  They also show that when the direction
of dependence is unpredicatable then there exist non-linear monontonic
increasing constitutive law functions for which the corresponding dependency is 
not monotonic.
}

\extra{
Despite the intuitive appeal of common covectors, it is not resolved
whether search for a common covector or an algorithm based on
determinant signs is more efficient either in practice or theoretically.
In \cite{sdcOMP} is is shown how the existence of a common covector
in the cases of supplemental subspace pairs with balanced rank and 
free union is as hard as the problem of telling if a digraph has
an even directed circuit.  The latter problem was recently shown to 
have a polynomial time solution in \cite{EvenCircSol:RobSeyTho}}

\subsection{Single Oriented Matroids}

See \cite{OMBOOK} for full details about OMs;
%oriented matroids;
\cite{BachemKern} is a good introduction to our point of view.
Other ways to apply matroids to 
electrical and other systems are given in \cite{Recski,Murota,RigidityBook}
and EE literature on symbolic simulation\cite{SymSim}.  A full survey is omitted.

We think of the oriented matroid $\mathcal{M}(M)$ 
\textit{represented} by matrix $M$ 
as its finite set of \textit{covectors} $\mathcal{L}(M)$, where
each covector 
is the tuple of the \textit{signs} $\{+,-,0\}$
or \textit{signature} $X=\sigma(l)$ of the real coordinates 
of a member $l$ of the \textit{linear subspace} $L(M)$ in $\Reals^U$
spanned by the rows of $M$. 
%(We use italics to denote a 
(Italics denote a 
\textit{term or symbol being defined}.)  Hence 
$\mathcal{L}(M)=\mathcal{L}(L(M))$ has at most 
$3^{|U|}$ covectors.  
For example,
when $M$ is the signed incidence matrix of a network graph, each covector
represents a combination of branch voltage drop 
signs feasible under Kirchhoff's
voltage law; the finite \textit{ground set} $U$ labels the branches.
One can call $X$ a \textit{signed set}, 
(formally speaking, a function $X:U\rightarrow \{+,-,0\}$ denoted by
$e\rightarrow X_e$) in
which elements of subset $X^+$ occur with $+$ sign and those in $X^-$ have
$-$ sign. 
The \textit{support} $\supp{X}$ is the subset
of $e\in U$ for which $X_e \neq 0$, i.e., $\supp{X}=X^+\cup X^-$.
Bland \cite{BlandLP} 
defined the oriented matroid represented by a linear subspace
this way in order to abstract the combinatorics of linear 
programming.  
%Bachem and Kern's book \cite{BachemKern}
%is a good exposition of this point of view.
%For the sake of brevity,
For brevity's sake,
we define oriented matroids 
using some ``sign algebra'' operation properties 
we will then use.

Given two sign tuples $X^1$, $X^2$, their \textit{composition}
$Z=X^1 \circ X^2$ has $\supp{Z}$ $=$ 
$\supp{X^1}\cup \supp{X^2}$
and 
for $e\in\supp{Z}$,
$X_e=X^i_e$ where $i$ is the smallest index for which $X^i_e\neq 0$.
Note that if $X^i=\sigma(l^i)$ for $l^i\in\Reals^U$, then $X^1\circ X^2$
$=$ $\sigma(l^1 + \epsilon l^2)$ for some sufficiently small $\epsilon >0$.
Hence $\mathcal{L}(M)$ is closed under 
%the 
$\circ$.
% operation.

\label{OMDEFsec}
\begin{definition}
\label{OMDEF}
The collection $\mathcal{L}(\mathcal{M})$ 
of signed sets with ground set $U$ is the set of 
\textit{covectors} of an oriented matroid 
$\mathcal{M}$ if it satisfies:\\
\textbf{(L0)} $0\in\mathcal{L}$. \textbf{(L1-2)} If $X,Y\in
      \mathcal{L}$ then $-X$ and $X\circ Y\in\mathcal{L}$.\\
\textbf{(L3)} For all $X$, $Y \in \mathcal{L}$ and $e\in X^+\cap Y^-$
there is
$Z\in\mathcal{L}$ such that 
$Z^+\subset(X^+\cup Y^+)\setminus\{e\}$,
$Z^-\subset(X^-\cup Y^-)\setminus\{e\}$,    %\\
and 
$(\supp{X}\setminus \supp{Y})\cup(\supp{Y}\setminus \supp{X})\cup$ %\\
%\hspace*{0.5in}
$(X^+\cup Y^+)\cup(X^-\cup Y^-)\subset\supp{Z}$.
\end{definition}
%Note that property 
Property (L3) says $Z_e=0$ and it predicts $Z_g$ for all
$g\neq e$ except those with $X_gY_g = -$; i.e., $g$
having opposite signs in
$X$ and $Y$.  The logical
equivalence of this definition to various apparently weaker
axiomatizations is due to work of Edmonds, Fukada and 
Mandel cited and surveyed in \cite{OMBOOK}.

Other oriented matroid 
notions such as duality (orthogonality) and 
independence can be expressed by properties of covector sets
that are motivated by linear algebra.
The covectors $\mathcal{L}(L^\perp)$ of the 
\textit{orthogonal complement}
$L^\perp$
of linear subspace $L\subset\Reals^U$ form another oriented matroid.
We say $X\perp Y$ for signed sets $X,Y$ when 
either $\supp{X}\cap\supp{Y}=\emptyset$ or
there are $e,f\in U$
with $X_fY_f = -X_eY_e \neq 0$.
This abstracts a necessary condition for 
two real vectors to be orthogonal
under the usual dot product.
In fact, for every 
%oriented matroid $\mathcal{M}$ with covectors
covector set
$\mathcal{L}(\mathcal{M})$,
the set $\mathcal{V}=\mathcal{L}^\perp$ defined by
$\{Y | Y\perp X \mbox{\ for all\ }X\in\mathcal{L}\}$ satisfies the
covector axioms (\cite{OMBOOK}, Prop. 3.7.12); 
$\mathcal{V}(\mathcal{M})$ is called the 
set of \textit{vectors} of $\mathcal{M}$ and is the set of covectors
of the \textit{dual}  oriented matroid 
$\mathcal{M}^\perp$. The OM vectors display 
%code 
all combinations of coefficient 
sign that occur among all linear dependencies of the columns of $M$,
when $\mathcal{M}=\mathcal{M}(M)$.
\extra{More directly, an independent set $I\subset U$ is characterized by: 
for all $3^{|I|}$ ``input'' assignments $i:I\rightarrow\{+,-,0\}$, 
there exists a covector $X\in \mathcal{L}(\mathcal{M})$ for which
$X_e = i_e$ for all $e\in I$.}  Abstractly, an \textit{independent set}
$I\subset U$ satisfies $\supp{V}\not\subset I$ for all non-zero 
vectors $V\in\mathcal{V}(\mathit{M})$.

We will use the terms \textit{orthogonal} for linear subspaces
(and signed sets for which $X\perp Y$)
but reserve \textit{dual} for dual oriented matroids although some authors
use orthogonal for the latter.  %The following theorem relates duality
%to orthogonallity.

%\begin{theorem} (\cite{BachemKern}, p. \textbf{???}) Let $\mathcal{S}$ be
%a collection of cocircuits or a collection of covectors of an oriented 
%matroid $\mathcal{M}$.  Then 
%$\mathcal{S}^{\perp}=\{X| X\perp S \mbox{\ for all\ } S\in\mathcal{S}\}$ 
%is the collection of covectors of the $\mathcal{M}^*$, the dual of 
%$\mathcal{M}$.  (Thus, $\mathcal{S}^{\perp}$ satisfies Definition
%\ref{OMDEF}.)
%\end{theorem}


KVL, KCL and analogous mechanical structural or geometric laws 
are each formulated by a constraint of the form 
$v\in L$ $=$ $\mbox{row space}(M)$
where $L\subset\Reals^U$.  To
reformulate this law by a system of linear equations,
a maximal subset $B\subset U$ corresponding to
a linearly independent set of columns of $M$ is found.  Such a $B$ is
a maximal independent set,
called a \textit{basis in the matroid} $\mathcal{M}(M)$.
The collection of all bases in $\mathcal{M}$ is denoted by 
$\mathcal{B}(\mathcal{M})$.
Row operations and possibly deletion of zero rows can transform
$M$ to $\hmat{I}{M^{\overline{B}}}$ (after column permutation)
where $I$ is the $r\times r$ identity matrix, where $r$ $=$ 
$\mbox{rank}(M)$ $=$ $\mbox{dim}(L)$ $=$ $\mbox{rank}(L)$
$=$ $\mbox{rank}(\mathcal{M}(L))$.    
%It is now clear that 
Thus $v\in L$ 
if and only if 
$v_{\overline{B}}$ $=$ $v_{B}M^{\overline{B}}$.
For each independently chosen
$v_{B}\in\Reals^B,$ $v=(v_B;v_{\overline{B}})\in L$ is unique
with its $B$ coordinates equal to $v_B$.

The \textit{cocircuits} (resp. \textit{circuits} $\mathcal{C}$)
of an oriented matroid are the non-zero covectors (resp. vectors)
whose support is minimal.  
\extra{Minty's painting property,
most popularly known as a theorem about 
directed graphs\cite{VandewalleChua}, is generally true about
the cocircuit $\mathcal{C}^*(\mathcal{M})$ and circuit 
$\mathcal{C}(\mathcal{M})$ collections.}
Note $\mathcal{C}$ is the
cocircuits of the dual oriented matroid $\mathcal{M}^\perp$.
\extra{In fact, when the 
simple non-triviality, symmetry, and minimal support properties are assumed, 
the painting property characterizes when 
$\mathcal{C}^*$ and $\mathcal{C}$ are the cocircuit/circuit collections of 
an oriented matroid.}

\extra{
\begin{theorem}
(\cite{OMBOOK}, Th. 3.4.4(4P); \cite{BachemKern}, Prop. 5.12) 
For every partition $U=R\cup G \cup B \cup W$ and for every $e\in R\cup G$,
\textbf{either} %\\
%\begin{itemize}
%\item[(a)] 
(a) There exists $X\in\mathcal{C}^*$ so $e\in\mbox{Supp}{X}$,
$X_R\geq 0$, $X_G\leq 0$, $X_B$ unrestricted and $X_W=0$ %\\
%\item[]
\textbf{or}\\
%\item[(b)]
(b) There exists $Y\in\mathcal{C}$ so $e\in\mbox{Supp}{Y}$,
$Y_R\geq 0$, $Y_G\leq 0$, $Y_B = 0$ and $Y_W$ unrestricted %\\
%\item[]
\textbf{but not both.}
%\end{itemize}
\end{theorem}
}

\subsection{Tableaux}

We will use the tableau notation (defined and used for oriented matroid
programs developed in Chapter 10 of 
\cite{OMBOOK}\footnote{Our notation 
differs slightly as we include the current basis elements in the column
set of the tableau.})
to express fundamental cocircuits associated with
a given oriented matroid basis $B$ so that, when the oriented matroid
$\mathcal{M}$ is realizable, the tableau encodes the sign pattern of 
the matrix $M=\hmat{I}{M^{\overline{B}}}$, with $B$ indexing the first
$|B|$ columns.  This $M$ is called the 
fundamental cocycle matrix associated with basis $B$.  $M$ is illustrated 
by the tableau:\\
\begin{center}
\input{onetableau.pstex_t}
\end{center}




\subsection{Pseudosphere Arrangements}

We first review the geometric interpretation of oriented matroid
$\mathcal{M}(M)$ as an arrangement of hyperplanes with distinguished
sides that is described in
section 1.2 of \cite{OMBOOK}.
Next, the topological representation theory of Folkman, Lawrence, Edmonds
and Mandel is described.  Finally, a heuristic explanation is given
for the significance of oriented matroid objects to systems defined and 
observed in terms of  states and functions of states.


A \textit{central hyperplane} $H_e$ in $\Reals^{r}$ is the 
solution subset $\{y\in\Reals^{r}|y m_e =0\}$ for the linear
constraint $y m_e = 0$ where $m_e$ is column $e$ of say 
the $r\times U$ matrix $M$.  The closed halfspace
$H^+_e=\{y\in\Reals^{r}|y m_e \geq 0\}$ is called the closed
\textit{positive side} of $H_e$.  
In this way, matrix $M$ 
defines
an \textit{arrangement of halfspaces}.
The oriented matroid 
$\mathcal{M}(M)$

\begin{quote}
encodes a lot of geometric information about the hyperplane arrangement in 
a very simple and explicit way.  For this, observe that every hyperplane
arrangement decomposes its ambient space into a collection of pieces
(that are in fact relatively open topological cells [homeomorphic to balls]
of various dimensions).  Each of these \textit{cells} is exactly determined
by the information whether for hyperplane $H_e$, the cell is on its positive
side, its negative side, or on the hyperplane itself.  This leads us to 
associate a sign vector with every cell.

These sign vectors are exactly the covectors of the oriented matroid
\end{quote}
$\mathcal{M}(M)$.  
This is simply another way at looking at how a realizable oriented matroid is 
defined by its covectors from the row space $L(M)\subset\Reals^U$ of 
matrix $M$.  As $y$ ranges over $\Reals^r$, the covector $\sigma(yM)$,
which is the tuple of $(\sigma(ym_e))_{e\in U}$, is a code for the cell
that contains $y$.  There are two advantages to this point of view:
(1) When we think of $y$ as an exact ``state,'' the covector
$\sigma(yM)$ indicates the combination of signs of coordinates and/or 
any other linear functions of $y$.  (2) The Topological Representation
Theorem of Folkman and Lawrence says that \textit{every} oriented matroid
(system defined by Definition \ref{OMDEF}) has a signed arrangement of 
suitable generalizations of hyperplanes whose cells 
are coded by its covector
set $\mathcal{L}$.  

Every signed arrangement of hyperplanes $(H_e|e\in U)$ is determined by its
intersections $S_e^\sigma =H_e^\sigma \cap S^{(r-1)}$ within the sphere 
$S^{(r-1)}=\{y\in\Reals^r| |y|=1\}$.  $S_e^0$ is a sphere of dimension $r-2$ 
in $S^{(r-1)}$ and $S_e^+$ is (topologically) a $(r-1)$ dimension ball in
$S^{(r-1)}$.  It is convenient to 
state the Folkman and Lawrence Theorem in terms of a 
a signed arrangement of generalized spheres (called pseudospheres) within
$S^{(r-1)}$.  The relevent definitions and statements are 
paraphrased from
\cite{OMBOOK} below.  

A subset $S$ of $S^{r-1}$ is called a \textit{pseudosphere} if $S=h(S^{r-2})$ 
for some homeomorphism $h:S^{(r-1)}\rightarrow S^{(r-1)}$ 
[continuous one-to-one correspondance with a continuous inverse].  A 
pseudosphere $S$ has two sides (hemispheres) $S^+$ and $S^-$, and is, of
course, topologically indistinguishable from a linear $(r-2)$-subsphere.
We define an \textit{arrangement of pseudospheres} 
$\mathcal{A}=(S_e)_{e\in U}$ to be a finite set of pseudospheres
$S_e$ in $S^{r-1}$ such that 

\begin{definition}(Arrangement of Pseudospheres)\\
\begin{itemize}
\item[A1] Every non-empty intersection $S_A$ $=$ $\cap_{e\in A}S_e$ is
homeomorphic to a sphere of some dimension, for $A\subseteq U$.
\item[A2] For every non-empty intersection $S_A$ and every $e\in U$ such
that $S_A\not\subseteq S_e$, the intersection $S_A\cap S_e$ is a pseudosphere
in $S_A$ with sides $S_A\cap S_e^+$ and $S_A\cap S_e^-$.
\item[A3] The intersection of a arbitrary collection of closed sides is 
either a sphere or a ball.  (This axiom is redundant.
Edmonds and Mandel \cite{} showed 
that A3 is implied by A1 and A2 but the proof is difficult.)
\end{itemize}
\end{definition}

The stronger version, proved by Edmonds and Mandel,
of the Topological Representation Theorem can now be stated:

\begin{theorem}
Let $\mathcal{L}\subseteq\{+,-,0\}^U$.  Then the following three conditions
are equivalent:
\begin{itemize}
\item[(i)] $\mathcal{L}$ is the set of covectors of a loop-free oriented
matroid of rank $r$.  (I.e., $\mathcal{L}$ satisfies Definition
\ref{OMDEF} and for no
$e$ is $X_e=0$ for all $X\in\mathcal{L}$.  For $\mathcal{L}=\mathcal{L}(M)$,
this means $M$ has no zero columns.)
\item[(ii)] $\mathcal{L}=\mathcal{L}(\mathcal{A})$ for some signed arrangement
$\mathcal{A}=(S_e)_{e\in U}$ of pseudospheres in 
$S^{r+k}$, such that $\mathrm{dim}(\cap_{e\in U}S_e) = k$.  (Note 
$\mathrm{dim}(\emptyset) = -1$, so the arrangement is in $S^{r-1}$ when
$\cap_{e\in U}S_e=\emptyset$.  Such an arrangement is 
called \textit{essential}.)
\item[(iii)]
$\mathcal{L}=\mathcal{L}(\mathcal{A})$ for some signed arrangement
$\mathcal{A}$ of pseudospheres in $S^{r-1}$, which is essential and centrally
symmetric (and whose induced cell complex $\Delta(\mathcal{A})$ is 
shellable.)
\end{itemize}
\end{theorem}

We can give a heuristic description of situations modeled by a pseudosphere
arrangement $\mathcal{A}$.  Each $e\in U$ corresponds to a scalar
function 
$m_e$ on a set of states.
The open pseudohemisphere models the 
states $y$ for which $m_e(y)>0$.   $S_e$ models the set of states for
which $m_e(y)=0$.  Hence $S_A$ models the states that satisfy all the 
constraints $(m_e(y)=0)_{e\in A}$.  The set $A$ is not independent if 
there is an element $e\in A$ for which $m_e(y)=0$ is satisfied for
every $y$ that satisfies the constraints $m_f(y)=0$ for 
all $f\in A\setminus e$.
The axiom (A1) means that all combinations of equality constraints have
feasible sets homeomorphic to spheres.   Axiom (A2) means that 
for every 
feasible combination of equality constraints, the remaining constraints
together define a (lower dimensional) oriented matroid situation.

The rank of the oriented matroid is the minimum size of a set $I\subseteq E$
for which $m_e(y)=0$ for all $e\in I$ implies $m_f(y)=0$ for all $f\in E$.
When the hyperplane arrangement is central, that is, 
$\cap_{e\in E}H^0_e=0$, this implies $y=0$.  The essential 
pseudosphere arrangement is the generalization of this.
For this case, the rank corresponds heuristically to  the number of 
degrees of freedom; one of freedom corresponds to a state vector's length.
Hence the dimension of the ambient topological sphere 
is the rank of the oriented matroid minus 1\footnote{An 
\textit{affine}
oriented matroid is an oriented matroid with one distinguished element
$e_0$ that is neither a loop nor an isthmus.  We can take the homogeneous
coordinates with $h_{e_0}=1$ to be the state space.  The number of
degrees of freedom equals the oriented matroid rank $r$ minus 1.  Then,
the state space is homeomorphic to a dimension $r-1$ ball and to 
dimension $r-1$ hemisphere $H_e^+$.}.



We can give the following heuristic description of oriented matroid 
\textit{circuits} and \textit{cocircuit}.  This description is precise
when the constraints are actually central linear hyperplanes.
\begin{itemize}
\item A circuit is the signature of a minimal dependency among constraints.
More specifically, suppose $X$ is a circuit.  The support $S=\supp{X}$ 
is the minimal $S\subseteq U$ for which the following is true:  Let $e\in S$.
For all states $y$, if for all $f\in S\setminus e$, $\sigma(m_f(y))=0$
then $\sigma(m_e(y))=0$.  

Furthermore, if for all $f\in S\setminus e$, 
$\sigma(m_f(y))=X_f$ or $\sigma(m_f(y))=0$ with at least 
one $f\in S\setminus e$ for which $\sigma(m_f(y))=X_f$, 
then $\sigma(m_e(y))=-X_e$.

\item A cocircuit $X$ is a signature of a feasible one-dimensional behavior.
It is a setwise minimal non-zero intersection of equality constraints
$m_f(y)=0$.  When these constraints $m_f(y)=0$ are satisfied for all
$f\in U\setminus\supp{X}$, the signs of $m_e(y)$ are all predicted by
$\sigma(m_e(y))=X_e$ for all $e\in\supp{X}$.  In general,
a covector is the signature of some feasible behavior.
\end{itemize}








\section{Pairing of Subspaces and Oriented Matroids}

%\begin{comment}
A \textit{subspace pair} $(L_V, L_I)$ is a pair of linear subspaces of
$\Reals^U$, where the elements of finite set $U$ index the coordinates.
The scalar product $v\cdot w = \sum_{e\in U}v_e w_e$ is used to define that
$v, w \in \Reals^U$ are \textit{orthogonal} when $v\cdot w = 0$.  An 
\textit{orthogonal subspace pair}  satisfies $v\cdot w = 0$ for
all $v\in L_V$ and $w\in L_I$.  A subspace pair has \textit{full rank} when
$\mathrm{rank}(L_V)+\mathrm{rank}(L_I)= |U|$.  Hence an orthogonal full rank 
subspace pair is a linear subspace $L$ paired with its orthogonal complement
$L^\perp$.

%\end{comment}

It is convenient to illustrate subspace and oriented matroid pairs
with a cocycle matrix $M_V$ or tableau for 
$\mathcal{M}(M_V)$ $=$ $\mathcal{M}_V$ printed above
a  cocycle matrix or tableau for
$\mathcal{M}(M_I)$ $=$ $\mathcal{M}_I$.  Thus an orthogonal pair of 
subspaces is illustrated by
\begin{center}
\input{dualmatrices.pstex_t}
\end{center}
The columns $B$, $\overline{B}$ form an identity submatrices in $M_V$,
$M_I$ respectively.



%Now that the basic terminology of subspace pairs is established,
%we give some matroid theoretic abstractions.

An oriented matroid pair is a pair of oriented matroids with common
ground set.  Each subspace pair as defined above naturally 
represents the oriented matroid pair $\mathcal{M}(L_V,L_I)$
$=$ $(\mathcal{M}(L_V),\mathcal{M}(L_I))$
$=$ $(\mathcal{M}_V,\mathcal{M}_I)$.  The ground set is denoted
$U$ $=$ $U(\mathcal{M}_V,\mathcal{M}_I)$ $=$
$U(\mathcal{M}_V)$ $=$ $U(\mathcal{M}_I)$.

When $\mathcal{M}_V,\mathcal{M}_I$ are duals as oriented matroids we
say $(\mathcal{M}_V,\mathcal{M}_I)$ is a dual oriented matroid pair.
It is almost immediate from the definition of oriented matroid dual that
a dual oriented matroid pair has no common covector and has at least one
complementary pair of bases.  A key theme of this subject is
how the latter property generalizes duality.  

\textbf{???}
The \textit{rank excess} of pair $(\mathcal{M}_V,\mathcal{M}_I)$ is 
$\rank(\mathcal{M}_V)+\rank(\mathcal{M}_I)-|U|$.  

\subsection{Subspace Pair Examples}

Some applications begin with basic structural laws
modeled by an orthogonal complementary pair of 
linear subspaces $(L_V, L_I)$, i.e., $L_I=L_V^\perp$.  This pair 
is then modified by port insertion, deletion, contraction, and 
nullator/norator insertion operations which can 
destroy the original orthogonality and/or 
$\mbox{dim\ }(L_V)+\mbox{dim\ }(L_I)=|U|$ properties.
Other applications can begin with one or both of $L_V,L_I$ 
expressing some constitutive laws too (see our example).

Electrical network structure is defined 
%beginning 
with the \textit{network graph}
$\mathcal{N}$ with \textit{nodes} $N$  and \textit{arcs} $U$.  
%(The generality obtainable by port, nullator/norator or nullor, and 
%device characteristic insertions will be treated later.)
%Each arc has 
%a fixed but arbitrary direction to define the sign of its voltage drop and
%current flow.  
The \textit{incidence matrix} $M_V$ has rows indexed by $N$,
columns indexed by $U$, and $M_V(n,e)=+1$ when the tail of $e$ is
$n$, $-1$ if the head of $e$ is $n$, and $0$ if $n$ and $e$ are not incident.
When $L_V=L(M_V)$ and $L_I=L_V^\perp$,
each $u\in L_V$ is a combination of voltage drops in $\mathcal{N}$ 
feasible under KVL and each $w\in L_I$ is 
an arc current flow feasible under KCL.  These facts 
restate Kirchhoff's laws and Tellegen's theorem.  

\extra{ 
We can determine
$(L_V, L_I)$ from one of these subspaces given and Tellegen's theorem: The
role of nodes here is not strictly necessary.
Kirchhoff's voltage law can be expressed by the statement:  
The feasible voltage drops are the image of the map 
$\Reals^N\rightarrow\Reals^U$ given by $y\rightarrow y M_V$.  
Kirchhoff's current law says the feasible current flows are the kernel
of the map $\Reals^U\rightarrow\Reals^N$ given by $u\rightarrow u M_I^t$.
Tellegen's theorem is the observation that $M_V$ and $M_I^t$ are adjoints.
See \cite{WyattTele}.}

\extra{
A practical and familiar way to generate $M_V$ and $M_I$ matrices whose rows 
provide some covectors to use for qualitative analysis is to 
chose a spanning set of (directed) cuts and cycles respectively, say
the fundamental cutsets and cycles of a spanning tree.  A single spanning
tree however
for both $M_V$ and $M_I$ is not necessary.  There is no advantage, at 
least for hand calculations, for $M_V$ and $M_I$ to have the minimum
number of rows.  Particular cutsets and cycles, or linear combinations of
them, can be devised and used to provide covectors that represent
approximations of particular ``modes'' of system operation or change 
from one operating point to another.}


The definition of mechanical elastic structure begins with the 
(undirected) \textit{framework graph} $\mathcal{F}$ with \textit{vertices}
$N$ and \textit{edges} $U$.  \textit{Framework} $\mathcal{F}(\mathbf{p})$ 
%in d $dimentions$ 
is 
%a framework graph 
$\mathcal{F}$ and an \textit{embedding} 
$\mathbf{p}:N\rightarrow \Reals^d$.  
%The embedding assigns each vertex to a point in
%$d$-dimensional space.  
The \textit{rigidity matrix} $M_V$ 
has $d|N|$ rows in groups of $d$ corresponding to the vertices.
%each indexed by one coordinate in $\Reals^d$ 
%of the point that embeds one vertex.
For $e=(i,j)\in U$, column $M_V(e)$ 
%of the rigidity matrix 
is defined \cite{RigidityBook}\\
%(when vertices are numbered 0 through $|N|-1$):\\
$(0, \ldots,0, \mathbf{p}(n_i)-\mathbf{p}(n_j),0,\ldots,
\mathbf{p}(n_j)-\mathbf{p}(n_i), 0, \ldots,0)^T$\\
where $\mathbf{p}(n_i)-\mathbf{p}(n_j)$ occupies 
%positions $di\ldots di+d-1$ and 
$n_i$'s group of positions
and $\mathbf{p}(n_j)-\mathbf{p}(n_i)$
occupies $n_j$'s.
%positions $dj\ldots dj+d-1$.
\extra{This definition is echoed from the 
literature \cite{RigidityBook} on rigidity theory, except
we interchange rows and columns.  Just as we deemphasized nodes of electrical
networks, we will use merely the row space of $M_V$ for most of what follows.
}
\extra{The rigidity matrix as a 
function of the embedding $\mathbf{p}$ is denoted
$M_V(\mathbf{p})$.  The row vector $\mathbf{p}$ left multiplied with
$M_V(\mathbf{p})$
is the row tuple denoted $\mathbf{L}=\mathbf{p}M_V(\mathbf{p})$  Then, 
$\mathbf{L}_e$ $=$ 
$(\mathbf{p}M_V(\mathbf{p}))_e$ $=$ $|\mathbf{p}_i-\mathbf{p}_j|^2$ for
each edge $e=(i,j)$.  Now if each $\mathbf{p}_i$ is a differentiable function
of $t$, $d\mathbf{L}/dt$ $=$ $2\mathbf{p}'M_V(\mathbf{p})$.  Framework
$\mathcal{F}(\mathbf{p})$ is \textit{first-order rigid} when 
$d\mathbf{L}/dt$ $=$ $0$ for all $\mathbf{p}'$ implies 
$|\mathbf{p}_i-\mathbf{p}_j|^2$ is constant for all pairs $i$, $j$, not just
endpoints of edges. 
}%extra
%Let $\mathbf{p}(i)-\mathbf{p}(j)$ for edge $e=\{i,j\}$ be called the
%\textit{vector from } $j$ \textit{to} $i$.  
From \cite{RigidityBook},
$u\in L_V=L(M_V)\subset\Reals^U$ 
iff 
for some combination of \textit{vertex velocities}
$\mathbf{v}:N\rightarrow\Reals^d$,
$u_e=(\mathbf{v}(i)-\mathbf{v}(j))\cdot(\mathbf{p}(i)-\mathbf{p}(j))$
for each $e=(i,j)\in U$.
Also,
the \textit{self-stress subspace} $L_I=L_V^\perp$%
%, the orthogonal complement of $L_V$, 
is shown to be all $w\in\Reals^U$
for which the framework is in static
equilibrium when each edge $e=(i,j)$ exerts force 
$w_e(\mathbf{p}(j)-\mathbf{p}(i))$ on vertex $i$.  By this convention, 
$w_e>0$ means $e$ is under tension and $w_e<0$ means $e$ is under 
compression.  
%(\cite{RigidityBook} actually defines $M_V^T$, not $M_V$.)
(The rigidity matrix in \cite{RigidityBook} is $M_V^T$, not $M_V$.)

Under this analogy, 
(1) KVL corresponds to geometric consistency of first order
bar length changes under changes in the embedding, (2) KCL corresponds to
Newton's laws of static equilibrium, and (3) Tellegen's theorem corresponds
to a virtual work principle, that static equilibrium is
characterized by 
the internal forces against every virtual embedding change 
doing zero virtual work.

\subsubsection{Display Elements}
We can also declare ``display'' quantities, say for the difference of two
independent quantities, by appending the appropriate column to $M_V$ or
$M_I$.  Any dependency of the display quantity on element quantities will
appear as a dependency in the oriented matroid.  The parallel extension 
operation (defined below) inserts an element whose quantity will duplicate
the element quantity of the parallel element.




\subsection{Reduction Operations}
The matroid contraction and contraction operations are defined on oriented
matroid pairs by $\mathcal{M}/e$ $=$ $(\mathcal{M}_V,\mathcal{M}_I)/e$
$=$ (by definition of \textit{contraction}) 
$(\mathcal{M}_V/e,\mathcal{M}_I\setminus e)$;
and 
$\mathcal{M}\setminus e$ 
$=$ (by definition of \textit{deletion}) 
$(\mathcal{M}_V\setminus e,\mathcal{M}_I/ e)$.
In other words, the operation is applied to the V part and its dual is applied
to the I part.  Oriented matroid pair deletion and contraction preserve
dual pairs.

\begin{theorem}
\label{NoCCMinors}
Suppose $(\mathcal{M}_V,\mathcal{M}_I)$ have complementary base pair 
$B_V \cup B_I=E$, $B_V \cap B_I=\emptyset$ 
(denoted $B_V \dunion B_I=E$)
and no
common covector.  Let $e\in E$.
If $e\in B_V$ then $(\mathcal{M}_V,\mathcal{M}_I)/e$
also has a complementary base pair 
($(B_V\setminus e) \dunion B_I$) and no common covector.
Dually, if $e\in B_I$ 
then $(\mathcal{M}_V,\mathcal{M}_I)\setminus e$ has a complementary base pair
and no common covector.
\end{theorem}

\begin{proof}
The smaller oriented matroid pair has complementary
base pairs because of the condition on which basis contains $e$ and 
the definition of deletion and contraction.  Suppose say $e\in B_V$ but
$(\mathcal{M}_V,\mathcal{M}_I)/e$ $=$ 
$(\mathcal{M}_V/e,\mathcal{M}_I\setminus e)$ have a common covector
$X$.  Let $A=\supp{X}$, $S=(B_V\setminus A)\cup\{e\}$, $R=(B_I\setminus A)$
and $Z=\emptyset$.  Let signed set $C_0$ be defined by
$(C_0)_f=X_f$ for all $f\neq e$ and $(C_0)_e=0$.  
$C_0$ is a covector in $\mathcal{M}_V$ 
by definition of the contraction $\mathcal{M}_V/e$.
Let signed set $D_0$ be a covector of $\mathcal{M}_I$
$(D_0)_f=X_f$ for all $f\neq e$ which exists since $X$ is a covector
of $\mathcal{M}_I\setminus e$.  If $(D_0)_e=0$ then
$X=C_0=D_0$ is a common covector of 
$(\mathcal{M}_V,\mathcal{M}_I)$ and we have a 
contradiction.  Otherwise, the following Theorem \ref{CCCtheorem}
applied to the above 
sets and covectors demonstrates a contradiction with $C$.


\begin{theorem}
\label{CCCtheorem}
Let ${\mathcal M}_1$ and ${\mathcal M}_2$ be oriented matroids 
on 
%
% EDITdone
% insert: the
%
the ground set $A$ $\dunion$
%
% EDITdone
% I'm guessing that you intend this for disjoint union.  
% Union with a dot inside (which is easy to create) is
% the common symbol, and this is used in the Oriented Matroids book.
%
$Z$ $\dunion$ $S$ $\dunion$ $R$.
Assume the 
covectors 
$C_0\in {\mathcal L}({\mathcal M}_1)$ and
$D_0\in {\mathcal L}({\mathcal M}_2)$ satisfy the 
%
% EDITdone
% insert: the
% Also, since the properties are in the middle of a sentence,
% they should end with commas, not periods, and at the end of the
% second to last, there should be an ``and''.
%
% A preferable alternative it to break this long sentence into two:
% Assume ... satisfy these properties: 1)..., 2).., and 3)...
% Then ...
% REPLY done
properties:
\begin{enumerate}
\item $A\neq\emptyset$ and $C_0(a)=D_0(a)\neq 0$ for all $a\in A$,
\item $S\dunion Z$ is independent in 
${\mathcal M}_1$ and $C_0(S\dunion Z)=0$, and
\item $R\dunion Z$ is independent in ${\mathcal M}_2$ and $D_0(R\dunion Z)=0$.
\end{enumerate}
Then 
\OMP{${\mathcal M}_1$}{${\mathcal M}_2$} have a common (non-zero) covector 
$C\in {\mathcal L}({\mathcal M}_1)\cap {\mathcal L}({\mathcal M}_2)$ that is
compatible with both $C_0$ and $D_0$.  In other words, $C_0\preceq C$ and 
$D_0\preceq C$.
\end{theorem}
The proof of Theorem \ref{CCCtheorem} given in \cite{sdcOMP}
contains an algorithm to construct
a common covector efficiently
by composing 
covectors with cocircuits.   
Figure \ref{ccctableaux} illustrates the setup.

\begin{figure}[tb]
\[
\mbox{\psfig{figure=ccctableaux.eps}}
\]
\caption{The tableau for ${\mathcal M}_1$
shows the cocircuits $c_e$ for $e\in S$, other cocircuits
for $e\in Z$,
%
% EDITdone
% insert a comma
%
and the covector $C_0$.  These cocircuits are fundamental
with respect to some basis that extends $Z\cup S$.  The algorithm
constructs 
%
% EDITdone
% insert ``a'' or ``the''
%
a covector $C$ by starting with $C\leftarrow C_0$ and composing 
$C\leftarrow C\circ (\pm c_e)$ to make $C(e)=D(e)$ when necessary, while
similar operations are applied to $D$ using cocircuits from 
${\mathcal M}_2$.
}
\label{ccctableaux}
\end{figure}



The dual case is similar.
\end{proof}

Given a subspace $L\subset\Reals^U$ and $e\in U$, the subspace 
``$L$ \textit{with} $e$ \textit{deleted}''
is
$L\setminus e$ $=$ 
$\{ l(U\setminus e) | l\in L\}$
$\subset\Reals^{U\setminus \{e\}}$, 
where $l(U\setminus e)$ denotes the $l\in\Reals^U$
with the component labeled by $e$ dropped.  
Thus, $L\setminus e$ is the \textit{projection} of $L$ into 
$\Reals^{U\setminus \{e\}}$.
If $L=L(M)$ 
then $L\setminus e$ $=$ $L(M(U\setminus \{e\}))$ is the row space of 
$M(U\setminus \{e\})$, which is $M$ with column $e$ deleted.
The subspace  ``$L$ \textit{with} $e$ \textit{contracted}'' 
$L/e$ $=$ 
$\{ l(U\setminus e) | l(U)\in L \mathrm{\ and\ } l(e)=0\}$.  
So, $L/e$ is the 
intersection of $L$ with
the (hyperplane) subspace of $\Reals^U$ with $l(e)=0$,
projected into 
$\Reals^{U\setminus \{e\}}$.

We now define \textit{deletion} and \textit{contraction} on subspace pairs:  
$(L_V, L_I)\setminus e$ $=$ $(L_V\setminus e, L_I/e)$ and %\\
$(L_V, L_I)/e$ $=$ $(L_V/e, L_I\setminus e)$.  
Deleting $e$ in the electrical application
corresponds to \textit{opening} the corresponding branch.
Dually, 
contraction corresponds to \textit{shorting}.  
Mechanically,
deletion of an edge corresponds to ``breaking'' the corresponding bar:
ignore any distance change between its ends and transmit no force.
Contraction corresponds to 
declaring the bar to be inelastic, which rules out 
all (first order) distance changes between the endpoints and 
rules out any constitutive law referring to 
tension or compression in that bar.

\subsubsection{Nullor Elements}

A \textit{nullator} element $e\in E$ 
expresses the ideal constitutive law $u_{Ve}=0$
and $u_{Ie}=0$ which approximates conditions at the input
to a high-gain amplifier when a system is stabilized by feedback.
Hence a nullator is declared by 
\textit{contracting $e$ in both
$L_V$ and $L_I$.}  
In this way, the effects of an ideal constitutive law are modeled by
the subspace rather than the constitutive law part of the subspace
pair model.

In non-degenerate situations, the variables associated with $e$ are
not already constrained to zero so $e$ is not a loop in 
either $\mathcal{M}_V$ or $\mathcal{M}_I$.  When this is true,
nullator insertion reduces both ranks by 1.
%
A \textit{norator} element $e\in E$ indicates that the constitutive
law puts no direct constraint on $u_{Ve}$ or $u_{Ie}$; the amplifiers
approximately adjust the output state so the feedback results in zero input.
Hence a norator is declared by 
\textit{deleting $e$ in both
$L_V$ and $L_I$.}
In non-degenerate situations, $e$ is not an isthmus 
in either $\mathcal{M}_V$ or $\mathcal{M}_I$, so their ranks don't change.



\subsection{Extension Operations}


\subsubsection{Series and Parallel Extension}

The parallel extension of a non-loop element $e$ in oriented matroid
$\mathcal{M}$ over $E$ has ground set $E\dunion e'$ 
and covectors $\mathcal{L}'$ defined by $X'\in\mathcal{L}'$
with $X'_e=X'_{e'}=X_e$ and $X'_f=X_f$ for all $f\in E\setminus e$
and $X\in\mathcal{L}$.  \cite[p.??]{OMBOOK} proves it is an oriented 
matroid.  Series extension is the dual operation:  If
$e$ is not an isthmus, then the series extension of $e$ is
$((\mathcal{M}^*)')^*$. 
We define parallel extension of $e$ in 
pair $(\mathcal{M}_V,\mathcal{M}_I)$ when $e$ is not a loop
in $\mathcal{M}_V$ and not an isthmus in 
$\mathcal{M}_I$ by parallel extending $e$ in $\mathcal{M}_V$
and series extending $e$ in $\mathcal{M}_I$.  Series extension 
is defined dually.  It is clear that series and parallel extension
preserves dual oriented matroid pairs.  

\begin{figure}
\input{parallel.pstex_t}
\caption{Tableaux for an oriented matroid pair and its parallel extension
at element $e$.}
\end{figure}


\begin{fuzz}
Does series/parallel extension preserve the complementary
base and no-common-covector property of a non-dual oriented matroid 
pair?
\end{fuzz}


\subsubsection{Ports}
A single \textit{port} in a single oriented matroid or subspace of 
$\Reals^U$ is a 
single distinguished element or coordinate $p\in U$.  
We will define a port in a pair of oriented matroids or subspaces to 
be a pair $\{p_V,p_I\}\subset U$.   Each member of the port will be
a matroid \textit{isthmus} (element not in any circuit\footnote{In 
matroid theory an element like $p_I$ of
$\mathcal{M}(L_V)$ that is independent of all others
is called an \textit{isthmus}.
Similarly, $p_V$ is
an isthmus of $\mathcal{M}(L_I)$.  In general, the matroid represented by
a matrix is characterized by the collection $\mathcal{I}$ 
of \textit{independent sets} 
of matrix columns, where a set of columns is called independent when it is
linearly independent.  (Matroid theory studies what can be deduced by 
the following three axioms satisfied by $\mathcal{I}$: (1) 
$\mathcal{I}\neq\emptyset$. 
(2) If $A\subset B\in\mathcal{I}$ then $A\in\mathcal{I}$.  (3) 
If $A$, $B$ $\in\mathcal{I}$ and $|A|<|B|$, then 
there exists $e\in B\setminus A$ 
for which $A\cup\{e\}\in\mathcal{I}$.  For example, an isthmus $e$ is 
characterized by $A\cup\{e\}\in\mathcal{I}$ for all $A\in\mathcal{I}$.
The \textit{rank} of a subset $C\in U$ is the size of the largest independent
subset of $C$.  
%A set $D$ is called \textit{coindependent} if $\rank(U\setminus D)$
%$=$ $\rank(U)$; in other words, removing $D$ does not diminish the original
%matroid's rank.)
})
in one of OMs or subspaces because this notation will
enable us to formulate some solution sets 
(when the constitutive laws are linear)
as intersections of subspaces.
We will construct oriented matroid or subspace pairs with one or more
ports by applying a \textit{port insertion operation} to such a pair.

Each port of a subspace pair provides two variables:  When one of them is 
considered a parameter, it is called an 
\textit{input} or a \textit{source}.
The non-input port variables are called \textit{outputs}.
Such a descriptions of variables can apply to the corresponding 
oriented matroid elements or coordinate names.

Environmental constraints in physical problems often 
occur
so 
that exactly one variable of each port is an input and the other
is an output\footnote{
Formulations in which both variables are inputs are used to
analyze interconnected systems.  The result of 
analyzing a linear network this way is called a 
transmission matrix.  \textbf{We should try to understand 
this distinction better.}
}.  Which of $\{p_V,p_I\}$ is for the input variable
is described by what \textit{kind of source} is \textit{driving}
the port:  electrical voltage (battery) or current sources; or their
mechanical analogs of displacement or force.
\extra{Unlike device variables, the two variables of each port
are not directly related by a constitutive law which is 
part of the system model.}
Ports are introduced so the response of an electrical network to current 
and/or voltage sources, and the mechanical analogs, can be formulated.
\extra{Questions of existence and 
uniqueness of solution for various combinations of 
kinds of sources are formulated after 
modeling devices.}

\extra{ Ports 
%also 
facilitate formal operations to compose larger systems
from smaller ones.  We believe ports are important for investigations of
rigidity because they model how a framework interacts with its environment,
for example, what a mechanical model ``feels like'' when you squeeze it.
We have also found that electrical port characteristics of unit resistance
ported electrical networks are ratios of coefficients in certain 
partial evaluations 
of a generalization of the Tutte polynomial\cite{sdcPorted}.}

\begin{figure}
\input{portext.pstex_t}
\caption{Tableaux for an oriented matroid pair and the result of 
inserting a port at element $p$}
\end{figure}



Here are the formal definitions.
Given $(L_V, L_I)$ and $p\in U$ not already a port, 
the 
operation of \textit{inserting a port at  $p$} 
defines a new subspace pair $(L'_V, L'_I)$
with $U'=U\setminus \{p\}\cup\{p_V,p_I\}$, $L'_V=L_V\oplus\Reals$ (direct 
sum) with $p$ replaced by $p_V$;
and the coordinate of the added $\mathbf{R}$ 
indexed by $p_I$, together with
$L'_I=L_I\oplus\Reals$ with $p$ now replaced by $p_I$
and the added subspace indexed by $p_V$.
\extra{Note that (going to $(L'_V, L'_I)$) the ranks of
$L_V$ and $L_I$ each increase by 1, and $|U'|$ $=$ $|U|+1$.}
\extra{After $p$ port
insertions, we denote the final $U$ by $E\cup P_V \cup P_I$ 
with pairwise disjoint
$E$, $P_V$ and $P_I$, $|P_V|$ $=$ $|P_I|$ $=$ $p$, $P_V\cup P_I$ being the 
replacement elements.}

%Ports are defined this way so the solution set is the
%intersection of linear subspaces when the constitutive laws
%are linear.

Inserting ports into a subspace pair is easily
abstracted to an oriented matroid pair.  An oriented matroid pair has ports
$P_V\dunion P_I$
when $U=E\dunion P_V\dunion P_I$ where $|P_V|=|P_I|$, $P_V$ is a set of
isthmuses in $\mathcal{M}_I$  $P_I$ is a set of isthmuses in
$\mathcal{M}_I$, and the elements of $P_V$ and $P_I$ are in one-to-one
correspondance.  A corresponding pair $(p_V,p_I)$ with
$p_V$ an isthmus in $\mathcal{M}_I$ and $p_I$ an isthmus 
in $\mathcal{M}_V$
is denoted briefly by $p\in P$.

Given $p\in E(\mathcal{M})$, not already a port, the oriented matroid
pair $\mathcal{M}'$ $=$ $\mbox{make-port}_p(\mathcal{M})$ is defined as
follows:  $E(\mathcal{M}')=E(\mathcal{M})\setminus p$,
$P_W(\mathcal{M}')=P_W(\mathcal{M})\dunion p_W$ for $W$ $=$ $V$ and $I$,
$\mathcal{M}'_V$ is the extension of $\mathcal{M}_V$ with 
isthmus $p_I$ and with element $p$ renamed by $p_V$, and 
$\mathcal{M}'_I$ is the extension of $\mathcal{M}_I$ with 
isthmus $p_V$ and with element $p$ renamed by $p_I$.
It is immediate that inserting a port causes the rank excess to increase
by 1.  The operation $\mbox{make-port}_p()$ may be iterated over each
$p\in P$  given $P\subseteq E$ with the same resulting
oriented matroid pair independent of the order of iteration.



\subsubsection{Zeroing of Ports}

Given port $p=(p_V,p_I)$ in 
$\mathcal{M}'$ $=$ $\mbox{make-port}_p(\mathcal{M})$ the operations
$\mbox{V-zero}_p(\mathcal{M}')$ and
$\mbox{I-zero}_p(\mathcal{M}')$ are defined as follows:
$\mbox{V-zero}_p(\mathcal{M}')$ $=$ 
$(\mathcal{M}'/p_V/p_I,\mathcal{M}'\setminus p_V \setminus p_I)$
$=$ $\mathcal{M}/p$.  The symbol $p$ in the last expression 
of course denotes the element of $E(\mathcal{M})$ that was made into a
port.  Dually,
$\mbox{I-zero}_p(\mathcal{M}')$ $=$ 
$(\mathcal{M}'\setminus p_V \setminus p_I,\mathcal{M}'/p_V/p_I)$
$=$ $\mathcal{M}\setminus p$. 

\begin{proposition}
If $p$ is in basis $B$ of $\mathcal{M}_V$ with 
$U\setminus B$ a basis of 
$\mathcal{M}_I$, then
$(\mathcal{M}_V,\mathcal{M}_I)/ p$ $=$
$\mbox{Vdrive-zero}_p(\mbox{make-port}_p(\mathcal{M}_V,\mathcal{M}_I))$.

If $p$ is in basis $B$ of $\mathcal{M}_I$ with 
$U\setminus B$ a basis of 
$\mathcal{M}_v$, then
$(\mathcal{M}_V,\mathcal{M}_I)\setminus p$ $=$
$\mbox{Idrive-zero}_p(\mbox{make-port}_p(\mathcal{M}_V,\mathcal{M}_I))$.
\end{proposition}



\subsection{Subspace and Matroid Combination Along Ports}



Suppose we can perform row operations on $M$ so it has the
its first two rows have the form $(1,0,r_1)$ and $(0,1,r_2)$, and 
the other $r-2$ rows have the form $(0,0,r_i)$.  Let $p_1$ and 
$p_2$ be the elements corresponding to the first two columns.

To combine $L$ along $p_1,p_2$ \textit{in parallel}, we replace the first
two rows by the single row $(1,1,r_1+r_2)$, delete the second
column and then identify elements $p_1,p_2$.
The resulting subspace is obtained by intersecting $L=L(M)$ with
the hyperplane defined by $u_{p_1}=u_{p_2}$ and then projecting into
$R^{U\setminus p_1}$.

To combine $L$ along $p_1,p_2$ \textit{in series}, we replace the first
two rows by the two rows  $(1,1,r_1)$ and $(1,1,r_2)$, delete the second
column and then identify elements $p_1,p_2$.  The resulting subspace is 
the image of $L=L(M)$ under the linear map 
$\Reals^U\rightarrow\Reals^{U\setminus p_2}$ by the map that takes 
$(1,0,0,\ldots,0)$ and $(0,1,0,\ldots,0)$ to $(1,\ldots,0)$.




\section{Parameters, Constitutive Laws and Solutions}

In this section, we will define the subspace pair model and its
solutions, give matroid
theoretic conditions necessary for solutions to exist or
to be unique, and show how no-common-covector conditions are
necessary for uniqueness.  The section concludes with
the main existance and uniqueness theorem, explained and proved
by demonstrating its equivalance to Sandberg and Willson's 
result.



The \textit{subspace pair model} $\mathbf{M}$ $=$ 
$(E, \Gamma, P, (L_V, L_I))$ consists of finite set $E$ of 
\textit{device elements}, \textit{constitutive law relations}
$\Gamma = \{\Gamma_e\subset\Reals\times\Reals | e \in E\}$, the finite set
$P=P_V \dunion P_I$ that results from inserting ports as defined above, 
and a subspace pair $(L_V, L_I)$ over $\Reals^U$ with $U=E\dunion P$.
Note that $P_V$ (resp. $P_I$) is a set of isthmuses
in $\mathcal{M}_I$
(resp. $P_V$).

The \textit{variables} of $\mathbf{M}$ are 
$\{u_{\mathit{Ve}}, u_{\mathit{Ie}} | e \in E\}$  $\cup$ %\\
$\{ u_{\mathit{pV}}, u_{\mathit{pI}} | p_I, p_V \in P \}$.  
(For brevity, subscript ``$\mathit{pV}$'' means port element
$p_V\in P_V$, etc.)
A \textit{subspace pair model with sources} $S$ 
is a subspace pair model $(E, \Gamma, P, (L_V, L_I), S)$
together with a subset $S$ of exactly $|P|$ of the $2|P|$ elements
in $P$.  A \textit{$V$-driven port} is 
%a port $p\in P$ 
one for which 
$p_V\in S$ and $p_I\not\in S$, then $u_{\mathit{pV}}$ is called 
an \textit{input variable}.   Reverse $V$ and $I$ to define an 
\textit{$I$-driven port} and its input variable.

A \textit{solution} of $\mathbf{M}$ with sources
is a real valued extension to \textit{all} variables of $\mathbf{M}$ 
of a given \textit{input assignment} to the input variables
that satisfies
$(u_{\mathit{PV}}, u_{\mathit{PI}}, u_V) \in L_V$,
$(u_{\mathit{PV}}, u_{\mathit{PI}}, u_I) \in L_I$
and
$(u_{\mathit{Ve}}, u_{\mathit{Ie}}) \in \Gamma_e$ for all
$e\in E$.
%  Note that in this model, 
The constraint 
$(u_{\mathit{PV}}, u_{\mathit{PI}}, u_V) \in L_V$ 
does not (by itself) imply any constraint
on a I-driven port variable $u_{\mathit{pI}}$, similarly, 
$u_{\mathit{pV}}$ is not constrained
by 
$(u_{\mathit{PV}}, u_{\mathit{PI}}, u_V) \in L_I$.
\extra{Port variables are not constrained
by the constitutive laws $\Gamma$ (by themselves) either.}

\textbf{MUST FIX THIS:::}Assume as usual 
%no branch is taken to be both a current and voltage source.
no port is both I-driven and V-driven.
The 
\extra{well-known necessary condition
for an electrical network to have a unique 
solution for all choices of source values is}
condition of no cycle of voltage source branches
in the ``voltage'' graph nor a
cutset of current source branches in the ``current graph'' (see, e.g.,
\cite{ChensBook})
generalizes to:


\begin{theorem}
\label{Feasiblity1}
\textbf{(1)} If all V input assignments are feasible under
the $L_V$ constraint then 
$\{p_V|p \mbox{\ is V-driven}\}$
is an independent set in the matroid $\mathcal{M}(L_V)$.
\textbf{(2)} If every solution is unique then %\\
$\{p_I|p \mbox{\ is V-driven}\}$
must be coindependent in $\mathcal{M}(L_I)$.
\textbf{(3)} 
If all I input assignments are feasible under
the $L_I$ constraint then 
$\{p_I|p \mbox{\ is I-driven}\}$
is an independent set in the matroid $\mathcal{M}(L_I)$.
\textbf{(4)} If every solution is unique 
then
$\{p_V|p \mbox{\ is I-driven}\}$
must be coindependent in $\mathcal{M}(L_V)$.
\end{theorem}

\begin{proof}
(1,3):
If set $S$ of input variables is dependent, then there is
some combination of input values that is not feasible.
(2,4):
$S$ is not a \textit{coindependent} set iff $S$ contains a cocircuit, so
%A non-coindependent set $S$ must contain a cocircuit, so
there is a non-zero covector supported by $S$.  Hence there is a 
feasible variable assignment that is non-zero on the some port output
variables only.
\end{proof}


\begin{figure}
\label{matroid-port-conditions}
%\[
\begin{tabular}{|c|c|} \hline \\
$S=\{p_V | p \mbox{\ is V-driven.}\} \in \mathcal{I}(\mathcal{M}_V)$& 
            \begin{minipage}{2.5in}
               $\leftarrow$ \ All of $\Reals^{|S|}$ feasible as V-input assignments.
            \end{minipage}
			\\  \hline  \hline
$S=\{p_V | p \mbox{\ is V-driven.}\} \in \mathcal{I}(\mathcal{M}_V^*)$& 
            \begin{minipage}{2.5in}
              $\leftrightarrow$
            All V-input assignments $\Reals^{|S|}$ make some other V variables non-zero.
            \end{minipage}
			\\  \hline  \hline
$S=\{p_I | p \mbox{\ is V-driven.}\}  \in \mathcal{I}(\mathcal{M}_I)$&
            \begin{minipage}{2.5in}
             $\leftrightarrow$
             No KCL dependencies among output variables
             \end{minipage}
			\\  \hline  \hline
$\{p_I | p \mbox{\ is V-driven.}\}  \in \mathcal{I}(\mathcal{M}_I^*)$ & 
            \begin{minipage}{2.5in}
            $\leftarrow$ I-output values unique.
	    \end{minipage}
			\\ \hline 
\end{tabular}
%\]
\caption{Electrical network application interpretation of matroid theoretic 
independence conditions on the V-driven port subset.  Corresponding dual 
statements
apply to the I-driven port subset.}
\end{figure}

Each port insertion increases the rank excess $\rank(L_V)+\rank(L_I)-|U|$ by 1.
When the constitutive laws are linear, the solutions of $\mathbf{M}$ are
found from the \textit{intersection} of two linear subspaces:  Let $G$ 
be the diagonal matrix with ``conductances'' $g_e$ in its positions
indexed by $e \in E$ (so $\Gamma_e$ $=$ $\{ (v, g_e v) | v \in \Reals \}$
and 1 in its other diagonal positions.  The solution set projected onto
the $u_I$ variables is $L_VG\cap L_I$.  
(Here, $L_V G$ means $L(M_V G)$.)

\subsection{Supplemental Subspace Pair and Uniqueness Theory}


\begin{comment}
Familiar topological conditions on dependencies among source values
pertain to the \textit{matroids} of the subspaces $L_V$ and $L_I$.
Questions about existence and uniqueness of solution
will be answered in terms of supplementary subspace pair models
which are obtained by the familiar operations of opening and shorting ports.
Finally, operations on subspace pairs that model nullor insertion are
defined, so that such ideal elements can be modeled combinatorially or
geometrically.  

The supplemental subspace pair derived after nullor 
insertion will typically not be orthogonal.  One might also choose to
model linearized CCCSs or VCVSs within one of the subspaces.  
\end{comment}

%\subsection{Deletion and contraction}

The \textit{supplemental subspace pair}
% of a subspace pair model
%with sources 
is constructed by \textit{zeroing} all the sources.
For each V-driven (resp. I-driven) port $p$, $p_V$ 
(resp. $p_I$) is 
contracted in both $L_V$ and $L_I$, and $p_I$ 
(resp. $p_V$) is deleted in
both $L_V$ and $L_I$.
The resulting subspaces, etc.  are denoted $L_V^0$, $L_I^0$, etc.

%\subsection{Uniqueness}

The following proposition expresses elementary properties of the 
subspace pair model formulation that have been observed by
Willson and Sandberg \cite{} and by Hasler, Nernyck \textit{et. al.}
\cite{} in their own formulations.

\begin{proposition}
\label{Uniquenessprop}
Let $\mathbf{M}=(E, \Gamma, P, (L_V, L_I), S)$ be subspace pair model
with sources in which the contitutive relations are monotone
increasing.  (a) If $\mathbf{M}$ has a solution and the supplemental
subspace pair $(L_V^0,L_I^0)$ has no common covector, then the solution
is unique. (b) If $(L_V^0,L_I^0)$ has a common covector, then there exist
linear monotone increasing constitutive relations $\Gamma$ for which
$\mathbf{M}$ has multiple solutions.
\end{proposition}


We note here that when $(L_V^0,L_I^0)$ have a common covector,
the existence of multiple solutions results from
either a ``degeneracy in the problem formulation''
(which we will see below as a failure of 
conditions \textbf{(1)} or \textbf{(2)} of Theorem \ref{SubspacePairTh},
below and is called a ``never-well-posed'' problem th Hasler and Neiryck)
or an exceptional choice of monotone constitutive laws.  A problem 
that is well-posed (has a unique solution) for almost all linear
constitutive laws, but which has non-unique solutions for particular
postive constitutive law coefficients is called 
``sometimes-ill-posed'' by Hasler and Neiryck.

The signs of each solution to a subspace pair problem with sources 
comprise a common covector of the subspace pair 
$(L_V,L_I)$.  Indeed, with linear constitutive laws, a solution 
will exist when $\mbox{dim\ } L_V+\mbox{dim\ } L_I> |U|$.  
The no-common-covector condition implies such a solution (a tuple of
reals) is unique.  The solution depends on the constitutive laws.
For different constitutive laws, there will typically be different
solutions.  Please note that the (non-zero) common covectors of
$(L_V,L_I)$ are not necessarilly unique (more precisely,
for common convectors $X_1,X_2\neq 0$, $X_1\neq X_2$ and $X_1\neq -X_2$,)
even when
$(L_V^0,L_I^0)$ have no common covector.




\subsection{Main Theorem on No-Common-Covectors and $\mathcal{W}_0$ Pairs}

%The following Theorem demonstrates, by means of 
Theorem \ref{SubspacePairTh} uses
Sandberg and Willson's 
%theory of 
$\mathcal{W}_0$ pairs
to show every 
subspace pair model 
\extra{(with  its separation of
geometric/topological and constitutive constraints)}
can be analyzed
for unique solvability from the oriented matroid pair it generates.
%By Thm.\ \ref{sandwillompairtheorem} 
Theorem \ref{sandwillompairtheorem} shows how
$(A,B)\in\mathcal{W}_0$ is characterized by a rank condition and a 
no-common-covector property.  
Theorem \ref{PairConclusionTheorem} is our unifying conclusion.

\begin{theorem}
\label{SubspacePairTh}
The subspace pair model has a unique solution for all 
%source 
input assignment 
values and positive monotone constitutive laws iff 
\textbf{(1)}, \textbf{(2)} and \textbf{(3)} are satisfied:\\
\textbf{(1).} There are bases $B_V\in\mathcal{B}(L_V)$,
$B_I\in\mathcal{B}(L_I)$ for which all V-driven 
ports $p$ satisfy $p_V\in B_V$ (note $p_I$ must be 
in $B_V$ since it's an isthmus in $\mathcal{M}_V$) 
and $p_I\not\in B_I$,
and all I-driven ports $p$ satisfy $p_I\in B_I$ and 
$p_V \not\in B_V$.  \\
\textbf{(2).}
$B_V\cup B_I = U$, for the bases in \textbf{(1)}.\\
\textbf{(3).} The oriented matroid pair of supplemental pair $(L_V^0,L_I^0)$ 
have no common (nonzero) covector.
(i.e., $\mathcal{L}(L_V^0)\cap\mathcal{L}(L_I^0)=\{0\}$.)
\end{theorem}


\begin{proof} (Sketch)
If:  When $(L_V^0,L_I^0)$ is constructed given
\textbf{(1-2)}, we find $\mbox{rank\ }(L_V^0)+\mbox{rank\ }(L_I^0)\geq|E|$. By
Theorem 1.3 of \cite{sdcOMP}
and $|E|\geq$ rank of the matroid union 
(\cite{Welsh}, 8.3) of $\mathcal{M}_V^0\vee\mathcal{M}_I^0$,
we must have equality else \textbf{(3)} would be 
contradicted.  
%We then verify from (2) the square matrices
%$A$ $=$ $\left(\begin{array}{c}M_V^0\\ 0\end{array}\right)$ and
%$B$ $=$ $\left(\begin{array}{c}0\\ -M_I^0\end{array}\right)$
%satisfy the rank part of condition (2) of Theorem
Form square matrices
$A$ $=$ $\left[\begin{array}{c}M_V^0\\ 0\end{array}\right]$,
$B$ $=$ $\left[\begin{array}{c}0\\ -M_I^0\end{array}\right]$ and 
verify from \textbf{(2)} they satisfy the rank part of cond. \textbf{(2)} of
Th. \ref{sandwillompairtheorem}. 
${\mathcal L}\hmat{A}{B} \cap {\mathcal L}\hmat{I}{-I}=\{0\}.$
follows directly from hypothesis \textbf{(3)}.  We can then
formulate the monotonic 
subspace pair problem as a case of 
Theorem \ref{sandwillompairtheorem} part \textbf{(5)} to complete the proof.

Only if:  Use Theorem \ref{Feasiblity1} to verify \textbf{(1)} and 
\textbf{(2)}.  
If \textbf{(3)} were violated,
positive linear constitutive laws could be constructed
(as in \cite{HaslerNeirynck,SWExistancePf}) for which 
a non-zero solution for zero input exists.
\end{proof}

\begin{theorem}
\label{sandwillompairtheorem}
(From \cite{sdcOMP}) For a pair 
%
% EDITdone
% delete ``order'' --- we understand that when we see 
% ``$n\times n$'', and this avoids confusing with order as in
% a linear ordering
%
of $n\times n$ 
matrices $(A,B)$, the following conditions are 
equivalent.\\
%
% EDITdone
% period, not colon
%
\textbf{(1)}
$(A,B)\in {\mathcal W}_0$ in the sense of 
Sandberg and 
Willson~\cite{SWExistancePf};
e.g., $|AD+B|\neq 0$ for all positive diagonal $D$, etc.\\
\textbf{(2)}
$\rank{\mathcal M}\hmat{A}{B}=n$ and 
${\mathcal L}\hmat{A}{B} \cap {\mathcal L}\hmat{I}{-I}=\{0\}.$\\
\textbf{(3)}
$\rank{\mathcal M}\hmat{A}{B}=n$ and 
${\mathcal V}\hmat{A}{B} \cap {\mathcal V}\hmat{I}{-I}=\{0\}.$\\
\textbf{(4)}
\textrm{(}Fundamental theorem of Sandberg and 
Willson~\cite{SWExistancePf,W0APPLpaper}\textrm{)}\ 
%
% EDITdone
% use ~ before \cite so a space is left between the word and the reference
%
For all functions $F:{\bf R}^n\rightarrow {\bf R}^n$ of the form
$F(x)_k=f_k(x_k)$ where each $f_k$ is a strictly 
monotone increasing 
function from 
${\bf R}$ {\em onto} $\bf R$ and for all $c\in {\bf R}^n$, the 
equation $AF(x) + Bx = c$\\
has a unique solution $x$. \cite{W0APPLpaper}.\\
\textbf{(5)}
For all functions $G:{\bf R}^n\rightarrow {\bf R}^n$ of the form
$G(w)_k=g_k(w_k)$ where each $g_k$ is a strictly monotone increasing 
function from ${\bf R}$ {\em onto} $\bf R$ and for all 
$d^{\prime}, d^{\prime\prime}\in {\bf R}^n$, the 
eqns.\ $u^t=z^tA+d^{\prime},\;\; w^t=z^tB+d^{\prime\prime},\;\; u=-G(w)$
have a unique solution.% $(u,w,z)$.
\end{theorem}

\begin{remark}
Our proof in \cite{sdcOMP} 
of equivalence of \textbf{(2)} and \textbf{(3)} depends on
general properties of relative orientations of complementary 
base pairs in
realizable oriented matroid pairs (i.e., subspace pairs).
The rank condition is equivalent to $|AD+B|\neq 0$ for some
diagonal $D$.  The no-common-vector condition of \textbf{(3)} means
that whenever $Ax+By=0$ and $\sigma(x)=\sigma(y)$, $x$ and $y$
must both be $0\in\Reals^n$.  The $\mathcal{W}_0$ property follows
from\cite{SWExistanceSIAM} a  stronger property that 
$(A,B)$ is a passive pair: $Ax+By=0$ implies $x\cdot y \leq 0$.
An example of a non-passive $\mathcal{W}_0$ pair is
$A=\left[\begin{array}{cc}1 & -1 \\ 0 & 0\end{array}\right]$ and 
$B=\left[\begin{array}{cc}0 & 0 \\ 0 & 1\end{array}\right]$.
This matrix pair comes from the elementary operational amplifier
example in \cite{sdcOMP}.
\end{remark}


\begin{remark}
The direct inductive proof given in section
\ref{DirectProofSection} which generalizes
\cite{HaslerNeirynck,Fosseprez}'s has the advantage
of revealing circuit theoretic concepts that occur.
One step uses Theorem \ref{NoCCMinors} to prove that 
the no-common-covector hypothesis is preserved 
by replacing one of the nonlinear elements by 
a source.
\end{remark}



\begin{theorem}
\label{PairConclusionTheorem}
With constitutive laws given by monotone increasing functions from 
$\Reals$ onto
$\Reals$, every subspace pair problem can be posed as a case of Theorem
\ref{sandwillompairtheorem} (see Theorem \ref{SubspacePairTh}), 
and every case of Theorem
\ref{sandwillompairtheorem} can be posed as a subspace pair problem
(with $\hmat{A}{B}=M_V^0$, $\hmat{I}{-I}=M_I^0$.)
\end{theorem}



\section{Survey of Formulations}




We survey various formulations and results about them
to seek relations and generalizations.  An assortment of
mathematical tools have been used to prove existence 
theorems:
\begin{itemize}
\item Fill in here:  Minty.
\item Global implicit function theorems and approximation of 
continuous non-decreasing functions by differential
strictly increasing function: DeSoer and Wu\cite{DesoerWu}, Sandberg and
Willson\cite{SWExistanceSIAM}
\item Induction: Hasler and Neirynck \cite{HaslerNeirynck}, Sandberg and
Willson\cite[???]{SWExistancePf}, our work.
\item Convex complementarity and optiomization:  
Rockafellar\cite{RockafellarConvProg}.
\end{itemize}

\extra{
\subsection{Pairs with a distinguished complementary base pair,
``Laplacians'', Elastic and Electrical problems in terms
of nodes.}



We demonstrate here how the subspace pair model applies to 
the reduced nodal admittance matrix formulation of the resistive
electrical network analysis problem.  

The Laplacian of the connected graph with nodes $\{0,1,\ldots,n\}$, edges
$E$ and edge weight $g_e$ on each $e\in E$ is the 
$n \times n$ matrix of coefficients from
the following equations with variables $(v_i)_{i=1,\ldots,n}$ and constants
$v_0=0$ and $(y_i)_{i=1,\ldots,n}$:
\begin{equation}
\label{KCLNodeEq}
\sum_{j:e=\{i,j\}\in E} g_e(v_i-v_j)=y_i\mathrm{\ \ for\ \ }i=1,\ldots,n.
\end{equation}

Recall that 
the Laplacian matrix can be factored as $MGM^T$ where $M$ is the 
incidence matrix with row $0$ deleted (\textit{reduced})
and $G$ is the diagonal
matrix with entries $(g_e)_{e\in E}$.  
In electrical network theory, the Laplacian is called the 
\textit{nodal admittance matrix} when the $v_i$ denote node voltages
relative to the ground node $0$ (hence $v_0=0$), 
the $g_e$ denote edge conductance and each
$y_i$ denotes current through an external path flowing from node
$0$ into node $i$.  
KVL implies that the voltage across edge $e=ij$ is $v_i-v_j$ for some
node potential $v:\{0,\ldots,n\}\rightarrow\Reals$.  Ohm's law means the
current through edge $e$ from $i$ to $j$ is $g_e(v_i-v_j)$.  Equation
(\ref{KCLNodeEq}) expresses KCL at node $i$.
%It is well-known that 
The Laplacian can be shown
to be non-singular when each $g_e>0$ by means of the 
Matrix Tree Theorem\footnote{The determinant of the Laplacian equals
$\sum g_T$ where $g_T$ is the product of the conductances $g_{ij}$ of
the edges in spanning forest $T$.}.
The inverse of the Laplacian is called the 
\textit{nodal resistance matrix} $R$. 
Given externally imposed currents $y$ into the nodes $1,\ldots,n$
(so their sum flows out of node $0$),  $v=Ry$ 
gives the voltages at nodes $1,\ldots,n$ relative to node $0$.

Voltage sources (batteries) are more common in elementary electrical
problems.  
Suppose one or more voltages sources are connected between node $0$ and other
nodes which establish some \textit{voltage input} nodes to be have particular
electrical potential relative to one another.
Quotients of minors of the Laplacian can also be used
to solve for the other node voltages.
Note that for this problem, the external 
current into the voltage input nodes is
unspecified and the external current into the other nodes is zero.


To model an electrical network interacting with its environment 
by means of currents and voltages at its nodes in both these
situations, and to draw analogies to elastic frameworks, 
we take for the matrix $M_V$ 
the graph's incidence matrix (not reduced, so it has $n$ rows but
its rank is $n-\mbox{\#connected components}$)
with 
the columns of an identity matrix appended.  We write 
$U=E\dunion N$; $N$ is the set of $n$ nodes.
($M_V$ is the reduced incidence matrix of the
graph constructed by appending a new node plus a new edge directed from
this node to each original node.)
Writing 
$M_V=\hmat{M}{I}$, we can take $M_I=\hmat{I}{-M^T}$ so
$L(M_V)=L(M_I)^{\perp}$ and 
both 
$M_I$ and $M_V$ have full row rank.  Thus the Laplacian or nodal admittance
matrix is $M_I^TGM_V$.


\subsubsection{Elastic Analog of the Indefinite Nodal Admittance Matrix}

The analogous matrices for a framework are constructed with $M$ being the
$dn\times |E|$ rigidity matrix,
$M_V=\hmat{M}{I}$ and $M_I=\hmat{I}{-M^T}$.  For brevity, we let
$N^d$ denote the row index set of $M_V$ and so $U=E\dunion N^d$.  For the
electrical network, $d=1$.  



Here are some observations about some vectors
and covectors supported by subsets of $E$ and $N^d$.

\begin{itemize}
\item Suppose $x\in L(M_V)$ and 
$\supp{X}\subseteq N^d$ where covector $X=\sigma(x)\in\mathcal{L}_V$.
Electrically, $x$ is a node potential that is 
constant on connected components of the network graph.  Mechanically,
$x=\mathbf{v}$ is a combination of node velocities for which
$0=u_e=(\mathbf{v}(i)-\mathbf{v}(j))\cdot(\mathbf{p}(i)-\mathbf{p}(j))$
for each $e=(i,j)\in E$.

%\item 
By duality, such $x\in V(M_I)=L(M_V)^\perp$, so $X\in\mathcal{V}_I$ is a 
vector.  Electrically, $x$ is a linear dependence among external 
node currents.  When the network is connected, every such $x=c(1,\ldots,1)$
for $c\in\Reals$.  When $c\neq 0$ this 
expresses the result of KCL that the sum of node currents is $0$.  
Mechanically, $x$ is a linear dependence among externally supplied
forces on the nodes necessary for equilibrium.

\item Suppose $y\in L(M_I)$ and
$\supp{Y}\subseteq E$ where covector $Y=\sigma(y)\in\mathcal{L}_I$.
Electrically, $y$ is a current flow in the 
edges that satisfies KCL with zero external current.  Mechanically,
$y$ is called a self-stress.

%\item 
By duality, such $y\in V(M_V)=L(M_V)^\perp$, so
$Y\in\mathcal{V}_V$ is a vector.  Electrically, $y$ is a linear dependence
among edge voltages that is a case of KVL.
Mechanically, it is a dependence among first order relative 
changes of bar lengths ($(\Delta l)/l$)
necessary to maintain geometric consistency of the framework embedding.

\end{itemize}

We can now explain how to modify the subspace pair $(L_V,L_I)$ to 
exclude from $\mathcal{L}_V$ covectors for electrical potential offsets that
are constant on connected components and mechanical node velocities that
result in zero first order length changes in the bars.  Choose a basis for
$E$ in the matroid $\mathcal{M}_V$ and independent set $T\subset N^d$ that 
extends the basis for $E$ to a basis for $S$.  The modified 
$(L_V',L_I')$ $=$ $(L_V,L_I)/T$ $=$ $(L_V/T,L_I\setminus T)$.  Let
$N^{d'}$ denote $N^d\setminus T$.  We observe $L_V'$ and $L_I'$ are orthogonal
(so $\mathcal{M}(L_V')=\mathcal{M}_V'$ and 
$\mathcal{M}(L_I')=\mathcal{M}_I'$ are duals)
and $N^d\setminus T$ is both independent and coindependent in both
$\mathcal{M}_V'$ and $\mathcal{M}_I'$.





Let us now introduce $dn$ ports at the elements of $N^d$.
}



\subsection{DeSoer and Wu's Formulation}

Desoer and Wu presented results\cite{DesoerWu} 
on existance and uniqueness of solutions of
electrical network problems with monotone non-decreasing but not necessarilly
onto nonlinear resistance or conductance functions for the constitutive
laws of two-terminal resistors.  Their formulation
led to a function $G:\Reals^m\times\Reals^r\rightarrow\Reals^m$.  A solution
$x\in\Reals^m$ for which $G(x,u)=y$ determines all the currents and
voltages from Kirchhoff's laws.  
$u\in\Reals^r$ and $y\in\Reals^m$ are parameters.  A global implicit function
theorem was used to prove a solution $x$ exists for all $u,y$ from 
certain graph theoretic conditions on the resistors
classified according to the directions ($x\rightarrow +\infty$ or
$x\rightarrow -\infty$) in which the domain and range
of the resistors' constitutive relations are bounded.
Ports or outputs were not used explicitly.  We will describe Desoer and
Wu's formulation.  It will illustrate how the equations of the 
formulation abstract to series and parallel extensions followed by 
port insertion.

The starting point is the matrices representing an orthogonally complementary
pair of subspaces, illustrated below:
\begin{center}
\input{DWdualmatrices.pstex_t}
\end{center}
Graph theoretically, the distinguished base $E_t$ in 
$\mathcal{M}_V$ is a spanning tree\footnote{Assuming the graph is connected.}.

Each resistor is assumed to have a constitutive relation of the form either
$v=r(i)$ or $i=g(v)$ (or both) where the functions are monotone non-decreasing.
A resistor characterized by
$v=r(i)$ or $i=g(v)$ is called \textit{current controlled} (CC) or
\textit{voltage controlled} (VC) respectively.  The spanning tree is chosen so
$E_t$ contains a maximum sized set of VC resistor edges $E_{vt}$.  A matroid
theoretic consequence is the other VC resistors $E_{vl}$ are spanned in 
$\mathcal{M}_V$ by $E_{vt}$.  So $F$ is further partitioned:

Desoer and Wu use variables for each voltage across resistors in $E_t$ and
currents through $E_l$, and write a real linear equation for each row in this
tableau.   The equations and variables corresponding to $E_{ct}\dunion E_{vl}$
are then eliminated.  The result is the equation $G(x,u)=y$ in variables
$x$.  The describe the parameters in terms of our subspace pair model, we 
do the following operations which result in a tableau description in figure
():
\begin{enumerate}
\item Parallel extend $E_t$ and series extend $E_l$ in 
$(\mathcal{M}_V,\mathcal{M}_I)$ to $P_t$ and $P_l$ respectively.  The result is 
illustrated in figure \ref{DEdualmatrices2}.
\item Introduce ports at $P_t\dunion P_l$.  The result is shown in figure 
\ref{DEsubspacepair}.
\end{enumerate}

\begin{figure}
\label{DEdualmatrices2}
\input{DEdualmatrices2.pstex_t}
\end{figure}

\begin{figure}
\label{DEsubspacepair}
\input{DEsubspacepair.pstex_t}
\end{figure}


By lemma (), operations (1) preserve the complementary base and 
orthogonallity properties of the original subspace pair.  By construction,
$V-zeroing$ of $P_l$ and $I-zeroing$ of $P_t$ produces the original subspace
pair.

Observation:  $P=P_t\dunion P_l$ is a coindependent base in both
${\mathcal{M}_V}'$ and ${\mathcal{M}_I}'$.  ($P$ is in fact a cobase in each).
Note the similarity to the Laplace section.



\extra{
\begin{figure}[htb]

\begin{minipage}[c]{.48\linewidth}
  \centering
 \centerline{\input{2res.pstex_t}}
%  \vspace{2.0cm}
\end{minipage}
%
\hfill
\begin{minipage}[c]{.48\linewidth}
\[
\begin{array}{cccc}
\multicolumn{4}{c}{\mbox{($M_V$ matrix)}} \\
0   &  1  &  0  &  0  \\
1   &  0  &  1  &  1  \\ \hline \hline
p_V & p_I & e_1 & e_2 \\ \hline \hline
1   &  0  &  0  &  0  \\
0   &  1  & -1  &  0  \\
0   &  0  &  1  &  -1 \\
\multicolumn{4}{c}{\mbox{($M_I$ matrix)}} \\
\end{array}
\]
\end{minipage}
%
\begin{minipage}[b]{.48\linewidth}
\[
\begin{array}{ccc}
  1  &  0  &  0  \\ \hline
 p_I & e_1 & e_2 \\ \hline
  0  &  0  &  0  \\
  1  & -1  &  0  \\
  0  &  1  &  -1
\end{array}
\]
ZIR analysis for voltage source input:
$p_V$ contracted.  Zero response (unique solution) even if
negative resistances $\neq 0$ are allowed.
\end{minipage}
\hfill
\begin{minipage}[b]{.48\linewidth}
\[
\begin{array}{ccc}
0   &  0  &  0  \\
1   &  1  &  1  \\ \hline
p_V & e_1 & e_2 \\ \hline
1   &  0  &  0  \\
0   &  1  &  -1
\end{array}
\]
ZIR analysis for current source input:
$p_I$ deleted.
Zero response (unique solution) \textit{unless}
$g_1 = -g_2$.  
\end{minipage}
\caption{Simple example that illustrates the solution is unique when the 
port is V-driven provided each $g_e\neq 0$, but the I-driven system has a
unique solution provided each $g_e>0$ because the resulting supplemental
oriented matroid pair has complementary bases and no common covector.}
\label{Simple}
%
\end{figure}
}


\subsection{Rockefellar's Formulation}

\subsection{Sandberg and Willson}

\subsection{Bott-Duffin Constrained Inverse}



\section{Direct Existence Proof}
\label{DirectProofSection}
We develop a direct proof of Theorem \ref{SubspacePairTh} by generalizing the 
work of \cite{HaslerNeirynck,Fosseprez}.  
In this prior work, what we see now as oriented matroid pairs
were pairs of graphic and cographic oriented matroids that were
obtained from an original electrical network graph by 
nullator, norator and V-driven and I-driven port insertions
along original edges.  A monotonicity condition on the behavior of the
system ``looking into'' a port that was obtained by replacing a resistor
was used.  In our more general context, the original subspace pair
problem might be given in which the direction of monotonicty at an original
port might be unpredicatable from the supplemental oriented matroid pair.
Indeed, if a non-port element is reversed (reoriened) in one oriented matroid
but not the other and then transformed to a port, the common covector property
of the supplemental oriented matroid pair will be unchanged, but the direction
of any monotonicity at the inserted port will be reversed.

We will therefore need to give an induction proof
which distinguishes between the port elements that 
were given in the original subspace pair model and the port
elements that arose by port insertions at elements that remained in the
original supplemental subspace pair.   The original supplemental subspace
pair is assumed by hypothesis of Theorem 
\ref{SubspacePairTh} to have no common covector.
We will see that this hypothesis is preserved by new port insertions.

We begin by stating the theorem 
in a stronger form that supports the inductive proof.

Recall that a subspace pair model with sources $|S|$ satisfies
$|S|=|P|$ and for each pair $\{p_V, p_I\}\subset P$, exactly one of
$p_V$ or $p_I$ is in $S$.  For a ``V-driven''  port $p$, 
$p_V\in S$, $p_I\not\in S$, and $u_{\mathit{pV}}$ is called the ``input
variable''.  Let $u_{\mathit{pI}}$ be called the 
\textit{corresponding output variable}.  For an ``I-driven'' port,
$u_{\mathit{pV}}$ is called the 
corresponding output variable.



\begin{theorem}
\label{SubspacePairThInd}
Let $\mathbf{M}$ be a subspace pair model with sources where 
that satisfies the following properties:

\textbf{(1).} There are bases $B_V\in\mathcal{B}(L_V)$,
$B_I\in\mathcal{B}(L_I)$ for which all V-driven 
ports $p$ satisfy $p_V\in B_V$ (note $p_I$ must be 
in $B_V$ since it's an isthmus in $\mathcal{M}_V$) 
and $p_I\not\in B_I$,
and all I-driven ports $p$ satisfy $p_I\in B_I$ and 
$p_V \not\in B_V$.  \\
\textbf{(2).}
$B_V\cup B_I = U$, for the bases in \textbf{(1)}.\\
\textbf{(3).} The oriented matroid pair of supplemental pair $(L_V^0,L_I^0)$ 
have no common (nonzero) covector.
(i.e., $\mathcal{L}(L_V^0)\cap\mathcal{L}(L_I^0)=\{0\}$.)\\
\textbf{(4).} 
All constitutive laws are increasing monotone onto functions 
$\Reals\rightarrow\Reals$.


Let $\Pnew\subset E$, $\Pnew_V=\Pnew\cap B_V$, 
$\Pnew_I=\Pnew\cap B_I$, and $\Mnew$ be the subspace pair problem with
sources $S\cup\Pnew$ obtained by inserting into $\mathbf{M}$ the 
new I-driven ports $\Pnew_I$ and V-driven ports $\Pnew_V$.

Suppose that $|E\setminus\Pnew|\leq n$.

Then, for every input assignment to $\Mnew$
there is a unique solution that depends 
continuously on the input assignment.  Furthermore, whenever one 
\textit{new} input variable's 
assigned value $u$ increases with the other
inputs constant,
the corresponding output variable's value $u'$ does not increase.

\end{theorem}

We get Theorem \ref{SubspacePairTh}'s existence conclusion by taking
$\Pnew=\emptyset$ above.


\subsection{Some Lemmas}

\begin{lemma}
\label{B0lemma}
Under the assumptions and terminology of Theorem \ref{SubspacePairThInd},
let $B_V^0=B_V\cap E$ and $B_I^0=B_I\cap E$.  Then 
$B_V^0\cap B_I^0=\emptyset$ and $B_V^0\cup B_I^0=E$.  (Denoted
$E=B_V^0\dunion B_I^0$ for brevity.)
\end{lemma}

\begin{lemma}
\label{PortProplemma}
Under the assumptions and terminology of Theorem \ref{SubspacePairThInd},
$\{p_V|p \mbox{\ is V-driven}\}$
is an independent set in the matroid $\mathcal{M}(L_V)$,
$\{p_I|p \mbox{\ is I-driven}\}$
is an independent set in the  $\mathcal{M}(L_I)$,
$\{p_I|p \mbox{\ is V-driven}\}$
is coindependent in $\mathcal{M}(L_I)$, and
$\{p_V|p \mbox{\ is I-driven}\}$
is coindependent in $\mathcal{M}(L_V)$.
\end{lemma}

\begin{lemma}
\label{PortInsertlemma}
Suppose $e\in B_V\cap E$. Let $\mathbf{M}'$ be obtained by inserting
$e$ as a V-driven port in $\mathbf{M}$.  Then the supplemental
subspace pair  $(L_V'^0,L_I'^0)$ of $\mathbf{M}'$ is 
$(L_V^0,L_I^0)/e$.  Similarly, if $e\in B_I\cap E$ and $e$ is inserted
as an I-driven port, then $(L_V'^0,L_I'^0)=(L_V^0,L_I^0)\setminus e$.  
\end{lemma}

\begin{lemma}
\label{TopoIndlemma}
Suppose $(M_V^0,M_I^0)$ have complementary base pair 
$B_V^0 \cup B_I^0=E$, $B_V^0 \cap B_I^0=\emptyset$ and no
common covector.  Let $e\in E$.
If $e\in B_V^0$ then $(M_V^0,M_I^0)/e$
also has a complementary base pair 
($(B_V^0\setminus e) \cup B_I^0$) and no common covector.
Dually, if $e\in B_I^0$ 
then $(M_V^0,M_I^0)\setminus e$ has complementary base pair
and no common covector.
\end{lemma}

This lemma expresses the combinatorial argument on which the 
induction is based.  It is an application of Theorem \ref{NoCCMinors}.

\begin{proof}

\textbf{Proof of Theorem 
\ref{SubspacePairThInd}, basis}

$n = 0$, so all elements of $\Mnew$ are ports.
Lemma \ref{PortProplemma} shows the input values can be assigned 
arbitrarilly.  Each output
value is in fact independent of the corresponding input; but it depends on 
other inputs or could be identically $0$.  
Lemma \ref{PortProplemma}
also shows the output values are unique.
Notice this argument holds for
any number of port elements.  Thus, each new output value is a non-increasing
function of the corresponding input because all such functions are constant.

\textbf{Proof of Theorem 
\ref{SubspacePairThInd}, induction}

Assume Theorem \ref{SubspacePairThInd}
for given $n$.  Let $\textbf{M}$ be a subspace pair
problem that satisfies the hypotheses, and $\Pnew\subset E$ be given
as specified with $|E\setminus\Pnew|\leq n+1$.

Suppose $e\in B_V\setminus\Pnew$ 
(the dual case of $e\in B_I$ is similar).  
(In the terminology of \cite{HaslerNeirynck,Fosseprez}, ``resistor 
$e$ is current-controlled.'')  
Let $\Mnew'$ be obtained by inserting a new V-driven port at 
$e$ together with new ports $\Pnew$.  

By induction, for each choice of 
$u_{\mathit{Ve}}$ and other input values $U$, there is a unique solution
to the subspace pair problem $\Mnew'$
that depends continuously on $u_{\mathit{Ve}}$ and other input values $U$
and the function
for the new output value 
$u_{\mathit{Ie}}=\phi(u_{\mathit{Ve}},U)$ 
is non-increasing in $u_{\mathit{Ve}}$ for fixed $U$.  

The constitutive law $\Gamma_e$ is a monotone increasing function
$u_{\mathit{Ve}}=r_e(u_{\mathit{Ie}})$.  Hence 
$f(u_{\mathit{Ie}})=\phi(r_e(u_{\mathit{Ie}}),U) - u_{\mathit{Ie}}$ is a
continuous decreasing function of $u_{\mathit{Ie}}$ with $f(-\infty)=+\infty$
and $f(+\infty)=-\infty$.  Therefore $f(u)=0$ has a solution.
Thus, the unique solution to $\Mnew'$ with new input value $u_{\mathit{Ie}}=u$
and other input values $U$ is a solution to $\Mnew$ with input values $U$.

It remains to show that for each choice of 
\textit{other} 
input values (new and old, for elements $\neq e$)
of an $\Mnew$ in Theorem \ref{SubspacePairThInd}
(which might have $|E\setminus \Pnew|=n+1$), the solution
is unique, it depends continuously on each input value, and each new output
value depends non-increasingly on its corresponding input variable when the 
other inputs are constant.

The uniqueness is proven from 
Lemma \ref{PortInsertlemma} followed by Lemma
\ref{TopoIndlemma} and then Proposition \ref{Uniquenessprop}.

Continuity follows from the fact that $\phi$ is continuous...

Finally, suppose $p\in\Pnew$, $u_i$ is the corresponding input variable
and $u_o$ is the corresponding output variable, and for particular values
of the other inputs, there are $u_i$, $u_o$ values satisfying
$(u_i-u_i')(u_o-u_o')>0$.  
We can construct a supplemental pair $(L_V^p,L_I^p)$ by zeroing
all the sources except for $p$.
An argument based on the signs of the solutions
with giving $u_i$, $u_i'$, $u_o$, and $u_o'$ similar to that of 
Proposition \ref{Uniquenessprop} demonstrates the existence of a 
common covector of $(L_V^p,L_I^p)$.
On the other hand, Lemma 
\ref{PortInsertlemma} and Lemma 
\ref{TopoIndlemma} combined with hypotheses \textbf{(3)} and \textbf{(4)}
demonstrate $(L_V^p,L_I^p)$ have no common covector.  Hence
$(u_i-u_i')(u_o-u_o')>0$ is impossible.
\end{proof}



\begin{fuzz}
\section{Other Stuff (Orphaned)}

\extra{\textbf{???}
Hence we do not assume any rank or orthogonality conditions on subspace 
pairs in the definitions below.}



%
Theorem \ref{Feasiblity1} applies to the subspace pair
obtained from all declarations of nullators, norators, opens and shorts.


There is a subtle difference between declaring a V-source with 0
input value and contracting the same element.  If a set $S$ 
of $k$ such elements
is not independent in $\mathcal{M}(L_V)$, then the rank of $\mathcal{M}(L_V/S)$
will be more than
$\mbox{rank}(\mathcal{M}(L_V))-k$ but the given combination of input values
will still be feasible.  If they are not coindependent in
$\mathcal{M}(L_I)$ (which will certainly be true when there are no nullors).
then the rank of $\mathcal{M}(L_I\setminus S)$ will be less than 
$\mbox{rank}(\mathcal{M}(L_I))$ but there will be a non-zero combination of
output only variables.  (Physically, that corresponds to non-zero current 
circulating in a loop of ideal wires; or a non-zero self-stress in an
overbraced subframework of rigid bars.  The dual physical situation
is that is possible for the disconnected parts of an electrical network
to differ in electrical potential when a cut-set of branches are removed;
mechanically, more flexes of the framework can exist when some edges are
removed. For this reason, we are careful to distinguish deletion/contraction
from port insertion.)


It's yet to be done to handle simutaneous application of force to more 
than 2 vertices....


\subsubsection{Topological Formulas}

They can come out of the subspace pair formulation three ways:
\begin{itemize}
\item  Besides the ``$g$'' or ``$r$'' variables, do not insert ports but
do use an extra variable ``$x$'' to relate the voltage to current of one port.
Then the equation $\det(M_VG;M_I)=0$ has the form $Ax+B=0$, so $x=-B/A$.
\item Use the Pl\"{u}cker coordinate formulation of the intersection subspace to 
identify a minor of a hybrid or other description matrix of as the determinant
of the solution submatrix for a system of equations; then use the Cramer's
rule generalization to find the ratios of minors of the equation 
matrix to analyzed.  
This was done for my ISCAS 98 paper.
\item Use Rota's Grassmann-Cayley algebra meet formula to extract expansion
directly from subspace pair matrices, together with the previous way to
identify Pl\"{u}cker coordinate ratios with description matrix minors.
\end{itemize}

\subsubsection{Series-Parallel}
For some $(L_V,L_I)$, the common covectors are unique up to 
multiplication by $-1$.  Such subspace pairs are from  
electrical networks in which the direction of each current flow 
and voltage drop is predictable from the network structure alone:
These are networks whose graph is \textit{series-parallel}.

\textbf{???coordinate w. Theorem (\ref{Feasiblity1})}
A subspace pair model with sources $S$
is called \textit{structurally feasible}
if $S\cap P_V$ is an independent set in $\mathcal{M}_V$ and
$S\cap P_I$ is an independent set in $\mathcal{M}_I$.


We say a subspace pair problem with sources $S$ 
is \textit{well-posed} when for all input assignments there is a unique 
solution.

\textbf{re zero-ing a port:
I'M NOT SURE THIS IS RIGHT...WHAT IF say there is a response
to $x_{p_V}=0$ in which the only non-zero value is  $x_{p_I}$ ???
I think the topological conditions (no cocircuits on port response
elements) exclude this, but the point must be made clear.}

Reduced nodal admittance matrix.  Nodal resistance matrix.
Interaction with a physical framework with it's environment.
A framework is first order rigid iff it ``resolves all applications of
static equilibrium forces''.  However, every physical bar has some
elasticity:  An ideal rigid bar is analogous to an ideal voltage 
source.  Hence, given an elastic framework, for every application
of static equilibrium forces on the vertices, the vertex positions
will change as the bars stretch or shrink under the forces they now
carry to resolve the applied force.  These first order vertex position 
changes are given by $Z\mathbf{f}$.

The environment might interact by ``forcing'' some vertices to change position
relative to one another.
Intuitively, the framework will ``push back''.  The other vertices are free
to move as adjacent vertices move and incident bars change length in
response to the forces developed in them to resolve the forces required
to hold the framework in its new position.  The position changes of the
free vertices $V$ can be calculated by solving for the unknown position changes
in the system of equations $(Y\mathbf{v}_V)(V)=0$.

For our purposes, we insert port elements in order to make interactions 
with the environment explicit.  This enables a coordinate of an 
environmental interaction quantity to correspond to an oriented matroid
element, so that its sign can be read off from the corresponding entry
in a covector.




(3) OMP from ``leaky electricity'' in digraph.  No common covector property
proven graph theoretically.  Citations of my ISCAS-95 paper and Tutte's
triangulation of a triangle aapplication.

(5) Port insertion on a initially dual pair.

(7) Finish researching applicability to DeSoer and Wu's formulation
of monotone non-linear systems.

(8) Application to decomposing a rectangle into rectangles with
non-linear aspect ratio functions..




\extra{
\section{Old Examples}
\subsection{Using an oriented matroid to reason about inequalities}

We illustrate an application 
of Definition \ref{OMDEF} together with the linear algebra
operation of (1) combining members in $L$ and (2) defining new a coordinate
to be a specific linear combination of existing coordinates to derive
inequalities on parameters necessary for known conditions on a
covector set.  Willson (\cite{WillsonNoGain}, Theorem 5)
gave the following conditions
for a (common ground node) 2-port
to be a no-gain element when it is described by continuous functions 
$i_1=F_1(v_1,v_2)$ and $i_2=F_2(v_1,v_2)$:
\[
v_1F_1(v_1,v_2) > 0 \mathrm{\ for\ } \frac{v_2}{v_1} < 1 
\]
\[
v_2F_2(v_1,v_2) > 0 \mathrm{\ for\ } \frac{v_1}{v_2} < 1 
\]
\[
(v_1+v_2)(F_1(v_1,v_2)+F_2(v_1,v_2)) > 0 \mathrm{\ for\ } v_1 v_2 > 0
\]

Notice these conditions can be expressed as restrictions on the 
combinations of the signs 
$\sigma(v_1),\sigma(v_2),\sigma(i_1),\sigma(i_2),\sigma(v_1-v_2)$
and $\sigma(i_1+i_2)$, since the third condition is relevant when
$\sigma(v_1)=\sigma(v_2)\neq 0$ and in that case, 
$\sigma(v_1+v_2)=\sigma(v_1)$.  We will derive conditions on the $g_{ij}$
of a linear admittance matrix description when the $F_i$ are linear.  First,
we write a matrix whose rows span the space of 
$(v_1,v_2,g_{11}v_1+g_{12}v_2,g_{21}v_1+g_{22}v_2)$:
\[
\begin{array}{cccc} 
v_1 & v_2 & i_1 & i_2 \\ \hline
1   &  0  &  g_{11} & g_{21} \\
0   &  1  &  g_{12} & g_{22} 
\end{array}
\]
We append two more columns, for 
$(v_1-v_2)$ and $(i_1+i_2)$:
\[
\begin{array}{cccccc} 
v_1 & v_2 & v_1-v_2 & i_1 & i_2 & i_1+i_2\\ \hline
1   &  0  & 1  & g_{11} & g_{21} &g_{11} + g_{21} \\
0   &  1  & -1 & g_{12} & g_{22} &  g_{12} + g_{22} 
\end{array}
\]

Suppose $v_1<<0$, which we abbreviate ``$v_1--$'' and $v_2$ is ``small'' and
positive, abbreviated $v_2+$.  In that case, the second condition applies.
However relevant covector is \\
$(-,+,*,*,\sigma(-g_{21})\circ\sigma(g_{22}),*)$
$=$ $(-,+,*,*,+,*)$
(where ``$*$'' denotes ``we don't care''), so 
$\sigma(-g_{21})\circ\sigma(g_{22})=+$, and we conclude $g_{21}\leq 0$ 
(and if $g_{21}=0$ then $g_{22}>0$).  We conclude $g_{12}\leq 0$ from the first 
condition and covector $(+,-,*,+,*)$ in similar fashion.



On the other hand, if $v_1++$ and $v_2+$, we get the composition
of the two covectors (from the first two rows of the matrix)
$(\sigma(1), 0 ,\sigma(1), \sigma(g_{11}),\sigma(g_{21}),\sigma(g_{11} + g_{21}))$
$\circ$\\
$(0, \sigma(1) ,\sigma(-1), \sigma(g_{12}),\sigma(g_{22}),\sigma(g_{12} + g_{22}))$
which is\\
$(+, + ,+, \sigma(g_{11})\circ\sigma(g_{12}),\ldots)$.  Note the first condition 
applies since $\sigma(v_1-v_2)=+\circ-=+$.  Therefore, 
$\sigma(i_1)= \sigma(g_{11})\circ\sigma(g_{12})$ must be $+$.  We conclude that
$g_{11}>0$ because we have already concluded $g_{12}\leq 0$.  In the same way, 
we conclude $g_{22}>0$ from the composition of the same covectors in the 
opposite order and the second condition.

Let us analyze the sign of $i_1+i_2$ in the same two compositions:
$(+,+,+,\ldots,\sigma(g_{11}+g_{21})\circ\sigma(g_{22}+g_{12}))$
and 
$(+,+,-,\ldots,\sigma(g_{22}+g_{12})\circ\sigma(g_{11}+g_{21}))$.
The third condition tells us that the last sign in both covectors must
be $+$.  We conclude $g_{11}+g_{21}\geq 0$ and $g_{22}+g_{12}\geq 0$ and
$g_{11}+g_{21}+ g_{22}+g_{12}> 0$.


To investigate the situations when $v_1$ and $v_2$ are close, we do exact 
calculation of the matrix rows to extend the matrix to:

\[
\begin{array}{cccccc} 
v_1 & v_2 & v_1-v_2 & i_1 & i_2 & i_1+i_2\\ \hline
1   &  0  & 1  & g_{11} & g_{21} &g_{11} + g_{21} \\
0   &  1  & -1 & g_{12} & g_{22} &  g_{12} + g_{22} \\
1   &  1  & 0 & g_{11}+g_{12} & g_{21}+g_{22} &  g_{11}+g_{21}+g_{12} + g_{22} 
\end{array}
\]

The covector obtained by composing the covector from the third 
row with the first produces a covector to which the first condition applies.
We conclude $g_{11}+g_{12}\geq 0$.  Similarly, the second condition
implies $g_{22}+g_{21}\geq 0$.

}

\subsection{A feedback structure case of 
Trajkovi\`{c} and Willson \cite{TrajWillNDR}}

%\begin{comment}
We illustrate the oriented matroid approach by reproducing the result of
\cite{TrajWillNDR} that a particular configuration of a ``feedback structure''
with two Ebers-Moll transistors and one port
cannot exhibit negative differential resistance by itself, and it can 
exhibit NDR if one resistor is added and $\alpha_1+\alpha_2-1>0$.  
Under KVL,
the voltages across the port, resistor
and two 
Ebers-Moll diodes are given by the row space member of $M_V$ when the
three rows are multiplied by the 3 independent voltages
$V_1$, $V_2$ and the port voltage $V$.  The space of current values
feasible in the same 4 elements under KCL and the two 
Ebers-Moll linear CCCS's
is only one dimensional; it is spanned by the one
row of $M_I$.  Beginning with the signatures of the rows of the matrices,
we can apply the covector axioms to explore what common covectors are possible
under several variations.
%\end{comment}





%$M_V$ is written above; $M_I$ below.  
We used the rules of 
Definition \ref{OMDEF} to figure out common covectors in 
$\mathcal{L}(M_V)$, $\mathcal{L}(M_I)$ with 
their first two 
signs,
of port elements pV, pI given,
for the 
tangents and differences drawn boldly on the 
$I_{\mbox{p}}$/$V_{\mbox{p}}$ curve.  
(The algorithm ideas used appear in \cite{sdcOMP}.)
In three cases, (unique)
extensions of the given signs exist when $1-\Sigma\alpha\le 0$ but there is
only one such case otherwise, verifying a condition for NDR from 
\cite{TrajWillNDR}.
Multiple operating points with an I-driven port are impossible.  
The magnitude inequalities come from ``$\epsilon<<1$'' in explaining
``$\circ$'' above Definition \ref{OMDEF}.

\noindent 
Fig. 1: Example.  
$M_V$/$M_I$ are the upper/lower right arrays.\hspace*{-20pt}
\[
\begin{array}{c|ccccc}
\mbox{(row mults.)}u_{\mbox{pI}}
                           & 0 & 1 & 0 & 0 & 0 \\
u_{\mbox{pV}}              & 1 & 0 & 0 & 0 & +1 \\
u_{\mbox{VR}}              & 0 & 0 & 1 & 0 & +1 \\
u_{\mbox{VD1}}             & 0 & 0 & 0 & 1 &-1 \\ \hline
\Gamma\mbox{\ coeffs.}\rightarrow
                           & 1 & 1 & g_{\mbox{R}}
                                       & g_{\mbox{D1}}
                                           & g_{\mbox{D1}} \\ \hline
\mbox{elements}\rightarrow & \mbox{pV}
                               & \mbox{pI}
                                   & \mbox{R} 
                                       & \mbox{D1}
                                           & \mbox{D2} \\ \hline
u_{\mbox{pI}}=-I_{\mbox{p}}& 0 & 1 
                                   & 1-\alpha_1-\alpha_2
                                       & \alpha_1-1
                                           &\alpha_2-1  \\
u_{\mbox{pV}}              & 1 & 0 & 0 & 0 & 0
\end{array}
\]

\noindent
\input{NDR.pstex_t}\vspace*{0.5em}

\noindent
\input{NDRgraph.pstex_t}

\end{fuzz}


% References should be produced using the bibtex program from suitable
% BiBTeX files (here: strings, refs, manuals). The IEEEbib.bst bibliography
% style file from IEEE produces unsorted bibliography list.
% -------------------------------------------------------------------------
%\bibliographystyle{IEEEbib}
\bibliographystyle{amsalpha}
%%\bibliography{strings,refs,manuals}
\bibliography{OMPEMech}
%%%\setlength{\itemsep}{-2pt} %Put me after \begin{thebibliography}..
\begin{comment} %comment out our hand hacked bibliography
\begin{thebibliography}{10}

\setlength{\itemsep}{-2pt}
%%%%%%%insert .bbl file here
\end{thebibliography}
\end{comment}


\end{document}


