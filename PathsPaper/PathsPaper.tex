% Sample article for the Electronic Journal of Combinatorics.
%
% EJC papers *must* begin with the following two lines. 
\documentclass[12pt]{article}
\usepackage{generic-e-jc}

% GEOMETRY
% Please remove all commands that change parameters such as
%    margins or page sizes.  The style file sets them

% PACKAGES
% Packages amssymb, amsthm and hyperref are already loaded. 
% We recommend also these packages for mathematics and images:
\usepackage{amsmath,graphicx}

\usepackage{tkz-graph}
\DeclareMathOperator{\sign}{sign}

% Override stuff in style file.
\specs{??}{??}{??}{??}


% THEOREM ENVIRONMENTS
% Theorem-like environments that are declared in the style file are:
% theorem, lemma, corollary, proposition, fact, observation, claim,
% definition, example, conjecture, open, problem, question, remark, note

% CHARACTER CODES
% Please do not use non-ascii characters in this file, but instead use
% the LaTex macros for characters with diacritical marks, such as
% G\"{o}del, R\'{e}nyi, Erd\H{o}s.  Don't use the package "babel".
% Note that this is the opposite of the rule for the article metadata
% that you enter on the web page; sorry for the confusion!

% DATES
% Give the submission and acceptance dates in the format shown.
% The editors will insert the publication date in the third argument.
\dateline{July ??, 2022}{TBD}{TBD}

% SUBJECT CODES
% Give one or more subject codes separated by commas.
% Codes are available from http://www.ams.org/mathscinet/freeTools.html
\MSC{05C88, 05C89}

% COPYRIGHT NOTICE
% Uncomment exactly one of the following copyright statements.  Alternatively,
% you can write your own copyright statement subject to the approval of the journal.
% See https://creativecommons.org/licenses/ for a full explanation of the 
% Creative Commons licenses.
%
% We strongly recommend CC BY-ND or CC BY. Both these licenses allow others
% to freely distribute your work while giving credit to you. The difference between
% them is that CC BY-ND only allows distribution unchanged and in whole, while
% CC BY also allows remixing, tweaking and building upon your work.  
%
%    One author:  ==========================
%\Copyright{The author.}
%\Copyright{The author. Released under the CC BY license (International 4.0).}
%\Copyright{The author. Released under the CC BY-ND license (International 4.0).}
%    More than one author: ===================
%\Copyright{The authors.}
%\Copyright{The authors. Released under the CC BY license (International 4.0).}
\Copyright{The authors. Released under the CC BY-ND license (International 4.0).}

% TITLE
% If needed, include a line break (\\) at an appropriate place in the title.
\title{Determinants and Inversion Properties of Digraph Path Matrices}


% AUTHORS
% Input author, affiliation, address and support information as follows;
% The address should include the country, but does not have to include
%    the street address. Give at least one email address.

\author{Seth Chaiken\\
\small Department of Computer Science (retired)\\[-0.8ex]
\small University at Albany\\[-0.8ex] 
\small Albany, New York, U.S.A.\\
\small\tt schaiken@albany.edu\\
\and
Jay Sulzberger\\
%\small School of Hard Knocks\\[-0.8ex]
%\small University of Western Nowhere\\[-0.8ex]
\small Yonkers, New York, U.S.A.\\
\small\tt jays@panix.com}

\begin{document}

\maketitle

% ABSTRACT
% E-JC papers must include an abstract. The abstract should consist of a
% succinct statement of background followed by a listing of the
% principal new results that are to be found in the paper. The abstract
% should be informative, clear, and as complete as possible. Phrases
% like "we investigate..." or "we study..." should be kept to a minimum
% in favor of "we prove that..."  or "we show that...".  Do not
% include equation numbers, unexpanded citations (such as "[23]"), or
% any other references to things in the paper that are not defined in
% the abstract. The abstract may be distributed without the rest of the
% paper so it must be entirely self-contained.  Try to include all words
% and phrases that someone might search for when looking for your paper.

\begin{abstract}
\end{abstract}

\section{Introduction}


\section{Setup and the Tree Theorem}

Matrices $M$ are indexed by the vertex set $V$ of a loopless, simple
directed graph $G$,
which is assumed to be linearly ordered, and, for submatrices,
subsets of $V$ with the same order.
When $R,C\subseteq V$, $M[R,C]$ denotes the submatrix with rows $R$ and columns $C$. The
complementary submatrix with rows $V\setminus R$ and columns $V\setminus C$ is denoted by
$M(R,C)$.  Using $ab$  in short for $\{a,b\}$ etc., $M(ab,cd)$ is the submatrix obtained by deleting
rows $a$ and $b$, and columns $c$ and $d$ from $M$, etc.  We remind the reader that the $[a,b]$ entry
of the $M$'s adjugate is $\sign(a,b)\det M(b,a)$.   Here $\sign(a,b)$ is $+$ or $-$ depending on
whether the parity of the number of vertices between and including $a,b$ in the order of $V$
is odd or even,
respectively.  When $V$ is the familiar $\{0, 1, \ldots, n\}$, $\sign(a,b)=(-1)^{(a+b)}$.
Cramer's rule is thus $M^{-1}[a,b]=(\sign(a,b)\det M(b,a))/\det M$ if $\det M \neq 0$.

Edges $ab$ of digraph $G$ are associated with distinct algebraically independent weights or
variables $x_{ab}$.  Some readers may choose to use a setup with variables $x_{ab}$
for all $a\neq b$ and consider digraphs to be determined by which $x_{ab}$ are not set to 0.  One
might also extend our setup and results
to digraphs with multiple edges by taking our $x_{ab}$ to be the
sum of the weights of all the edges from $a$ to $b$.  These we won't pursue. 

A \emph{walk} from vertex $a$ to vertex $b$ is a sequence of alternatingly
incident vertices and (directed) edges that begins with $a$, ends with $b$.  A walk may have
repeated vertices; if it has a repeated edge it must have repeated vertices.  A \emph{(simple) path}
is a walk that has no vertex appearing more than once.  For us, the term \emph{path} will always
mean \emph{simple path}. We will use \emph{simple} to indicate that this property is applied
in a proof.
Path 
$(a, ab, b, bc, \cdots, yz, z)$ is abbreviated
$(a, b, c, \cdots, z)$. The one and only path from $a$ to $a$ is $(a)$, a \emph{no-edge}
path.  It's unique since paths are simple.


\begin{definition}
  The \emph{path matrix}\cite{MR311496} $PM$ of digraph $G$ has rows and columns labeled by the vertices of $G$,
  and for $i,j$ vertices of $G$, entry
  \[ PM[i,j] = \Sigma_px_p
  \]
  where the sum is over all paths $p$ from $i$ to $j$ in $G$, and $x_p$ is the product of
  the weight variables over the edges of $p$.  The convention that all paths are simple implies
  $PM[i,i] = 1$.
\end{definition}

\begin{proposition}\label{Prop:GivenEdge}
  Suppose $ab$ is an edge in $G$. Then $\det PM(b,a) \neq 0$. Specifically,
  \[
  \det PM(b,a) = x_{ab}(1 + P) + Q
  \]
  where polynomial $P$ has no constant terms, and no term in $Q$ is a constant
  multiple of $x_{ab}$.  Thus, $\det PM(a,b)$ expanded into terms is
  $x_{ab} + \cdots \neq 0$.
\end{proposition}

\begin{proof}
  Taking an order where $a<b$ are the first two vertices, $PM$ has the following
  form.  Consider places of appearances of monomials within the entries of the matrix.
  Each monomial corresponds to a simple path.
  \[
  PM=\left[
    \begin{matrix}
      1      & x_{ab}+P & *      & \cdots & *  \\
      *      & 1       &  *     & \cdots & *  \\
      *      & *       &  1     & \cdots & *  \\
      \vdots & \vdots  & \vdots & \ddots & \vdots \\
      *      & *       & *      & \cdots &  1
    \end{matrix}
  \right],
  \]
  The only place a constant multiple of $x_{ab}$ appears is within $PM[ab]$. This is because
  every simple path whose endpoints are not $a,b$ must either be a no-edge path or
  contain at least one edge different from $ab$.  The only places
  1 appears are on the main diagonal, and no other constants appear anywhere else, since every
  no-edge path starts and ends at the same vertex. 
  When we now delete the first column ($a$) and the second row ($b$), the resulting diagonal
  is $(x_{ab} +P, 1, \cdots, 1)$.
  Therefore, when the deteminant is expanded into terms, no other term is a
  constant multiple of $x_{ab}$ besides $x_{ab}$ multiplied by the $1$s from the main diagonal.
  ($x_{ab}$ may of course appear linearly in other terms within the matrix.)
  \end{proof}


We are interested in situations where a kind of converse to Proposition~\ref{Prop:GivenEdge} is true.
If, for a particular digraph, $Q=0$ for all $a,b$ in the Proposition, then we would have
$\det PM(b,a) \neq 0$ \emph{if and only if} $ab$ is an edge in $G$!  Since $\pm\det PM(b,a)$ is the
$[a,b]$ entry of $PM$'s adjugate, and $\det PM\neq 0$, $PM^{-1}$ would have the same pattern of
zero versus non-zero entries in one variation of a weighted adjacency matrix of $G$, defined here.

\begin{definition}
  The weighted adjacency matrix $AM$ for digraph $G$ has entries $AM[i,j]=x_{ij}$ if $ij\in E(G)$,
  $AM[i,j]=0$ if $ij\not\in E(G)$, and $AM[i,i]=1$.
\end{definition}


\begin{definition}
  A digraph $G$ with path matrix $PM$  has the
  \emph{path matrix inversion property} when for all vertex pairs $[a,b]$,
  $det PM(b,a)\neq 0$ (equivalently, $PM^{-1}[a,b]\neq 0$) if and only if
  $a=b$ or $ab$ is an edge in $G$.
\end{definition}
  

\begin{definition}
  The \emph{digraph} of an undirected graph $G$ has the same vertices, with
  the edges occurring in pairs $ab$ and $ba$ if and only if $\{a,b\}$
  is an edge of $G$.  In other words, each undirected edge replaced by two
  oppositely directed edges.
\end{definition}

\begin{theorem}
  Digraphs of (undirected) trees all have the path matrix inversion property.
\end{theorem}

The smallest non-trivial case is, with $02\not\in\{01,10,12,21\}$,
\begin{center}
\resizebox{0.3\textwidth}{0.3\textwidth}{
\begin{tikzpicture}
\definecolor{cv0}{rgb}{0.0,0.0,0.0}
\definecolor{cfv0}{rgb}{1.0,1.0,1.0}
\definecolor{clv0}{rgb}{0.0,0.0,0.0}
\definecolor{cv1}{rgb}{0.0,0.0,0.0}
\definecolor{cfv1}{rgb}{1.0,1.0,1.0}
\definecolor{clv1}{rgb}{0.0,0.0,0.0}
\definecolor{cv2}{rgb}{0.0,0.0,0.0}
\definecolor{cfv2}{rgb}{1.0,1.0,1.0}
\definecolor{clv2}{rgb}{0.0,0.0,0.0}
\definecolor{cv0v1}{rgb}{0.0,0.0,0.0}
\definecolor{cv1v0}{rgb}{0.0,0.0,0.0}
\definecolor{cv1v2}{rgb}{0.0,0.0,0.0}
\definecolor{cv2v1}{rgb}{0.0,0.0,0.0}
%
\Vertex[style={minimum size=1.0cm,draw=cv0,fill=cfv0,text=clv0,shape=circle},LabelOut=false,L=\hbox{$0$},x=5.0cm,y=5.0cm]{v0}
\Vertex[style={minimum size=1.0cm,draw=cv1,fill=cfv1,text=clv1,shape=circle},LabelOut=false,L=\hbox{$1$},x=2.4445cm,y=2.4445cm]{v1}
\Vertex[style={minimum size=1.0cm,draw=cv2,fill=cfv2,text=clv2,shape=circle},LabelOut=false,L=\hbox{$2$},x=0.0cm,y=0.0cm]{v2}
%
\Edge[lw=0.1cm,style={post, bend right,color=cv0v1,},](v0)(v1)
\Edge[lw=0.1cm,style={post, bend right,color=cv1v0,},](v1)(v0)
\Edge[lw=0.1cm,style={post, bend right,color=cv1v2,},](v1)(v2)
\Edge[lw=0.1cm,style={post, bend right,color=cv2v1,},](v2)(v1)
%
\end{tikzpicture}
}
\end{center}
\[
PM=\left[
  \begin{matrix}
    1          & x_{01} & x_{01}x_{12} \\
    x_{10}      & 1     & x_{12}      \\
    x_{21}x_{10} & x_{21} & 1
  \end{matrix}
  \right];
\hspace{5mm}
PM(2,0) =
\left[
  \begin{matrix}
     x_{01} & x_{01}x_{12} \\
       1     & x_{12}      
  \end{matrix}
  \right]
\text{\ is singular.}
\]
In all,
\[
\det PM= 1  - x_{12}x_{21}  - x_{01}x_{10} + x_{01}x_{10}x_{12}x_{21}
\]
\[
(\det PM)PM^{-1}=\left[ \begin{matrix}
    1 - x_{12}x_{21}           & x_{01}(x_{12}x_{21} - 1)     & 0 \\
    x_{10}(x_{12}*x_{21} - 1) & 1 - x_{01}x_{10}x_{12}x_{21}  & x_{12}(x_{01}x_{10} - 1) \\
    0                        &  x_{21}(x_{01}x_{10} - 1)    & 1 - x_{01}x_{10}
  \end{matrix} \right]
\]
\[
AM = \left[ \begin{matrix}
    1     & x_{01}  & 0 \\
    x_{10} & 1      & x_{12} \\
    0     & x_{21}  & 1
  \end{matrix} \right]
\]
It's of interest compare $PM$ with the adjugate of $AM$:
\[
(\det AM)AM^{-1}= \left[ \begin{matrix}
    1 - x_{12}x_{21} & -x_{01}    & x_{01}x_{12} \\
    -x_{10}         &   1       &  -x_{12}      \\
    x_{21}x_{10}     &   -x_{21}  &  1 - x_{01}x_{10}
  \end{matrix} \right].
\]

\begin{proof}
  For any digraph with path matrix $PM$, if $ab$ is an edge,
  then $det PM(b,a)= x_{ab} + \cdots \neq 0$ is the result of Proposition~\ref{Prop:GivenEdge}.
  If $a=b$, then $det PM(a,a)= 1 + \cdots \neq 0$.

  Therefore, what remains to prove is that if
  $G$ is the digraph of a given tree, and $a\neq b$,
  if $ab$ is not an edge of $G$ (equivalently, $\{a,b\}$
  is not an edge in the tree) then $det PM(b,a) = 0$.  This will be proved partly by induction. For
  notational convenience, we take $V=\{0, 1, \ldots, n\}$ for the vertices.

  A \emph{leaf vertex} is a vertex that is a leaf in the tree.  For convenience, we take $0$ to
  be a leaf vertex and $1$ the (unique) vertex it is adjacent to.  The only simple paths between
  $0$ and $1$ are $(0,1)$ and $(1,0)$.  Further, the simple paths from $0$ to each $x\geq 2$
  are exactly $(0,1)$ followed by the simple paths from 
  $1$ to $x$.  The analogous property holds for simple paths from each $x\geq 2$ to $0$.
  So
 \begin{equation}\label{Eq:LeafPMForm}
   PM  = \left[
  \begin{matrix}
      1 & x_{01} & x_{01}[PM_{12}\;\;\; PM_{13} \;\; \cdots ] \\
      x_{10} & 1     & \;\;\;\;\; [PM_{12}\;\;\; PM_{13} \;\; \cdots ]\\
      x_{10}\hspace{-2mm}\left[ \begin{matrix}
          PM_{21} \\
          PM_{31} \\
          \vdots
        \end{matrix}
        \right]          &        
           \left[ \begin{matrix}
          PM_{21} \\
          PM_{31} \\
          \vdots
        \end{matrix}
        \right] &    \ddots
    \end{matrix}
  \right].
 \end{equation}
 Variables $x_{01}$ and $x_{10}$ appear in the polynomials of $PM$ only as explicitly shown
 in form (\ref{Eq:LeafPMForm}), because the enumerated paths are all simple.  Note then that
 $PM(0,0)$ is the path matrix of $G$ with vertex $0$ and edges $01$ and $10$ deleted.

 The cases of the theorem when $\{a,b\}=\{0,1\}$ hold because of Proposition~\ref{Prop:GivenEdge}.
   The cases when $a\neq 0$ and $b\neq 0$ hold by induction applied to $PM(0,0)$.

   The remaining cases, when $1\not\in\{a,b\}$, and $a=0$ or $b=0$, are proven without
   induction. The claim, when $a=0$,
   so $b\geq 2$ and $ab$ is not an edge, is $\det PM(b,0)=0$.
   Form~(\ref{Eq:LeafPMForm}) shows this because when column $0$ is deleted, row $0$
   becomes row $1$ multiplied by $x_{01}$. The $b=0$ case where $a\geq 2$ is analogous
   with the edge directions, and rows and columns, reversed.
\end{proof}

This proof also shows:
\begin{proposition}\label{Prop:PendantEdge}
  The path matrix inversion property of each digraph is unaffected by gluing, or ungluing and removing,
  a separate 2-cycle along any vertex.
\end{proposition}

%\begin{figure}[ht]
%  \centering
    % Use \includegraphics to import figures; for example 
    %      \includegraphics[scale=0.6]{filename}
%  \caption{Here is an informative figure.\label{fig:InformativeFigure}}
%\end{figure}

%%%%%%%%%%%%%%%%%%%%%%%%%%%%%%%%%%%%%%%%%%%%%%%%%
\subsection*{Acknowledgements}

%BIBLIOGRAPHY
% You do not have to use the same format for your references, but 
%    include everything in this file.
% If you use BibTeX to create a bibliography, copy the .bbl file into here.
% \newblock is optional (it adds a little space)

\bibliographystyle{amsplain}
\bibliography{PathsPaper}
\end{document}


\end{document}

\begin{thebibliography}{99}

\end{thebibliography}

\end{document}

\bibitem{Bollobas} B. Bollob{\'a}s. \newblock Almost every
  graph has reconstruction number three. \newblock \emph{J. Graph Theory},
  14(1):1--4, 1990.

\bibitem{BH} J.~A. Bondy and R. Hemminger,
\newblock Graph reconstruction---a survey.
\emph{J. Graph Theory}, 1:227--268, 1977. \doi{10.1002/jgt.3190010306}.

\bibitem{FGH} J.~Fisher, R.~L. Graham, and F.~Harary. \newblock A
  simpler counterexample to the reconstruction conjecture for
  denumerable graphs. \newblock \emph{J. Combinatorial Theory, Ser. B},
  12:203--204, 1972.

\bibitem{HHRT} E. Hemaspaandra, L.~A. Hemaspaandra,
  S.~P. Radziszowski, and R. Tripathi. \newblock
  Complexity results in graph reconstruction. \newblock \emph{Discrete
    Appl. Math.}, 155(2):103--118, 2007.

\bibitem{Kelly} P.~J. Kelly. \newblock A congruence theorem for
  trees. \newblock \emph{Pacific J. Math.}, 7:961--968, 1957.

\bibitem{KSU} M. Kiyomi, T. Saitoh, and R. Uehara.
  \newblock Reconstruction of interval graphs. \newblock In 
    \emph{Computing and combinatorics}, volume 5609 of
    \emph{Lecture Notes in Comput. Sci.}, pages 106--115. Springer, 2009.

\bibitem{Stockmeyer} P.~K. Stockmeyer. \newblock The falsity of the
  reconstruction conjecture for tournaments. \newblock \emph{J. Graph
    Theory}, 1(1):19--25, 1977.

\bibitem{Ulam} S.~M. Ulam. \newblock \newblock {A collection of
    mathematical problems}. \newblock Interscience Tracts in Pure and
  Applied Mathematics, no. 8.  Interscience Publishers, New
  York-London, 1960.
  
\bibitem{WS} D.~B. West and H. Spinoza.
 \newblock Reconstruction from $k$-decks for graphs with maximum degree~2.
 \newblock \arxiv{1609.00284vi}, 2016.




